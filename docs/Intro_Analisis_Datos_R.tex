% Options for packages loaded elsewhere
\PassOptionsToPackage{unicode}{hyperref}
\PassOptionsToPackage{hyphens}{url}
%
\documentclass[
]{book}
\usepackage{amsmath,amssymb}
\usepackage{lmodern}
\usepackage{ifxetex,ifluatex}
\ifnum 0\ifxetex 1\fi\ifluatex 1\fi=0 % if pdftex
  \usepackage[T1]{fontenc}
  \usepackage[utf8]{inputenc}
  \usepackage{textcomp} % provide euro and other symbols
\else % if luatex or xetex
  \usepackage{unicode-math}
  \defaultfontfeatures{Scale=MatchLowercase}
  \defaultfontfeatures[\rmfamily]{Ligatures=TeX,Scale=1}
\fi
% Use upquote if available, for straight quotes in verbatim environments
\IfFileExists{upquote.sty}{\usepackage{upquote}}{}
\IfFileExists{microtype.sty}{% use microtype if available
  \usepackage[]{microtype}
  \UseMicrotypeSet[protrusion]{basicmath} % disable protrusion for tt fonts
}{}
\makeatletter
\@ifundefined{KOMAClassName}{% if non-KOMA class
  \IfFileExists{parskip.sty}{%
    \usepackage{parskip}
  }{% else
    \setlength{\parindent}{0pt}
    \setlength{\parskip}{6pt plus 2pt minus 1pt}}
}{% if KOMA class
  \KOMAoptions{parskip=half}}
\makeatother
\usepackage{xcolor}
\IfFileExists{xurl.sty}{\usepackage{xurl}}{} % add URL line breaks if available
\IfFileExists{bookmark.sty}{\usepackage{bookmark}}{\usepackage{hyperref}}
\hypersetup{
  pdftitle={Introducción al Análisis de Datos con R},
  pdfauthor={Rubén Fernández Casal (ruben.fcasal@udc.es); Javier Roca-Pardiñas (roca@uvigo.es); Julián Costa Bouzas (julian.costa@udc.es); Manuel Oviedo de la Fuente (manuel.oviedo@udc.es)},
  hidelinks,
  pdfcreator={LaTeX via pandoc}}
\urlstyle{same} % disable monospaced font for URLs
\usepackage{color}
\usepackage{fancyvrb}
\newcommand{\VerbBar}{|}
\newcommand{\VERB}{\Verb[commandchars=\\\{\}]}
\DefineVerbatimEnvironment{Highlighting}{Verbatim}{commandchars=\\\{\}}
% Add ',fontsize=\small' for more characters per line
\usepackage{framed}
\definecolor{shadecolor}{RGB}{248,248,248}
\newenvironment{Shaded}{\begin{snugshade}}{\end{snugshade}}
\newcommand{\AlertTok}[1]{\textcolor[rgb]{0.94,0.16,0.16}{#1}}
\newcommand{\AnnotationTok}[1]{\textcolor[rgb]{0.56,0.35,0.01}{\textbf{\textit{#1}}}}
\newcommand{\AttributeTok}[1]{\textcolor[rgb]{0.77,0.63,0.00}{#1}}
\newcommand{\BaseNTok}[1]{\textcolor[rgb]{0.00,0.00,0.81}{#1}}
\newcommand{\BuiltInTok}[1]{#1}
\newcommand{\CharTok}[1]{\textcolor[rgb]{0.31,0.60,0.02}{#1}}
\newcommand{\CommentTok}[1]{\textcolor[rgb]{0.56,0.35,0.01}{\textit{#1}}}
\newcommand{\CommentVarTok}[1]{\textcolor[rgb]{0.56,0.35,0.01}{\textbf{\textit{#1}}}}
\newcommand{\ConstantTok}[1]{\textcolor[rgb]{0.00,0.00,0.00}{#1}}
\newcommand{\ControlFlowTok}[1]{\textcolor[rgb]{0.13,0.29,0.53}{\textbf{#1}}}
\newcommand{\DataTypeTok}[1]{\textcolor[rgb]{0.13,0.29,0.53}{#1}}
\newcommand{\DecValTok}[1]{\textcolor[rgb]{0.00,0.00,0.81}{#1}}
\newcommand{\DocumentationTok}[1]{\textcolor[rgb]{0.56,0.35,0.01}{\textbf{\textit{#1}}}}
\newcommand{\ErrorTok}[1]{\textcolor[rgb]{0.64,0.00,0.00}{\textbf{#1}}}
\newcommand{\ExtensionTok}[1]{#1}
\newcommand{\FloatTok}[1]{\textcolor[rgb]{0.00,0.00,0.81}{#1}}
\newcommand{\FunctionTok}[1]{\textcolor[rgb]{0.00,0.00,0.00}{#1}}
\newcommand{\ImportTok}[1]{#1}
\newcommand{\InformationTok}[1]{\textcolor[rgb]{0.56,0.35,0.01}{\textbf{\textit{#1}}}}
\newcommand{\KeywordTok}[1]{\textcolor[rgb]{0.13,0.29,0.53}{\textbf{#1}}}
\newcommand{\NormalTok}[1]{#1}
\newcommand{\OperatorTok}[1]{\textcolor[rgb]{0.81,0.36,0.00}{\textbf{#1}}}
\newcommand{\OtherTok}[1]{\textcolor[rgb]{0.56,0.35,0.01}{#1}}
\newcommand{\PreprocessorTok}[1]{\textcolor[rgb]{0.56,0.35,0.01}{\textit{#1}}}
\newcommand{\RegionMarkerTok}[1]{#1}
\newcommand{\SpecialCharTok}[1]{\textcolor[rgb]{0.00,0.00,0.00}{#1}}
\newcommand{\SpecialStringTok}[1]{\textcolor[rgb]{0.31,0.60,0.02}{#1}}
\newcommand{\StringTok}[1]{\textcolor[rgb]{0.31,0.60,0.02}{#1}}
\newcommand{\VariableTok}[1]{\textcolor[rgb]{0.00,0.00,0.00}{#1}}
\newcommand{\VerbatimStringTok}[1]{\textcolor[rgb]{0.31,0.60,0.02}{#1}}
\newcommand{\WarningTok}[1]{\textcolor[rgb]{0.56,0.35,0.01}{\textbf{\textit{#1}}}}
\usepackage{longtable,booktabs,array}
\usepackage{calc} % for calculating minipage widths
% Correct order of tables after \paragraph or \subparagraph
\usepackage{etoolbox}
\makeatletter
\patchcmd\longtable{\par}{\if@noskipsec\mbox{}\fi\par}{}{}
\makeatother
% Allow footnotes in longtable head/foot
\IfFileExists{footnotehyper.sty}{\usepackage{footnotehyper}}{\usepackage{footnote}}
\makesavenoteenv{longtable}
\usepackage{graphicx}
\makeatletter
\def\maxwidth{\ifdim\Gin@nat@width>\linewidth\linewidth\else\Gin@nat@width\fi}
\def\maxheight{\ifdim\Gin@nat@height>\textheight\textheight\else\Gin@nat@height\fi}
\makeatother
% Scale images if necessary, so that they will not overflow the page
% margins by default, and it is still possible to overwrite the defaults
% using explicit options in \includegraphics[width, height, ...]{}
\setkeys{Gin}{width=\maxwidth,height=\maxheight,keepaspectratio}
% Set default figure placement to htbp
\makeatletter
\def\fps@figure{htbp}
\makeatother
\setlength{\emergencystretch}{3em} % prevent overfull lines
\providecommand{\tightlist}{%
  \setlength{\itemsep}{0pt}\setlength{\parskip}{0pt}}
\setcounter{secnumdepth}{5}
\usepackage{booktabs}
\usepackage[a4paper, top=3.25cm, bottom=2.5cm, left=3cm, right=2.5cm]{geometry}
% \usepackage{fontspec}
% \setmainfont{Arial}
% \usepackage{amsthm}
% Espacio después de teorema
% Basado en https://tex.stackexchange.com/questions/37797/theorem-environment-line-break-after-label
% \newtheoremstyle{break}
%   {\topsep}{\topsep}% Space above and below
%   {\itshape}{}%       Body font, Indent amoun
%   {\bfseries}{}%      Theorem head font, Punctuation after theorem head
%   {\newline}%         Space after theorem head
%   {}%                 Theorem head spec (can be left empty, meaning ‘normal’ )
% Problemas con listas  https://tex.stackexchange.com/questions/8110/is-it-possible-to-skip-the-first-line-in-a-theorem-environment

\usepackage{ntheorem}
\theoremstyle{break}
\theoremheaderfont{\normalfont\bfseries}
\theorembodyfont{\normalfont}
\theorempreskip{\bigskipamount}
\theorempostskip{\smallskipamount}
\theoremprework{\bigskip\hrule\leavevmode}
\theoremseparator{\smallskip}
\theorempostwork{\bigskip\hrule\bigskip}
\newtheorem{theorem}{Teorema}[chapter]
\theoremprework{\bigskip\hrule\leavevmode}
\theorempostwork{\bigskip\hrule\bigskip}
\newtheorem{conjecture}{Algoritmo}[chapter]
\newtheorem{lemma}{Lema}[chapter]
\newtheorem{corollary}{Corolario}[chapter]
\newtheorem{proposition}{Proposición}[chapter]
\newtheorem{definition}{Definición}[chapter]
\newtheorem{hypothesis}{Hipótesis}[chapter]
\newtheorem{exercise}{Ejercicio}[chapter]
%\theoremprework{\bigskip\leavevmode}
%\theorempostwork{\vspace*{-\bigskipamount}\vspace*{-\medskipamount}}
\newtheorem{example}{Ejemplo}[chapter]
\theoremstyle{nonumberplain}
\theoremheaderfont{\normalfont\itshape}
\theoremseparator{:}
% \theorempostwork{\hrule}
\newtheorem{remark}{Nota}
\newtheorem{solution}{Solución}
\newtheorem{proof}{Demostración}


\ifxetex
  \usepackage{polyglossia}
  \setmainlanguage{spanish}
  % Tabla en lugar de cuadro
  \gappto\captionsspanish{\renewcommand{\tablename}{Tabla}
          \renewcommand{\listtablename}{Índice de tablas}}

\else
  \usepackage[spanish,es-tabla]{babel}
\fi
\makeatletter
\def\thm@space@setup{%
  \thm@preskip=8pt plus 2pt minus 4pt
  \thm@postskip=\thm@preskip
}
\makeatother
\ifluatex
  \usepackage{selnolig}  % disable illegal ligatures
\fi
\usepackage[]{natbib}
\bibliographystyle{apalike}

\title{Introducción al Análisis de Datos con \texttt{R}}
\author{Rubén Fernández Casal (\href{mailto:ruben.fcasal@udc.es}{\nolinkurl{ruben.fcasal@udc.es}}) \and Javier Roca-Pardiñas (\href{mailto:roca@uvigo.es}{\nolinkurl{roca@uvigo.es}}) \and Julián Costa Bouzas (\href{mailto:julian.costa@udc.es}{\nolinkurl{julian.costa@udc.es}}) \and Manuel Oviedo de la Fuente (\href{mailto:manuel.oviedo@udc.es}{\nolinkurl{manuel.oviedo@udc.es}})}
\date{Edición: Junio de 2022. Impresión: 2023-01-30. ISBN: 978-84-09-41823-7}

\begin{document}
\maketitle

{
\setcounter{tocdepth}{1}
\tableofcontents
}
\hypertarget{pruxf3logo}{%
\chapter*{Prólogo}\label{pruxf3logo}}
\addcontentsline{toc}{chapter}{Prólogo}

Este es un libro introductorio al análisis de datos con R.

En el Apéndice \ref{instalacion} se detallan los pasos para la instalación de \texttt{R} y el entorno de desarrollo RStudio.
En la Sección \protect\hyperlink{links}{Enlaces} de las Referencias se incluyen recursos adicionales, incluyendo algunos que pueden ser útiles para el aprendizaje de R.

Este libro ha sido escrito en \href{http://rmarkdown.rstudio.com}{R-Markdown} empleando el paquete \href{https://bookdown.org/yihui/bookdown/}{\texttt{bookdown}} y está disponible en el repositorio Github: \href{https://github.com/rubenfcasal/book_remuestreo}{rubenfcasal/intror}.
Se puede acceder a la versión en línea a través del siguiente enlace:

\url{https://rubenfcasal.github.io/intror}.

donde puede descargarse en formato \href{https://rubenfcasal.github.io/intror/Intro_Analisis_Datos_R.pdf}{pdf}.

Para ejecutar los ejemplos mostrados en el libro sería necesario tener instalados los siguientes paquetes:
\href{https://CRAN.R-project.org/package=lattice}{\texttt{lattice}}, \href{https://CRAN.R-project.org/package=ggplot2}{\texttt{ggplot2}}, \href{https://CRAN.R-project.org/package=foreign}{\texttt{foreign}}, \href{https://CRAN.R-project.org/package=car}{\texttt{car}}, \href{https://CRAN.R-project.org/package=leaps}{\texttt{leaps}}, \href{https://CRAN.R-project.org/package=MASS}{\texttt{MASS}}, \href{https://CRAN.R-project.org/package=RcmdrMisc}{\texttt{RcmdrMisc}}, \href{https://CRAN.R-project.org/package=lmtest}{\texttt{lmtest}}, \href{https://CRAN.R-project.org/package=glmnet}{\texttt{glmnet}}, \href{https://CRAN.R-project.org/package=mgcv}{\texttt{mgcv}}, \href{https://CRAN.R-project.org/package=rmarkdown}{\texttt{rmarkdown}}, \href{https://CRAN.R-project.org/package=knitr}{\texttt{knitr}}, \href{https://CRAN.R-project.org/package=dplyr}{\texttt{dplyr}}, \href{https://CRAN.R-project.org/package=tidyr}{\texttt{tidyr}}.
Por ejemplo mediante los siguientes comandos:

\begin{Shaded}
\begin{Highlighting}[]
\NormalTok{pkgs }\OtherTok{\textless{}{-}} \FunctionTok{c}\NormalTok{(}\StringTok{"lattice"}\NormalTok{, }\StringTok{"ggplot2"}\NormalTok{, }\StringTok{"foreign"}\NormalTok{, }\StringTok{"car"}\NormalTok{, }\StringTok{"leaps"}\NormalTok{, }\StringTok{"MASS"}\NormalTok{, }\StringTok{"RcmdrMisc"}\NormalTok{, }
          \StringTok{"lmtest"}\NormalTok{, }\StringTok{"glmnet"}\NormalTok{, }\StringTok{"mgcv"}\NormalTok{, }\StringTok{"rmarkdown"}\NormalTok{, }\StringTok{"knitr"}\NormalTok{, }\StringTok{"dplyr"}\NormalTok{, }\StringTok{"tidyr"}\NormalTok{)}
\FunctionTok{install.packages}\NormalTok{(}\FunctionTok{setdiff}\NormalTok{(pkgs, }\FunctionTok{installed.packages}\NormalTok{()[,}\StringTok{"Package"}\NormalTok{]), }\AttributeTok{dependencies =} \ConstantTok{TRUE}\NormalTok{)}
\end{Highlighting}
\end{Shaded}

(puede que haya que seleccionar el repositorio de descarga, e.g.~\emph{Spain (Madrid)}).

El código anterior no reinstala los paquetes ya instalados, por lo que podrían aparecer problemas debidos a incompatibilidades entre versiones (aunque no suele ocurrir, salvo que nuestra instalación de R esté muy desactualizada).
Si es el caso, en lugar de la última línea se puede ejecutar:

\begin{Shaded}
\begin{Highlighting}[]
\FunctionTok{install.packages}\NormalTok{(pkgs, }\AttributeTok{dependencies =} \ConstantTok{TRUE}\NormalTok{) }\CommentTok{\# Instala todos...}
\end{Highlighting}
\end{Shaded}

Para generar el libro (compilar) serán necesarios paquetes adicionales,
para lo que se recomendaría consultar el libro de \href{https://rubenfcasal.github.io/bookdown_intro}{``Escritura de libros con bookdown''} en castellano.

Este obra está bajo una licencia de \href{https://creativecommons.org/licenses/by-nc-nd/4.0/deed.es_ES}{Creative Commons Reconocimiento-NoComercial-SinObraDerivada 4.0 Internacional}
(esperamos poder liberarlo bajo una licencia menos restrictiva más adelante\ldots).

\includegraphics[width=1.22in]{by-nc-nd-88x31}

Para citar este libro se puede emplear la referencia:

Fernández-Casal R., Roca-Pardiñas J., Costa J. y Oviedo-de la Fuente M. (2022). \emph{Introducción al Análisis de Datos con R}. ISBN: 978-84-09-41823-7. \url{https://rubenfcasal.github.io/intror}.

También puede resultar de utilidad la siguiente entrada BibTeX:

\begin{verbatim}
@book{fernandezetal2022,
    title        = {Introducción al Análisis de Datos con R},
    author       = {Fernández-Casal, R.; Roca-Pardiñas, J.; Costa, J.;Oviedo-de la Fuente, M.},
    year         = {2022},
    note         = {ISBN 978-84-09-41823-7},
    url          = {https://rubenfcasal.github.io/intror/}
}
\end{verbatim}

\hypertarget{introducciuxf3n}{%
\chapter{Introducción}\label{introducciuxf3n}}

El entorno estadístico \texttt{R} puede ser una herramienta de gran
utilidad a lo largo de todo el proceso de obtención de información
a partir de datos (normalmente con el objetivo final de ayudar a tomar decisiones).

\begin{figure}[!htb]

{\centering \includegraphics[width=0.7\linewidth]{figuras/esquema2} 

}

\caption{Etapas del proceso}\label{fig:esquema}
\end{figure}

\hypertarget{el-lenguaje-y-entorno-estaduxedstico-r}{%
\section{\texorpdfstring{El lenguaje y entorno estadístico \texttt{R}}{El lenguaje y entorno estadístico R}}\label{el-lenguaje-y-entorno-estaduxedstico-r}}

\texttt{R} es un lenguaje de programación desarrollado específicamente para el
análisis estadístico y la visualización de datos.

\begin{itemize}
\item
  El lenguaje \texttt{R} es interpretado (similar a Matlab o Phyton) pero orientado al
  análisis estadístico (fórmulas modelos, factores,\ldots).

  \begin{itemize}
  \tightlist
  \item
    derivado del S (Laboratorios Bell).
  \end{itemize}
\item
  \texttt{R} es un \textbf{Software Libre} bajo las condiciones de licencia GPL de
  GNU, con código fuente de libre acceso.

  \begin{itemize}
  \tightlist
  \item
    Además de permitir crear \textbf{nuevas funciones},
    se pueden examinar y modificar las ya existentes.
  \end{itemize}
\item
  Multiplataforma,
  disponible para los sistemas operativos más populares (Linux, Windows, MacOS X, \ldots).
\end{itemize}

\hypertarget{principales-caracteruxedsticas}{%
\subsection{Principales características}\label{principales-caracteruxedsticas}}

Se pueden destacar las siguientes características del entorno \texttt{R}:

\begin{itemize}
\item
  Dispone de numerosos complementos (librerías, paquetes) que cubren ``literalmente'' todos los campos del análisis de datos.
\item
  Repositorios:

  \begin{itemize}
  \item
    \href{https://cran.r-project.org}{CRAN} (9705, 14972, 19122, \ldots)
  \item
    \href{https://www.bioconductor.org}{Bioconductor} (1289, 1741, 2183, \ldots),
  \item
    \href{https://github.com/trending/r?since=monthly}{GitHub}, \ldots{}
  \end{itemize}
\item
  Existe una comunidad de usuarios (programadores) muy dinámica
  (multitud de paquetes adicionales).
\item
  Muy bien documentado y con numerosos foros de ayuda.
\item
  Puntos débiles (a priori): velocidad, memoria, \ldots{}
\end{itemize}

Aunque inicialmente fue un lenguaje desarrollado por estadísticos
para estadísticos:

\begin{figure}[!htb]

{\centering \includegraphics[width=0.4\linewidth]{figuras/rexer2016} 

}

\caption{Rexer Data Miner Survey 2007-2015}\label{fig:rexer}
\end{figure}

Hoy en día es muy popular:

\begin{figure}[!htb]

{\centering \includegraphics[width=0.35\linewidth]{figuras/IEEE-top-programming-languages-of-2019} 

}

\caption{[IEEE Spectrum](https://spectrum.ieee.org) Top Programming Languages, 2019}\label{fig:ieee}
\end{figure}

\texttt{R} destaca especialmente en:

\begin{itemize}
\item
  Representaciones gráficas.
\item
  Métodos estadísticos ``avanzados'':

  \begin{itemize}
  \item
    \emph{Data Science}: \emph{Statistical Learning}, \emph{Data Mining},
    \emph{Machine Learning}, \emph{Business Intelligence}, \ldots{}
  \item
    Datos funcionales.
  \item
    Estadística espacial.
  \item
    \ldots{}
  \end{itemize}
\item
  Análisis de datos ``complejos'':

  \begin{itemize}
  \item
    Big Data.
  \item
    Lenguaje natural (\emph{Text Mining}).
  \item
    Análisis de redes.
  \item
    \ldots{}
  \end{itemize}
\end{itemize}

En el Apéndice \ref{instalacion} se detallan los pasos para la instalación de \texttt{R} y el entorno de desarrollo RStudio.
En la Sección \protect\hyperlink{links}{Enlaces} de las Referencias se incluyen recursos adicionales, incluyendo algunos que pueden ser útiles para el aprendizaje de R.

\hypertarget{interfaz-de-comandos}{%
\section{Interfaz de comandos}\label{interfaz-de-comandos}}

Normalmente se trabaja en \texttt{R} de forma interactiva empleando una \textbf{interfaz de comandos} donde se teclean las instrucciones que se pretenden ejecutar.
En Linux se trabaja directamente en el terminal de comandos y se inicia ejecutando el comando \texttt{R}.
En Windows se puede emplear el menú de inicio para ejecutar \texttt{R} (e.g.~abriendo \emph{R x64 X.Y.Z}) y se mostrará una ventana de consola que permite trabajar de modo interactivo (ver Figura \ref{fig:consola}).

\begin{figure}[!htb]

{\centering \includegraphics[width=0.7\linewidth]{figuras/consola} 

}

\caption{Consola de `R` en Windows (modo MDI).}\label{fig:consola}
\end{figure}

En la línea de comandos \texttt{R} muestra el carácter \texttt{\textgreater{}} (el \emph{prompt}) para indicar que está a la espera de instrucciones.
Para ejecutar una línea de instrucciones hay que pulsar \emph{Retorno} (y por defecto se imprime el resultado).

Por ejemplo, para obtener una secuencia de números desde el 1 hasta el 10, se utilizará la sentencia:

\begin{Shaded}
\begin{Highlighting}[]
\DecValTok{1}\SpecialCharTok{:}\DecValTok{10}
\end{Highlighting}
\end{Shaded}

obteniéndose el resultado

\begin{verbatim}
##  [1]  1  2  3  4  5  6  7  8  9 10
\end{verbatim}

Se pueden escribir varias instrucciones en una misma línea separándolas por ``;''.

\begin{Shaded}
\begin{Highlighting}[]
\DecValTok{2}\SpecialCharTok{+}\DecValTok{2}\NormalTok{; }\DecValTok{1}\SpecialCharTok{+}\DecValTok{2}\SpecialCharTok{*}\DecValTok{4}
\end{Highlighting}
\end{Shaded}

\begin{verbatim}
## [1] 4
\end{verbatim}

\begin{verbatim}
## [1] 9
\end{verbatim}

Si no se completó algún comando, el prompt cambia a \texttt{+} (habría que completar la instrucción anterior antes de escribir una nueva, o pulsar \emph{Escape} para cancelarla).

Se pueden recuperar líneas de instrucciones introducidas anteriormente pulsando la tecla \emph{Arriba}, a fin de re-ejecutarlas o modificarlas.

La ventana consola ejecuta de forma automática cada línea de comando.
Sin embargo, suele interesar guardar un conjunto de instrucciones en un único archivo de texto para formar lo que se conoce como un \emph{script} (archivo de código).
Las instrucciones del script se pueden pegar en la ventana de comandos para ser ejecutadas, pero también hay editores o entornos de desarrollo que permiten interactuar directamente con R.

Por ejemplo, en la consola de R en Windows se puede abrir una ventana de código seleccionando el menú \emph{Archivo \textgreater{} Nuevo script}.
Posteriormente se pueden ejecutar líneas de código pulsando \emph{Ctrl+R}.

\begin{figure}[!htb]

{\centering \includegraphics[width=0.7\linewidth]{figuras/script} 

}

\caption{Ventanas de la consola y de comandos en Windows (modo MDI).}\label{fig:script}
\end{figure}

Sin embargo, nosotros recomendamos emplear \emph{RStudio Desktop}.

\hypertarget{rstudio}{%
\section{El entorno de desarrollo RStudio Desktop}\label{rstudio}}

Al ejecutar RStudio se muestra la ventana principal:

\begin{figure}[!htb]

{\centering \includegraphics[width=0.7\linewidth]{figuras/rstudio} 

}

\caption{Ventana principal de RStudio.}\label{fig:rstudio}
\end{figure}

Por defecto RStudio está organizado en cuatro paneles:

\begin{itemize}
\item
  Editor de código (normalmente un fichero \emph{.R} o \emph{.Rmd}).
\item
  Consola de R (y terminal de comandos del sistema operativo).
\item
  Explorador del entorno e historial.
\item
  Explorador de archivos, visor de gráficos, ayuda y navegador web integrado.
\end{itemize}

Primeros pasos:

\begin{itemize}
\item
  Presionar \emph{Ctrl-Enter} (\emph{Command-Enter} en OS X) para ejecutar la línea de código actual o el código seleccionado (también se puede emplear el botón \emph{Run} en la barra de herramientas del Editor o el menú \emph{Code}).
\item
  Presionar \emph{Tab} para autocompletado.
\item
  Pulsar en el nombre del objeto en la pestaña \emph{Environment}, o ejecutar \texttt{View(objeto)} en la consola, para visualizar el objeto en una nueva pestaña del editor.
\end{itemize}

Información adicional:

\begin{itemize}
\item
  \href{https://posit.co/wp-content/uploads/2022/10/rstudio-ide-1.pdf}{RStudio cheatsheet}
\item
  \href{https://support.posit.co/hc/en-us/sections/200107586-Using-the-RStudio-IDE}{Using the RStudio IDE}
\end{itemize}

\hypertarget{ayuda}{%
\section{Ayuda}\label{ayuda}}

Se puede acceder a la ayuda empleando el entorno de comandos o los menús correspondientes de la intefaz gráfica.
Por ejemplo en RStudio se puede emplear el menú \emph{Help}, y en la consola de \texttt{R} el menú \emph{Ayuda \textgreater{} Manuales (en PDF)}.
Para acceder a la ayuda desde la interfaz de comandos se puede ejecutar \texttt{help.start()} (también puede ser de interés la función \texttt{demo()}).

Todas las funciones de \texttt{R} están documentadas.
Para obtener la ayuda de una determinada función se utilizará \texttt{help(función)} o de forma equivalente \texttt{?función}.

Por ejemplo, la ayuda de la función \texttt{rnorm} (utilizada para la generación de datos con distribución normal) se obtiene con el código

\begin{Shaded}
\begin{Highlighting}[]
\FunctionTok{help}\NormalTok{(rnorm)}
\NormalTok{?rnorm}
\end{Highlighting}
\end{Shaded}

En muchas ocasiones no se conoce el nombre exacto de la función de la que queremos obtener la documentación.
En estos casos, la función \texttt{help.search()} realiza búsquedas en la documentación en todos los paquetes instalados, estén cargados o no.
Por ejemplo, si no conocemos la función que permite calcular la mediana de un conjunto de datos, se puede utilizar

\begin{Shaded}
\begin{Highlighting}[]
\FunctionTok{help.search}\NormalTok{(}\StringTok{"median"}\NormalTok{)}
\end{Highlighting}
\end{Shaded}

Para más detalles véase \texttt{?help.search}

\hypertarget{primera-sesion}{%
\section{Una primera sesión}\label{primera-sesion}}

Como ya se comentó, al emplear la interfaz de comandos, el usuario puede ir ejecutando instrucciones y se va imprimiendo el resultado.
Por ejemplo:

\begin{Shaded}
\begin{Highlighting}[]
\DecValTok{3}\SpecialCharTok{+}\DecValTok{5}
\end{Highlighting}
\end{Shaded}

\begin{verbatim}
## [1] 8
\end{verbatim}

\begin{Shaded}
\begin{Highlighting}[]
\FunctionTok{sqrt}\NormalTok{(}\DecValTok{16}\NormalTok{) }\CommentTok{\# raiz cuadrada de 16}
\end{Highlighting}
\end{Shaded}

\begin{verbatim}
## [1] 4
\end{verbatim}

\begin{Shaded}
\begin{Highlighting}[]
\NormalTok{pi }\CommentTok{\# R reconoce el número pi}
\end{Highlighting}
\end{Shaded}

\begin{verbatim}
## [1] 3.141593
\end{verbatim}

Nótese que en los comandos se pueden hacer comentarios utilizando el símbolo \texttt{\#}.

Los resultados obtenidos pueden guardarse en objetos empleando el operador asignación \texttt{\textless{}-} (o \texttt{=}).
Por ejemplo, al ejecutar

\begin{Shaded}
\begin{Highlighting}[]
\NormalTok{a }\OtherTok{\textless{}{-}} \DecValTok{3} \SpecialCharTok{+} \DecValTok{5}
\end{Highlighting}
\end{Shaded}

el resultado de la suma se guarda en el objeto \texttt{a} (se crea o se reescribe si ya existía previamente).
Se puede comprobar si la asignación se ha realizado correctamente escribiendo el nombre del objeto (equivalente a ejecutar \texttt{print(a)})

\begin{Shaded}
\begin{Highlighting}[]
\NormalTok{a}
\end{Highlighting}
\end{Shaded}

\begin{verbatim}
## [1] 8
\end{verbatim}

Es importante señalar que \texttt{R} diferencia entre mayúsculas y
minúsculas, de modo que los objetos \texttt{a} y \texttt{A} serán diferentes.

\begin{Shaded}
\begin{Highlighting}[]
\NormalTok{a }\OtherTok{\textless{}{-}} \DecValTok{1}\SpecialCharTok{:}\DecValTok{10} \CommentTok{\# secuencia de números}
\NormalTok{A }\OtherTok{\textless{}{-}} \StringTok{"casa"}
\NormalTok{a}
\end{Highlighting}
\end{Shaded}

\begin{verbatim}
##  [1]  1  2  3  4  5  6  7  8  9 10
\end{verbatim}

\begin{Shaded}
\begin{Highlighting}[]
\NormalTok{A}
\end{Highlighting}
\end{Shaded}

\begin{verbatim}
## [1] "casa"
\end{verbatim}

\textbf{Nota}: Habitualmente no habrá diferencia entre la utilización de las asignaciones hechas con \texttt{=} y \texttt{\textless{}-} (aunque nosotros emplearemos el segundo).
Las diferencias aparecen a nivel de programación y se tratarán en el Capítulo \ref{programacion}.

Veamos ahora un ejemplo de un análisis exploratorio muy básico (de una variable numérica).
En el siguiente código:

\begin{itemize}
\item
  Se carga el objeto \texttt{precip} (uno de los conjuntos de datos de ejemplo disponibles en el paquete base de \texttt{R}) que contiene el promedio de precipitación, en pulgadas de lluvia, de 70 ciudades de Estados Unidos.
\item
  Se hace un resumen estadístico de los datos.
\item
  Se hace el correspondiente histograma y gráfico de cajas.
\end{itemize}

\begin{Shaded}
\begin{Highlighting}[]
\FunctionTok{data}\NormalTok{(precip)    }\CommentTok{\# Datos de lluvia}
\CommentTok{\# ?precip       \# Mostrar ayuda?}
\CommentTok{\# precip        \# Imprimir?}
\FunctionTok{summary}\NormalTok{(precip) }\CommentTok{\# Resumen estadístico}
\end{Highlighting}
\end{Shaded}

\begin{verbatim}
##    Min. 1st Qu.  Median    Mean 3rd Qu.    Max. 
##    7.00   29.38   36.60   34.89   42.77   67.00
\end{verbatim}

\begin{Shaded}
\begin{Highlighting}[]
\FunctionTok{hist}\NormalTok{(precip)    }\CommentTok{\# Histograma}
\end{Highlighting}
\end{Shaded}

\begin{center}\includegraphics[width=0.7\linewidth]{01-Introduccion_files/figure-latex/unnamed-chunk-11-1} \end{center}

\begin{Shaded}
\begin{Highlighting}[]
\FunctionTok{boxplot}\NormalTok{(precip) }\CommentTok{\# Gráfico de cajas}
\end{Highlighting}
\end{Shaded}

\begin{center}\includegraphics[width=0.7\linewidth]{01-Introduccion_files/figure-latex/unnamed-chunk-11-2} \end{center}

\hypertarget{funciones-paquetes}{%
\section{Funciones y librerías (paquetes)}\label{funciones-paquetes}}

Al iniciar el programa \texttt{R} se cargan por defecto una serie de librerías básicas con las que se pueden realizar una gran cantidad de operaciones empleando las funciones que implementan.
Estas librerías conforman el llamado \textbf{paquete base}.

En otras ocasiones es necesario cargar librerías adicionales, empleando los denominados paquetes (packages).
Normalmente se emplean los disponibles en el repositorio CRAN oficial
\url{http://cran.r-project.org/web/packages/}.

\hypertarget{funciones-internas}{%
\subsection{Funciones internas}\label{funciones-internas}}

Las llamadas a una función son de la forma \texttt{nombre\_función(argumento1,\ argumento2,\ ...)} y típicamente al evaluarlas devuelven un objeto con los resultados (o generan un gráfico).
Los argumentos pueden tener nombres (se asignan por posición ó nombre) y valores por defecto (solo es necesario especificarlos para asignarles un valor distinto).
Un nombre seguido de paréntesis hace siempre referencia a una función (realmente es un tipo de objeto y si por ejemplo se introduce solo el nombre en la línea de comandos simplemente se imprime el código).

\begin{Shaded}
\begin{Highlighting}[]
\NormalTok{x }\OtherTok{\textless{}{-}} \FunctionTok{sin}\NormalTok{(pi}\SpecialCharTok{/}\DecValTok{2}\NormalTok{) }
\CommentTok{\# La función \textasciigrave{}sin()\textasciigrave{} y el objeto \textasciigrave{}pi\textasciigrave{} están en el paquete base}
\FunctionTok{cat}\NormalTok{(}\StringTok{"El objeto x contiene:"}\NormalTok{, x, }\StringTok{"}\SpecialCharTok{\textbackslash{}n}\StringTok{"}\NormalTok{)}
\end{Highlighting}
\end{Shaded}

\begin{verbatim}
## El objeto x contiene: 1
\end{verbatim}

El parámetro \texttt{...} aglutina los argumentos no definidos explícitamente (cuando la función puede operar sobre múltiples argumentos, e.g.~\texttt{cat(...)}, o para poder incluir parámetros de otra función a la que se llama internamente).

Algunas funciones se comportan de manera diferente dependiendo del tipo de objeto (la clase) de sus argumentos, son lo que se denominan \emph{funciones genéricas}.
Entre ellas \texttt{summary()}, \texttt{print()}, \texttt{plot()} (por ejemplo, al ejecutar \texttt{methods(plot)} se muestran los métodos asociados esta función; el método por defecto es \texttt{plot.default()}).

\hypertarget{instalacion-pkg}{%
\subsection{Paquetes}\label{instalacion-pkg}}

La instalación de un paquete se puede hacer de varias formas:

\begin{itemize}
\item
  Desde la interfaz de comandos utilizando la instrucción

\begin{Shaded}
\begin{Highlighting}[]
\FunctionTok{install.packages}\NormalTok{(}\StringTok{"nombre del paquete"}\NormalTok{)}
\end{Highlighting}
\end{Shaded}
\item
  Desde el correspondiente menú de la intefaz gráfica (\emph{Paquetes \textgreater{} Instalar paquete(s)\ldots{}} en la consola de R y \emph{Tools \textgreater{} Install packages\ldots{}} o la pestaña \emph{Packages} en RStudio).
\end{itemize}

Este proceso sólo es necesario realizarlo la primera vez que se utilice el paquete.

Para utilizar un paquete ya instalado será necesario cargarlo, lo cual se puede hacer de varias formas:

\begin{itemize}
\item
  Desde el menú \emph{Paquetes \textgreater{} Cargar paquete(s)\ldots{}}
\item
  Por consola, utilizando \texttt{library(paquete)} (también \texttt{require(paquete)})
\end{itemize}

Esta operación será necesario realizarla cada vez que se inicie una
sesión de R.

Finalmente, la ayuda de un paquete se puede obtener con la sentencia

\begin{Shaded}
\begin{Highlighting}[]
\FunctionTok{library}\NormalTok{(}\AttributeTok{help =} \StringTok{"nombre del paquete"}\NormalTok{) }
\end{Highlighting}
\end{Shaded}

\hypertarget{objetos-buxe1sicos}{%
\section{Objetos básicos}\label{objetos-buxe1sicos}}

\texttt{R} es un lenguaje \textbf{orientado a objetos} lo que significa que las variables, datos, funciones, resultados, etc., se guardan en la memoria del ordenador en forma de \emph{objetos} con un nombre específico.

Los principales tipos de valores básicos de \texttt{R} son:

\begin{itemize}
\item
  numéricos,
\item
  cadenas de caracteres, y
\item
  lógicos
\end{itemize}

\hypertarget{objetos-numuxe9ricos}{%
\subsection{Objetos numéricos}\label{objetos-numuxe9ricos}}

Los valores numéricos adoptan
la notación habitual en informática: punto decimal, notacion científica, \ldots{}

\begin{Shaded}
\begin{Highlighting}[]
\NormalTok{pi}
\end{Highlighting}
\end{Shaded}

\begin{verbatim}
## [1] 3.141593
\end{verbatim}

\begin{Shaded}
\begin{Highlighting}[]
\FloatTok{1e3}
\end{Highlighting}
\end{Shaded}

\begin{verbatim}
## [1] 1000
\end{verbatim}

Con este tipo de objetos se pueden hacer operaciones aritméticas
utilizando el operador correspondiente.

\begin{Shaded}
\begin{Highlighting}[]
\NormalTok{a }\OtherTok{\textless{}{-}} \FloatTok{3.4}
\NormalTok{b }\OtherTok{\textless{}{-}} \FloatTok{4.5}
\NormalTok{a }\SpecialCharTok{*}\NormalTok{ b}
\end{Highlighting}
\end{Shaded}

\begin{verbatim}
## [1] 15.3
\end{verbatim}

\begin{Shaded}
\begin{Highlighting}[]
\NormalTok{a }\SpecialCharTok{/}\NormalTok{ b}
\end{Highlighting}
\end{Shaded}

\begin{verbatim}
## [1] 0.7555556
\end{verbatim}

\begin{Shaded}
\begin{Highlighting}[]
\NormalTok{a }\SpecialCharTok{+}\NormalTok{ b}
\end{Highlighting}
\end{Shaded}

\begin{verbatim}
## [1] 7.9
\end{verbatim}

\begin{Shaded}
\begin{Highlighting}[]
\FunctionTok{min}\NormalTok{(a, b)}
\end{Highlighting}
\end{Shaded}

\begin{verbatim}
## [1] 3.4
\end{verbatim}

\hypertarget{objetos-tipo-caruxe1cter}{%
\subsection{Objetos tipo carácter}\label{objetos-tipo-caruxe1cter}}

Las cadenas de caracteres
se introducen delimitadas por comillas (``nombre'') o por apóstrofos
(`nombre').

\begin{Shaded}
\begin{Highlighting}[]
\NormalTok{a }\OtherTok{\textless{}{-}} \StringTok{"casa grande"}
\NormalTok{a}
\end{Highlighting}
\end{Shaded}

\begin{verbatim}
## [1] "casa grande"
\end{verbatim}

\begin{Shaded}
\begin{Highlighting}[]
\NormalTok{a }\OtherTok{\textless{}{-}} \StringTok{\textquotesingle{}casa grande\textquotesingle{}}
\NormalTok{a}
\end{Highlighting}
\end{Shaded}

\begin{verbatim}
## [1] "casa grande"
\end{verbatim}

\begin{Shaded}
\begin{Highlighting}[]
\NormalTok{a }\OtherTok{\textless{}{-}} \StringTok{\textquotesingle{}casa "grande"\textquotesingle{}}
\NormalTok{a}
\end{Highlighting}
\end{Shaded}

\begin{verbatim}
## [1] "casa \"grande\""
\end{verbatim}

\hypertarget{objetos-luxf3gicos}{%
\subsection{Objetos lógicos}\label{objetos-luxf3gicos}}

Los objetos lógicos sólo pueden
tomar dos valores \texttt{TRUE} (numéricamente toma el valor 1) y \texttt{FALSE}
(valor 0).

\begin{Shaded}
\begin{Highlighting}[]
\NormalTok{A }\OtherTok{\textless{}{-}} \ConstantTok{TRUE}
\NormalTok{B }\OtherTok{\textless{}{-}} \ConstantTok{FALSE}
\NormalTok{A}
\end{Highlighting}
\end{Shaded}

\begin{verbatim}
## [1] TRUE
\end{verbatim}

\begin{Shaded}
\begin{Highlighting}[]
\NormalTok{B}
\end{Highlighting}
\end{Shaded}

\begin{verbatim}
## [1] FALSE
\end{verbatim}

\begin{Shaded}
\begin{Highlighting}[]
\CommentTok{\# valores numéricos}
\FunctionTok{as.numeric}\NormalTok{(A)}
\end{Highlighting}
\end{Shaded}

\begin{verbatim}
## [1] 1
\end{verbatim}

\begin{Shaded}
\begin{Highlighting}[]
\FunctionTok{as.numeric}\NormalTok{(B)}
\end{Highlighting}
\end{Shaded}

\begin{verbatim}
## [1] 0
\end{verbatim}

\hypertarget{operadores-luxf3gicos}{%
\subsection{Operadores lógicos}\label{operadores-luxf3gicos}}

Existen varios operadores en
\texttt{R} que devuelven un valor de tipo lógico. Veamos algún ejemplo

\begin{Shaded}
\begin{Highlighting}[]
\NormalTok{a }\OtherTok{\textless{}{-}} \DecValTok{2}
\NormalTok{b }\OtherTok{\textless{}{-}} \DecValTok{3}
\NormalTok{a }\SpecialCharTok{==}\NormalTok{ b  }\CommentTok{\# compara a y b}
\end{Highlighting}
\end{Shaded}

\begin{verbatim}
## [1] FALSE
\end{verbatim}

\begin{Shaded}
\begin{Highlighting}[]
\NormalTok{a }\SpecialCharTok{==}\NormalTok{ a  }\CommentTok{\# compara a y a}
\end{Highlighting}
\end{Shaded}

\begin{verbatim}
## [1] TRUE
\end{verbatim}

\begin{Shaded}
\begin{Highlighting}[]
\NormalTok{a }\SpecialCharTok{\textless{}}\NormalTok{ b}
\end{Highlighting}
\end{Shaded}

\begin{verbatim}
## [1] TRUE
\end{verbatim}

\begin{Shaded}
\begin{Highlighting}[]
\NormalTok{b }\SpecialCharTok{\textless{}}\NormalTok{ a}
\end{Highlighting}
\end{Shaded}

\begin{verbatim}
## [1] FALSE
\end{verbatim}

\begin{Shaded}
\begin{Highlighting}[]
\SpecialCharTok{!}\NormalTok{ (b }\SpecialCharTok{\textless{}}\NormalTok{ a) }\CommentTok{\# ! niega la condición}
\end{Highlighting}
\end{Shaded}

\begin{verbatim}
## [1] TRUE
\end{verbatim}

\begin{Shaded}
\begin{Highlighting}[]
\DecValTok{2}\SpecialCharTok{**}\DecValTok{2} \SpecialCharTok{==} \DecValTok{2}\SpecialCharTok{\^{}}\DecValTok{2}
\end{Highlighting}
\end{Shaded}

\begin{verbatim}
## [1] TRUE
\end{verbatim}

\begin{Shaded}
\begin{Highlighting}[]
\DecValTok{3}\SpecialCharTok{*}\DecValTok{2} \SpecialCharTok{==} \DecValTok{3}\SpecialCharTok{\^{}}\DecValTok{2}
\end{Highlighting}
\end{Shaded}

\begin{verbatim}
## [1] FALSE
\end{verbatim}

Nótese la diferencia entre \texttt{=} (asignación) y \texttt{==} (operador lógico)

\begin{Shaded}
\begin{Highlighting}[]
\DecValTok{2} \SpecialCharTok{==} \DecValTok{3}
\end{Highlighting}
\end{Shaded}

\begin{verbatim}
## [1] FALSE
\end{verbatim}

\begin{Shaded}
\begin{Highlighting}[]
\CommentTok{\# 2 = 3 \# produce un error:}
\CommentTok{\# Error en 2 = 3 : lado izquierdo de la asignación inválida (do\_set)}
\end{Highlighting}
\end{Shaded}

Se pueden encadenar varias condiciones lógicas utiilizando
los operadores \texttt{\&} (y lógico) y \texttt{\textbar{}} (o lógico).

\begin{Shaded}
\begin{Highlighting}[]
\ConstantTok{TRUE} \SpecialCharTok{\&} \ConstantTok{TRUE}
\end{Highlighting}
\end{Shaded}

\begin{verbatim}
## [1] TRUE
\end{verbatim}

\begin{Shaded}
\begin{Highlighting}[]
\ConstantTok{TRUE} \SpecialCharTok{|} \ConstantTok{TRUE}
\end{Highlighting}
\end{Shaded}

\begin{verbatim}
## [1] TRUE
\end{verbatim}

\begin{Shaded}
\begin{Highlighting}[]
\ConstantTok{TRUE} \SpecialCharTok{\&} \ConstantTok{FALSE}
\end{Highlighting}
\end{Shaded}

\begin{verbatim}
## [1] FALSE
\end{verbatim}

\begin{Shaded}
\begin{Highlighting}[]
\ConstantTok{TRUE} \SpecialCharTok{|} \ConstantTok{FALSE}
\end{Highlighting}
\end{Shaded}

\begin{verbatim}
## [1] TRUE
\end{verbatim}

\begin{Shaded}
\begin{Highlighting}[]
\DecValTok{2} \SpecialCharTok{\textless{}} \DecValTok{3} \SpecialCharTok{\&} \DecValTok{3} \SpecialCharTok{\textless{}} \DecValTok{1}
\end{Highlighting}
\end{Shaded}

\begin{verbatim}
## [1] FALSE
\end{verbatim}

\begin{Shaded}
\begin{Highlighting}[]
\DecValTok{2} \SpecialCharTok{\textless{}} \DecValTok{3} \SpecialCharTok{|} \DecValTok{3} \SpecialCharTok{\textless{}} \DecValTok{1}
\end{Highlighting}
\end{Shaded}

\begin{verbatim}
## [1] TRUE
\end{verbatim}

\hypertarget{uxe1rea-de-trabajo}{%
\section{Área de trabajo}\label{uxe1rea-de-trabajo}}

Como ya se ha comentado con anterioridad es posible guardar los comandos que se han utilizado en una sesión en ficheros llamados \textbf{script}.
En ocasiones interesará además guardar todos los objetos que han sido generados a lo largo de una sesión de trabajo.

El \textbf{Workspace} o \textbf{Área de Trabajo} es el entorno en el que se almacenan todos los objetos creados en una sesión.
Se puede guardar este entorno en el disco de forma que la próxima vez que se inicie el programa, al cargar dicho entorno, se pueda acceder a lo objetos almacenados en él.

En primer lugar, para saber los objetos que tenemos en memoria se utiliza la función \texttt{ls()}.
Por ejemplo, supongamos que acabamos de iniciar una sesión de \texttt{R} y hemos escrito

\begin{Shaded}
\begin{Highlighting}[]
\NormalTok{a }\OtherTok{\textless{}{-}} \DecValTok{1}\SpecialCharTok{:}\DecValTok{10}
\NormalTok{b }\OtherTok{\textless{}{-}} \FunctionTok{log}\NormalTok{(}\DecValTok{50}\NormalTok{)}
\end{Highlighting}
\end{Shaded}

Entonces al utilizar \texttt{ls()} se obtendrá la siguiente lista de objetos en memoria

\begin{Shaded}
\begin{Highlighting}[]
\FunctionTok{ls}\NormalTok{()}
\end{Highlighting}
\end{Shaded}

\begin{verbatim}
## [1] "a" "b"
\end{verbatim}

Los objetos se pueden eliminar empleando la función \texttt{rm()}.

\begin{Shaded}
\begin{Highlighting}[]
\FunctionTok{rm}\NormalTok{(b)}
\FunctionTok{ls}\NormalTok{()}
\end{Highlighting}
\end{Shaded}

\begin{verbatim}
## [1] "a"
\end{verbatim}

Para borrar todos los objetos en memoria se puede utilizar \texttt{rm(list=ls())}.

\begin{Shaded}
\begin{Highlighting}[]
\FunctionTok{rm}\NormalTok{(}\AttributeTok{list =} \FunctionTok{ls}\NormalTok{())}
\end{Highlighting}
\end{Shaded}

\begin{verbatim}
## character(0)
\end{verbatim}

\texttt{character(0)} (cadena de texto vacía) significa que no hay objetos en memoria.

\hypertarget{guardar-y-cargar-objetos}{%
\subsection{Guardar y cargar objetos}\label{guardar-y-cargar-objetos}}

Para guardar el área de trabajo (Workspace) con todos los objetos de memoria (es decir, los que figuran al utilizar \texttt{ls()}) se utiliza la función \texttt{save.image(nombre\ archivo)}.

\begin{Shaded}
\begin{Highlighting}[]
\FunctionTok{rm}\NormalTok{(}\AttributeTok{list =} \FunctionTok{ls}\NormalTok{()) }\CommentTok{\# borramos todos los objetos en memoria}
\NormalTok{x }\OtherTok{\textless{}{-}} \DecValTok{20}
\NormalTok{y }\OtherTok{\textless{}{-}} \DecValTok{34}
\NormalTok{z }\OtherTok{\textless{}{-}} \StringTok{"casa"}
\FunctionTok{save.image}\NormalTok{(}\AttributeTok{file =} \StringTok{"prueba.RData"}\NormalTok{) }\CommentTok{\# guarda área de trabajo en prueba.RData}
\end{Highlighting}
\end{Shaded}

La función \texttt{save()} permite guardar los objetos especificados.

\begin{Shaded}
\begin{Highlighting}[]
\FunctionTok{save}\NormalTok{(x, y, }\AttributeTok{file =} \StringTok{"prueba2.RData"}\NormalTok{) }\CommentTok{\# guarda los objetos x e y}
\end{Highlighting}
\end{Shaded}

Para cargar los objetos almacenados en un archivo se utiliza la función \texttt{load()}.

\begin{Shaded}
\begin{Highlighting}[]
\FunctionTok{load}\NormalTok{(}\StringTok{"prueba2.RData"}\NormalTok{) }\CommentTok{\# carga los objetos x e y}
\end{Highlighting}
\end{Shaded}

\hypertarget{directorio-de-trabajo}{%
\subsection{Directorio de trabajo}\label{directorio-de-trabajo}}

Por defecto \texttt{R} utiliza una carpeta de trabajo donde guardará toda la información.
Dicha carpeta se puede obtener con la función

\begin{Shaded}
\begin{Highlighting}[]
\FunctionTok{getwd}\NormalTok{() }
\end{Highlighting}
\end{Shaded}

\begin{verbatim}
## [1] "d:/"
\end{verbatim}

El directorio de trabajo se puede cambiar utilizando \texttt{setwd(directorio)}.
Por ejemplo, para cambiar el directorio de trabajo a \texttt{c:\textbackslash{}datos}, se utiliza el comando

\begin{Shaded}
\begin{Highlighting}[]
\FunctionTok{setwd}\NormalTok{(}\StringTok{"c:/datos"}\NormalTok{)}
\CommentTok{\# Importante podemos emplear \textquotesingle{}/\textquotesingle{} o \textquotesingle{}\textbackslash{}\textbackslash{}\textquotesingle{} como separador en la ruta}
\CommentTok{\# NO funciona setwd("c:\textbackslash{}datos")}
\end{Highlighting}
\end{Shaded}

\hypertarget{estructuras-de-datos}{%
\chapter{Estructuras de datos}\label{estructuras-de-datos}}

En los ejemplos que hemos visto hasta ahora los objetos de \texttt{R} almacenaban un único valor cada uno.
Sin embargo, las estructuras de datos que proporciona \texttt{R} permiten almacenar en un mismo objeto varios valores.
Las principales estructuras son:

\begin{itemize}
\item
  Vectores
\item
  Matrices y Arrays
\item
  Data Frames
\item
  Listas
\end{itemize}

\hypertarget{vectores}{%
\section{Vectores}\label{vectores}}

Un vector es un conjunto de valores básicos del mismo tipo.
La forma más sencilla de crear vectores es a
través de la función \texttt{c()} que se usa para combinar (concatenar) valores.

\begin{Shaded}
\begin{Highlighting}[]
\NormalTok{x }\OtherTok{\textless{}{-}} \FunctionTok{c}\NormalTok{(}\DecValTok{3}\NormalTok{, }\DecValTok{5}\NormalTok{, }\DecValTok{7}\NormalTok{)}
\NormalTok{x}
\end{Highlighting}
\end{Shaded}

\begin{verbatim}
## [1] 3 5 7
\end{verbatim}

\begin{Shaded}
\begin{Highlighting}[]
\NormalTok{y }\OtherTok{\textless{}{-}} \FunctionTok{c}\NormalTok{(}\DecValTok{8}\NormalTok{, }\DecValTok{9}\NormalTok{)}
\NormalTok{y}
\end{Highlighting}
\end{Shaded}

\begin{verbatim}
## [1] 8 9
\end{verbatim}

\begin{Shaded}
\begin{Highlighting}[]
\FunctionTok{c}\NormalTok{(x, y)}
\end{Highlighting}
\end{Shaded}

\begin{verbatim}
## [1] 3 5 7 8 9
\end{verbatim}

\begin{Shaded}
\begin{Highlighting}[]
\NormalTok{z }\OtherTok{\textless{}{-}} \FunctionTok{c}\NormalTok{(}\StringTok{"Hola"}\NormalTok{, }\StringTok{"Adios"}\NormalTok{)}
\NormalTok{z}
\end{Highlighting}
\end{Shaded}

\begin{verbatim}
## [1] "Hola"  "Adios"
\end{verbatim}

\hypertarget{generaciuxf3n-de-secuencias}{%
\subsection{Generación de secuencias}\label{generaciuxf3n-de-secuencias}}

Existen varias funciones que pemiten obtener secuencias de números

\begin{Shaded}
\begin{Highlighting}[]
\NormalTok{x }\OtherTok{\textless{}{-}} \DecValTok{1}\SpecialCharTok{:}\DecValTok{5}
\NormalTok{x}
\end{Highlighting}
\end{Shaded}

\begin{verbatim}
## [1] 1 2 3 4 5
\end{verbatim}

\begin{Shaded}
\begin{Highlighting}[]
\FunctionTok{seq}\NormalTok{(}\DecValTok{1}\NormalTok{, }\DecValTok{5}\NormalTok{, }\FloatTok{0.5}\NormalTok{)}
\end{Highlighting}
\end{Shaded}

\begin{verbatim}
## [1] 1.0 1.5 2.0 2.5 3.0 3.5 4.0 4.5 5.0
\end{verbatim}

\begin{Shaded}
\begin{Highlighting}[]
\FunctionTok{seq}\NormalTok{(}\AttributeTok{from=}\DecValTok{1}\NormalTok{, }\AttributeTok{to=}\DecValTok{5}\NormalTok{, }\AttributeTok{length=}\DecValTok{9}\NormalTok{)}
\end{Highlighting}
\end{Shaded}

\begin{verbatim}
## [1] 1.0 1.5 2.0 2.5 3.0 3.5 4.0 4.5 5.0
\end{verbatim}

\begin{Shaded}
\begin{Highlighting}[]
\FunctionTok{rep}\NormalTok{(}\DecValTok{1}\NormalTok{, }\DecValTok{5}\NormalTok{)}
\end{Highlighting}
\end{Shaded}

\begin{verbatim}
## [1] 1 1 1 1 1
\end{verbatim}

\hypertarget{generaciuxf3n-secuencias-aleatorias}{%
\subsection{Generación secuencias aleatorias}\label{generaciuxf3n-secuencias-aleatorias}}

A continuación se obtiene una simulación de 10 lanzamientos de un dado

\begin{Shaded}
\begin{Highlighting}[]
\FunctionTok{sample}\NormalTok{(}\DecValTok{1}\SpecialCharTok{:}\DecValTok{6}\NormalTok{, }\AttributeTok{size=}\DecValTok{10}\NormalTok{, }\AttributeTok{replace =}\NormalTok{ T) }\CommentTok{\# lanzamiento de un dado}
\end{Highlighting}
\end{Shaded}

\begin{verbatim}
##  [1] 6 3 1 1 5 5 5 3 6 6
\end{verbatim}

Para simular el lanzamiento de una moneda podemos escribir

\begin{Shaded}
\begin{Highlighting}[]
\NormalTok{resultado }\OtherTok{\textless{}{-}} \FunctionTok{c}\NormalTok{(}\AttributeTok{cara =} \DecValTok{1}\NormalTok{, }\AttributeTok{cruz =} \DecValTok{0}\NormalTok{) }\CommentTok{\# se asignan nombres a los componentes}
\FunctionTok{print}\NormalTok{(resultado)}
\end{Highlighting}
\end{Shaded}

\begin{verbatim}
## cara cruz 
##    1    0
\end{verbatim}

\begin{Shaded}
\begin{Highlighting}[]
\FunctionTok{class}\NormalTok{(resultado)}
\end{Highlighting}
\end{Shaded}

\begin{verbatim}
## [1] "numeric"
\end{verbatim}

\begin{Shaded}
\begin{Highlighting}[]
\FunctionTok{attributes}\NormalTok{(resultado)}
\end{Highlighting}
\end{Shaded}

\begin{verbatim}
## $names
## [1] "cara" "cruz"
\end{verbatim}

\begin{Shaded}
\begin{Highlighting}[]
\FunctionTok{names}\NormalTok{(resultado)}
\end{Highlighting}
\end{Shaded}

\begin{verbatim}
## [1] "cara" "cruz"
\end{verbatim}

\begin{Shaded}
\begin{Highlighting}[]
\NormalTok{lanz }\OtherTok{\textless{}{-}} \FunctionTok{sample}\NormalTok{(resultado, }\AttributeTok{size=}\DecValTok{10}\NormalTok{, }\AttributeTok{replace =}\NormalTok{ T)}
\NormalTok{lanz}
\end{Highlighting}
\end{Shaded}

\begin{verbatim}
## cara cruz cara cruz cara cara cara cara cara cruz 
##    1    0    1    0    1    1    1    1    1    0
\end{verbatim}

\begin{Shaded}
\begin{Highlighting}[]
\FunctionTok{table}\NormalTok{(lanz)}
\end{Highlighting}
\end{Shaded}

\begin{verbatim}
## lanz
## 0 1 
## 3 7
\end{verbatim}

Otros ejemplos

\begin{Shaded}
\begin{Highlighting}[]
\FunctionTok{rnorm}\NormalTok{(}\DecValTok{10}\NormalTok{)  }\CommentTok{\# rnorm(10, mean = 0, sd = 1)}
\end{Highlighting}
\end{Shaded}

\begin{verbatim}
##  [1]  1.06057423 -1.13112660 -0.55005465 -0.06400286  0.37460541  0.90174865
##  [7]  0.60904681 -0.06503237 -0.72623274  0.99699218
\end{verbatim}

\begin{Shaded}
\begin{Highlighting}[]
\FunctionTok{runif}\NormalTok{(}\DecValTok{15}\NormalTok{, }\AttributeTok{min =} \DecValTok{2}\NormalTok{, }\AttributeTok{max =} \DecValTok{10}\NormalTok{)}
\end{Highlighting}
\end{Shaded}

\begin{verbatim}
##  [1] 4.133106 6.093477 7.363590 5.900827 7.166668 4.672351 8.618037 3.510272
##  [9] 9.658221 5.722486 5.455435 2.315432 3.362482 8.963233 7.090975
\end{verbatim}

Como ya se comentó, se puede utilizar \texttt{help(funcion)} (o \texttt{?funcion}) para mostrar la ayuda de las funciones anteriores.

\hypertarget{selecciuxf3n-de-elementos-de-un-vector}{%
\subsection{Selección de elementos de un vector}\label{selecciuxf3n-de-elementos-de-un-vector}}

Para acceder a los elementos de un vector se indica entre corchetes el
correspondiente vector de subíndices (enteros positivos).

\begin{Shaded}
\begin{Highlighting}[]
\NormalTok{x }\OtherTok{\textless{}{-}} \FunctionTok{seq}\NormalTok{(}\SpecialCharTok{{-}}\DecValTok{3}\NormalTok{, }\DecValTok{3}\NormalTok{, }\DecValTok{1}\NormalTok{)}
\NormalTok{x}
\end{Highlighting}
\end{Shaded}

\begin{verbatim}
## [1] -3 -2 -1  0  1  2  3
\end{verbatim}

\begin{Shaded}
\begin{Highlighting}[]
\NormalTok{x[}\DecValTok{1}\NormalTok{]  }\CommentTok{\# primer elemento}
\end{Highlighting}
\end{Shaded}

\begin{verbatim}
## [1] -3
\end{verbatim}

\begin{Shaded}
\begin{Highlighting}[]
\NormalTok{ii }\OtherTok{\textless{}{-}} \FunctionTok{c}\NormalTok{(}\DecValTok{1}\NormalTok{, }\DecValTok{5}\NormalTok{, }\DecValTok{7}\NormalTok{)}
\NormalTok{x[ii] }\CommentTok{\# posiciones 1, 5 y 7}
\end{Highlighting}
\end{Shaded}

\begin{verbatim}
## [1] -3  1  3
\end{verbatim}

\begin{Shaded}
\begin{Highlighting}[]
\NormalTok{ii }\OtherTok{\textless{}{-}}\NormalTok{ x }\SpecialCharTok{\textgreater{}} \DecValTok{0}
\NormalTok{ii}
\end{Highlighting}
\end{Shaded}

\begin{verbatim}
## [1] FALSE FALSE FALSE FALSE  TRUE  TRUE  TRUE
\end{verbatim}

\begin{Shaded}
\begin{Highlighting}[]
\NormalTok{x[ii]  }\CommentTok{\# valores positivos}
\end{Highlighting}
\end{Shaded}

\begin{verbatim}
## [1] 1 2 3
\end{verbatim}

\begin{Shaded}
\begin{Highlighting}[]
\NormalTok{ii }\OtherTok{\textless{}{-}} \DecValTok{1}\SpecialCharTok{:}\DecValTok{3}
\NormalTok{x[}\SpecialCharTok{{-}}\NormalTok{ii]  }\CommentTok{\# elementos de x salvo los 3 primeros}
\end{Highlighting}
\end{Shaded}

\begin{verbatim}
## [1] 0 1 2 3
\end{verbatim}

\hypertarget{ordenaciuxf3n-de-vectores}{%
\subsection{Ordenación de vectores}\label{ordenaciuxf3n-de-vectores}}

\begin{Shaded}
\begin{Highlighting}[]
\NormalTok{x }\OtherTok{\textless{}{-}} \FunctionTok{c}\NormalTok{(}\DecValTok{65}\NormalTok{, }\DecValTok{18}\NormalTok{, }\DecValTok{59}\NormalTok{, }\DecValTok{18}\NormalTok{, }\DecValTok{6}\NormalTok{, }\DecValTok{94}\NormalTok{, }\DecValTok{26}\NormalTok{)}
\FunctionTok{sort}\NormalTok{(x)}
\end{Highlighting}
\end{Shaded}

\begin{verbatim}
## [1]  6 18 18 26 59 65 94
\end{verbatim}

\begin{Shaded}
\begin{Highlighting}[]
\FunctionTok{sort}\NormalTok{(x, }\AttributeTok{decreasing =}\NormalTok{ T)}
\end{Highlighting}
\end{Shaded}

\begin{verbatim}
## [1] 94 65 59 26 18 18  6
\end{verbatim}

Otra posibilidad es utilizar un índice de ordenación.

\begin{Shaded}
\begin{Highlighting}[]
\NormalTok{ii }\OtherTok{\textless{}{-}} \FunctionTok{order}\NormalTok{(x)}
\NormalTok{ii  }\CommentTok{\# índice de ordenación}
\end{Highlighting}
\end{Shaded}

\begin{verbatim}
## [1] 5 2 4 7 3 1 6
\end{verbatim}

\begin{Shaded}
\begin{Highlighting}[]
\NormalTok{x[ii]  }\CommentTok{\# valores ordenados}
\end{Highlighting}
\end{Shaded}

\begin{verbatim}
## [1]  6 18 18 26 59 65 94
\end{verbatim}

La función \texttt{rev()} devuelve los valores del vector en orden inverso.

\begin{Shaded}
\begin{Highlighting}[]
\FunctionTok{rev}\NormalTok{(x)}
\end{Highlighting}
\end{Shaded}

\begin{verbatim}
## [1] 26 94  6 18 59 18 65
\end{verbatim}

\hypertarget{datos-faltantes}{%
\subsection{Datos faltantes}\label{datos-faltantes}}

Los datos faltantes (también denominados valores perdidos) aparecen normalmente cuando algún dato no ha sido registrado.
Este tipo de valores se registran como \texttt{NA} (abreviatura de \emph{Not Available}).

Por ejemplo, supongamos que tenemos registrado las alturas de 5 personas pero desconocemos la altura de la cuarta persona.
El vector sería registrado como sigue:

\begin{Shaded}
\begin{Highlighting}[]
\NormalTok{altura }\OtherTok{\textless{}{-}} \FunctionTok{c}\NormalTok{(}\DecValTok{165}\NormalTok{, }\DecValTok{178}\NormalTok{, }\DecValTok{184}\NormalTok{, }\ConstantTok{NA}\NormalTok{, }\DecValTok{175}\NormalTok{)}
\NormalTok{altura}
\end{Highlighting}
\end{Shaded}

\begin{verbatim}
## [1] 165 178 184  NA 175
\end{verbatim}

Es importante notar que cualquier operación aritmética sobre un vector
que contiene algún \texttt{NA} dará como resultado otro \texttt{NA}.

\begin{Shaded}
\begin{Highlighting}[]
\FunctionTok{mean}\NormalTok{(altura)}
\end{Highlighting}
\end{Shaded}

\begin{verbatim}
## [1] NA
\end{verbatim}

En muchas funciones para forzar a \texttt{R} a que ignore los valores perdidos se utiliza la opción \texttt{na.rm\ =\ TRUE}.

\begin{Shaded}
\begin{Highlighting}[]
\FunctionTok{mean}\NormalTok{(altura, }\AttributeTok{na.rm =} \ConstantTok{TRUE}\NormalTok{)}
\end{Highlighting}
\end{Shaded}

\begin{verbatim}
## [1] 175.5
\end{verbatim}

\texttt{R} permite gestionar otros tipos de valores especiales:

\begin{itemize}
\item
  \texttt{NaN} (\emph{Not a Number}): es resultado de una indeterminación.
\item
  \texttt{Inf}: \texttt{R} representa valores no finitos \(\pm \infty\) como \texttt{Inf} y \texttt{-Inf}.
\end{itemize}

\vspace*{0.3cm}

\begin{Shaded}
\begin{Highlighting}[]
\DecValTok{5}\SpecialCharTok{/}\DecValTok{0}  \CommentTok{\# Infinito}
\end{Highlighting}
\end{Shaded}

\begin{verbatim}
## [1] Inf
\end{verbatim}

\begin{Shaded}
\begin{Highlighting}[]
\FunctionTok{log}\NormalTok{(}\DecValTok{0}\NormalTok{)  }\CommentTok{\# {-}Infinito}
\end{Highlighting}
\end{Shaded}

\begin{verbatim}
## [1] -Inf
\end{verbatim}

\begin{Shaded}
\begin{Highlighting}[]
\DecValTok{0}\SpecialCharTok{/}\DecValTok{0}  \CommentTok{\# Not a Number}
\end{Highlighting}
\end{Shaded}

\begin{verbatim}
## [1] NaN
\end{verbatim}

\hypertarget{vectores-no-numuxe9ricos}{%
\subsection{Vectores no numéricos}\label{vectores-no-numuxe9ricos}}

Los vectores pueden ser no numéricos, aunque todas las componentes deben ser del mismo tipo:

\begin{Shaded}
\begin{Highlighting}[]
\NormalTok{a }\OtherTok{\textless{}{-}} \FunctionTok{c}\NormalTok{(}\StringTok{"A Coruña"}\NormalTok{, }\StringTok{"Lugo"}\NormalTok{, }\StringTok{"Ourense"}\NormalTok{, }\StringTok{"Pontevedra"}\NormalTok{)}
\NormalTok{a}
\end{Highlighting}
\end{Shaded}

\begin{verbatim}
## [1] "A Coruña"   "Lugo"       "Ourense"    "Pontevedra"
\end{verbatim}

\begin{Shaded}
\begin{Highlighting}[]
\NormalTok{letters[}\DecValTok{1}\SpecialCharTok{:}\DecValTok{10}\NormalTok{]  }\CommentTok{\# primeras 10 letas del abecedario}
\end{Highlighting}
\end{Shaded}

\begin{verbatim}
##  [1] "a" "b" "c" "d" "e" "f" "g" "h" "i" "j"
\end{verbatim}

\begin{Shaded}
\begin{Highlighting}[]
\NormalTok{LETTERS[}\DecValTok{1}\SpecialCharTok{:}\DecValTok{10}\NormalTok{]  }\CommentTok{\# lo mismo en mayúscula}
\end{Highlighting}
\end{Shaded}

\begin{verbatim}
##  [1] "A" "B" "C" "D" "E" "F" "G" "H" "I" "J"
\end{verbatim}

\begin{Shaded}
\begin{Highlighting}[]
\NormalTok{month.name[}\DecValTok{1}\SpecialCharTok{:}\DecValTok{6}\NormalTok{]  }\CommentTok{\# primeros 6 meses del año en inglés}
\end{Highlighting}
\end{Shaded}

\begin{verbatim}
## [1] "January"  "February" "March"    "April"    "May"      "June"
\end{verbatim}

\hypertarget{factores}{%
\subsection{Factores}\label{factores}}

Los factores se utilizan para representar datos categóricos.
Se puede pensar en ellos como vectores de enteros en los que cada entero tiene asociada una etiqueta (\emph{label}).
Los factores son muy importantes en la modelización estadística ya que \texttt{R}los trata de forma especial.

Utilizar factores con etiquetas es preferible a utilizar enteros porque las etiquetas son auto-descriptivas.

Veamos un ejemplo. Supongamos que el vector \texttt{sexo} indica el sexo de un persona codificado como 0 si hombre y 1 si mujer

\begin{Shaded}
\begin{Highlighting}[]
\NormalTok{sexo }\OtherTok{\textless{}{-}} \FunctionTok{c}\NormalTok{(}\DecValTok{0}\NormalTok{, }\DecValTok{1}\NormalTok{, }\DecValTok{1}\NormalTok{, }\DecValTok{1}\NormalTok{, }\DecValTok{0}\NormalTok{, }\DecValTok{0}\NormalTok{, }\DecValTok{1}\NormalTok{, }\DecValTok{0}\NormalTok{, }\DecValTok{1}\NormalTok{)}
\NormalTok{sexo}
\end{Highlighting}
\end{Shaded}

\begin{verbatim}
## [1] 0 1 1 1 0 0 1 0 1
\end{verbatim}

\begin{Shaded}
\begin{Highlighting}[]
\FunctionTok{table}\NormalTok{(sexo)}
\end{Highlighting}
\end{Shaded}

\begin{verbatim}
## sexo
## 0 1 
## 4 5
\end{verbatim}

El problema de introducir así los valores es que no queda reflejada la codificación de los mismos.
Para ello guardaremos los datos en una estructura tipo factor:

\begin{Shaded}
\begin{Highlighting}[]
\NormalTok{sexo2 }\OtherTok{\textless{}{-}} \FunctionTok{factor}\NormalTok{(sexo, }\AttributeTok{labels =} \FunctionTok{c}\NormalTok{(}\StringTok{"hombre"}\NormalTok{, }\StringTok{"mujer"}\NormalTok{)); sexo2}
\end{Highlighting}
\end{Shaded}

\begin{verbatim}
## [1] hombre mujer  mujer  mujer  hombre hombre mujer  hombre mujer 
## Levels: hombre mujer
\end{verbatim}

\begin{Shaded}
\begin{Highlighting}[]
\FunctionTok{levels}\NormalTok{(sexo2)  }\CommentTok{\# devuelve los niveles de un factor}
\end{Highlighting}
\end{Shaded}

\begin{verbatim}
## [1] "hombre" "mujer"
\end{verbatim}

\begin{Shaded}
\begin{Highlighting}[]
\FunctionTok{unclass}\NormalTok{(sexo2)  }\CommentTok{\# representación subyacente del factor}
\end{Highlighting}
\end{Shaded}

\begin{verbatim}
## [1] 1 2 2 2 1 1 2 1 2
## attr(,"levels")
## [1] "hombre" "mujer"
\end{verbatim}

\begin{Shaded}
\begin{Highlighting}[]
\FunctionTok{table}\NormalTok{(sexo2)}
\end{Highlighting}
\end{Shaded}

\begin{verbatim}
## sexo2
## hombre  mujer 
##      4      5
\end{verbatim}

Veamos otro ejemplo, en el que inicialmente tenemos datos categóricos.
Los niveles se toman automáticamente por orden alfabético

\begin{Shaded}
\begin{Highlighting}[]
\NormalTok{respuestas }\OtherTok{\textless{}{-}} \FunctionTok{factor}\NormalTok{(}\FunctionTok{c}\NormalTok{(}\StringTok{\textquotesingle{}si\textquotesingle{}}\NormalTok{, }\StringTok{\textquotesingle{}si\textquotesingle{}}\NormalTok{, }\StringTok{\textquotesingle{}no\textquotesingle{}}\NormalTok{, }\StringTok{\textquotesingle{}si\textquotesingle{}}\NormalTok{, }\StringTok{\textquotesingle{}si\textquotesingle{}}\NormalTok{, }\StringTok{\textquotesingle{}no\textquotesingle{}}\NormalTok{, }\StringTok{\textquotesingle{}no\textquotesingle{}}\NormalTok{))}
\NormalTok{respuestas}
\end{Highlighting}
\end{Shaded}

\begin{verbatim}
## [1] si si no si si no no
## Levels: no si
\end{verbatim}

Si deseásemos otro orden (lo cual puede ser importante en algunos casos, por ejemplo para representaciones gráficas), habría que indicarlo expresamente

\begin{Shaded}
\begin{Highlighting}[]
\NormalTok{respuestas }\OtherTok{\textless{}{-}} \FunctionTok{factor}\NormalTok{(}\FunctionTok{c}\NormalTok{(}\StringTok{\textquotesingle{}si\textquotesingle{}}\NormalTok{, }\StringTok{\textquotesingle{}si\textquotesingle{}}\NormalTok{, }\StringTok{\textquotesingle{}no\textquotesingle{}}\NormalTok{, }\StringTok{\textquotesingle{}si\textquotesingle{}}\NormalTok{, }\StringTok{\textquotesingle{}si\textquotesingle{}}\NormalTok{, }\StringTok{\textquotesingle{}no\textquotesingle{}}\NormalTok{, }\StringTok{\textquotesingle{}no\textquotesingle{}}\NormalTok{), }\AttributeTok{levels =} \FunctionTok{c}\NormalTok{(}\StringTok{\textquotesingle{}si\textquotesingle{}}\NormalTok{, }\StringTok{\textquotesingle{}no\textquotesingle{}}\NormalTok{))}
\NormalTok{respuestas}
\end{Highlighting}
\end{Shaded}

\begin{verbatim}
## [1] si si no si si no no
## Levels: si no
\end{verbatim}

\hypertarget{matrices-y-arrays}{%
\section{Matrices y arrays}\label{matrices-y-arrays}}

\hypertarget{matrices}{%
\subsection{Matrices}\label{matrices}}

Las \emph{matrices} son la extensión natural de los vectores a dos dimensiones.
Su generalización a más dimensiones se llama \emph{array}.

Las matrices se pueden crear concatenando vectores con las funciones \texttt{cbind()} o \texttt{rbind()}:

\begin{Shaded}
\begin{Highlighting}[]
\NormalTok{x }\OtherTok{\textless{}{-}} \FunctionTok{c}\NormalTok{(}\DecValTok{3}\NormalTok{, }\DecValTok{7}\NormalTok{, }\DecValTok{1}\NormalTok{, }\DecValTok{8}\NormalTok{, }\DecValTok{4}\NormalTok{)}
\NormalTok{y }\OtherTok{\textless{}{-}} \FunctionTok{c}\NormalTok{(}\DecValTok{7}\NormalTok{, }\DecValTok{5}\NormalTok{, }\DecValTok{2}\NormalTok{, }\DecValTok{1}\NormalTok{, }\DecValTok{0}\NormalTok{)}
\FunctionTok{cbind}\NormalTok{(x, y)  }\CommentTok{\# por columnas}
\end{Highlighting}
\end{Shaded}

\begin{verbatim}
##      x y
## [1,] 3 7
## [2,] 7 5
## [3,] 1 2
## [4,] 8 1
## [5,] 4 0
\end{verbatim}

\begin{Shaded}
\begin{Highlighting}[]
\FunctionTok{rbind}\NormalTok{(x, y)  }\CommentTok{\# por filas}
\end{Highlighting}
\end{Shaded}

\begin{verbatim}
##   [,1] [,2] [,3] [,4] [,5]
## x    3    7    1    8    4
## y    7    5    2    1    0
\end{verbatim}

Una matriz se puede crear con la función \texttt{matrix()} donde el parámetro \texttt{nrow} indica el número de filas y \texttt{ncol} el número de columnas.
Por defecto, los valores se colocan por columnas.

\begin{Shaded}
\begin{Highlighting}[]
\FunctionTok{matrix}\NormalTok{(}\DecValTok{1}\SpecialCharTok{:}\DecValTok{8}\NormalTok{, }\AttributeTok{nrow =} \DecValTok{2}\NormalTok{, }\AttributeTok{ncol =} \DecValTok{4}\NormalTok{)  }\CommentTok{\# equivalente a matrix(1:8, nrow=2)}
\end{Highlighting}
\end{Shaded}

\begin{verbatim}
##      [,1] [,2] [,3] [,4]
## [1,]    1    3    5    7
## [2,]    2    4    6    8
\end{verbatim}

Los nombres de los parámetros se pueden acortar siempre y cuando no haya ambigüedad, por lo que podríamos escribir

\begin{Shaded}
\begin{Highlighting}[]
\FunctionTok{matrix}\NormalTok{(}\DecValTok{1}\SpecialCharTok{:}\DecValTok{8}\NormalTok{, }\AttributeTok{nr =} \DecValTok{2}\NormalTok{, }\AttributeTok{nc =} \DecValTok{4}\NormalTok{)}
\end{Highlighting}
\end{Shaded}

\begin{verbatim}
##      [,1] [,2] [,3] [,4]
## [1,]    1    3    5    7
## [2,]    2    4    6    8
\end{verbatim}

Si queremos indicar que los valores se almacenen por filas

\begin{Shaded}
\begin{Highlighting}[]
\FunctionTok{matrix}\NormalTok{(}\DecValTok{1}\SpecialCharTok{:}\DecValTok{8}\NormalTok{, }\AttributeTok{nr =} \DecValTok{2}\NormalTok{, }\AttributeTok{byrow =} \ConstantTok{TRUE}\NormalTok{)}
\end{Highlighting}
\end{Shaded}

\begin{verbatim}
##      [,1] [,2] [,3] [,4]
## [1,]    1    2    3    4
## [2,]    5    6    7    8
\end{verbatim}

\hypertarget{nombres-en-matrices}{%
\subsection{Nombres en matrices}\label{nombres-en-matrices}}

Se pueden dar nombres a las filas y columnas de una matriz.

\begin{Shaded}
\begin{Highlighting}[]
\NormalTok{x }\OtherTok{\textless{}{-}} \FunctionTok{matrix}\NormalTok{(}\FunctionTok{c}\NormalTok{(}\DecValTok{1}\NormalTok{, }\DecValTok{2}\NormalTok{, }\DecValTok{3}\NormalTok{, }\DecValTok{11}\NormalTok{, }\DecValTok{12}\NormalTok{, }\DecValTok{13}\NormalTok{), }\AttributeTok{nrow =} \DecValTok{2}\NormalTok{, }\AttributeTok{byrow =} \ConstantTok{TRUE}\NormalTok{)}
\NormalTok{x}
\end{Highlighting}
\end{Shaded}

\begin{verbatim}
##      [,1] [,2] [,3]
## [1,]    1    2    3
## [2,]   11   12   13
\end{verbatim}

\begin{Shaded}
\begin{Highlighting}[]
\FunctionTok{rownames}\NormalTok{(x) }\OtherTok{\textless{}{-}} \FunctionTok{c}\NormalTok{(}\StringTok{"fila 1"}\NormalTok{, }\StringTok{"fila 2"}\NormalTok{)}
\FunctionTok{colnames}\NormalTok{(x) }\OtherTok{\textless{}{-}} \FunctionTok{c}\NormalTok{(}\StringTok{"col 1"}\NormalTok{, }\StringTok{"col 2"}\NormalTok{, }\StringTok{"col 3"}\NormalTok{)}
\NormalTok{x }
\end{Highlighting}
\end{Shaded}

\begin{verbatim}
##        col 1 col 2 col 3
## fila 1     1     2     3
## fila 2    11    12    13
\end{verbatim}

Obtenemos el mismo resultado si escribimos

\begin{Shaded}
\begin{Highlighting}[]
\FunctionTok{colnames}\NormalTok{(x) }\OtherTok{\textless{}{-}} \FunctionTok{paste}\NormalTok{(}\StringTok{"col"}\NormalTok{, }\DecValTok{1}\SpecialCharTok{:}\FunctionTok{ncol}\NormalTok{(x), }\AttributeTok{sep=}\StringTok{" "}\NormalTok{)}
\end{Highlighting}
\end{Shaded}

Internamente, las matrices son vectores con un atributo especial: la \emph{dimensión}.

\begin{Shaded}
\begin{Highlighting}[]
\FunctionTok{dim}\NormalTok{(x)}
\end{Highlighting}
\end{Shaded}

\begin{verbatim}
## [1] 2 3
\end{verbatim}

\begin{Shaded}
\begin{Highlighting}[]
\FunctionTok{attributes}\NormalTok{(x)}
\end{Highlighting}
\end{Shaded}

\begin{verbatim}
## $dim
## [1] 2 3
## 
## $dimnames
## $dimnames[[1]]
## [1] "fila 1" "fila 2"
## 
## $dimnames[[2]]
## [1] "col 1" "col 2" "col 3"
\end{verbatim}

\hypertarget{acceso-a-los-elementos-de-una-matriz}{%
\subsection{Acceso a los elementos de una matriz}\label{acceso-a-los-elementos-de-una-matriz}}

El acceso a los elementos de una matriz se realiza de forma análoga al acceso ya comentado para los vectores.

\begin{Shaded}
\begin{Highlighting}[]
\NormalTok{x }\OtherTok{\textless{}{-}} \FunctionTok{matrix}\NormalTok{(}\DecValTok{1}\SpecialCharTok{:}\DecValTok{6}\NormalTok{, }\DecValTok{2}\NormalTok{, }\DecValTok{3}\NormalTok{); x}
\end{Highlighting}
\end{Shaded}

\begin{verbatim}
##      [,1] [,2] [,3]
## [1,]    1    3    5
## [2,]    2    4    6
\end{verbatim}

\begin{Shaded}
\begin{Highlighting}[]
\NormalTok{x[}\DecValTok{1}\NormalTok{, }\DecValTok{1}\NormalTok{]}
\end{Highlighting}
\end{Shaded}

\begin{verbatim}
## [1] 1
\end{verbatim}

\begin{Shaded}
\begin{Highlighting}[]
\NormalTok{x[}\DecValTok{2}\NormalTok{, }\DecValTok{2}\NormalTok{]}
\end{Highlighting}
\end{Shaded}

\begin{verbatim}
## [1] 4
\end{verbatim}

\begin{Shaded}
\begin{Highlighting}[]
\NormalTok{x[}\DecValTok{2}\NormalTok{, ]  }\CommentTok{\# segunda fila}
\end{Highlighting}
\end{Shaded}

\begin{verbatim}
## [1] 2 4 6
\end{verbatim}

\begin{Shaded}
\begin{Highlighting}[]
\NormalTok{x[ ,}\DecValTok{2}\NormalTok{]  }\CommentTok{\# segunda columna}
\end{Highlighting}
\end{Shaded}

\begin{verbatim}
## [1] 3 4
\end{verbatim}

\begin{Shaded}
\begin{Highlighting}[]
\NormalTok{x[}\DecValTok{1}\NormalTok{, }\DecValTok{1}\SpecialCharTok{:}\DecValTok{2}\NormalTok{]  }\CommentTok{\# primera fila, columnas 1ª y 2ª }
\end{Highlighting}
\end{Shaded}

\begin{verbatim}
## [1] 1 3
\end{verbatim}

\hypertarget{ordenaciuxf3n-por-filas-y-columnas}{%
\subsection{Ordenación por filas y columnas}\label{ordenaciuxf3n-por-filas-y-columnas}}

En ocasiones, interesará ordenar los elementos de una matriz por los valores de una determinada columna o fila.

Por ejemplo, supongamos la matriz

\begin{Shaded}
\begin{Highlighting}[]
\NormalTok{x }\OtherTok{\textless{}{-}} \FunctionTok{c}\NormalTok{(}\DecValTok{79}\NormalTok{, }\DecValTok{100}\NormalTok{, }\DecValTok{116}\NormalTok{, }\DecValTok{121}\NormalTok{, }\DecValTok{52}\NormalTok{, }\DecValTok{134}\NormalTok{, }\DecValTok{123}\NormalTok{, }\DecValTok{109}\NormalTok{, }\DecValTok{80}\NormalTok{, }\DecValTok{107}\NormalTok{, }\DecValTok{66}\NormalTok{, }\DecValTok{118}\NormalTok{)}
\NormalTok{x }\OtherTok{\textless{}{-}} \FunctionTok{matrix}\NormalTok{(x, }\AttributeTok{ncol=}\DecValTok{4}\NormalTok{, }\AttributeTok{byrow=}\NormalTok{T); x}
\end{Highlighting}
\end{Shaded}

\begin{verbatim}
##      [,1] [,2] [,3] [,4]
## [1,]   79  100  116  121
## [2,]   52  134  123  109
## [3,]   80  107   66  118
\end{verbatim}

La matriz ordenada por los valores de la primera columna viene dada por

\begin{Shaded}
\begin{Highlighting}[]
\NormalTok{ii }\OtherTok{\textless{}{-}} \FunctionTok{order}\NormalTok{(x[ ,}\DecValTok{1}\NormalTok{])}
\NormalTok{x[ii, ]  }\CommentTok{\# ordenación columna 1}
\end{Highlighting}
\end{Shaded}

\begin{verbatim}
##      [,1] [,2] [,3] [,4]
## [1,]   52  134  123  109
## [2,]   79  100  116  121
## [3,]   80  107   66  118
\end{verbatim}

De igual modo, si queremos ordenar por los valores de la cuarta columna:

\begin{Shaded}
\begin{Highlighting}[]
\NormalTok{ii }\OtherTok{\textless{}{-}} \FunctionTok{order}\NormalTok{(x[ ,}\DecValTok{4}\NormalTok{]); x[ii, ]  }\CommentTok{\# ordenación columna 4}
\end{Highlighting}
\end{Shaded}

\begin{verbatim}
##      [,1] [,2] [,3] [,4]
## [1,]   52  134  123  109
## [2,]   80  107   66  118
## [3,]   79  100  116  121
\end{verbatim}

\hypertarget{operaciones-con-matrices-y-arrays}{%
\subsection{Operaciones con Matrices y Arrays}\label{operaciones-con-matrices-y-arrays}}

A continuación se muestran algunas funciones que se pueden emplear con
matrices

\begin{longtable}[]{@{}ll@{}}
\toprule
Función & Descripción \\
\midrule
\endhead
\texttt{dim(),\ nrow(),\ ncol()} & número de filas y/o columnas \\
\texttt{diag()} & diagonal de una matrix \\
\texttt{*} & multiplicación elemento a elemento \\
\texttt{\%*\%} & multiplicación matricial de matrices \\
\texttt{cbind(),\ rbind()} & encadenamiento de columnas o filas \\
\texttt{t()} & transpuesta \\
\texttt{solve(A)} & inversa de la matriz A \\
\texttt{solve(A,b)} & solución del sistema de ecuaciones \(Ax=b\) \\
\texttt{qr()} & descomposición de Cholesky \\
\texttt{eigen()} & autovalores y autovectores \\
\texttt{svd()} & descomposición singular \\
\bottomrule
\end{longtable}

\hypertarget{ejemplos}{%
\subsection{Ejemplos}\label{ejemplos}}

\begin{Shaded}
\begin{Highlighting}[]
\NormalTok{x }\OtherTok{\textless{}{-}} \FunctionTok{matrix}\NormalTok{(}\DecValTok{1}\SpecialCharTok{:}\DecValTok{6}\NormalTok{, }\AttributeTok{ncol =} \DecValTok{3}\NormalTok{)}
\NormalTok{x}
\end{Highlighting}
\end{Shaded}

\begin{verbatim}
##      [,1] [,2] [,3]
## [1,]    1    3    5
## [2,]    2    4    6
\end{verbatim}

\begin{Shaded}
\begin{Highlighting}[]
\FunctionTok{t}\NormalTok{(x)  }\CommentTok{\# matriz transpuesta}
\end{Highlighting}
\end{Shaded}

\begin{verbatim}
##      [,1] [,2]
## [1,]    1    2
## [2,]    3    4
## [3,]    5    6
\end{verbatim}

\begin{Shaded}
\begin{Highlighting}[]
\FunctionTok{dim}\NormalTok{(x)  }\CommentTok{\# dimensiones de la matriz}
\end{Highlighting}
\end{Shaded}

\begin{verbatim}
## [1] 2 3
\end{verbatim}

\hypertarget{inversiuxf3n-de-una-matriz}{%
\subsection{Inversión de una matriz}\label{inversiuxf3n-de-una-matriz}}

\begin{Shaded}
\begin{Highlighting}[]
\NormalTok{A }\OtherTok{\textless{}{-}} \FunctionTok{matrix}\NormalTok{(}\FunctionTok{c}\NormalTok{(}\DecValTok{2}\NormalTok{, }\DecValTok{4}\NormalTok{, }\DecValTok{0}\NormalTok{, }\DecValTok{2}\NormalTok{), }\AttributeTok{nrow =} \DecValTok{2}\NormalTok{); A}
\end{Highlighting}
\end{Shaded}

\begin{verbatim}
##      [,1] [,2]
## [1,]    2    0
## [2,]    4    2
\end{verbatim}

\begin{Shaded}
\begin{Highlighting}[]
\NormalTok{B }\OtherTok{\textless{}{-}} \FunctionTok{solve}\NormalTok{(A)}
\NormalTok{B  }\CommentTok{\# inversa}
\end{Highlighting}
\end{Shaded}

\begin{verbatim}
##      [,1] [,2]
## [1,]  0.5  0.0
## [2,] -1.0  0.5
\end{verbatim}

\begin{Shaded}
\begin{Highlighting}[]
\NormalTok{A }\SpecialCharTok{\%*\%}\NormalTok{ B  }\CommentTok{\# comprobamos que está bien}
\end{Highlighting}
\end{Shaded}

\begin{verbatim}
##      [,1] [,2]
## [1,]    1    0
## [2,]    0    1
\end{verbatim}

\hypertarget{data-frames}{%
\section{Data frames}\label{data-frames}}

Los \texttt{data.frames} (\emph{marcos de datos}) son el objeto más habitual para el almacenamiento de conjuntos de datos.
En este tipo de objetos cada individuo de la muestra se corresponde con una fila y cada una de las variables con una columna.
Para la creación de estas estructuras se utiliza la función \texttt{data.frame()}.

Este tipo de estructuras son en apariencia muy similares a las matrices, con la ventaja de que permiten que los valores de las distintas columnas sean de tipos diferentes.
Por ejemplo, supongamos que tenemos registrados los siguientes valores

\begin{Shaded}
\begin{Highlighting}[]
\NormalTok{Producto }\OtherTok{\textless{}{-}} \FunctionTok{c}\NormalTok{(}\StringTok{"Zumo"}\NormalTok{, }\StringTok{"Queso"}\NormalTok{, }\StringTok{"Yogourt"}\NormalTok{)}
\NormalTok{Seccion }\OtherTok{\textless{}{-}} \FunctionTok{c}\NormalTok{(}\StringTok{"Bebidas"}\NormalTok{, }\StringTok{"Lácteos"}\NormalTok{, }\StringTok{"Lácteos"}\NormalTok{)}
\NormalTok{Unidades }\OtherTok{\textless{}{-}} \FunctionTok{c}\NormalTok{(}\DecValTok{2}\NormalTok{, }\DecValTok{1}\NormalTok{, }\DecValTok{10}\NormalTok{)}
\end{Highlighting}
\end{Shaded}

Los valores anteriores se podrían guardar en una única matriz

\begin{Shaded}
\begin{Highlighting}[]
\NormalTok{x }\OtherTok{\textless{}{-}} \FunctionTok{cbind}\NormalTok{(Producto, Seccion, Unidades)}
\FunctionTok{class}\NormalTok{(x)}
\end{Highlighting}
\end{Shaded}

\begin{verbatim}
## [1] "matrix" "array"
\end{verbatim}

\begin{Shaded}
\begin{Highlighting}[]
\NormalTok{x}
\end{Highlighting}
\end{Shaded}

\begin{verbatim}
##      Producto  Seccion   Unidades
## [1,] "Zumo"    "Bebidas" "2"     
## [2,] "Queso"   "Lácteos" "1"     
## [3,] "Yogourt" "Lácteos" "10"
\end{verbatim}

Sin embargo, el resultado anterior no es satisfactorio ya que todos los valores se han transformado en caracteres.
Una solución mejor es utilizar un \texttt{data.frame}, con lo cual se mantiene el tipo original de las variables.

\begin{Shaded}
\begin{Highlighting}[]
\NormalTok{lista.compra }\OtherTok{\textless{}{-}} \FunctionTok{data.frame}\NormalTok{(Producto, Seccion, Unidades)}
\FunctionTok{class}\NormalTok{(lista.compra)}
\end{Highlighting}
\end{Shaded}

\begin{verbatim}
## [1] "data.frame"
\end{verbatim}

\begin{Shaded}
\begin{Highlighting}[]
\NormalTok{lista.compra}
\end{Highlighting}
\end{Shaded}

\begin{verbatim}
##   Producto Seccion Unidades
## 1     Zumo Bebidas        2
## 2    Queso Lácteos        1
## 3  Yogourt Lácteos       10
\end{verbatim}

A continuación se muestran ejemplos que ilustran la manera de acceder
a los valores de un data.frame.

\begin{Shaded}
\begin{Highlighting}[]
\NormalTok{lista.compra}\SpecialCharTok{$}\NormalTok{Unidades}
\end{Highlighting}
\end{Shaded}

\begin{verbatim}
## [1]  2  1 10
\end{verbatim}

\begin{Shaded}
\begin{Highlighting}[]
\NormalTok{lista.compra[ ,}\DecValTok{3}\NormalTok{]  }\CommentTok{\# de manera equivalente}
\end{Highlighting}
\end{Shaded}

\begin{verbatim}
## [1]  2  1 10
\end{verbatim}

\begin{Shaded}
\begin{Highlighting}[]
\NormalTok{lista.compra}\SpecialCharTok{$}\NormalTok{Seccion}
\end{Highlighting}
\end{Shaded}

\begin{verbatim}
## [1] "Bebidas" "Lácteos" "Lácteos"
\end{verbatim}

\begin{Shaded}
\begin{Highlighting}[]
\NormalTok{lista.compra}\SpecialCharTok{$}\NormalTok{Unidades[}\DecValTok{1}\SpecialCharTok{:}\DecValTok{2}\NormalTok{]  }\CommentTok{\# primeros dos valores de Unidades}
\end{Highlighting}
\end{Shaded}

\begin{verbatim}
## [1] 2 1
\end{verbatim}

\begin{Shaded}
\begin{Highlighting}[]
\NormalTok{lista.compra[}\DecValTok{2}\NormalTok{,]  }\CommentTok{\# segunda fila}
\end{Highlighting}
\end{Shaded}

\begin{verbatim}
##   Producto Seccion Unidades
## 2    Queso Lácteos        1
\end{verbatim}

La función \texttt{summary()} permite hacer un resumen estadístico de las
variables (columnas) del data.frame.

\begin{Shaded}
\begin{Highlighting}[]
\FunctionTok{summary}\NormalTok{(lista.compra)}
\end{Highlighting}
\end{Shaded}

\begin{verbatim}
##    Producto           Seccion             Unidades     
##  Length:3           Length:3           Min.   : 1.000  
##  Class :character   Class :character   1st Qu.: 1.500  
##  Mode  :character   Mode  :character   Median : 2.000  
##                                        Mean   : 4.333  
##                                        3rd Qu.: 6.000  
##                                        Max.   :10.000
\end{verbatim}

\hypertarget{listas}{%
\section{Listas}\label{listas}}

Las listas son colecciones ordenadas de cualquier tipo de objetos (en \texttt{R} las listas son un tipo especial de vectores).
Así, mientras que los elementos de los vectores, matrices y arrays deben ser del mismo tipo, en el caso de las listas se pueden tener elementos de tipos distintos.

\begin{Shaded}
\begin{Highlighting}[]
\NormalTok{x }\OtherTok{\textless{}{-}} \FunctionTok{c}\NormalTok{(}\DecValTok{1}\NormalTok{, }\DecValTok{2}\NormalTok{, }\DecValTok{3}\NormalTok{, }\DecValTok{4}\NormalTok{)}
\NormalTok{y }\OtherTok{\textless{}{-}} \FunctionTok{c}\NormalTok{(}\StringTok{"Hombre"}\NormalTok{, }\StringTok{"Mujer"}\NormalTok{)}
\NormalTok{z }\OtherTok{\textless{}{-}} \FunctionTok{matrix}\NormalTok{(}\DecValTok{1}\SpecialCharTok{:}\DecValTok{12}\NormalTok{, }\AttributeTok{ncol =} \DecValTok{4}\NormalTok{)}
\NormalTok{datos }\OtherTok{\textless{}{-}} \FunctionTok{list}\NormalTok{(}\AttributeTok{A =}\NormalTok{ x, }\AttributeTok{B =}\NormalTok{ y, }\AttributeTok{C =}\NormalTok{ z)}
\NormalTok{datos}
\end{Highlighting}
\end{Shaded}

\begin{verbatim}
## $A
## [1] 1 2 3 4
## 
## $B
## [1] "Hombre" "Mujer" 
## 
## $C
##      [,1] [,2] [,3] [,4]
## [1,]    1    4    7   10
## [2,]    2    5    8   11
## [3,]    3    6    9   12
\end{verbatim}

\hypertarget{graficos}{%
\chapter{Gráficos}\label{graficos}}

En el paquete base de \texttt{R} se dispone de múltiples funciones que permiten la generación de gráficos (los denominados gráficos estándar). Se dividen en dos grandes grupos:

\begin{itemize}
\item
  Gráficos de \textbf{alto nivel}: Crean un gráfico nuevo.

  \begin{itemize}
  \tightlist
  \item
    \texttt{plot}, \texttt{hist}, \texttt{boxplot}, \ldots{}
  \end{itemize}
\item
  Gráficos de \textbf{bajo nivel}: Permiten añadir elementos (líneas, puntos, \ldots) a un gráfico ya existente

  \begin{itemize}
  \tightlist
  \item
    \texttt{points}, \texttt{lines}, \texttt{legend}, \texttt{text}, \ldots{}
  \end{itemize}
\end{itemize}

El parámetro \texttt{add\ =\ TRUE} convierte una función de nivel alto a bajo.

Dentro de las funciones gráficas de alto nivel destaca la función \texttt{plot()} que tiene muchas variantes y dependiendo del tipo de datos que se le pasen como argumento actuará de modo distinto
(es una \emph{función genérica}, \texttt{methods(plot)} devuelve los métodos disponibles).

\hypertarget{funcion-plot}{%
\section{La función plot}\label{funcion-plot}}

Si ejecutamos \texttt{plot(x,\ y)} siendo \texttt{x} e \texttt{y} vectores, entonces \texttt{R} generará el denominado \textbf{gráfico de dispersión} que
representa en un sistema coordenado los pares de valores \((x,y)\).

Por ejemplo, utilizando el siguiente código

\begin{Shaded}
\begin{Highlighting}[]
\FunctionTok{data}\NormalTok{(cars)}
\FunctionTok{plot}\NormalTok{(cars}\SpecialCharTok{$}\NormalTok{speed, cars}\SpecialCharTok{$}\NormalTok{dist)    }\CommentTok{\# otra posibilidad plot(cars)}
\end{Highlighting}
\end{Shaded}

\begin{figure}[!htb]

{\centering \includegraphics[width=0.7\linewidth]{03-Graficos_files/figure-latex/cars1-1} 

}

\caption{Gráfico de dispersión de distancia frente a velocidad}\label{fig:cars1}
\end{figure}

{[}Figura \ref{fig:cars1}{]}

El comando \texttt{plot} incluye por defecto una elección automáticas de
títulos, ejes, escalas, etiquetas, etc., que pueden ser modificados
añadiendo parámetros gráficos al comando:

\begin{longtable}[]{@{}
  >{\raggedright\arraybackslash}p{(\columnwidth - 2\tabcolsep) * \real{0.22}}
  >{\raggedright\arraybackslash}p{(\columnwidth - 2\tabcolsep) * \real{0.64}}@{}}
\toprule
Parámetro & Descripción \\
\midrule
\endhead
\texttt{type} & tipo de gráfico:
\texttt{p}: puntos, \texttt{l}: líneas, \texttt{b}: puntos y líneas,
\texttt{n}: gráfico en blanco\ldots{} \\
\texttt{xlim}, \texttt{ylim} & límites de los ejes
(e.g.~\texttt{xlim=c(1,\ 10)} o \texttt{xlim=range(x)}) \\
\texttt{xlab}, \texttt{ylab} & títulos de los ejes \\
\texttt{main,\ sub} & título principal y subtítulo \\
\texttt{col} & color de los símbolos (véase \texttt{colors()}).
También \texttt{col.axis}, \texttt{col.lab}, \texttt{col.main}, \texttt{col.sub} \\
\texttt{lty} & tipo de línea \\
\texttt{lwd} & anchura de línea \\
\texttt{pch} & tipo de símbolo \\
\texttt{cex} & tamaño de los símbolos \\
\texttt{bg} & color de relleno (para \texttt{pch\ =\ 21:25}) \\
\bottomrule
\end{longtable}

Para obtener ayuda sobre estos parámetros ejecutar \texttt{help(par)}.

Veamos algún ejemplo:

\begin{Shaded}
\begin{Highlighting}[]
\FunctionTok{plot}\NormalTok{(cars, }\AttributeTok{xlab =} \StringTok{"velocidad"}\NormalTok{, }\AttributeTok{ylab =} \StringTok{"distancia"}\NormalTok{, }\AttributeTok{main =} \StringTok{"Título"}\NormalTok{)}
\end{Highlighting}
\end{Shaded}

\begin{figure}[!htb]

{\centering \includegraphics[width=0.7\linewidth]{03-Graficos_files/figure-latex/cars2-1} 

}

\caption{Gráfico de dispersión de distancia frente a velocidad, especificando título y etiquetas de los ejes}\label{fig:cars2}
\end{figure}

{[}Figura \ref{fig:cars2}{]}

\begin{Shaded}
\begin{Highlighting}[]
\FunctionTok{plot}\NormalTok{(cars, }\AttributeTok{pch =} \DecValTok{16}\NormalTok{, }\AttributeTok{col =} \StringTok{\textquotesingle{}blue\textquotesingle{}}\NormalTok{, }\AttributeTok{main =} \StringTok{\textquotesingle{}pch=16\textquotesingle{}}\NormalTok{)}
\end{Highlighting}
\end{Shaded}

\begin{figure}[!htb]

{\centering \includegraphics[width=0.7\linewidth]{03-Graficos_files/figure-latex/cars3-1} 

}

\caption{Gráfico de dispersión de distancia frente a velocidad, cambiando el color y el tipo de símbolo}\label{fig:cars3}
\end{figure}

{[}Figura \ref{fig:cars3}{]}

\hypertarget{funciones-gruxe1ficas-de-bajo-nivel}{%
\section{Funciones gráficas de bajo nivel}\label{funciones-gruxe1ficas-de-bajo-nivel}}

Las principales funciones gráficas de bajo nivel son:

\begin{longtable}[]{@{}
  >{\raggedright\arraybackslash}p{(\columnwidth - 2\tabcolsep) * \real{0.28}}
  >{\raggedright\arraybackslash}p{(\columnwidth - 2\tabcolsep) * \real{0.60}}@{}}
\toprule
Función & Descripción \\
\midrule
\endhead
\texttt{points}, \texttt{lines} & agregan puntos y líneas \\
\texttt{text} & agrega un texto \\
\texttt{mtext} & agrega texto en los márgenes \\
\texttt{segments} & dibuja trozos de líneas desde puntos
iniciales a finales \\
\texttt{abline} & dibuja líneas \\
\texttt{rect} & dibuja rectángulos \\
\texttt{polygon} & dibuja polígonos \\
\texttt{legend} & agrega una leyenda \\
\texttt{axis} & agrega ejes \\
\texttt{locator} & devuelve coordenadas de puntos \\
\texttt{identify} & similar a \texttt{locator} \\
\bottomrule
\end{longtable}

\hypertarget{ejemplos-1}{%
\section{Ejemplos}\label{ejemplos-1}}

\begin{Shaded}
\begin{Highlighting}[]
\FunctionTok{plot}\NormalTok{(cars)}
\FunctionTok{abline}\NormalTok{(}\AttributeTok{h =} \FunctionTok{c}\NormalTok{(}\DecValTok{20}\NormalTok{, }\DecValTok{40}\NormalTok{), }\AttributeTok{lty =} \DecValTok{2}\NormalTok{) }\CommentTok{\# líneas horizontales discontinuas (lty=2)}
\CommentTok{\# selecciona puntos y los dibuja en azul sólido}
\FunctionTok{points}\NormalTok{(}\FunctionTok{subset}\NormalTok{(cars, dist }\SpecialCharTok{\textgreater{}} \DecValTok{20} \SpecialCharTok{\&}\NormalTok{ dist }\SpecialCharTok{\textless{}} \DecValTok{40}\NormalTok{), }\AttributeTok{pch =} \DecValTok{16}\NormalTok{, }\AttributeTok{col =} \StringTok{\textquotesingle{}blue\textquotesingle{}}\NormalTok{) }
\end{Highlighting}
\end{Shaded}

\begin{center}\includegraphics[width=0.7\linewidth]{03-Graficos_files/figure-latex/unnamed-chunk-2-1} \end{center}

\begin{Shaded}
\begin{Highlighting}[]
\NormalTok{x }\OtherTok{\textless{}{-}} \FunctionTok{seq}\NormalTok{(}\DecValTok{0}\NormalTok{, }\DecValTok{2} \SpecialCharTok{*}\NormalTok{ pi, }\AttributeTok{length =} \DecValTok{100}\NormalTok{)}
\NormalTok{y1 }\OtherTok{\textless{}{-}} \FunctionTok{cos}\NormalTok{(x)}
\NormalTok{y2 }\OtherTok{\textless{}{-}} \FunctionTok{sin}\NormalTok{(x)}
\FunctionTok{plot}\NormalTok{( x, y1, }\AttributeTok{type =} \StringTok{"l"}\NormalTok{, }\AttributeTok{col =} \DecValTok{2}\NormalTok{, }\AttributeTok{lwd =} \DecValTok{3}\NormalTok{, }\AttributeTok{xlab =} \StringTok{"[0,2pi]"}\NormalTok{, }\AttributeTok{ylab =} \StringTok{""}\NormalTok{, }\AttributeTok{main =} \StringTok{"Seno y Coseno"}\NormalTok{)}
\FunctionTok{lines}\NormalTok{(x, y2, }\AttributeTok{col =} \DecValTok{3}\NormalTok{, }\AttributeTok{lwd =} \DecValTok{3}\NormalTok{, }\AttributeTok{lty =} \DecValTok{2}\NormalTok{)}
\FunctionTok{points}\NormalTok{(pi, }\DecValTok{0}\NormalTok{, }\AttributeTok{pch =} \DecValTok{17}\NormalTok{, }\AttributeTok{col =} \DecValTok{4}\NormalTok{)}
\FunctionTok{legend}\NormalTok{(}\DecValTok{0}\NormalTok{, }\SpecialCharTok{{-}}\FloatTok{0.5}\NormalTok{, }\FunctionTok{c}\NormalTok{(}\StringTok{"Coseno"}\NormalTok{, }\StringTok{"Seno"}\NormalTok{), }\AttributeTok{col =} \DecValTok{2}\SpecialCharTok{:}\DecValTok{3}\NormalTok{, }\AttributeTok{lty =} \DecValTok{1}\SpecialCharTok{:}\DecValTok{2}\NormalTok{, }\AttributeTok{lwd =} \DecValTok{3}\NormalTok{)}

\FunctionTok{abline}\NormalTok{(}\AttributeTok{v =}\NormalTok{ pi, }\AttributeTok{lty =} \DecValTok{3}\NormalTok{)}
\FunctionTok{abline}\NormalTok{(}\AttributeTok{h =} \DecValTok{0}\NormalTok{, }\AttributeTok{lty =} \DecValTok{3}\NormalTok{)}
\FunctionTok{text}\NormalTok{(pi, }\DecValTok{0}\NormalTok{, }\StringTok{"(pi,0)"}\NormalTok{, }\AttributeTok{adj =} \FunctionTok{c}\NormalTok{(}\DecValTok{0}\NormalTok{, }\DecValTok{0}\NormalTok{))}
\end{Highlighting}
\end{Shaded}

\begin{center}\includegraphics[width=0.7\linewidth]{03-Graficos_files/figure-latex/unnamed-chunk-3-1} \end{center}

Alternativamente se podría usar \texttt{curve()}:

\begin{Shaded}
\begin{Highlighting}[]
\FunctionTok{curve}\NormalTok{(cos, }\DecValTok{0}\NormalTok{, }\DecValTok{2}\SpecialCharTok{*}\NormalTok{pi, }\AttributeTok{col =} \DecValTok{2}\NormalTok{, }\AttributeTok{lwd =} \DecValTok{3}\NormalTok{, }
      \AttributeTok{xlab =} \StringTok{"[0,2pi]"}\NormalTok{, }\AttributeTok{ylab =} \StringTok{""}\NormalTok{, }\AttributeTok{main =} \StringTok{"Seno y Coseno"}\NormalTok{)}
\FunctionTok{curve}\NormalTok{(sin, }\AttributeTok{col =} \DecValTok{3}\NormalTok{, }\AttributeTok{lwd =} \DecValTok{3}\NormalTok{, }\AttributeTok{lty =} \DecValTok{2}\NormalTok{, }\AttributeTok{add =} \ConstantTok{TRUE}\NormalTok{)}
\end{Highlighting}
\end{Shaded}

\begin{center}\includegraphics[width=0.7\linewidth]{03-Graficos_files/figure-latex/unnamed-chunk-4-1} \end{center}

\hypertarget{bajo-nivel-plot}{%
\section{Parámetros gráficos}\label{bajo-nivel-plot}}

Como ya hemos visto, muchas funciones gráficas permiten establecer (temporalmente) opciones gráficas mediante estos parámetros.
Con la función \texttt{par()} se pueden obtener y establecer (de forma permanente) todas las opciones gráficas.
Algunas más de estas opciones son:

\begin{longtable}[]{@{}
  >{\raggedright\arraybackslash}p{(\columnwidth - 2\tabcolsep) * \real{0.17}}
  >{\raggedright\arraybackslash}p{(\columnwidth - 2\tabcolsep) * \real{0.67}}@{}}
\toprule
Parámetro & Descripción \\
\midrule
\endhead
\texttt{adj} & justificación del texto \\
\texttt{axes} & si es \texttt{FALSE} no dibuja los ejes ni la caja \\
\texttt{bg} & color del fondo \\
\texttt{bty} & tipo de caja alrededor del gráfico \\
\texttt{font} & estilo del texto
(1: normal, 2: cursiva, 3:negrita, 4: negrita cursiva) \\
\texttt{las} & orientación de los caracteres en los ejes \\
\texttt{mar} & márgenes \\
\texttt{mfcol} & divide la pantalla gráfica por columnas \\
\texttt{mfrow} & lo mismo que \texttt{mfcol} pero por filas \\
\bottomrule
\end{longtable}

Ejecutar \texttt{help(par)} para obtener la lista completa.

\hypertarget{mfrow-plot}{%
\section{Múltiples gráficos por ventana}\label{mfrow-plot}}

En \texttt{R} se pueden hacer varios gráficos por ventana.
Para ello, antes de ejecutar la función \texttt{plot()}, se puede ejecutar:

\begin{Shaded}
\begin{Highlighting}[]
\FunctionTok{par}\NormalTok{(}\AttributeTok{mfrow =} \FunctionTok{c}\NormalTok{(filas, columnas))}
\end{Highlighting}
\end{Shaded}

Los gráficos se irán mostrando en pantalla por filas. En caso de que se
quieran mostrar por columnas en la función anterior se sustituye \texttt{mfrow}
por \texttt{mfcol}.

Por ejemplo:

\begin{Shaded}
\begin{Highlighting}[]
\NormalTok{old.par }\OtherTok{\textless{}{-}} \FunctionTok{par}\NormalTok{(}\AttributeTok{mfrow =} \FunctionTok{c}\NormalTok{(}\DecValTok{2}\NormalTok{, }\DecValTok{3}\NormalTok{))}
\FunctionTok{plot}\NormalTok{(cars, }\AttributeTok{pch =} \DecValTok{1}\NormalTok{, }\AttributeTok{main =} \StringTok{"pch = 1"}\NormalTok{)}
\FunctionTok{plot}\NormalTok{(cars, }\AttributeTok{pch =} \DecValTok{2}\NormalTok{, }\AttributeTok{main =} \StringTok{"pch = 2"}\NormalTok{)}
\FunctionTok{plot}\NormalTok{(cars, }\AttributeTok{pch =} \DecValTok{3}\NormalTok{, }\AttributeTok{main =} \StringTok{"pch = 3"}\NormalTok{)}

\FunctionTok{plot}\NormalTok{(cars, }\AttributeTok{col =} \StringTok{"red"}\NormalTok{, }\AttributeTok{main =} \StringTok{"col = red"}\NormalTok{)}
\FunctionTok{plot}\NormalTok{(cars, }\AttributeTok{col =} \StringTok{"blue"}\NormalTok{, }\AttributeTok{main =} \StringTok{"col = blue"}\NormalTok{)}
\FunctionTok{plot}\NormalTok{(cars, }\AttributeTok{col =} \StringTok{"brown"}\NormalTok{, }\AttributeTok{main =} \StringTok{"col = brown"}\NormalTok{)}
\end{Highlighting}
\end{Shaded}

\begin{center}\includegraphics[width=0.7\linewidth]{03-Graficos_files/figure-latex/unnamed-chunk-6-1} \end{center}

\begin{Shaded}
\begin{Highlighting}[]
\FunctionTok{par}\NormalTok{(old.par)}
\end{Highlighting}
\end{Shaded}

La función \texttt{par()} devuelve la anterior configuración de parámetros,
lo que permite volverlos a establecer.

Para estructuras gráficas más complicadas véase \texttt{help(layout)}.

\hypertarget{exportar-gruxe1ficos}{%
\section{Exportar gráficos}\label{exportar-gruxe1ficos}}

Para guardar gráficos, en Windows, se puede usar el menú \texttt{Archivo\ -\textgreater{}\ Guardar\ como} de la ventana gráfica (seleccionando el formato deseado: bitmap, postscript,\ldots)
y también mediante código ejecutando \texttt{savePlot(filename,\ type)}.
Alternativamente, se pueden emplear ficheros como dispositivos gráficos. Por ejemplo, a continuación guardamos un gráfico en el fichero \emph{car.pdf}:

\begin{Shaded}
\begin{Highlighting}[]
\FunctionTok{pdf}\NormalTok{(}\StringTok{"cars.pdf"}\NormalTok{)   }\CommentTok{\# abrimos el dispositivo gráfico}
\FunctionTok{plot}\NormalTok{(cars)}
\FunctionTok{dev.off}\NormalTok{()         }\CommentTok{\# cerramos el dispositivo}
\end{Highlighting}
\end{Shaded}

Con el siguiente código guardaremos el gráfico en formato jpeg:

\begin{Shaded}
\begin{Highlighting}[]
\FunctionTok{jgeg}\NormalTok{(}\StringTok{"cars.jpg"}\NormalTok{)  }\CommentTok{\# abrimos el dispositivo gráfico}
\FunctionTok{plot}\NormalTok{(cars)}
\FunctionTok{dev.off}\NormalTok{()         }\CommentTok{\# cerramos el dispositivo}
\end{Highlighting}
\end{Shaded}

Otros formatos disponibles son \texttt{bmp}, \texttt{png} y \texttt{tiff}. Para más detalles ejecutar \texttt{help(Devices)}.

Sin embargo para exportar resultados, incluyendo gráficos, suele ser preferible emplear informes en RMarkdown cono se muestra en el Capítulo \ref{informes}.

\hypertarget{otras-libreruxedas-gruxe1ficas}{%
\section{Otras librerías gráficas}\label{otras-libreruxedas-gruxe1ficas}}

Además de los gráficos estándar, en \texttt{R} están disponibles muchas librerías gráficas adicionales:

\begin{itemize}
\item
  Gráficos Lattice (Trellis)

  \begin{itemize}
  \item
    Especialmente adecuados para gráficas condicionales múltiples.
  \item
    No se pueden combinar con las funciones estándar.
  \item
    Generalmente el argumento principal es una formula:

    \begin{itemize}
    \item
      \texttt{y\ \textasciitilde{}\ x\ \textbar{}\ a} gráficas de \texttt{y} sobre \texttt{x} condicionadas por \texttt{a}
    \item
      \texttt{y\ \textasciitilde{}\ x\ \textbar{}\ a*b} gráficas condicionadas por \texttt{a} y \texttt{b}
    \end{itemize}
  \item
    Devuelven un objeto con el que se puede interactuar.
  \end{itemize}
\item
  \href{https://ggplot2.tidyverse.org}{ggplot2}: Create Elegant Data Visualisations Using the Grammar of Graphics.
\item
  \href{https://dmurdoch.github.io/rgl}{rgl}: 3D visualization device system for R using OpenGL.
\end{itemize}

Para más detalles ver \href{http://cran.r-project.org/web/views/Graphics.html}{CRAN Task View: Graphics}

\hypertarget{ejemplos-2}{%
\subsection{Ejemplos}\label{ejemplos-2}}

\begin{Shaded}
\begin{Highlighting}[]
\FunctionTok{load}\NormalTok{(}\StringTok{"datos/empleados.RData"}\NormalTok{)}
\FunctionTok{library}\NormalTok{(lattice)}
\FunctionTok{xyplot}\NormalTok{(}\FunctionTok{log}\NormalTok{(salario) }\SpecialCharTok{\textasciitilde{}} \FunctionTok{log}\NormalTok{(salini) }\SpecialCharTok{|}\NormalTok{ sexoraza, }\AttributeTok{data =}\NormalTok{ empleados)}
\end{Highlighting}
\end{Shaded}

\begin{center}\includegraphics[width=0.7\linewidth]{03-Graficos_files/figure-latex/unnamed-chunk-9-1} \end{center}

\begin{Shaded}
\begin{Highlighting}[]
\CommentTok{\# Equivalente a xyplot(log(salario) \textasciitilde{} log(salini) | sexo*minoria, data = empleados)}
\end{Highlighting}
\end{Shaded}

\begin{Shaded}
\begin{Highlighting}[]
\FunctionTok{library}\NormalTok{(ggplot2)}
\FunctionTok{ggplot}\NormalTok{(empleados, }\FunctionTok{aes}\NormalTok{(}\FunctionTok{log}\NormalTok{(salini), }\FunctionTok{log}\NormalTok{(salario), }\AttributeTok{col =}\NormalTok{ sexo)) }\SpecialCharTok{+}
  \FunctionTok{geom\_point}\NormalTok{() }\SpecialCharTok{+}
  \FunctionTok{geom\_smooth}\NormalTok{(}\AttributeTok{method =} \StringTok{"lm"}\NormalTok{, }\AttributeTok{se =} \ConstantTok{FALSE}\NormalTok{)}
\end{Highlighting}
\end{Shaded}

\begin{center}\includegraphics[width=0.7\linewidth]{03-Graficos_files/figure-latex/unnamed-chunk-10-1} \end{center}

\begin{Shaded}
\begin{Highlighting}[]
\FunctionTok{ggplot}\NormalTok{(empleados, }\FunctionTok{aes}\NormalTok{(salario, }\AttributeTok{fill =}\NormalTok{ sexo)) }\SpecialCharTok{+}
  \FunctionTok{geom\_density}\NormalTok{(}\AttributeTok{alpha=}\NormalTok{.}\DecValTok{5}\NormalTok{)}
\end{Highlighting}
\end{Shaded}

\begin{center}\includegraphics[width=0.7\linewidth]{03-Graficos_files/figure-latex/unnamed-chunk-10-2} \end{center}

\hypertarget{manipulaciuxf3n-de-datos-con-r}{%
\chapter{Manipulación de datos con R}\label{manipulaciuxf3n-de-datos-con-r}}

La mayoría de los estudios estadísticos
requieren disponer de un conjunto de datos.

\hypertarget{lectura-importaciuxf3n-y-exportaciuxf3n-de-datos}{%
\section{Lectura, importación y exportación de datos}\label{lectura-importaciuxf3n-y-exportaciuxf3n-de-datos}}

Además de la introducción directa, R es capaz de
importar datos externos en múltiples formatos:

\begin{itemize}
\item
  bases de datos disponibles en librerías de R
\item
  archivos de texto en formato ASCII
\item
  archivos en otros formatos: Excel, SPSS, \ldots{}
\item
  bases de datos relacionales: MySQL, Oracle, \ldots{}
\item
  formatos web: HTML, XML, JSON, \ldots{}
\item
  \ldots.
\end{itemize}

\hypertarget{formato-de-datos-de-r}{%
\subsection{Formato de datos de R}\label{formato-de-datos-de-r}}

El formato de archivo en el que habitualmente se almacena objetos (datos)
R es binario y está comprimido (en formato \texttt{"gzip"} por defecto).
Para cargar un fichero de datos se emplea normalmente \href{https://www.rdocumentation.org/packages/base/versions/3.6.1/topics/load}{\texttt{load()}}:

\begin{Shaded}
\begin{Highlighting}[]
\NormalTok{res }\OtherTok{\textless{}{-}} \FunctionTok{load}\NormalTok{(}\StringTok{"datos/empleados.RData"}\NormalTok{)}
\NormalTok{res}
\end{Highlighting}
\end{Shaded}

\begin{verbatim}
## [1] "empleados"
\end{verbatim}

\begin{Shaded}
\begin{Highlighting}[]
\FunctionTok{ls}\NormalTok{()}
\end{Highlighting}
\end{Shaded}

\begin{verbatim}
##  [1] "cite_fsimres" "cite_simres"  "citefig"      "citefig2"     "citepkg"     
##  [6] "empleados"    "fig.path"     "inline"       "inline2"      "is_html"     
## [11] "is_latex"     "latexfig"     "res"
\end{verbatim}

y para guardar \href{https://www.rdocumentation.org/packages/base/versions/3.6.1/topics/save}{\texttt{save()}}:

\begin{Shaded}
\begin{Highlighting}[]
\CommentTok{\# Guardar}
\FunctionTok{save}\NormalTok{(empleados, }\AttributeTok{file =} \StringTok{"datos/empleados\_new.RData"}\NormalTok{)}
\end{Highlighting}
\end{Shaded}

El objeto empleado normalmente en R para almacenar datos en memoria
es el \href{https://www.rdocumentation.org/packages/base/versions/3.6.1/topics/data.frame}{\texttt{data.frame}}.

\hypertarget{acceso-a-datos-en-paquetes}{%
\subsection{Acceso a datos en paquetes}\label{acceso-a-datos-en-paquetes}}

R dispone de múltiples conjuntos de datos en distintos paquetes, especialmente en el paquete \texttt{datasets}
que se carga por defecto al abrir R.
Con el comando \texttt{data()} podemos obtener un listado de las bases de datos disponibles.

Para cargar una base de datos concreta se utiliza el comando
\texttt{data(nombre)} (aunque en algunos casos se cargan automáticamente al emplearlos).
Por ejemplo, \texttt{data(cars)} carga la base de datos llamada \texttt{cars} en el entorno de trabajo (\texttt{".GlobalEnv"})
y \texttt{?cars} muestra la ayuda correspondiente con la descripición de la base de datos.

\hypertarget{cap4-texto}{%
\subsection{Lectura de archivos de texto}\label{cap4-texto}}

En R para leer archivos de texto se suele utilizar la función \texttt{read.table()}.
Supóngase, por ejemplo, que en el directorio actual está el fichero
\emph{empleados.txt}. La lectura de este fichero vendría dada por el código:

\begin{Shaded}
\begin{Highlighting}[]
\CommentTok{\# Session \textgreater{} Set Working Directory \textgreater{} To Source...?}
\NormalTok{datos }\OtherTok{\textless{}{-}} \FunctionTok{read.table}\NormalTok{(}\AttributeTok{file =} \StringTok{"datos/empleados.txt"}\NormalTok{, }\AttributeTok{header =} \ConstantTok{TRUE}\NormalTok{)}
\CommentTok{\# head(datos)}
\FunctionTok{str}\NormalTok{(datos)}
\end{Highlighting}
\end{Shaded}

\begin{verbatim}
## 'data.frame':    474 obs. of  10 variables:
##  $ id      : int  1 2 3 4 5 6 7 8 9 10 ...
##  $ sexo    : chr  "Hombre" "Hombre" "Mujer" "Mujer" ...
##  $ fechnac : chr  "2/3/1952" "5/23/1958" "7/26/1929" "4/15/1947" ...
##  $ educ    : int  15 16 12 8 15 15 15 12 15 12 ...
##  $ catlab  : chr  "Directivo" "Administrativo" "Administrativo" "Administrativo" ...
##  $ salario : num  57000 40200 21450 21900 45000 ...
##  $ salini  : int  27000 18750 12000 13200 21000 13500 18750 9750 12750 13500 ...
##  $ tiempemp: int  98 98 98 98 98 98 98 98 98 98 ...
##  $ expprev : int  144 36 381 190 138 67 114 0 115 244 ...
##  $ minoria : chr  "No" "No" "No" "No" ...
\end{verbatim}

Si el fichero estuviese en el directorio \emph{c:\textbackslash datos} bastaría con especificar
\texttt{file\ =\ "c:/datos/empleados.txt"}.
Nótese también que para la lectura del fichero anterior se ha
establecido el argumento \texttt{header=TRUE} para indicar que la primera línea del
fichero contiene los nombres de las variables.

Los argumentos utilizados habitualmente para esta función son:

\begin{itemize}
\item
  \texttt{header}: indica si el fichero tiene cabecera (\texttt{header=TRUE}) o no
  (\texttt{header=FALSE}). Por defecto toma el valor \texttt{header=FALSE}.
\item
  \texttt{sep}: carácter separador de columnas que por defecto es un espacio
  en blanco (\texttt{sep=""}). Otras opciones serían: \texttt{sep=","} si el separador es
  un ``;'', \texttt{sep="*"} si el separador es un ``*'', etc.
\item
  \texttt{dec}: carácter utilizado en el fichero para los números decimales.
  Por defecto se establece \texttt{dec\ =\ "."}. Si los decimales vienen dados
  por ``,'' se utiliza \texttt{dec\ =\ ","}
\end{itemize}

Resumiendo, los (principales) argumentos por defecto de la función
\texttt{read.table} son los que se muestran en la siguiente línea:

\begin{Shaded}
\begin{Highlighting}[]
\FunctionTok{read.table}\NormalTok{(file, }\AttributeTok{header =} \ConstantTok{FALSE}\NormalTok{, }\AttributeTok{sep =} \StringTok{""}\NormalTok{, }\AttributeTok{dec =} \StringTok{"."}\NormalTok{)  }
\end{Highlighting}
\end{Shaded}

Para más detalles sobre esta función véase
\texttt{help(read.table)}.

Estan disponibles otras funciones con valores por defecto de los parámetros
adecuados para otras situaciones. Por ejemplo, para ficheros separados por tabuladores
se puede utilizar \texttt{read.delim()} o \texttt{read.delim2()}:

\begin{Shaded}
\begin{Highlighting}[]
\FunctionTok{read.delim}\NormalTok{(file, }\AttributeTok{header =} \ConstantTok{TRUE}\NormalTok{, }\AttributeTok{sep =} \StringTok{"}\SpecialCharTok{\textbackslash{}t}\StringTok{"}\NormalTok{, }\AttributeTok{dec =} \StringTok{"."}\NormalTok{)}
\FunctionTok{read.delim2}\NormalTok{(file, }\AttributeTok{header =} \ConstantTok{TRUE}\NormalTok{, }\AttributeTok{sep =} \StringTok{"}\SpecialCharTok{\textbackslash{}t}\StringTok{"}\NormalTok{, }\AttributeTok{dec =} \StringTok{","}\NormalTok{)}
\end{Highlighting}
\end{Shaded}

\hypertarget{alternativa-tidyverse}{%
\subsection{\texorpdfstring{Alternativa \texttt{tidyverse}}{Alternativa tidyverse}}\label{alternativa-tidyverse}}

Para leer archivos de texto en distintos formatos también se puede emplear el paquete \href{https://readr.tidyverse.org}{\texttt{readr}}
(colección \href{https://www.tidyverse.org/}{\texttt{tidyverse}}), para lo que se recomienda
consultar el \href{https://r4ds.had.co.nz/data-import.html}{Capítulo 11} del libro \href{http://r4ds.had.co.nz}{R for Data Science}.

\hypertarget{importaciuxf3n-desde-spss}{%
\subsection{Importación desde SPSS}\label{importaciuxf3n-desde-spss}}

El programa R permite
lectura de ficheros de datos en formato SPSS (extensión \emph{.sav}) sin
necesidad de tener instalado dicho programa en el ordenador. Para ello
se necesita:

\begin{itemize}
\item
  cargar la librería \texttt{foreign}
\item
  utilizar la función \texttt{read.spss}
\end{itemize}

Por ejemplo:

\begin{Shaded}
\begin{Highlighting}[]
\FunctionTok{library}\NormalTok{(foreign)}
\NormalTok{datos }\OtherTok{\textless{}{-}} \FunctionTok{read.spss}\NormalTok{(}\AttributeTok{file =} \StringTok{"datos/Employee data.sav"}\NormalTok{, }\AttributeTok{to.data.frame =} \ConstantTok{TRUE}\NormalTok{)}
\CommentTok{\# head(datos)}
\FunctionTok{str}\NormalTok{(datos)}
\end{Highlighting}
\end{Shaded}

\begin{verbatim}
## 'data.frame':    474 obs. of  10 variables:
##  $ id      : num  1 2 3 4 5 6 7 8 9 10 ...
##  $ sexo    : Factor w/ 2 levels "Hombre","Mujer": 1 1 2 2 1 1 1 2 2 2 ...
##  $ fechnac : num  1.17e+10 1.19e+10 1.09e+10 1.15e+10 1.17e+10 ...
##  $ educ    : Factor w/ 10 levels "8","12","14",..: 4 5 2 1 4 4 4 2 4 2 ...
##  $ catlab  : Factor w/ 3 levels "Administrativo",..: 3 1 1 1 1 1 1 1 1 1 ...
##  $ salario : Factor w/ 221 levels "15750","15900",..: 179 137 28 31 150 101 121 31 71 45 ...
##  $ salini  : Factor w/ 90 levels "9000","9750",..: 60 42 13 21 48 23 42 2 18 23 ...
##  $ tiempemp: Factor w/ 36 levels "63","64","65",..: 36 36 36 36 36 36 36 36 36 36 ...
##  $ expprev : Factor w/ 208 levels "Ausente","10",..: 38 131 139 64 34 181 13 1 14 91 ...
##  $ minoria : Factor w/ 2 levels "No","Sí": 1 1 1 1 1 1 1 1 1 1 ...
##  - attr(*, "variable.labels")= Named chr [1:10] "Código de empleado" "Sexo" "Fecha de nacimiento" "Nivel educativo" ...
##   ..- attr(*, "names")= chr [1:10] "id" "sexo" "fechnac" "educ" ...
##  - attr(*, "codepage")= int 1252
\end{verbatim}

\textbf{Nota}: Si hay fechas, puede ser recomendable emplear la función \texttt{spss.get()} del paquete \texttt{Hmisc}.

\hypertarget{importaciuxf3n-desde-excel}{%
\subsection{Importación desde Excel}\label{importaciuxf3n-desde-excel}}

Se pueden leer fichero de
Excel (con extensión \emph{.xlsx}) utilizando por ejemplo los paquetes \href{https://cran.r-project.org/web/packages/openxlsx/index.html}{\texttt{openxlsx}}, \href{https://readxl.tidyverse.org}{\texttt{readxl}} (colección \href{https://www.tidyverse.org/}{\texttt{tidyverse}}), \texttt{XLConnect} o
\href{https://cran.r-project.org/web/packages/RODBC/index.html}{\texttt{RODBC}} (este paquete se empleará más adelante para acceder a bases de datos),
entre otros.

Sin embargo, un procedimiento sencillo consiste en exportar los datos desde Excel a un archivo
de texto separado por tabuladores (extensión \emph{.csv}).
Por ejemplo, supongamos que queremos leer el fichero \emph{coches.xls}:

\begin{itemize}
\item
  Desde Excel se selecciona el menú
  \texttt{Archivo\ -\textgreater{}\ Guardar\ como\ -\textgreater{}\ Guardar\ como} y en \texttt{Tipo} se escoge la opción de
  archivo CSV. De esta forma se guardarán los datos en el archivo
  \emph{coches.csv}.
\item
  El fichero \emph{coches.csv} es un fichero de texto plano (se puede
  editar con Notepad), con cabecera, las columnas separadas por ``;'', y
  siendo ``,'' el carácter decimal.
\item
  Por lo tanto, la lectura de este fichero se puede hacer con:

\begin{Shaded}
\begin{Highlighting}[]
\NormalTok{datos }\OtherTok{\textless{}{-}} \FunctionTok{read.table}\NormalTok{(}\StringTok{"coches.csv"}\NormalTok{, }\AttributeTok{header =} \ConstantTok{TRUE}\NormalTok{, }\AttributeTok{sep =} \StringTok{";"}\NormalTok{, }\AttributeTok{dec =} \StringTok{","}\NormalTok{)}
\end{Highlighting}
\end{Shaded}
\end{itemize}

Otra posibilidad es utilizar la función \texttt{read.csv2}, que es
una adaptación de la función general \texttt{read.table} con las siguientes
opciones:

\begin{Shaded}
\begin{Highlighting}[]
\FunctionTok{read.csv2}\NormalTok{(file, }\AttributeTok{header =} \ConstantTok{TRUE}\NormalTok{, }\AttributeTok{sep =} \StringTok{";"}\NormalTok{, }\AttributeTok{dec =} \StringTok{","}\NormalTok{)}
\end{Highlighting}
\end{Shaded}

Por lo tanto, la lectura del fichero \emph{coches.csv} se puede hacer de modo
más directo con:

\begin{Shaded}
\begin{Highlighting}[]
\NormalTok{datos }\OtherTok{\textless{}{-}} \FunctionTok{read.csv2}\NormalTok{(}\StringTok{"coches.csv"}\NormalTok{)}
\end{Highlighting}
\end{Shaded}

\hypertarget{exportaciuxf3n-de-datos}{%
\subsection{Exportación de datos}\label{exportaciuxf3n-de-datos}}

Puede ser de interés la
exportación de datos para que puedan leídos con otros programas. Para
ello, se puede emplear la función \texttt{write.table()}. Esta función es
similar, pero funcionando en sentido inverso, a \texttt{read.table()}
(Sección \ref{cap4-texto}).

Veamos un ejemplo:

\begin{Shaded}
\begin{Highlighting}[]
\NormalTok{tipo }\OtherTok{\textless{}{-}} \FunctionTok{c}\NormalTok{(}\StringTok{"A"}\NormalTok{, }\StringTok{"B"}\NormalTok{, }\StringTok{"C"}\NormalTok{)}
\NormalTok{longitud }\OtherTok{\textless{}{-}} \FunctionTok{c}\NormalTok{(}\FloatTok{120.34}\NormalTok{, }\FloatTok{99.45}\NormalTok{, }\FloatTok{115.67}\NormalTok{)}
\NormalTok{datos }\OtherTok{\textless{}{-}} \FunctionTok{data.frame}\NormalTok{(tipo, longitud)}
\NormalTok{datos}
\end{Highlighting}
\end{Shaded}

\begin{verbatim}
##   tipo longitud
## 1    A   120.34
## 2    B    99.45
## 3    C   115.67
\end{verbatim}

Para guardar el data.frame \texttt{datos} en un fichero de texto se
puede utilizar:

\begin{Shaded}
\begin{Highlighting}[]
\FunctionTok{write.table}\NormalTok{(datos, }\AttributeTok{file =} \StringTok{"datos.txt"}\NormalTok{)}
\end{Highlighting}
\end{Shaded}

Otra posibilidad es utilizar la función:

\begin{Shaded}
\begin{Highlighting}[]
\FunctionTok{write.csv2}\NormalTok{(datos, }\AttributeTok{file =} \StringTok{"datos.csv"}\NormalTok{)}
\end{Highlighting}
\end{Shaded}

que dará lugar al fichero \emph{datos.csv} importable directamente desde Excel.

\hypertarget{manipulaciuxf3n-de-datos}{%
\section{Manipulación de datos}\label{manipulaciuxf3n-de-datos}}

Una vez cargada una (o varias) bases
de datos hay una series de operaciones que serán de interés para el
tratamiento de datos:

\begin{itemize}
\tightlist
\item
  Operaciones con variables:

  \begin{itemize}
  \tightlist
  \item
    crear
  \item
    recodificar (e.g.~categorizar)
  \item
    \ldots{}
  \end{itemize}
\item
  Operaciones con casos:

  \begin{itemize}
  \tightlist
  \item
    ordenar
  \item
    filtrar
  \item
    \ldots{}
  \end{itemize}
\end{itemize}

A continuación se tratan algunas operaciones \emph{básicas}.

\hypertarget{operaciones-con-variables}{%
\subsection{Operaciones con variables}\label{operaciones-con-variables}}

\hypertarget{creaciuxf3n-y-eliminaciuxf3n-de-variables}{%
\subsubsection{Creación (y eliminación) de variables}\label{creaciuxf3n-y-eliminaciuxf3n-de-variables}}

Consideremos de nuevo la
base de datos \texttt{cars} incluida en el paquete \texttt{datasets}:

\begin{Shaded}
\begin{Highlighting}[]
\FunctionTok{data}\NormalTok{(cars)}
\CommentTok{\# str(cars)}
\FunctionTok{head}\NormalTok{(cars)}
\end{Highlighting}
\end{Shaded}

\begin{verbatim}
##   speed dist
## 1     4    2
## 2     4   10
## 3     7    4
## 4     7   22
## 5     8   16
## 6     9   10
\end{verbatim}

Utilizando el comando \texttt{help(cars)}
se obtiene que \texttt{cars} es un data.frame con 50 observaciones y dos
variables:

\begin{itemize}
\item
  \texttt{speed}: Velocidad (millas por hora)
\item
  \texttt{dist}: tiempo hasta detenerse (pies)
\end{itemize}

Recordemos que, para acceder a la variable \texttt{speed} se puede
hacer directamente con su nombre o bien utilizando notación
``matricial''.

\begin{Shaded}
\begin{Highlighting}[]
\NormalTok{cars}\SpecialCharTok{$}\NormalTok{speed}
\end{Highlighting}
\end{Shaded}

\begin{verbatim}
##  [1]  4  4  7  7  8  9 10 10 10 11 11 12 12 12 12 13 13 13 13 14 14 14 14 15 15
## [26] 15 16 16 17 17 17 18 18 18 18 19 19 19 20 20 20 20 20 22 23 24 24 24 24 25
\end{verbatim}

\begin{Shaded}
\begin{Highlighting}[]
\NormalTok{cars[, }\DecValTok{1}\NormalTok{]  }\CommentTok{\# Equivalente}
\end{Highlighting}
\end{Shaded}

\begin{verbatim}
##  [1]  4  4  7  7  8  9 10 10 10 11 11 12 12 12 12 13 13 13 13 14 14 14 14 15 15
## [26] 15 16 16 17 17 17 18 18 18 18 19 19 19 20 20 20 20 20 22 23 24 24 24 24 25
\end{verbatim}

Supongamos ahora que queremos transformar la variable original \texttt{speed}
(millas por hora) en una nueva variable \texttt{velocidad} (kilómetros por
hora) y añadir esta nueva variable al data.frame \texttt{cars}.
La transformación que permite pasar millas a kilómetros es
\texttt{kilómetros=millas/0.62137} que en R se hace directamente con:

\begin{Shaded}
\begin{Highlighting}[]
\NormalTok{cars}\SpecialCharTok{$}\NormalTok{speed}\SpecialCharTok{/}\FloatTok{0.62137}
\end{Highlighting}
\end{Shaded}

Finalmente, incluimos la nueva variable que llamaremos
\texttt{velocidad} en \texttt{cars}:

\begin{Shaded}
\begin{Highlighting}[]
\NormalTok{cars}\SpecialCharTok{$}\NormalTok{velocidad }\OtherTok{\textless{}{-}}\NormalTok{ cars}\SpecialCharTok{$}\NormalTok{speed }\SpecialCharTok{/} \FloatTok{0.62137}
\FunctionTok{head}\NormalTok{(cars)}
\end{Highlighting}
\end{Shaded}

\begin{verbatim}
##   speed dist velocidad
## 1     4    2  6.437388
## 2     4   10  6.437388
## 3     7    4 11.265430
## 4     7   22 11.265430
## 5     8   16 12.874777
## 6     9   10 14.484124
\end{verbatim}

También transformaremos la variable \texttt{dist} (en pies) en una nueva
variable \texttt{distancia} (en metros). Ahora la transformación deseada es
\texttt{metros=pies/3.2808}:

\begin{Shaded}
\begin{Highlighting}[]
\NormalTok{cars}\SpecialCharTok{$}\NormalTok{distancia }\OtherTok{\textless{}{-}}\NormalTok{ cars}\SpecialCharTok{$}\NormalTok{dis }\SpecialCharTok{/} \FloatTok{3.2808}
\FunctionTok{head}\NormalTok{(cars)}
\end{Highlighting}
\end{Shaded}

\begin{verbatim}
##   speed dist velocidad distancia
## 1     4    2  6.437388 0.6096074
## 2     4   10  6.437388 3.0480371
## 3     7    4 11.265430 1.2192148
## 4     7   22 11.265430 6.7056815
## 5     8   16 12.874777 4.8768593
## 6     9   10 14.484124 3.0480371
\end{verbatim}

Ahora, eliminaremos las variables originales \texttt{speed} y
\texttt{dist}, y guardaremos el data.frame resultante con el nombre \texttt{coches}.
En primer lugar, veamos varias formas de acceder a las variables de
interés:

\begin{Shaded}
\begin{Highlighting}[]
\NormalTok{cars[, }\FunctionTok{c}\NormalTok{(}\DecValTok{3}\NormalTok{, }\DecValTok{4}\NormalTok{)]}
\NormalTok{cars[, }\FunctionTok{c}\NormalTok{(}\StringTok{"velocidad"}\NormalTok{, }\StringTok{"distancia"}\NormalTok{)]}
\NormalTok{cars[, }\SpecialCharTok{{-}}\FunctionTok{c}\NormalTok{(}\DecValTok{1}\NormalTok{, }\DecValTok{2}\NormalTok{)]}
\end{Highlighting}
\end{Shaded}

Utilizando alguna de las opciones anteriores se obtiene el \texttt{data.frame}
deseado:

\begin{Shaded}
\begin{Highlighting}[]
\NormalTok{coches }\OtherTok{\textless{}{-}}\NormalTok{ cars[, }\FunctionTok{c}\NormalTok{(}\StringTok{"velocidad"}\NormalTok{, }\StringTok{"distancia"}\NormalTok{)]}
\CommentTok{\# head(coches)}
\FunctionTok{str}\NormalTok{(coches)}
\end{Highlighting}
\end{Shaded}

\begin{verbatim}
## 'data.frame':    50 obs. of  2 variables:
##  $ velocidad: num  6.44 6.44 11.27 11.27 12.87 ...
##  $ distancia: num  0.61 3.05 1.22 6.71 4.88 ...
\end{verbatim}

Finalmente los datos anteriores podrían ser guardados en un fichero
exportable a Excel con el siguiente comando:

\begin{Shaded}
\begin{Highlighting}[]
\FunctionTok{write.csv2}\NormalTok{(coches, }\AttributeTok{file =} \StringTok{"coches.csv"}\NormalTok{)}
\end{Highlighting}
\end{Shaded}

\hypertarget{operaciones-con-casos}{%
\subsection{Operaciones con casos}\label{operaciones-con-casos}}

\hypertarget{ordenaciuxf3n}{%
\subsubsection{Ordenación}\label{ordenaciuxf3n}}

Continuemos con el data.frame \texttt{cars}.
Se puede comprobar que los datos disponibles están ordenados por
los valores de \texttt{speed}. A continuación haremos la ordenación utilizando
los valores de \texttt{dist}. Para ello utilizaremos el conocido como vector de
índices de ordenación.
Este vector establece el orden en que tienen que ser elegidos los
elementos para obtener la ordenación deseada.
Veamos un ejemplo sencillo:

\begin{Shaded}
\begin{Highlighting}[]
\NormalTok{x }\OtherTok{\textless{}{-}} \FunctionTok{c}\NormalTok{(}\FloatTok{2.5}\NormalTok{, }\FloatTok{4.3}\NormalTok{, }\FloatTok{1.2}\NormalTok{, }\FloatTok{3.1}\NormalTok{, }\FloatTok{5.0}\NormalTok{) }\CommentTok{\# valores originales}
\NormalTok{ii }\OtherTok{\textless{}{-}} \FunctionTok{order}\NormalTok{(x)}
\NormalTok{ii    }\CommentTok{\# vector de ordenación}
\end{Highlighting}
\end{Shaded}

\begin{verbatim}
## [1] 3 1 4 2 5
\end{verbatim}

\begin{Shaded}
\begin{Highlighting}[]
\NormalTok{x[ii] }\CommentTok{\# valores ordenados}
\end{Highlighting}
\end{Shaded}

\begin{verbatim}
## [1] 1.2 2.5 3.1 4.3 5.0
\end{verbatim}

En el caso de vectores, el procedimiento anterior se podría
hacer directamente con:

\begin{Shaded}
\begin{Highlighting}[]
\FunctionTok{sort}\NormalTok{(x)}
\end{Highlighting}
\end{Shaded}

Sin embargo, para ordenar data.frames será necesario la utilización del
vector de índices de ordenación. A continuación, los datos de \texttt{cars}
ordenados por \texttt{dist}:

\begin{Shaded}
\begin{Highlighting}[]
\NormalTok{ii }\OtherTok{\textless{}{-}} \FunctionTok{order}\NormalTok{(cars}\SpecialCharTok{$}\NormalTok{dist) }\CommentTok{\# Vector de índices de ordenación}
\NormalTok{cars2 }\OtherTok{\textless{}{-}}\NormalTok{ cars[ii, ]    }\CommentTok{\# Datos ordenados por dist}
\FunctionTok{head}\NormalTok{(cars2)}
\end{Highlighting}
\end{Shaded}

\begin{verbatim}
##    speed dist velocidad distancia
## 1      4    2  6.437388 0.6096074
## 3      7    4 11.265430 1.2192148
## 2      4   10  6.437388 3.0480371
## 6      9   10 14.484124 3.0480371
## 12    12   14 19.312165 4.2672519
## 5      8   16 12.874777 4.8768593
\end{verbatim}

\hypertarget{filtrado}{%
\subsubsection{Filtrado}\label{filtrado}}

El filtrado de datos consiste en
elegir una submuestra que cumpla determinadas condiciones. Para ello se
puede utilizar la función \href{https://www.rdocumentation.org/packages/base/versions/3.6.1/topics/subset}{\texttt{subset()}}
(que además permite seleccionar variables).

A continuación se muestran un par de ejemplos:

\begin{Shaded}
\begin{Highlighting}[]
\FunctionTok{subset}\NormalTok{(cars, dist }\SpecialCharTok{\textgreater{}} \DecValTok{85}\NormalTok{) }\CommentTok{\# datos con dis\textgreater{}85}
\end{Highlighting}
\end{Shaded}

\begin{verbatim}
##    speed dist velocidad distancia
## 47    24   92  38.62433  28.04194
## 48    24   93  38.62433  28.34674
## 49    24  120  38.62433  36.57644
\end{verbatim}

\begin{Shaded}
\begin{Highlighting}[]
\FunctionTok{subset}\NormalTok{(cars, speed }\SpecialCharTok{\textgreater{}} \DecValTok{10} \SpecialCharTok{\&}\NormalTok{ speed }\SpecialCharTok{\textless{}} \DecValTok{15} \SpecialCharTok{\&}\NormalTok{ dist }\SpecialCharTok{\textgreater{}} \DecValTok{45}\NormalTok{) }\CommentTok{\# speed en (10,15) y dist\textgreater{}45}
\end{Highlighting}
\end{Shaded}

\begin{verbatim}
##    speed dist velocidad distancia
## 19    13   46  20.92151  14.02097
## 22    14   60  22.53086  18.28822
## 23    14   80  22.53086  24.38430
\end{verbatim}

También se pueden hacer el filtrado empleando directamente los
correspondientes vectores de índices:

\begin{Shaded}
\begin{Highlighting}[]
\NormalTok{ii }\OtherTok{\textless{}{-}}\NormalTok{ cars}\SpecialCharTok{$}\NormalTok{dist }\SpecialCharTok{\textgreater{}} \DecValTok{85}
\NormalTok{cars[ii, ]   }\CommentTok{\# dis\textgreater{}85}
\end{Highlighting}
\end{Shaded}

\begin{verbatim}
##    speed dist velocidad distancia
## 47    24   92  38.62433  28.04194
## 48    24   93  38.62433  28.34674
## 49    24  120  38.62433  36.57644
\end{verbatim}

\begin{Shaded}
\begin{Highlighting}[]
\NormalTok{ii }\OtherTok{\textless{}{-}}\NormalTok{ cars}\SpecialCharTok{$}\NormalTok{speed }\SpecialCharTok{\textgreater{}} \DecValTok{10} \SpecialCharTok{\&}\NormalTok{ cars}\SpecialCharTok{$}\NormalTok{speed }\SpecialCharTok{\textless{}} \DecValTok{15} \SpecialCharTok{\&}\NormalTok{ cars}\SpecialCharTok{$}\NormalTok{dist }\SpecialCharTok{\textgreater{}} \DecValTok{45}
\NormalTok{cars[ii, ]  }\CommentTok{\# speed en (10,15) y dist\textgreater{}45}
\end{Highlighting}
\end{Shaded}

\begin{verbatim}
##    speed dist velocidad distancia
## 19    13   46  20.92151  14.02097
## 22    14   60  22.53086  18.28822
## 23    14   80  22.53086  24.38430
\end{verbatim}

En este caso puede ser de utilidad la función \href{https://www.rdocumentation.org/packages/base/versions/3.6.1/topics/which}{\texttt{which()}}:

\begin{Shaded}
\begin{Highlighting}[]
\NormalTok{it }\OtherTok{\textless{}{-}} \FunctionTok{which}\NormalTok{(ii)}
\FunctionTok{str}\NormalTok{(it)}
\end{Highlighting}
\end{Shaded}

\begin{verbatim}
##  int [1:3] 19 22 23
\end{verbatim}

\begin{Shaded}
\begin{Highlighting}[]
\NormalTok{cars[it, }\DecValTok{1}\SpecialCharTok{:}\DecValTok{2}\NormalTok{]}
\end{Highlighting}
\end{Shaded}

\begin{verbatim}
##    speed dist
## 19    13   46
## 22    14   60
## 23    14   80
\end{verbatim}

\begin{Shaded}
\begin{Highlighting}[]
\CommentTok{\# rownames(cars[it, 1:2])}

\NormalTok{id }\OtherTok{\textless{}{-}} \FunctionTok{which}\NormalTok{(}\SpecialCharTok{!}\NormalTok{ii)}
\FunctionTok{str}\NormalTok{(cars[id, }\DecValTok{1}\SpecialCharTok{:}\DecValTok{2}\NormalTok{])}
\end{Highlighting}
\end{Shaded}

\begin{verbatim}
## 'data.frame':    47 obs. of  2 variables:
##  $ speed: num  4 4 7 7 8 9 10 10 10 11 ...
##  $ dist : num  2 10 4 22 16 10 18 26 34 17 ...
\end{verbatim}

\begin{Shaded}
\begin{Highlighting}[]
\CommentTok{\# Se podría p.e. emplear cars[id, ] para predecir cars[it, ]$speed}
\CommentTok{\# ?which.min}
\end{Highlighting}
\end{Shaded}

\hypertarget{anuxe1lisis-exploratorio-de-datos}{%
\chapter{Análisis exploratorio de datos}\label{anuxe1lisis-exploratorio-de-datos}}

El objetivo del \emph{análisis exploratorio de datos} es presentar una descripción de los
mismos que faciliten su análisis mediante procedimientos que permitan:

\begin{itemize}
\tightlist
\item
  Organizar los datos
\item
  Resumirlos
\item
  Representarlos gráficamente
\item
  Análizar la información
\end{itemize}

\hypertarget{medidas-resumen}{%
\section{Medidas resumen}\label{medidas-resumen}}

\hypertarget{datos-de-ejemplo}{%
\subsection{Datos de ejemplo}\label{datos-de-ejemplo}}

El fichero \emph{empleados.RData} contiene datos de empleados de un banco que utilizaremos,
entre otros, a modo de ejemplo.

\begin{Shaded}
\begin{Highlighting}[]
\FunctionTok{load}\NormalTok{(}\StringTok{"datos/empleados.RData"}\NormalTok{)}
\FunctionTok{data.frame}\NormalTok{(}\AttributeTok{Etiquetas =} \FunctionTok{attr}\NormalTok{(empleados, }\StringTok{"variable.labels"}\NormalTok{))  }\CommentTok{\# Listamos las etiquetas}
\end{Highlighting}
\end{Shaded}

\begin{verbatim}
##                              Etiquetas
## id                  Código de empleado
## sexo                              Sexo
## fechnac            Fecha de nacimiento
## educ            Nivel educativo (años)
## catlab               Categoría Laboral
## salario                 Salario actual
## salini                 Salario inicial
## tiempemp       Meses desde el contrato
## expprev     Experiencia previa (meses)
## minoria           Clasificación étnica
## sexoraza Clasificación por sexo y raza
\end{verbatim}

Para hacer referencia directamente a las variables de \emph{empleados}

\begin{Shaded}
\begin{Highlighting}[]
\FunctionTok{attach}\NormalTok{(empleados)}
\end{Highlighting}
\end{Shaded}

\hypertarget{tablas-de-frecuencias}{%
\subsection{Tablas de frecuencias}\label{tablas-de-frecuencias}}

\begin{Shaded}
\begin{Highlighting}[]
\FunctionTok{table}\NormalTok{(sexo)}
\end{Highlighting}
\end{Shaded}

\begin{verbatim}
## sexo
## Hombre  Mujer 
##    258    216
\end{verbatim}

\begin{Shaded}
\begin{Highlighting}[]
\FunctionTok{prop.table}\NormalTok{(}\FunctionTok{table}\NormalTok{(sexo))}
\end{Highlighting}
\end{Shaded}

\begin{verbatim}
## sexo
##    Hombre     Mujer 
## 0.5443038 0.4556962
\end{verbatim}

\begin{Shaded}
\begin{Highlighting}[]
\FunctionTok{table}\NormalTok{(sexo,catlab)}
\end{Highlighting}
\end{Shaded}

\begin{verbatim}
##         catlab
## sexo     Administrativo Seguridad Directivo
##   Hombre            157        27        74
##   Mujer             206         0        10
\end{verbatim}

\begin{Shaded}
\begin{Highlighting}[]
\FunctionTok{prop.table}\NormalTok{(}\FunctionTok{table}\NormalTok{(sexo,catlab))}
\end{Highlighting}
\end{Shaded}

\begin{verbatim}
##         catlab
## sexo     Administrativo  Seguridad  Directivo
##   Hombre     0.33122363 0.05696203 0.15611814
##   Mujer      0.43459916 0.00000000 0.02109705
\end{verbatim}

\begin{Shaded}
\begin{Highlighting}[]
\FunctionTok{prop.table}\NormalTok{(}\FunctionTok{table}\NormalTok{(sexo,catlab), }\DecValTok{1}\NormalTok{)}
\end{Highlighting}
\end{Shaded}

\begin{verbatim}
##         catlab
## sexo     Administrativo Seguridad Directivo
##   Hombre      0.6085271 0.1046512 0.2868217
##   Mujer       0.9537037 0.0000000 0.0462963
\end{verbatim}

\begin{Shaded}
\begin{Highlighting}[]
\FunctionTok{prop.table}\NormalTok{(}\FunctionTok{table}\NormalTok{(sexo,catlab), }\DecValTok{2}\NormalTok{)}
\end{Highlighting}
\end{Shaded}

\begin{verbatim}
##         catlab
## sexo     Administrativo Seguridad Directivo
##   Hombre      0.4325069 1.0000000 0.8809524
##   Mujer       0.5674931 0.0000000 0.1190476
\end{verbatim}

\begin{Shaded}
\begin{Highlighting}[]
\FunctionTok{table}\NormalTok{(catlab,educ,sexo)}
\end{Highlighting}
\end{Shaded}

\begin{verbatim}
## , , sexo = Hombre
## 
##                 educ
## catlab             8  12  14  15  16  17  18  19  20  21
##   Administrativo  10  48   6  78  10   2   2   1   0   0
##   Seguridad       13  13   0   1   0   0   0   0   0   0
##   Directivo        0   1   0   4  25   8   7  26   2   1
## 
## , , sexo = Mujer
## 
##                 educ
## catlab             8  12  14  15  16  17  18  19  20  21
##   Administrativo  30 128   0  33  14   1   0   0   0   0
##   Seguridad        0   0   0   0   0   0   0   0   0   0
##   Directivo        0   0   0   0  10   0   0   0   0   0
\end{verbatim}

\begin{Shaded}
\begin{Highlighting}[]
\FunctionTok{round}\NormalTok{(}\FunctionTok{prop.table}\NormalTok{(}\FunctionTok{table}\NormalTok{(catlab,educ,sexo)),}\DecValTok{2}\NormalTok{)}
\end{Highlighting}
\end{Shaded}

\begin{verbatim}
## , , sexo = Hombre
## 
##                 educ
## catlab              8   12   14   15   16   17   18   19   20   21
##   Administrativo 0.02 0.10 0.01 0.16 0.02 0.00 0.00 0.00 0.00 0.00
##   Seguridad      0.03 0.03 0.00 0.00 0.00 0.00 0.00 0.00 0.00 0.00
##   Directivo      0.00 0.00 0.00 0.01 0.05 0.02 0.01 0.05 0.00 0.00
## 
## , , sexo = Mujer
## 
##                 educ
## catlab              8   12   14   15   16   17   18   19   20   21
##   Administrativo 0.06 0.27 0.00 0.07 0.03 0.00 0.00 0.00 0.00 0.00
##   Seguridad      0.00 0.00 0.00 0.00 0.00 0.00 0.00 0.00 0.00 0.00
##   Directivo      0.00 0.00 0.00 0.00 0.02 0.00 0.00 0.00 0.00 0.00
\end{verbatim}

Si la variable es ordinal, entonces también son de interés las
frecuencias acumuladas

\begin{Shaded}
\begin{Highlighting}[]
\FunctionTok{table}\NormalTok{(educ)}
\end{Highlighting}
\end{Shaded}

\begin{verbatim}
## educ
##   8  12  14  15  16  17  18  19  20  21 
##  53 190   6 116  59  11   9  27   2   1
\end{verbatim}

\begin{Shaded}
\begin{Highlighting}[]
\FunctionTok{prop.table}\NormalTok{(}\FunctionTok{table}\NormalTok{(educ))}
\end{Highlighting}
\end{Shaded}

\begin{verbatim}
## educ
##           8          12          14          15          16          17 
## 0.111814346 0.400843882 0.012658228 0.244725738 0.124472574 0.023206751 
##          18          19          20          21 
## 0.018987342 0.056962025 0.004219409 0.002109705
\end{verbatim}

\begin{Shaded}
\begin{Highlighting}[]
\FunctionTok{cumsum}\NormalTok{(}\FunctionTok{table}\NormalTok{(educ))}
\end{Highlighting}
\end{Shaded}

\begin{verbatim}
##   8  12  14  15  16  17  18  19  20  21 
##  53 243 249 365 424 435 444 471 473 474
\end{verbatim}

\begin{Shaded}
\begin{Highlighting}[]
\FunctionTok{cumsum}\NormalTok{(}\FunctionTok{prop.table}\NormalTok{(}\FunctionTok{table}\NormalTok{(educ)))}
\end{Highlighting}
\end{Shaded}

\begin{verbatim}
##         8        12        14        15        16        17        18        19 
## 0.1118143 0.5126582 0.5253165 0.7700422 0.8945148 0.9177215 0.9367089 0.9936709 
##        20        21 
## 0.9978903 1.0000000
\end{verbatim}

\hypertarget{media-y-varianza}{%
\subsection{Media y varianza}\label{media-y-varianza}}

La media es la medida de centralización por excelencia. Para su cálculo
se utiliza la instrucción \emph{mean}

\begin{Shaded}
\begin{Highlighting}[]
\NormalTok{consumo}\OtherTok{\textless{}{-}}\FunctionTok{c}\NormalTok{(}\FloatTok{6.9}\NormalTok{, }\FloatTok{6.3}\NormalTok{, }\FloatTok{6.2}\NormalTok{, }\FloatTok{6.5}\NormalTok{ ,}\FloatTok{6.4}\NormalTok{, }\FloatTok{6.8}\NormalTok{, }\FloatTok{6.6}\NormalTok{)}
\FunctionTok{mean}\NormalTok{(consumo)}
\end{Highlighting}
\end{Shaded}

\begin{verbatim}
## [1] 6.528571
\end{verbatim}

\begin{Shaded}
\begin{Highlighting}[]
\FunctionTok{dotchart}\NormalTok{(consumo,}\AttributeTok{pch=}\DecValTok{16}\NormalTok{)}
\FunctionTok{text}\NormalTok{(}\FunctionTok{mean}\NormalTok{(consumo),}\FloatTok{2.5}\NormalTok{, }\AttributeTok{pos=}\DecValTok{3}\NormalTok{,}\FunctionTok{expression}\NormalTok{(}\FunctionTok{bar}\NormalTok{(X)}\SpecialCharTok{==}\FloatTok{6.53}\NormalTok{))}
\FunctionTok{arrows}\NormalTok{(}\FunctionTok{mean}\NormalTok{(consumo),}\DecValTok{0}\NormalTok{,}\FunctionTok{mean}\NormalTok{(consumo),}\FloatTok{2.5}\NormalTok{,}\AttributeTok{length =} \FloatTok{0.15}\NormalTok{,}\AttributeTok{col=}\StringTok{\textquotesingle{}red\textquotesingle{}}\NormalTok{)}
\end{Highlighting}
\end{Shaded}

\begin{center}\includegraphics[width=0.7\linewidth]{05-AnalisisExploratorio_files/figure-latex/unnamed-chunk-7-1} \end{center}

\begin{Shaded}
\begin{Highlighting}[]
\FunctionTok{mean}\NormalTok{(salario)}
\end{Highlighting}
\end{Shaded}

\begin{verbatim}
## [1] 34419.57
\end{verbatim}

\begin{Shaded}
\begin{Highlighting}[]
\FunctionTok{mean}\NormalTok{(}\FunctionTok{subset}\NormalTok{(empleados,catlab}\SpecialCharTok{==}\StringTok{\textquotesingle{}Directivo\textquotesingle{}}\NormalTok{)}\SpecialCharTok{$}\NormalTok{salario)}
\end{Highlighting}
\end{Shaded}

\begin{verbatim}
## [1] 63977.8
\end{verbatim}

También se puede utilizar la función \emph{tapply}, que se estudiará
con detalle más adelante

\begin{Shaded}
\begin{Highlighting}[]
\FunctionTok{tapply}\NormalTok{(salario, catlab, mean)}
\end{Highlighting}
\end{Shaded}

\begin{verbatim}
## Administrativo      Seguridad      Directivo 
##       27838.54       30938.89       63977.80
\end{verbatim}

La principal medida de dispersión es la varianza. En la práctica, cuando
se trabaja con datos muestrales, se sustituye por la \emph{cuasi}-varianza
(también llamada varianza muestral corregida), que se calcula mediante
el comando \emph{var}

\begin{Shaded}
\begin{Highlighting}[]
\FunctionTok{var}\NormalTok{(consumo)}
\end{Highlighting}
\end{Shaded}

\begin{verbatim}
## [1] 0.06571429
\end{verbatim}

\begin{Shaded}
\begin{Highlighting}[]
\FunctionTok{var}\NormalTok{(salario)}
\end{Highlighting}
\end{Shaded}

\begin{verbatim}
## [1] 291578214
\end{verbatim}

La \emph{cuasi}-desviación típica se calcula

\begin{Shaded}
\begin{Highlighting}[]
\FunctionTok{sd}\NormalTok{(consumo)}
\end{Highlighting}
\end{Shaded}

\begin{verbatim}
## [1] 0.256348
\end{verbatim}

\begin{Shaded}
\begin{Highlighting}[]
\FunctionTok{sd}\NormalTok{(salario)}
\end{Highlighting}
\end{Shaded}

\begin{verbatim}
## [1] 17075.66
\end{verbatim}

o, equivalentemente,

\begin{Shaded}
\begin{Highlighting}[]
\FunctionTok{sqrt}\NormalTok{(}\FunctionTok{var}\NormalTok{(consumo))}
\end{Highlighting}
\end{Shaded}

\begin{verbatim}
## [1] 0.256348
\end{verbatim}

\begin{Shaded}
\begin{Highlighting}[]
\FunctionTok{sqrt}\NormalTok{(}\FunctionTok{var}\NormalTok{(salario))}
\end{Highlighting}
\end{Shaded}

\begin{verbatim}
## [1] 17075.66
\end{verbatim}

La media de dispersión adimensional (relativa) más utilizada
es el \emph{coeficiente de variación} (de Pearson)

\begin{Shaded}
\begin{Highlighting}[]
\FunctionTok{sd}\NormalTok{(consumo)}\SpecialCharTok{/}\FunctionTok{abs}\NormalTok{(}\FunctionTok{mean}\NormalTok{(consumo))}
\end{Highlighting}
\end{Shaded}

\begin{verbatim}
## [1] 0.03926555
\end{verbatim}

que también podemos expresar en tanto por cien

\begin{Shaded}
\begin{Highlighting}[]
\DecValTok{100}\SpecialCharTok{*}\FunctionTok{sd}\NormalTok{(consumo)}\SpecialCharTok{/}\FunctionTok{abs}\NormalTok{(}\FunctionTok{mean}\NormalTok{(consumo))}
\end{Highlighting}
\end{Shaded}

\begin{verbatim}
## [1] 3.926555
\end{verbatim}

El coeficiente de variación nos permite, entre otras cosas, comparar dispersiones de
variables medidas en diferentes unidades

\begin{Shaded}
\begin{Highlighting}[]
\DecValTok{100}\SpecialCharTok{*}\FunctionTok{sd}\NormalTok{(salini)}\SpecialCharTok{/}\FunctionTok{abs}\NormalTok{(}\FunctionTok{mean}\NormalTok{(salini))}
\end{Highlighting}
\end{Shaded}

\begin{verbatim}
## [1] 46.2541
\end{verbatim}

\begin{Shaded}
\begin{Highlighting}[]
\DecValTok{100}\SpecialCharTok{*}\FunctionTok{sd}\NormalTok{(salario)}\SpecialCharTok{/}\FunctionTok{abs}\NormalTok{(}\FunctionTok{mean}\NormalTok{(salario))}
\end{Highlighting}
\end{Shaded}

\begin{verbatim}
## [1] 49.61033
\end{verbatim}

\begin{Shaded}
\begin{Highlighting}[]
\DecValTok{100}\SpecialCharTok{*}\FunctionTok{sd}\NormalTok{(expprev)}\SpecialCharTok{/}\FunctionTok{abs}\NormalTok{(}\FunctionTok{mean}\NormalTok{(expprev))}
\end{Highlighting}
\end{Shaded}

\begin{verbatim}
## [1] 109.1022
\end{verbatim}

\hypertarget{mediana-y-cuantiles}{%
\subsection{Mediana y cuantiles}\label{mediana-y-cuantiles}}

La mediana es una medida de centralización robusta. Se calcula mediante \emph{median}

\begin{Shaded}
\begin{Highlighting}[]
\NormalTok{diametro }\OtherTok{\textless{}{-}} \FunctionTok{c}\NormalTok{(}\FloatTok{3.88}\NormalTok{,}\FloatTok{4.09}\NormalTok{,}\FloatTok{3.92}\NormalTok{,}\FloatTok{3.97}\NormalTok{,}\FloatTok{4.02}\NormalTok{,}\FloatTok{3.95}\NormalTok{, }\FloatTok{4.03}\NormalTok{,}\FloatTok{3.92}\NormalTok{,}\FloatTok{3.98}\NormalTok{,}\FloatTok{5.60}\NormalTok{)}
\FunctionTok{dotchart}\NormalTok{(diametro,}\AttributeTok{pch=}\DecValTok{16}\NormalTok{,}\AttributeTok{xlab=}\StringTok{"diámetro"}\NormalTok{)}
\FunctionTok{abline}\NormalTok{(}\AttributeTok{v=}\FunctionTok{mean}\NormalTok{(diametro),}\AttributeTok{col=}\StringTok{\textquotesingle{}red\textquotesingle{}}\NormalTok{,}\AttributeTok{lwd=}\DecValTok{2}\NormalTok{)}
\FunctionTok{abline}\NormalTok{(}\AttributeTok{v=}\FunctionTok{median}\NormalTok{(diametro),}\AttributeTok{col=}\StringTok{\textquotesingle{}blue\textquotesingle{}}\NormalTok{,}\AttributeTok{lty=}\DecValTok{2}\NormalTok{,}\AttributeTok{lwd=}\DecValTok{2}\NormalTok{)}
\FunctionTok{legend}\NormalTok{(}\StringTok{"bottomright"}\NormalTok{,}\FunctionTok{c}\NormalTok{(}\StringTok{"media"}\NormalTok{,}\StringTok{"mediana"}\NormalTok{),}
       \AttributeTok{col=}\FunctionTok{c}\NormalTok{(}\StringTok{"red"}\NormalTok{,}\StringTok{"blue"}\NormalTok{),}\AttributeTok{lty=}\FunctionTok{c}\NormalTok{(}\DecValTok{1}\NormalTok{,}\DecValTok{2}\NormalTok{),}\AttributeTok{lwd=}\FunctionTok{c}\NormalTok{(}\DecValTok{2}\NormalTok{,}\DecValTok{2}\NormalTok{),}\AttributeTok{box.lty=}\DecValTok{0}\NormalTok{,}\AttributeTok{cex=}\FloatTok{1.5}\NormalTok{)}
\end{Highlighting}
\end{Shaded}

\begin{center}\includegraphics[width=0.7\linewidth]{05-AnalisisExploratorio_files/figure-latex/unnamed-chunk-15-1} \end{center}

Podemos comprobar que la variable \emph{salario} presenta una
asimetría derecha

\begin{Shaded}
\begin{Highlighting}[]
\FunctionTok{mean}\NormalTok{(salario); }\FunctionTok{median}\NormalTok{(salario)}
\end{Highlighting}
\end{Shaded}

\begin{verbatim}
## [1] 34419.57
\end{verbatim}

\begin{verbatim}
## [1] 28875
\end{verbatim}

Calculemos cuántos empleados tienen un salario inferior al salario medio

\begin{Shaded}
\begin{Highlighting}[]
\FunctionTok{mean}\NormalTok{(salario }\SpecialCharTok{\textless{}} \FunctionTok{mean}\NormalTok{(salario))}
\end{Highlighting}
\end{Shaded}

\begin{verbatim}
## [1] 0.6940928
\end{verbatim}

\begin{Shaded}
\begin{Highlighting}[]
\FunctionTok{paste}\NormalTok{(}\StringTok{\textquotesingle{}El \textquotesingle{}}\NormalTok{, }\FunctionTok{round}\NormalTok{(}\DecValTok{100}\SpecialCharTok{*}\FunctionTok{mean}\NormalTok{(salario }\SpecialCharTok{\textless{}} \FunctionTok{mean}\NormalTok{(salario)),}\DecValTok{0}\NormalTok{), }\StringTok{\textquotesingle{}\%\textquotesingle{}}\NormalTok{,}
      \StringTok{\textquotesingle{} de los empleados tienen un salario inferior al salario medio\textquotesingle{}}\NormalTok{, }\AttributeTok{sep=}\StringTok{\textquotesingle{}\textquotesingle{}}\NormalTok{)}
\end{Highlighting}
\end{Shaded}

\begin{verbatim}
## [1] "El 69% de los empleados tienen un salario inferior al salario medio"
\end{verbatim}

Como sabemos, la mitad de los empleados tienen un salario inferior a la mediana

\begin{Shaded}
\begin{Highlighting}[]
\FunctionTok{mean}\NormalTok{(salario }\SpecialCharTok{\textless{}} \FunctionTok{median}\NormalTok{(salario))}
\end{Highlighting}
\end{Shaded}

\begin{verbatim}
## [1] 0.5
\end{verbatim}

Los cuantiles son una generalización de la mediana, que se corresponde con el
cuantil de orden 0.5. \emph{R} contempla distintas formas
de calcular los cuantiles

\begin{Shaded}
\begin{Highlighting}[]
\FunctionTok{median}\NormalTok{(}\FunctionTok{c}\NormalTok{(}\DecValTok{1}\NormalTok{,}\DecValTok{2}\NormalTok{,}\DecValTok{3}\NormalTok{,}\DecValTok{4}\NormalTok{))}
\end{Highlighting}
\end{Shaded}

\begin{verbatim}
## [1] 2.5
\end{verbatim}

\begin{Shaded}
\begin{Highlighting}[]
\FunctionTok{quantile}\NormalTok{(}\FunctionTok{c}\NormalTok{(}\DecValTok{1}\NormalTok{,}\DecValTok{2}\NormalTok{,}\DecValTok{3}\NormalTok{,}\DecValTok{4}\NormalTok{),}\FloatTok{0.5}\NormalTok{)}
\end{Highlighting}
\end{Shaded}

\begin{verbatim}
## 50% 
## 2.5
\end{verbatim}

\begin{Shaded}
\begin{Highlighting}[]
\FunctionTok{quantile}\NormalTok{(}\FunctionTok{c}\NormalTok{(}\DecValTok{1}\NormalTok{,}\DecValTok{2}\NormalTok{,}\DecValTok{3}\NormalTok{,}\DecValTok{4}\NormalTok{),}\FloatTok{0.5}\NormalTok{,}\AttributeTok{type=}\DecValTok{1}\NormalTok{)}
\end{Highlighting}
\end{Shaded}

\begin{verbatim}
## 50% 
##   2
\end{verbatim}

Calculemos los \emph{cuartiles} y los \emph{deciles} de la variable \emph{salario}

\begin{Shaded}
\begin{Highlighting}[]
\FunctionTok{quantile}\NormalTok{(salario)}
\end{Highlighting}
\end{Shaded}

\begin{verbatim}
##       0%      25%      50%      75%     100% 
##  15750.0  24000.0  28875.0  36937.5 135000.0
\end{verbatim}

\begin{Shaded}
\begin{Highlighting}[]
\FunctionTok{quantile}\NormalTok{(salario, }\AttributeTok{probs=}\FunctionTok{c}\NormalTok{(}\FloatTok{0.25}\NormalTok{,}\FloatTok{0.5}\NormalTok{,}\FloatTok{0.75}\NormalTok{))}
\end{Highlighting}
\end{Shaded}

\begin{verbatim}
##     25%     50%     75% 
## 24000.0 28875.0 36937.5
\end{verbatim}

\begin{Shaded}
\begin{Highlighting}[]
\FunctionTok{quantile}\NormalTok{(salario, }\AttributeTok{probs=}\FunctionTok{seq}\NormalTok{(}\FloatTok{0.1}\NormalTok{, }\FloatTok{0.9}\NormalTok{, }\FloatTok{0.1}\NormalTok{))}
\end{Highlighting}
\end{Shaded}

\begin{verbatim}
##     10%     20%     30%     40%     50%     60%     70%     80%     90% 
## 21045.0 22950.0 24885.0 26700.0 28875.0 30750.0 34500.0 40920.0 59392.5
\end{verbatim}

El \emph{rango} y el \emph{rango intercuartílico}

\begin{Shaded}
\begin{Highlighting}[]
\FunctionTok{data.frame}\NormalTok{(}\AttributeTok{Rango=}\FunctionTok{max}\NormalTok{(salario)}\SpecialCharTok{{-}}\FunctionTok{min}\NormalTok{(salario),}
           \AttributeTok{RI=}\FunctionTok{as.numeric}\NormalTok{(}\FunctionTok{quantile}\NormalTok{(salario, }\FloatTok{0.75}\NormalTok{) }\SpecialCharTok{{-}} \FunctionTok{quantile}\NormalTok{(salario, }\FloatTok{0.25}\NormalTok{)))}
\end{Highlighting}
\end{Shaded}

\begin{verbatim}
##    Rango      RI
## 1 119250 12937.5
\end{verbatim}

\hypertarget{summary}{%
\subsection{\texorpdfstring{\emph{Summary}}{Summary}}\label{summary}}

\begin{Shaded}
\begin{Highlighting}[]
\FunctionTok{summary}\NormalTok{(empleados)}
\end{Highlighting}
\end{Shaded}

\begin{verbatim}
##        id            sexo        fechnac                educ      
##  Min.   :  1.0   Hombre:258   Min.   :1929-02-10   Min.   : 8.00  
##  1st Qu.:119.2   Mujer :216   1st Qu.:1948-01-03   1st Qu.:12.00  
##  Median :237.5                Median :1962-01-23   Median :12.00  
##  Mean   :237.5                Mean   :1956-10-08   Mean   :13.49  
##  3rd Qu.:355.8                3rd Qu.:1965-07-06   3rd Qu.:15.00  
##  Max.   :474.0                Max.   :1971-02-10   Max.   :21.00  
##                               NA's   :1                           
##             catlab       salario           salini         tiempemp    
##  Administrativo:363   Min.   : 15750   Min.   : 9000   Min.   :63.00  
##  Seguridad     : 27   1st Qu.: 24000   1st Qu.:12488   1st Qu.:72.00  
##  Directivo     : 84   Median : 28875   Median :15000   Median :81.00  
##                       Mean   : 34420   Mean   :17016   Mean   :81.11  
##                       3rd Qu.: 36938   3rd Qu.:17490   3rd Qu.:90.00  
##                       Max.   :135000   Max.   :79980   Max.   :98.00  
##                                                                       
##     expprev       minoria           sexoraza  
##  Min.   :  0.00   No:370   Blanca varón :194  
##  1st Qu.: 19.25   Sí:104   Minoría varón: 64  
##  Median : 55.00            Blanca mujer :176  
##  Mean   : 95.86            Minoría mujer: 40  
##  3rd Qu.:138.75                               
##  Max.   :476.00                               
## 
\end{verbatim}

\begin{Shaded}
\begin{Highlighting}[]
\FunctionTok{summary}\NormalTok{(}\FunctionTok{subset}\NormalTok{(empleados,catlab}\SpecialCharTok{==}\StringTok{\textquotesingle{}Directivo\textquotesingle{}}\NormalTok{))}
\end{Highlighting}
\end{Shaded}

\begin{verbatim}
##        id            sexo       fechnac                educ      
##  Min.   :  1.0   Hombre:74   Min.   :1937-07-12   Min.   :12.00  
##  1st Qu.:102.5   Mujer :10   1st Qu.:1954-08-09   1st Qu.:16.00  
##  Median :233.5               Median :1961-05-29   Median :17.00  
##  Mean   :234.1               Mean   :1958-11-26   Mean   :17.25  
##  3rd Qu.:344.2               3rd Qu.:1963-10-03   3rd Qu.:19.00  
##  Max.   :468.0               Max.   :1966-04-05   Max.   :21.00  
##             catlab      salario           salini         tiempemp    
##  Administrativo: 0   Min.   : 34410   Min.   :15750   Min.   :64.00  
##  Seguridad     : 0   1st Qu.: 51956   1st Qu.:23063   1st Qu.:73.00  
##  Directivo     :84   Median : 60500   Median :28740   Median :81.00  
##                      Mean   : 63978   Mean   :30258   Mean   :81.15  
##                      3rd Qu.: 71281   3rd Qu.:34058   3rd Qu.:91.00  
##                      Max.   :135000   Max.   :79980   Max.   :98.00  
##     expprev       minoria          sexoraza 
##  Min.   :  3.00   No:80   Blanca varón :70  
##  1st Qu.: 19.75   Sí: 4   Minoría varón: 4  
##  Median : 52.00           Blanca mujer :10  
##  Mean   : 77.62           Minoría mujer: 0  
##  3rd Qu.:125.25                             
##  Max.   :285.00
\end{verbatim}

\hypertarget{gruxe1ficos}{%
\section{Gráficos}\label{gruxe1ficos}}

\hypertarget{diagrama-de-barras-y-gruxe1fico-de-sectores}{%
\subsection{Diagrama de barras y gráfico de sectores}\label{diagrama-de-barras-y-gruxe1fico-de-sectores}}

\begin{Shaded}
\begin{Highlighting}[]
\FunctionTok{table}\NormalTok{(catlab)}
\end{Highlighting}
\end{Shaded}

\begin{verbatim}
## catlab
## Administrativo      Seguridad      Directivo 
##            363             27             84
\end{verbatim}

\begin{Shaded}
\begin{Highlighting}[]
\FunctionTok{par}\NormalTok{(}\AttributeTok{mfrow =} \FunctionTok{c}\NormalTok{(}\DecValTok{1}\NormalTok{, }\DecValTok{3}\NormalTok{))}
\FunctionTok{barplot}\NormalTok{(}\FunctionTok{table}\NormalTok{(catlab),}\AttributeTok{main=}\StringTok{"frecuencia absoluta"}\NormalTok{)}
\FunctionTok{barplot}\NormalTok{(}\DecValTok{100}\SpecialCharTok{*}\FunctionTok{prop.table}\NormalTok{(}\FunctionTok{table}\NormalTok{(catlab)),}\AttributeTok{main=}\StringTok{"frecuencia relativa (\%)"}\NormalTok{)}
\FunctionTok{pie}\NormalTok{(}\FunctionTok{table}\NormalTok{(catlab))}
\end{Highlighting}
\end{Shaded}

\begin{center}\includegraphics[width=0.7\linewidth]{05-AnalisisExploratorio_files/figure-latex/unnamed-chunk-23-1} \end{center}

\begin{Shaded}
\begin{Highlighting}[]
\NormalTok{nj }\OtherTok{\textless{}{-}} \FunctionTok{table}\NormalTok{(educ)}
\NormalTok{fj }\OtherTok{\textless{}{-}} \FunctionTok{prop.table}\NormalTok{(nj)}
\NormalTok{Nj }\OtherTok{\textless{}{-}} \FunctionTok{cumsum}\NormalTok{(nj)}
\NormalTok{Fj }\OtherTok{\textless{}{-}} \FunctionTok{cumsum}\NormalTok{(fj)}
\FunctionTok{layout}\NormalTok{(}\FunctionTok{matrix}\NormalTok{(}\FunctionTok{c}\NormalTok{(}\DecValTok{1}\NormalTok{,}\DecValTok{2}\NormalTok{,}\DecValTok{5}\NormalTok{,}\DecValTok{3}\NormalTok{,}\DecValTok{4}\NormalTok{,}\DecValTok{5}\NormalTok{), }\DecValTok{2}\NormalTok{, }\DecValTok{3}\NormalTok{, }\AttributeTok{byrow=}\ConstantTok{TRUE}\NormalTok{), }\AttributeTok{respect=}\ConstantTok{TRUE}\NormalTok{)}
\FunctionTok{barplot}\NormalTok{(nj,}\AttributeTok{main=}\StringTok{"frecuencia absolutas"}\NormalTok{,}\AttributeTok{xlab=}\StringTok{\textquotesingle{}años de estudio\textquotesingle{}}\NormalTok{)}
\FunctionTok{barplot}\NormalTok{(fj,}\AttributeTok{main=}\StringTok{"frecuencia relativas"}\NormalTok{,}\AttributeTok{xlab=}\StringTok{\textquotesingle{}años de estudio\textquotesingle{}}\NormalTok{)}
\FunctionTok{barplot}\NormalTok{(Nj,}\AttributeTok{main=}\StringTok{"frecuencia absolutas acumuladas"}\NormalTok{,}\AttributeTok{xlab=}\StringTok{\textquotesingle{}años de estudio\textquotesingle{}}\NormalTok{)}
\FunctionTok{barplot}\NormalTok{(Fj,}\AttributeTok{main=}\StringTok{"frecuencia relativas acumuladas"}\NormalTok{,}\AttributeTok{xlab=}\StringTok{\textquotesingle{}años de estudio\textquotesingle{}}\NormalTok{)}
\FunctionTok{pie}\NormalTok{(nj,}\AttributeTok{col=}\FunctionTok{rainbow}\NormalTok{(}\DecValTok{6}\NormalTok{),}\AttributeTok{main=}\StringTok{\textquotesingle{}años de estudio\textquotesingle{}}\NormalTok{)}
\end{Highlighting}
\end{Shaded}

\begin{center}\includegraphics[width=0.7\linewidth]{05-AnalisisExploratorio_files/figure-latex/unnamed-chunk-23-2} \end{center}

\begin{Shaded}
\begin{Highlighting}[]
\FunctionTok{par}\NormalTok{(}\AttributeTok{mfrow =} \FunctionTok{c}\NormalTok{(}\DecValTok{1}\NormalTok{, }\DecValTok{1}\NormalTok{))}
\end{Highlighting}
\end{Shaded}

Con datos continuos, podemos hacer uso de la función \emph{cut}
(más adelante veremos como se representa el histograma)

\begin{Shaded}
\begin{Highlighting}[]
\FunctionTok{table}\NormalTok{(}\FunctionTok{cut}\NormalTok{(expprev, }\AttributeTok{breaks=}\DecValTok{5}\NormalTok{))}
\end{Highlighting}
\end{Shaded}

\begin{verbatim}
## 
## (-0.476,95.2]    (95.2,190]     (190,286]     (286,381]     (381,476] 
##           312            81            46            22            13
\end{verbatim}

\begin{Shaded}
\begin{Highlighting}[]
\FunctionTok{barplot}\NormalTok{(}\FunctionTok{table}\NormalTok{(}\FunctionTok{cut}\NormalTok{(expprev,}\AttributeTok{breaks=}\DecValTok{5}\NormalTok{)),}\AttributeTok{xlab=}\StringTok{"Experiencia previa"}\NormalTok{,}
        \AttributeTok{main=}\StringTok{"Categorización en 5 clases"}\NormalTok{)}
\end{Highlighting}
\end{Shaded}

\begin{center}\includegraphics[width=0.7\linewidth]{05-AnalisisExploratorio_files/figure-latex/unnamed-chunk-24-1} \end{center}

Debemos ser muy cuidadosos a la hora de valorar gráficas como la siguiente

\begin{Shaded}
\begin{Highlighting}[]
\NormalTok{tt }\OtherTok{\textless{}{-}} \FunctionTok{table}\NormalTok{(}\FunctionTok{cut}\NormalTok{(expprev, }\AttributeTok{breaks=}\FunctionTok{c}\NormalTok{(}\DecValTok{0}\NormalTok{,}\DecValTok{40}\NormalTok{,}\DecValTok{80}\NormalTok{,}\DecValTok{150}\NormalTok{,}\DecValTok{250}\NormalTok{,}\DecValTok{400}\NormalTok{)))}
\FunctionTok{barplot}\NormalTok{(tt,}\AttributeTok{xlab=}\StringTok{"Experiencia previa"}\NormalTok{, }\AttributeTok{main=}\StringTok{"Categorización en 5 clases"}\NormalTok{)}
\end{Highlighting}
\end{Shaded}

\begin{center}\includegraphics[width=0.7\linewidth]{05-AnalisisExploratorio_files/figure-latex/unnamed-chunk-25-1} \end{center}

\hypertarget{gruxe1fico-de-puntos}{%
\subsection{Gráfico de puntos}\label{gruxe1fico-de-puntos}}

\begin{Shaded}
\begin{Highlighting}[]
\FunctionTok{dotchart}\NormalTok{(salario, }\AttributeTok{xlab=}\StringTok{\textquotesingle{}salarios\textquotesingle{}}\NormalTok{)}
\end{Highlighting}
\end{Shaded}

\begin{center}\includegraphics[width=0.7\linewidth]{05-AnalisisExploratorio_files/figure-latex/unnamed-chunk-26-1} \end{center}

\begin{Shaded}
\begin{Highlighting}[]
\FunctionTok{stripchart}\NormalTok{(salario}\SpecialCharTok{\textasciitilde{}}\NormalTok{sexo, }\AttributeTok{method=}\StringTok{\textquotesingle{}jitter\textquotesingle{}}\NormalTok{)}
\end{Highlighting}
\end{Shaded}

\begin{center}\includegraphics[width=0.7\linewidth]{05-AnalisisExploratorio_files/figure-latex/unnamed-chunk-26-2} \end{center}

\hypertarget{uxe1rbol-de-tallo-y-hojas}{%
\subsection{Árbol de tallo y hojas}\label{uxe1rbol-de-tallo-y-hojas}}

Esta representación puede ser útil cuando se dispone de pocos datos.

\begin{Shaded}
\begin{Highlighting}[]
\FunctionTok{stem}\NormalTok{(salario)}
\end{Highlighting}
\end{Shaded}

\begin{verbatim}
## 
##   The decimal point is 4 digit(s) to the right of the |
## 
##    1 | 666666777777777778888999
##    2 | 00000000000000111111111111111111122222222222222222222222233333333333+148
##    3 | 00000000000000000001111111111111111111111111122222222222223333333333+36
##    4 | 0000000001112222334445555666778899
##    5 | 0111123344555556677778999
##    6 | 0001122355566777888999
##    7 | 00134455889
##    8 | 01346
##    9 | 1127
##   10 | 044
##   11 | 1
##   12 | 
##   13 | 5
\end{verbatim}

\begin{Shaded}
\begin{Highlighting}[]
\FunctionTok{stem}\NormalTok{(tiempemp)}
\end{Highlighting}
\end{Shaded}

\begin{verbatim}
## 
##   The decimal point is at the |
## 
##   62 | 000
##   64 | 00000000000000000000000
##   66 | 000000000000000000000000000000000
##   68 | 0000000000000000000000000000000
##   70 | 0000000000000000
##   72 | 00000000000000000000000000
##   74 | 000000000000000
##   76 | 00000000000000000000000
##   78 | 000000000000000000000000000000000000
##   80 | 00000000000000000000000000000000000000
##   82 | 0000000000000000000000000000000000
##   84 | 000000000000000000000000
##   86 | 000000000000000000000000
##   88 | 00000000000000000000
##   90 | 00000000000000000000000000000
##   92 | 00000000000000000000000000000000000000
##   94 | 00000000000000000000
##   96 | 000000000000000000000000000
##   98 | 00000000000000
\end{verbatim}

\hypertarget{histograma}{%
\subsection{Histograma}\label{histograma}}

Este gráfico es uno de los más habituales para representar datos continuos

\begin{Shaded}
\begin{Highlighting}[]
\FunctionTok{hist}\NormalTok{(salario, }\AttributeTok{main=}\StringTok{\textquotesingle{}número de clases por defecto\textquotesingle{}}\NormalTok{)}
\end{Highlighting}
\end{Shaded}

\begin{center}\includegraphics[width=0.7\linewidth]{05-AnalisisExploratorio_files/figure-latex/unnamed-chunk-28-1} \end{center}

\begin{Shaded}
\begin{Highlighting}[]
\FunctionTok{hist}\NormalTok{(salario, }\AttributeTok{breaks=}\DecValTok{3}\NormalTok{, }\AttributeTok{main=}\StringTok{\textquotesingle{}3 intervalos de clase\textquotesingle{}}\NormalTok{)}
\end{Highlighting}
\end{Shaded}

\begin{center}\includegraphics[width=0.7\linewidth]{05-AnalisisExploratorio_files/figure-latex/unnamed-chunk-28-2} \end{center}

\begin{Shaded}
\begin{Highlighting}[]
\FunctionTok{hist}\NormalTok{(salario, }\AttributeTok{breaks=}\DecValTok{100}\NormalTok{, }\AttributeTok{main=}\StringTok{\textquotesingle{}100 intervalos de clase\textquotesingle{}}\NormalTok{)}
\end{Highlighting}
\end{Shaded}

\begin{center}\includegraphics[width=0.7\linewidth]{05-AnalisisExploratorio_files/figure-latex/unnamed-chunk-28-3} \end{center}

\begin{Shaded}
\begin{Highlighting}[]
\NormalTok{cl1 }\OtherTok{\textless{}{-}} \FunctionTok{seq}\NormalTok{(}\DecValTok{15000}\NormalTok{,}\DecValTok{40000}\NormalTok{,}\DecValTok{5000}\NormalTok{)}
\NormalTok{cl2 }\OtherTok{\textless{}{-}} \FunctionTok{seq}\NormalTok{(}\DecValTok{50000}\NormalTok{,}\DecValTok{80000}\NormalTok{,}\DecValTok{10000}\NormalTok{)}
\NormalTok{cl3 }\OtherTok{\textless{}{-}} \FunctionTok{seq}\NormalTok{(}\DecValTok{100000}\NormalTok{,}\DecValTok{140000}\NormalTok{,}\DecValTok{20000}\NormalTok{)}
\FunctionTok{hist}\NormalTok{(salario, }\AttributeTok{breaks=}\FunctionTok{c}\NormalTok{(cl1,cl2,cl3),}\AttributeTok{main=}\StringTok{\textquotesingle{}intervalos de clase de distinta amplitud\textquotesingle{}}\NormalTok{)}
\end{Highlighting}
\end{Shaded}

\begin{center}\includegraphics[width=0.7\linewidth]{05-AnalisisExploratorio_files/figure-latex/unnamed-chunk-28-4} \end{center}

\hypertarget{gruxe1fico-de-densidad}{%
\subsection{Gráfico de densidad}\label{gruxe1fico-de-densidad}}

Es una versión suavizada del histograma.

\begin{Shaded}
\begin{Highlighting}[]
\FunctionTok{plot}\NormalTok{(}\FunctionTok{density}\NormalTok{(salario))}
\end{Highlighting}
\end{Shaded}

\begin{center}\includegraphics[width=0.7\linewidth]{05-AnalisisExploratorio_files/figure-latex/unnamed-chunk-29-1} \end{center}

\begin{Shaded}
\begin{Highlighting}[]
\FunctionTok{hist}\NormalTok{(salario, }\AttributeTok{freq=}\NormalTok{F, }\AttributeTok{main=}\StringTok{\textquotesingle{}\textquotesingle{}}\NormalTok{)}
\FunctionTok{lines}\NormalTok{(}\FunctionTok{density}\NormalTok{(salario), }\AttributeTok{lwd=}\DecValTok{3}\NormalTok{, }\AttributeTok{col=}\StringTok{\textquotesingle{}red\textquotesingle{}}\NormalTok{)}
\end{Highlighting}
\end{Shaded}

\begin{center}\includegraphics[width=0.7\linewidth]{05-AnalisisExploratorio_files/figure-latex/unnamed-chunk-29-2} \end{center}

El paquete \emph{car} nos da acceso a la instrucción \emph{densityPlot}:

\begin{Shaded}
\begin{Highlighting}[]
\FunctionTok{library}\NormalTok{(car)  }\CommentTok{\# help(car)}
\FunctionTok{densityPlot}\NormalTok{(salario}\SpecialCharTok{\textasciitilde{}}\NormalTok{sexo)}
\end{Highlighting}
\end{Shaded}

\begin{center}\includegraphics[width=0.7\linewidth]{05-AnalisisExploratorio_files/figure-latex/unnamed-chunk-30-1} \end{center}

\hypertarget{diagrama-de-cajas}{%
\subsection{Diagrama de cajas}\label{diagrama-de-cajas}}

Se trata de un gráfico muy polivalente

\begin{Shaded}
\begin{Highlighting}[]
\FunctionTok{boxplot}\NormalTok{(salario, }\AttributeTok{horizontal=}\NormalTok{T, }\AttributeTok{axes=}\NormalTok{F)}
\FunctionTok{axis}\NormalTok{(}\DecValTok{1}\NormalTok{)}
\end{Highlighting}
\end{Shaded}

\begin{center}\includegraphics[width=0.7\linewidth]{05-AnalisisExploratorio_files/figure-latex/unnamed-chunk-31-1} \end{center}

\begin{Shaded}
\begin{Highlighting}[]
\FunctionTok{par}\NormalTok{(}\AttributeTok{mfrow=}\FunctionTok{c}\NormalTok{(}\DecValTok{1}\NormalTok{,}\DecValTok{2}\NormalTok{))}
\FunctionTok{boxplot}\NormalTok{(salario}\SpecialCharTok{\textasciitilde{}}\NormalTok{catlab)}
\FunctionTok{boxplot}\NormalTok{(salario}\SpecialCharTok{\textasciitilde{}}\NormalTok{sexo)}
\end{Highlighting}
\end{Shaded}

\begin{center}\includegraphics[width=0.7\linewidth]{05-AnalisisExploratorio_files/figure-latex/unnamed-chunk-31-2} \end{center}

\begin{Shaded}
\begin{Highlighting}[]
\FunctionTok{par}\NormalTok{(}\AttributeTok{mfrow=}\FunctionTok{c}\NormalTok{(}\DecValTok{1}\NormalTok{,}\DecValTok{1}\NormalTok{))}
\FunctionTok{boxplot}\NormalTok{(salario}\SpecialCharTok{\textasciitilde{}}\NormalTok{sexo}\SpecialCharTok{*}\NormalTok{catlab)}
\end{Highlighting}
\end{Shaded}

\begin{center}\includegraphics[width=0.7\linewidth]{05-AnalisisExploratorio_files/figure-latex/unnamed-chunk-31-3} \end{center}

\begin{Shaded}
\begin{Highlighting}[]
\FunctionTok{boxplot}\NormalTok{(salini, salario)}
\end{Highlighting}
\end{Shaded}

\begin{center}\includegraphics[width=0.7\linewidth]{05-AnalisisExploratorio_files/figure-latex/unnamed-chunk-31-4} \end{center}

\begin{Shaded}
\begin{Highlighting}[]
\FunctionTok{hist}\NormalTok{(salario,}\AttributeTok{probability=}\NormalTok{T,}\AttributeTok{ylab=}\StringTok{""}\NormalTok{,}\AttributeTok{col=}\StringTok{\textquotesingle{}grey\textquotesingle{}}\NormalTok{,}\AttributeTok{axes=}\NormalTok{F,}\AttributeTok{main=}\StringTok{""}\NormalTok{); }\FunctionTok{axis}\NormalTok{(}\DecValTok{1}\NormalTok{)}
\FunctionTok{lines}\NormalTok{(}\FunctionTok{density}\NormalTok{(salario),}\AttributeTok{col=}\StringTok{\textquotesingle{}red\textquotesingle{}}\NormalTok{,}\AttributeTok{lwd=}\DecValTok{2}\NormalTok{)}
\FunctionTok{par}\NormalTok{(}\AttributeTok{new=}\NormalTok{T)}
\FunctionTok{boxplot}\NormalTok{(salario,}\AttributeTok{horizontal=}\NormalTok{T,}\AttributeTok{axes=}\NormalTok{F,}\AttributeTok{lwd=}\DecValTok{2}\NormalTok{)}
\end{Highlighting}
\end{Shaded}

\begin{center}\includegraphics[width=0.7\linewidth]{05-AnalisisExploratorio_files/figure-latex/unnamed-chunk-31-5} \end{center}

\hypertarget{gruxe1fica-de-dispersiuxf3n}{%
\subsection{Gráfica de dispersión}\label{gruxe1fica-de-dispersiuxf3n}}

Permite ver la relación entre dos variables:

\begin{Shaded}
\begin{Highlighting}[]
\FunctionTok{plot}\NormalTok{(educ,salario)}
\end{Highlighting}
\end{Shaded}

\begin{center}\includegraphics[width=0.7\linewidth]{05-AnalisisExploratorio_files/figure-latex/unnamed-chunk-32-1} \end{center}

\begin{Shaded}
\begin{Highlighting}[]
\FunctionTok{plot}\NormalTok{(tiempemp,salario)}
\end{Highlighting}
\end{Shaded}

\begin{center}\includegraphics[width=0.7\linewidth]{05-AnalisisExploratorio_files/figure-latex/unnamed-chunk-32-2} \end{center}

\begin{Shaded}
\begin{Highlighting}[]
\FunctionTok{plot}\NormalTok{(salini,salario)}
\end{Highlighting}
\end{Shaded}

\begin{center}\includegraphics[width=0.7\linewidth]{05-AnalisisExploratorio_files/figure-latex/unnamed-chunk-32-3} \end{center}

En el caso de una serie temporal

\begin{Shaded}
\begin{Highlighting}[]
\NormalTok{AirPassengers}
\end{Highlighting}
\end{Shaded}

\begin{verbatim}
##      Jan Feb Mar Apr May Jun Jul Aug Sep Oct Nov Dec
## 1949 112 118 132 129 121 135 148 148 136 119 104 118
## 1950 115 126 141 135 125 149 170 170 158 133 114 140
## 1951 145 150 178 163 172 178 199 199 184 162 146 166
## 1952 171 180 193 181 183 218 230 242 209 191 172 194
## 1953 196 196 236 235 229 243 264 272 237 211 180 201
## 1954 204 188 235 227 234 264 302 293 259 229 203 229
## 1955 242 233 267 269 270 315 364 347 312 274 237 278
## 1956 284 277 317 313 318 374 413 405 355 306 271 306
## 1957 315 301 356 348 355 422 465 467 404 347 305 336
## 1958 340 318 362 348 363 435 491 505 404 359 310 337
## 1959 360 342 406 396 420 472 548 559 463 407 362 405
## 1960 417 391 419 461 472 535 622 606 508 461 390 432
\end{verbatim}

\begin{Shaded}
\begin{Highlighting}[]
\FunctionTok{plot}\NormalTok{(AirPassengers)}
\end{Highlighting}
\end{Shaded}

\begin{center}\includegraphics[width=0.7\linewidth]{05-AnalisisExploratorio_files/figure-latex/unnamed-chunk-33-1} \end{center}

Y un último ejemplo utilizando los datos \emph{iris} de Fisher:

\begin{Shaded}
\begin{Highlighting}[]
\FunctionTok{plot}\NormalTok{(iris[,}\DecValTok{3}\NormalTok{],iris[,}\DecValTok{4}\NormalTok{],}\AttributeTok{main=}\StringTok{"Longitud y anchura de pétalos de lirios"}\NormalTok{,}
     \AttributeTok{xlab=}\StringTok{"Longitud de pétalo"}\NormalTok{,}\AttributeTok{ylab=}\StringTok{"Anchura de pétalo"}\NormalTok{)}
\end{Highlighting}
\end{Shaded}

\begin{center}\includegraphics[width=0.7\linewidth]{05-AnalisisExploratorio_files/figure-latex/unnamed-chunk-34-1} \end{center}

\begin{Shaded}
\begin{Highlighting}[]
\NormalTok{iris.color}\OtherTok{\textless{}{-}}\FunctionTok{c}\NormalTok{(}\StringTok{"red"}\NormalTok{,}\StringTok{"green"}\NormalTok{,}\StringTok{"blue"}\NormalTok{)[iris}\SpecialCharTok{$}\NormalTok{Species]}
\FunctionTok{plot}\NormalTok{(iris[,}\DecValTok{3}\NormalTok{],iris[,}\DecValTok{4}\NormalTok{],}\AttributeTok{col=}\NormalTok{iris.color,}\AttributeTok{main=}\StringTok{"Longitud y anchura}
\StringTok{     de pétalo según especies"}\NormalTok{,}\AttributeTok{xlab=}\StringTok{"Longitud de pétalo"}\NormalTok{,}
     \AttributeTok{ylab=}\StringTok{"Anchura de pétalo"}\NormalTok{)}
\FunctionTok{legend}\NormalTok{(}\StringTok{"topleft"}\NormalTok{,}\FunctionTok{c}\NormalTok{(}\StringTok{"Setosa"}\NormalTok{,}\StringTok{"Versicolor"}\NormalTok{,}\StringTok{"Virginica"}\NormalTok{),}\AttributeTok{pch=}\DecValTok{1}\NormalTok{,}
       \AttributeTok{col=}\FunctionTok{c}\NormalTok{(}\StringTok{"red"}\NormalTok{,}\StringTok{"green"}\NormalTok{,}\StringTok{"blue"}\NormalTok{),}\AttributeTok{box.lty=}\DecValTok{0}\NormalTok{)}
\end{Highlighting}
\end{Shaded}

\begin{center}\includegraphics[width=0.7\linewidth]{05-AnalisisExploratorio_files/figure-latex/unnamed-chunk-34-2} \end{center}

\begin{Shaded}
\begin{Highlighting}[]
\FunctionTok{pairs}\NormalTok{(iris[,}\DecValTok{1}\SpecialCharTok{:}\DecValTok{4}\NormalTok{],}\AttributeTok{col=}\NormalTok{iris.color)}
\end{Highlighting}
\end{Shaded}

\begin{center}\includegraphics[width=0.7\linewidth]{05-AnalisisExploratorio_files/figure-latex/unnamed-chunk-34-3} \end{center}

\hypertarget{inferencia-estaduxedstica}{%
\chapter{Inferencia estadística}\label{inferencia-estaduxedstica}}

El objetivo de este capítulo es ofrecer un primer acercamiento a la inferencia estadística,
cubriendo de forma somera los siguientes apartados:

\begin{itemize}
\tightlist
\item
  contrastes de normalidad
\item
  contrastes paramétricos y no paramétricos, con una y dos muestras
\item
  regresión y correlación
\item
  análisis de la varianza con un factor
\end{itemize}

En este capítulo utilizaremos como ejemplo los datos de clientes de una compañía de distribución industrial (HATCO)
contenidos en el fichero \emph{hatco.RData}.

\begin{Shaded}
\begin{Highlighting}[]
\FunctionTok{load}\NormalTok{(}\StringTok{\textquotesingle{}datos/hatco.RData\textquotesingle{}}\NormalTok{)}
\end{Highlighting}
\end{Shaded}

Listado de etiquetas

\begin{Shaded}
\begin{Highlighting}[]
\FunctionTok{as.data.frame}\NormalTok{(}\FunctionTok{attr}\NormalTok{(hatco, }\StringTok{"variable.labels"}\NormalTok{))}
\end{Highlighting}
\end{Shaded}

\begin{verbatim}
##          attr(hatco, "variable.labels")
## empresa                         Empresa
## tamano             Tamaño de la empresa
## adquisic      Estructura de adquisición
## tindustr              Tipo de industria
## tsitcomp    Tipo de situación de compra
## velocida           Velocidad de entrega
## precio                 Nivel de precios
## flexprec        Flexibilidad de precios
## imgfabri          Imagen del fabricante
## servconj              Servicio conjunto
## imgfvent     Imagen de fuerza de ventas
## calidadp            Calidad de producto
## fidelida   Porcentaje de compra a HATCO
## satisfac            Satisfacción global
## nfidelid        Nivel de compra a HATCO
## nsatisfa          Nivel de satisfacción
\end{verbatim}

\hypertarget{normalidad}{%
\section{Normalidad}\label{normalidad}}

Queremos hacer un estudio inferencial de la variable \emph{satisfac} (satisfacción global). Lo primero
que vamos a hacer es comprobar si, visualmente, los datos parecen razonablemente
simétricos y si se pueden ajustar por una distribución normal

\begin{Shaded}
\begin{Highlighting}[]
\FunctionTok{hist}\NormalTok{(hatco}\SpecialCharTok{$}\NormalTok{satisfac)}
\end{Highlighting}
\end{Shaded}

\begin{center}\includegraphics[width=0.7\linewidth]{06-Inferencia_files/figure-latex/unnamed-chunk-4-1} \end{center}

\begin{Shaded}
\begin{Highlighting}[]
\FunctionTok{qqnorm}\NormalTok{(hatco}\SpecialCharTok{$}\NormalTok{satisfac)}
\end{Highlighting}
\end{Shaded}

\begin{center}\includegraphics[width=0.7\linewidth]{06-Inferencia_files/figure-latex/unnamed-chunk-4-2} \end{center}

\begin{Shaded}
\begin{Highlighting}[]
\FunctionTok{shapiro.test}\NormalTok{(hatco}\SpecialCharTok{$}\NormalTok{satisfac)}
\end{Highlighting}
\end{Shaded}

\begin{verbatim}
## 
##  Shapiro-Wilk normality test
## 
## data:  hatco$satisfac
## W = 0.97608, p-value = 0.06813
\end{verbatim}

\hypertarget{contrastes}{%
\section{Contrastes}\label{contrastes}}

\hypertarget{una-muestra}{%
\subsection{Una muestra}\label{una-muestra}}

Obtenemos un intervalo de confianza de \emph{satisfac}

\begin{Shaded}
\begin{Highlighting}[]
\FunctionTok{t.test}\NormalTok{(hatco}\SpecialCharTok{$}\NormalTok{satisfac)  }\CommentTok{\# with(hatco, t.test(satisfac))}
\end{Highlighting}
\end{Shaded}

\begin{verbatim}
## 
##  One Sample t-test
## 
## data:  hatco$satisfac
## t = 55.301, df = 98, p-value < 2.2e-16
## alternative hypothesis: true mean is not equal to 0
## 95 percent confidence interval:
##  4.603406 4.946089
## sample estimates:
## mean of x 
##  4.774747
\end{verbatim}

Contrastamos si es razonable suponer que la media es 5

\begin{Shaded}
\begin{Highlighting}[]
\FunctionTok{t.test}\NormalTok{(hatco}\SpecialCharTok{$}\NormalTok{satisfac, }\AttributeTok{mu=}\DecValTok{5}\NormalTok{)}
\end{Highlighting}
\end{Shaded}

\begin{verbatim}
## 
##  One Sample t-test
## 
## data:  hatco$satisfac
## t = -2.6089, df = 98, p-value = 0.01051
## alternative hypothesis: true mean is not equal to 5
## 95 percent confidence interval:
##  4.603406 4.946089
## sample estimates:
## mean of x 
##  4.774747
\end{verbatim}

Utilizando una confianza del 99\%

\begin{Shaded}
\begin{Highlighting}[]
\FunctionTok{t.test}\NormalTok{(hatco}\SpecialCharTok{$}\NormalTok{satisfac, }\AttributeTok{mu=}\DecValTok{5}\NormalTok{, }\AttributeTok{conf.level=}\FloatTok{0.99}\NormalTok{)}
\end{Highlighting}
\end{Shaded}

\begin{verbatim}
## 
##  One Sample t-test
## 
## data:  hatco$satisfac
## t = -2.6089, df = 98, p-value = 0.01051
## alternative hypothesis: true mean is not equal to 5
## 99 percent confidence interval:
##  4.547935 5.001560
## sample estimates:
## mean of x 
##  4.774747
\end{verbatim}

Veamos si podemos afirmar que la media es menor que 5

\begin{Shaded}
\begin{Highlighting}[]
\FunctionTok{t.test}\NormalTok{(hatco}\SpecialCharTok{$}\NormalTok{satisfac, }\AttributeTok{mu=}\DecValTok{5}\NormalTok{, }\AttributeTok{alternative =} \StringTok{\textquotesingle{}less\textquotesingle{}}\NormalTok{)}
\end{Highlighting}
\end{Shaded}

\begin{verbatim}
## 
##  One Sample t-test
## 
## data:  hatco$satisfac
## t = -2.6089, df = 98, p-value = 0.005253
## alternative hypothesis: true mean is less than 5
## 95 percent confidence interval:
##      -Inf 4.918122
## sample estimates:
## mean of x 
##  4.774747
\end{verbatim}

¿Y mayor que 4.65?

\begin{Shaded}
\begin{Highlighting}[]
\FunctionTok{t.test}\NormalTok{(hatco}\SpecialCharTok{$}\NormalTok{satisfac, }\AttributeTok{mu=}\FloatTok{4.65}\NormalTok{, }\AttributeTok{alternative =} \StringTok{\textquotesingle{}greater\textquotesingle{}}\NormalTok{)}
\end{Highlighting}
\end{Shaded}

\begin{verbatim}
## 
##  One Sample t-test
## 
## data:  hatco$satisfac
## t = 1.4448, df = 98, p-value = 0.07585
## alternative hypothesis: true mean is greater than 4.65
## 95 percent confidence interval:
##  4.631373      Inf
## sample estimates:
## mean of x 
##  4.774747
\end{verbatim}

El test de los rangos con signo de Wilcoxon es un contraste no paramétrico
(exige que la distribución sea simétrica) que se puede utilizar como
alternativa al contraste \emph{t} de Student

\begin{Shaded}
\begin{Highlighting}[]
\FunctionTok{with}\NormalTok{(hatco, }\FunctionTok{wilcox.test}\NormalTok{(satisfac, }\AttributeTok{mu=}\DecValTok{5}\NormalTok{))}
\end{Highlighting}
\end{Shaded}

\begin{verbatim}
## 
##  Wilcoxon signed rank test with continuity correction
## 
## data:  satisfac
## V = 1574, p-value = 0.01303
## alternative hypothesis: true location is not equal to 5
\end{verbatim}

\hypertarget{dos-muestras}{%
\subsection{Dos muestras}\label{dos-muestras}}

Disponemos de dos muestras independientes, el porcentaje de compra
en las empresas con nivel de satisfacción bajo y alto,
y asumimos que las varianzas son iguales

\begin{Shaded}
\begin{Highlighting}[]
\FunctionTok{t.test}\NormalTok{(fidelida }\SpecialCharTok{\textasciitilde{}}\NormalTok{ nsatisfa, }\AttributeTok{data =}\NormalTok{ hatco, }\AttributeTok{var.equal=}\ConstantTok{TRUE}\NormalTok{)}
\end{Highlighting}
\end{Shaded}

\begin{verbatim}
## 
##  Two Sample t-test
## 
## data:  fidelida by nsatisfa
## t = -6.5833, df = 97, p-value = 2.363e-09
## alternative hypothesis: true difference in means between group bajo and group alto is not equal to 0
## 95 percent confidence interval:
##  -12.915013  -6.931653
## sample estimates:
## mean in group bajo mean in group alto 
##           41.72778           51.65111
\end{verbatim}

Si no se asume igualdad de varianzas, se calcula la variante Welch del test \emph{t}

\begin{Shaded}
\begin{Highlighting}[]
\FunctionTok{t.test}\NormalTok{(fidelida }\SpecialCharTok{\textasciitilde{}}\NormalTok{ nsatisfa, }\AttributeTok{data =}\NormalTok{ hatco)}
\end{Highlighting}
\end{Shaded}

\begin{verbatim}
## 
##  Welch Two Sample t-test
## 
## data:  fidelida by nsatisfa
## t = -6.6901, df = 96.995, p-value = 1.437e-09
## alternative hypothesis: true difference in means between group bajo and group alto is not equal to 0
## 95 percent confidence interval:
##  -12.86727  -6.97940
## sample estimates:
## mean in group bajo mean in group alto 
##           41.72778           51.65111
\end{verbatim}

Comparemos visualmente las varianzas

\begin{Shaded}
\begin{Highlighting}[]
\FunctionTok{boxplot}\NormalTok{(fidelida }\SpecialCharTok{\textasciitilde{}}\NormalTok{ nsatisfa, }\AttributeTok{data =}\NormalTok{ hatco)}
\end{Highlighting}
\end{Shaded}

\begin{center}\includegraphics[width=0.7\linewidth]{06-Inferencia_files/figure-latex/unnamed-chunk-13-1} \end{center}

La comparación de las varianzas puede hacerse con el test \emph{F}

\begin{Shaded}
\begin{Highlighting}[]
\FunctionTok{var.test}\NormalTok{(fidelida }\SpecialCharTok{\textasciitilde{}}\NormalTok{ nsatisfa, }\AttributeTok{data =}\NormalTok{ hatco)}
\end{Highlighting}
\end{Shaded}

\begin{verbatim}
## 
##  F test to compare two variances
## 
## data:  fidelida by nsatisfa
## F = 1.4248, num df = 53, denom df = 44, p-value = 0.2292
## alternative hypothesis: true ratio of variances is not equal to 1
## 95 percent confidence interval:
##  0.797925 2.505462
## sample estimates:
## ratio of variances 
##           1.424804
\end{verbatim}

Una alternativa no paramétrica

\begin{Shaded}
\begin{Highlighting}[]
\FunctionTok{bartlett.test}\NormalTok{(fidelida }\SpecialCharTok{\textasciitilde{}}\NormalTok{ nsatisfa, }\AttributeTok{data =}\NormalTok{ hatco)}
\end{Highlighting}
\end{Shaded}

\begin{verbatim}
## 
##  Bartlett test of homogeneity of variances
## 
## data:  fidelida by nsatisfa
## Bartlett's K-squared = 1.4675, df = 1, p-value = 0.2257
\end{verbatim}

También puede utilizarse el test de Wilcoxon como alternativa al test \emph{t}

\begin{Shaded}
\begin{Highlighting}[]
\FunctionTok{wilcox.test}\NormalTok{(fidelida }\SpecialCharTok{\textasciitilde{}}\NormalTok{ nsatisfa, }\AttributeTok{data =}\NormalTok{ hatco)}
\end{Highlighting}
\end{Shaded}

\begin{verbatim}
## 
##  Wilcoxon rank sum test with continuity correction
## 
## data:  fidelida by nsatisfa
## W = 430.5, p-value = 3.504e-08
## alternative hypothesis: true location shift is not equal to 0
\end{verbatim}

Si disponemos de datos apareados, por ejemplo nivel de precios e imagen
de fuerza de ventas

\begin{Shaded}
\begin{Highlighting}[]
\FunctionTok{with}\NormalTok{(hatco, }\FunctionTok{t.test}\NormalTok{(precio, imgfvent, }\AttributeTok{paired =} \ConstantTok{TRUE}\NormalTok{))}
\end{Highlighting}
\end{Shaded}

\begin{verbatim}
## 
##  Paired t-test
## 
## data:  precio and imgfvent
## t = -2.2347, df = 98, p-value = 0.02771
## alternative hypothesis: true difference in means is not equal to 0
## 95 percent confidence interval:
##  -0.55114759 -0.03269079
## sample estimates:
## mean of the differences 
##              -0.2919192
\end{verbatim}

Y la correspondiente alternativa no paramétrica

\begin{Shaded}
\begin{Highlighting}[]
\FunctionTok{with}\NormalTok{(hatco, }\FunctionTok{wilcox.test}\NormalTok{(precio, imgfvent, }\AttributeTok{paired =} \ConstantTok{TRUE}\NormalTok{))}
\end{Highlighting}
\end{Shaded}

\begin{verbatim}
## 
##  Wilcoxon signed rank test with continuity correction
## 
## data:  precio and imgfvent
## V = 1789.5, p-value = 0.02431
## alternative hypothesis: true location shift is not equal to 0
\end{verbatim}

\hypertarget{regresiuxf3n-y-correlaciuxf3n}{%
\section{Regresión y correlación}\label{regresiuxf3n-y-correlaciuxf3n}}

\hypertarget{regresiuxf3n-lineal-simple}{%
\subsection{Regresión lineal simple}\label{regresiuxf3n-lineal-simple}}

Utilizando la función \emph{lm} (modelo lineal) se puede llevar a cabo, entre otras
muchas cosas, una regresión lineal simple

\begin{Shaded}
\begin{Highlighting}[]
\FunctionTok{lm}\NormalTok{(satisfac }\SpecialCharTok{\textasciitilde{}}\NormalTok{ fidelida, }\AttributeTok{data =}\NormalTok{ hatco)}
\end{Highlighting}
\end{Shaded}

\begin{verbatim}
## 
## Call:
## lm(formula = satisfac ~ fidelida, data = hatco)
## 
## Coefficients:
## (Intercept)     fidelida  
##      1.6074       0.0685
\end{verbatim}

\begin{Shaded}
\begin{Highlighting}[]
\NormalTok{modelo }\OtherTok{\textless{}{-}} \FunctionTok{lm}\NormalTok{(satisfac }\SpecialCharTok{\textasciitilde{}}\NormalTok{ fidelida, }\AttributeTok{data =}\NormalTok{ hatco, }\AttributeTok{na.action=}\NormalTok{na.exclude)}
\FunctionTok{summary}\NormalTok{(modelo)}
\end{Highlighting}
\end{Shaded}

\begin{verbatim}
## 
## Call:
## lm(formula = satisfac ~ fidelida, data = hatco, na.action = na.exclude)
## 
## Residuals:
##      Min       1Q   Median       3Q      Max 
## -1.47492 -0.37341  0.09358  0.38258  1.25258 
## 
## Coefficients:
##             Estimate Std. Error t value Pr(>|t|)    
## (Intercept) 1.607399   0.322436   4.985 2.71e-06 ***
## fidelida    0.068500   0.006848  10.003  < 2e-16 ***
## ---
## Signif. codes:  0 '***' 0.001 '**' 0.01 '*' 0.05 '.' 0.1 ' ' 1
## 
## Residual standard error: 0.6058 on 97 degrees of freedom
##   (1 observation deleted due to missingness)
## Multiple R-squared:  0.5078, Adjusted R-squared:  0.5027 
## F-statistic: 100.1 on 1 and 97 DF,  p-value: < 2.2e-16
\end{verbatim}

\begin{Shaded}
\begin{Highlighting}[]
\FunctionTok{plot}\NormalTok{(hatco}\SpecialCharTok{$}\NormalTok{fidelida, hatco}\SpecialCharTok{$}\NormalTok{satisfac)      }\CommentTok{\# Cuidado con el orden de las variables}
\CommentTok{\# with(hatco, plot(fidelida, satisfac))   \# Alternativa empleando with}
\CommentTok{\# plot(satisfac \textasciitilde{} fidelida, data = hatco) \# Alternativa empleando fórmulas}
\FunctionTok{abline}\NormalTok{(modelo)}
\end{Highlighting}
\end{Shaded}

\begin{center}\includegraphics[width=0.7\linewidth]{06-Inferencia_files/figure-latex/unnamed-chunk-19-1} \end{center}

Valores ajustados

\begin{Shaded}
\begin{Highlighting}[]
\FunctionTok{fitted}\NormalTok{(modelo)}
\end{Highlighting}
\end{Shaded}

\begin{verbatim}
##        1        2        3        4        5        6        7        8 
## 3.799412 4.552917 4.895419 3.799412 5.580423 4.689918 4.758418 4.621417 
##        9       10       11       12       13       14       15       16 
## 5.922925 5.306421 3.799412 4.826919 4.278915 4.210415 5.306421 4.963919 
##       17       18       19       20       21       22       23       24 
## 4.210415 4.347416 5.306421 5.374922 4.415916 4.004913 5.374922 4.073414 
##       25       26       27       28       29       30       31       32 
## 4.963919 4.963919 4.073414 5.306421 4.963919 4.758418 4.552917 5.237921 
##       33       34       35       36       37       38       39       40 
## 5.717424 4.847469 4.004913 4.278915 4.621417 4.758418 3.593911 3.525410 
##       41       42       43       44       45       46       47       48 
## 4.347416 5.580423 5.237921 4.895419 4.210415 5.306421 5.374922 4.552917 
##       49       50       51       52       53       54       55       56 
## 5.511923 5.237921 4.415916 5.237921 5.032420 3.799412 4.278915 4.826919 
##       57       58       59       60       61       62       63       64 
## 5.854425 6.059926 4.758418 5.032420 5.306421 5.717424 4.826919 4.073414 
##       65       66       67       68       69       70       71       72 
## 4.347416 4.689918 5.648924 4.758418 5.580423 4.963919 5.032420 5.374922 
##       73       74       75       76       77       78       79       80 
## 5.100920 5.717424 4.415916 4.963919 4.484416 4.826919 4.278915 5.443422 
##       81       82       83       84       85       86       87       88 
## 5.648924 4.847469 4.415916 4.141914 5.237921 4.552917 5.100920 4.073414 
##       89       90       91       92       93       94       95       96 
## 3.936413 5.717424 4.963919 4.278915 4.552917 4.073414 3.730912 3.319909 
##       97       98       99      100 
## 5.717424 4.210415 4.484416       NA
\end{verbatim}

Residuos

\begin{Shaded}
\begin{Highlighting}[]
\FunctionTok{head}\NormalTok{(}\FunctionTok{resid}\NormalTok{(modelo))}
\end{Highlighting}
\end{Shaded}

\begin{verbatim}
##          1          2          3          4          5          6 
##  0.4005878 -0.2529168  0.3045811  0.1005878  1.2195769 -0.2899177
\end{verbatim}

\begin{Shaded}
\begin{Highlighting}[]
\FunctionTok{qqnorm}\NormalTok{(}\FunctionTok{resid}\NormalTok{(modelo))}
\end{Highlighting}
\end{Shaded}

\begin{center}\includegraphics[width=0.7\linewidth]{06-Inferencia_files/figure-latex/unnamed-chunk-21-1} \end{center}

\begin{Shaded}
\begin{Highlighting}[]
\FunctionTok{shapiro.test}\NormalTok{(}\FunctionTok{resid}\NormalTok{(modelo))}
\end{Highlighting}
\end{Shaded}

\begin{verbatim}
## 
##  Shapiro-Wilk normality test
## 
## data:  resid(modelo)
## W = 0.98515, p-value = 0.3325
\end{verbatim}

\begin{Shaded}
\begin{Highlighting}[]
\FunctionTok{plot}\NormalTok{(hatco}\SpecialCharTok{$}\NormalTok{fidelida, hatco}\SpecialCharTok{$}\NormalTok{satisfac)    }

\FunctionTok{abline}\NormalTok{(modelo)}
\CommentTok{\# segments(hatco$fidelida, fitted(modelo), hatco$fidelida, hatco$satisfac)}
\FunctionTok{with}\NormalTok{(hatco, }\FunctionTok{segments}\NormalTok{(fidelida, }\FunctionTok{fitted}\NormalTok{(modelo), fidelida, satisfac))}
\end{Highlighting}
\end{Shaded}

\begin{center}\includegraphics[width=0.7\linewidth]{06-Inferencia_files/figure-latex/unnamed-chunk-21-2} \end{center}

\begin{Shaded}
\begin{Highlighting}[]
\FunctionTok{plot}\NormalTok{(}\FunctionTok{fitted}\NormalTok{(modelo), }\FunctionTok{resid}\NormalTok{(modelo))}
\end{Highlighting}
\end{Shaded}

\begin{center}\includegraphics[width=0.7\linewidth]{06-Inferencia_files/figure-latex/unnamed-chunk-21-3} \end{center}

Banda de confianza

\begin{Shaded}
\begin{Highlighting}[]
\FunctionTok{predict}\NormalTok{(modelo, }\AttributeTok{interval=}\StringTok{\textquotesingle{}confidence\textquotesingle{}}\NormalTok{)}
\end{Highlighting}
\end{Shaded}

\begin{verbatim}
##          fit      lwr      upr
## 1   3.799412 3.571263 4.027561
## 2   4.552917 4.424306 4.681528
## 3   4.895419 4.772225 5.018613
## 4   3.799412 3.571263 4.027561
## 5   5.580423 5.380031 5.780815
## 6   4.689918 4.567906 4.811929
## 7   4.758418 4.637529 4.879307
## 8   4.621417 4.496801 4.746033
## 9   5.922925 5.665048 6.180803
## 10  5.306421 5.146011 5.466832
## 11  3.799412 3.571263 4.027561
## 12  4.826919 4.705631 4.948206
## 13  4.278915 4.123089 4.434741
## 14  4.210415 4.045670 4.375159
## 15  5.306421 5.146011 5.466832
## 16  4.963919 4.837379 5.090459
## 17  4.210415 4.045670 4.375159
## 18  4.347416 4.199793 4.495038
## 19  5.306421 5.146011 5.466832
## 20  5.374922 5.205264 5.544580
## 21  4.415916 4.275658 4.556174
## 22  4.004913 3.810147 4.199680
## 23  5.374922 5.205264 5.544580
## 24  4.073414 3.889113 4.257714
## 25  4.963919 4.837379 5.090459
## 26  4.963919 4.837379 5.090459
## 27  4.073414 3.889113 4.257714
## 28  5.306421 5.146011 5.466832
## 29  4.963919 4.837379 5.090459
## 30  4.758418 4.637529 4.879307
## 31  4.552917 4.424306 4.681528
## 32  5.237921 5.086103 5.389740
## 33  5.717424 5.494745 5.940103
## 34  4.847469 4.725765 4.969172
## 35  4.004913 3.810147 4.199680
## 36  4.278915 4.123089 4.434741
## 37  4.621417 4.496801 4.746033
## 38  4.758418 4.637529 4.879307
## 39  3.593911 3.330292 3.857530
## 40  3.525410 3.249642 3.801179
## 41  4.347416 4.199793 4.495038
## 42  5.580423 5.380031 5.780815
## 43  5.237921 5.086103 5.389740
## 44  4.895419 4.772225 5.018613
## 45  4.210415 4.045670 4.375159
## 46  5.306421 5.146011 5.466832
## 47  5.374922 5.205264 5.544580
## 48  4.552917 4.424306 4.681528
## 49  5.511923 5.322196 5.701650
## 50  5.237921 5.086103 5.389740
## 51  4.415916 4.275658 4.556174
## 52  5.237921 5.086103 5.389740
## 53  5.032420 4.901205 5.163635
## 54  3.799412 3.571263 4.027561
## 55  4.278915 4.123089 4.434741
## 56  4.826919 4.705631 4.948206
## 57  5.854425 5.608471 6.100378
## 58  6.059926 5.777748 6.342104
## 59  4.758418 4.637529 4.879307
## 60  5.032420 4.901205 5.163635
## 61  5.306421 5.146011 5.466832
## 62  5.717424 5.494745 5.940103
## 63  4.826919 4.705631 4.948206
## 64  4.073414 3.889113 4.257714
## 65  4.347416 4.199793 4.495038
## 66  4.689918 4.567906 4.811929
## 67  5.648924 5.437531 5.860316
## 68  4.758418 4.637529 4.879307
## 69  5.580423 5.380031 5.780815
## 70  4.963919 4.837379 5.090459
## 71  5.032420 4.901205 5.163635
## 72  5.374922 5.205264 5.544580
## 73  5.100920 4.963837 5.238003
## 74  5.717424 5.494745 5.940103
## 75  4.415916 4.275658 4.556174
## 76  4.963919 4.837379 5.090459
## 77  4.484416 4.350544 4.618289
## 78  4.826919 4.705631 4.948206
## 79  4.278915 4.123089 4.434741
## 80  5.443422 5.263964 5.622881
## 81  5.648924 5.437531 5.860316
## 82  4.847469 4.725765 4.969172
## 83  4.415916 4.275658 4.556174
## 84  4.141914 3.967647 4.316181
## 85  5.237921 5.086103 5.389740
## 86  4.552917 4.424306 4.681528
## 87  5.100920 4.963837 5.238003
## 88  4.073414 3.889113 4.257714
## 89  3.936413 3.730815 4.142011
## 90  5.717424 5.494745 5.940103
## 91  4.963919 4.837379 5.090459
## 92  4.278915 4.123089 4.434741
## 93  4.552917 4.424306 4.681528
## 94  4.073414 3.889113 4.257714
## 95  3.730912 3.491126 3.970697
## 96  3.319909 3.006980 3.632839
## 97  5.717424 5.494745 5.940103
## 98  4.210415 4.045670 4.375159
## 99  4.484416 4.350544 4.618289
## 100       NA       NA       NA
\end{verbatim}

Banda de predicción

\begin{Shaded}
\begin{Highlighting}[]
\FunctionTok{head}\NormalTok{(}\FunctionTok{predict}\NormalTok{(modelo, }\AttributeTok{interval=}\StringTok{\textquotesingle{}prediction\textquotesingle{}}\NormalTok{))}
\end{Highlighting}
\end{Shaded}

\begin{verbatim}
##        fit      lwr      upr
## 1 3.799412 2.575563 5.023261
## 2 4.552917 3.343663 5.762171
## 3 4.895419 3.686729 6.104109
## 4 3.799412 2.575563 5.023261
## 5 5.580423 4.361444 6.799403
## 6 4.689918 3.481348 5.898487
\end{verbatim}

Representación gráfica de las bandas

\begin{Shaded}
\begin{Highlighting}[]
\NormalTok{bandas.frame }\OtherTok{\textless{}{-}} \FunctionTok{data.frame}\NormalTok{(}\AttributeTok{fidelida=}\DecValTok{24}\SpecialCharTok{:}\DecValTok{66}\NormalTok{)}
\NormalTok{bc }\OtherTok{\textless{}{-}} \FunctionTok{predict}\NormalTok{(modelo, }\AttributeTok{interval =} \StringTok{\textquotesingle{}confidence\textquotesingle{}}\NormalTok{, }\AttributeTok{newdata =}\NormalTok{ bandas.frame)}
\NormalTok{bp }\OtherTok{\textless{}{-}} \FunctionTok{predict}\NormalTok{(modelo, }\AttributeTok{interval =} \StringTok{\textquotesingle{}prediction\textquotesingle{}}\NormalTok{, }\AttributeTok{newdata =}\NormalTok{ bandas.frame)}
\FunctionTok{plot}\NormalTok{(hatco}\SpecialCharTok{$}\NormalTok{fidelida, hatco}\SpecialCharTok{$}\NormalTok{satisfac, }\AttributeTok{ylim =} \FunctionTok{range}\NormalTok{(hatco}\SpecialCharTok{$}\NormalTok{satisfac, bp, }\AttributeTok{na.rm =} \ConstantTok{TRUE}\NormalTok{))}
\FunctionTok{matlines}\NormalTok{(bandas.frame}\SpecialCharTok{$}\NormalTok{fidelida, bc, }\AttributeTok{lty=}\FunctionTok{c}\NormalTok{(}\DecValTok{1}\NormalTok{,}\DecValTok{2}\NormalTok{,}\DecValTok{2}\NormalTok{), }\AttributeTok{col=}\StringTok{\textquotesingle{}black\textquotesingle{}}\NormalTok{)}
\FunctionTok{matlines}\NormalTok{(bandas.frame}\SpecialCharTok{$}\NormalTok{fidelida, bp, }\AttributeTok{lty=}\FunctionTok{c}\NormalTok{(}\DecValTok{0}\NormalTok{,}\DecValTok{3}\NormalTok{,}\DecValTok{3}\NormalTok{), }\AttributeTok{col=}\StringTok{\textquotesingle{}red\textquotesingle{}}\NormalTok{)}
\end{Highlighting}
\end{Shaded}

\begin{center}\includegraphics[width=0.7\linewidth]{06-Inferencia_files/figure-latex/unnamed-chunk-24-1} \end{center}

\hypertarget{correlaciuxf3n}{%
\subsection{Correlación}\label{correlaciuxf3n}}

Coeficiente de correlación de Pearson

\begin{Shaded}
\begin{Highlighting}[]
\FunctionTok{cor}\NormalTok{(hatco}\SpecialCharTok{$}\NormalTok{fidelida, hatco}\SpecialCharTok{$}\NormalTok{satisfac, }\AttributeTok{use=}\StringTok{\textquotesingle{}complete.obs\textquotesingle{}}\NormalTok{)}
\end{Highlighting}
\end{Shaded}

\begin{verbatim}
## [1] 0.712581
\end{verbatim}

\begin{Shaded}
\begin{Highlighting}[]
\FunctionTok{cor}\NormalTok{(hatco[,}\DecValTok{6}\SpecialCharTok{:}\DecValTok{14}\NormalTok{], }\AttributeTok{use=}\StringTok{\textquotesingle{}complete.obs\textquotesingle{}}\NormalTok{)}
\end{Highlighting}
\end{Shaded}

\begin{verbatim}
##             velocida      precio    flexprec    imgfabri    servconj
## velocida  1.00000000 -0.35439461  0.51879732  0.04885481  0.60908594
## precio   -0.35439461  1.00000000 -0.48550163  0.27150666  0.51134698
## flexprec  0.51879732 -0.48550163  1.00000000 -0.11472112  0.07496499
## imgfabri  0.04885481  0.27150666 -0.11472112  1.00000000  0.29800272
## servconj  0.60908594  0.51134698  0.07496499  0.29800272  1.00000000
## imgfvent  0.08084452  0.18873090 -0.03801323  0.79015164  0.24641510
## calidadp -0.48984768  0.46822563 -0.44542562  0.19904126 -0.06152068
## fidelida  0.67428681  0.07682487  0.57807750  0.22442574  0.69802972
## satisfac  0.64981476  0.02636286  0.53057615  0.47553688  0.63054720
##             imgfvent    calidadp    fidelida    satisfac
## velocida  0.08084452 -0.48984768  0.67428681  0.64981476
## precio    0.18873090  0.46822563  0.07682487  0.02636286
## flexprec -0.03801323 -0.44542562  0.57807750  0.53057615
## imgfabri  0.79015164  0.19904126  0.22442574  0.47553688
## servconj  0.24641510 -0.06152068  0.69802972  0.63054720
## imgfvent  1.00000000  0.18052945  0.26674626  0.34349253
## calidadp  0.18052945  1.00000000 -0.20401261 -0.28687427
## fidelida  0.26674626 -0.20401261  1.00000000  0.71258104
## satisfac  0.34349253 -0.28687427  0.71258104  1.00000000
\end{verbatim}

\begin{Shaded}
\begin{Highlighting}[]
\FunctionTok{cor.test}\NormalTok{(hatco}\SpecialCharTok{$}\NormalTok{fidelida, hatco}\SpecialCharTok{$}\NormalTok{satisfac)}
\end{Highlighting}
\end{Shaded}

\begin{verbatim}
## 
##  Pearson's product-moment correlation
## 
## data:  hatco$fidelida and hatco$satisfac
## t = 10.003, df = 97, p-value < 2.2e-16
## alternative hypothesis: true correlation is not equal to 0
## 95 percent confidence interval:
##  0.5995024 0.7977691
## sample estimates:
##      cor 
## 0.712581
\end{verbatim}

El coeficiente de correlación de Spearman es una variante no paramétrica

\begin{Shaded}
\begin{Highlighting}[]
\FunctionTok{cor.test}\NormalTok{(hatco}\SpecialCharTok{$}\NormalTok{fidelida, hatco}\SpecialCharTok{$}\NormalTok{satisfac, }\AttributeTok{method=}\StringTok{\textquotesingle{}spearman\textquotesingle{}}\NormalTok{)}
\end{Highlighting}
\end{Shaded}

\begin{verbatim}
## 
##  Spearman's rank correlation rho
## 
## data:  hatco$fidelida and hatco$satisfac
## S = 46601, p-value < 2.2e-16
## alternative hypothesis: true rho is not equal to 0
## sample estimates:
##       rho 
## 0.7118039
\end{verbatim}

\hypertarget{anuxe1lisis-de-la-varianza}{%
\section{Análisis de la varianza}\label{anuxe1lisis-de-la-varianza}}

\hypertarget{anova-con-un-factor}{%
\subsection{ANOVA con un factor}\label{anova-con-un-factor}}

Vamos a estudiar si hay diferencias en las medias de la variable \emph{satisfac}
(satisfacción global) entre los diferentes grupos definidos por \emph{nfidelid}
(nivel de compra), utilizando el procedimiento clásico de análisis de la
varianza. Este procedimiento exige normalidad y homocedasticidad.

\begin{Shaded}
\begin{Highlighting}[]
\FunctionTok{table}\NormalTok{(hatco}\SpecialCharTok{$}\NormalTok{nfidelid)}
\end{Highlighting}
\end{Shaded}

\begin{verbatim}
## 
##  bajo medio  alto 
##     3    64    33
\end{verbatim}

\begin{Shaded}
\begin{Highlighting}[]
\FunctionTok{tapply}\NormalTok{(hatco}\SpecialCharTok{$}\NormalTok{satisfac, hatco}\SpecialCharTok{$}\NormalTok{nfidelid, mean, }\AttributeTok{na.rm =} \ConstantTok{TRUE}\NormalTok{)}
\end{Highlighting}
\end{Shaded}

\begin{verbatim}
##     bajo    medio     alto 
## 3.533333 4.498437 5.443750
\end{verbatim}

La variable explicativa tiene que ser obligatoriamente de tipo \emph{factor}.
Por coherencia con la función (general) \emph{lm}, la variación entre grupos
está etiquetada \emph{nfidelid}, y la variación dentro de los grupos como
\emph{Residuals}

\begin{Shaded}
\begin{Highlighting}[]
\FunctionTok{anova}\NormalTok{(}\FunctionTok{lm}\NormalTok{(satisfac}\SpecialCharTok{\textasciitilde{}}\NormalTok{nfidelid, }\AttributeTok{data =}\NormalTok{ hatco))}
\end{Highlighting}
\end{Shaded}

\begin{verbatim}
## Analysis of Variance Table
## 
## Response: satisfac
##           Df Sum Sq Mean Sq F value    Pr(>F)    
## nfidelid   2 23.832 11.9158  23.588 4.647e-09 ***
## Residuals 96 48.495  0.5052                      
## ---
## Signif. codes:  0 '***' 0.001 '**' 0.01 '*' 0.05 '.' 0.1 ' ' 1
\end{verbatim}

Como alternativa, se puede utilizar la función \emph{aov}

\begin{Shaded}
\begin{Highlighting}[]
\FunctionTok{aov}\NormalTok{(satisfac}\SpecialCharTok{\textasciitilde{}}\NormalTok{nfidelid, }\AttributeTok{data =}\NormalTok{ hatco)}
\end{Highlighting}
\end{Shaded}

\begin{verbatim}
## Call:
##    aov(formula = satisfac ~ nfidelid, data = hatco)
## 
## Terms:
##                 nfidelid Residuals
## Sum of Squares  23.83161  48.49526
## Deg. of Freedom        2        96
## 
## Residual standard error: 0.7107454
## Estimated effects may be unbalanced
## 1 observation deleted due to missingness
\end{verbatim}

\begin{Shaded}
\begin{Highlighting}[]
\FunctionTok{summary}\NormalTok{(}\FunctionTok{aov}\NormalTok{(satisfac}\SpecialCharTok{\textasciitilde{}}\NormalTok{nfidelid, }\AttributeTok{data =}\NormalTok{ hatco))}
\end{Highlighting}
\end{Shaded}

\begin{verbatim}
##             Df Sum Sq Mean Sq F value   Pr(>F)    
## nfidelid     2  23.83  11.916   23.59 4.65e-09 ***
## Residuals   96  48.50   0.505                     
## ---
## Signif. codes:  0 '***' 0.001 '**' 0.01 '*' 0.05 '.' 0.1 ' ' 1
## 1 observation deleted due to missingness
\end{verbatim}

Comparaciones entre pares de variables

\begin{Shaded}
\begin{Highlighting}[]
\FunctionTok{pairwise.t.test}\NormalTok{(hatco}\SpecialCharTok{$}\NormalTok{satisfac, hatco}\SpecialCharTok{$}\NormalTok{nfidelid)}
\end{Highlighting}
\end{Shaded}

\begin{verbatim}
## 
##  Pairwise comparisons using t tests with pooled SD 
## 
## data:  hatco$satisfac and hatco$nfidelid 
## 
##       bajo    medio  
## medio 0.024   -      
## alto  4.6e-05 5.5e-08
## 
## P value adjustment method: holm
\end{verbatim}

Relajamos la hipótesis de varianzas iguales

\begin{Shaded}
\begin{Highlighting}[]
\FunctionTok{oneway.test}\NormalTok{(satisfac}\SpecialCharTok{\textasciitilde{}}\NormalTok{nfidelid, }\AttributeTok{data =}\NormalTok{ hatco)}
\end{Highlighting}
\end{Shaded}

\begin{verbatim}
## 
##  One-way analysis of means (not assuming equal variances)
## 
## data:  satisfac and nfidelid
## F = 35.013, num df = 2.0000, denom df = 6.7661, p-value = 0.0002697
\end{verbatim}

Podemos utilizar el test de Bartlett para contrastar la igualdad de varianzas

\begin{Shaded}
\begin{Highlighting}[]
\FunctionTok{bartlett.test}\NormalTok{(satisfac}\SpecialCharTok{\textasciitilde{}}\NormalTok{nfidelid, }\AttributeTok{data =}\NormalTok{ hatco)}
\end{Highlighting}
\end{Shaded}

\begin{verbatim}
## 
##  Bartlett test of homogeneity of variances
## 
## data:  satisfac by nfidelid
## Bartlett's K-squared = 1.4922, df = 2, p-value = 0.4742
\end{verbatim}

Representación gráfica

\begin{Shaded}
\begin{Highlighting}[]
\NormalTok{medias }\OtherTok{\textless{}{-}} \FunctionTok{tapply}\NormalTok{(hatco}\SpecialCharTok{$}\NormalTok{satisfac, hatco}\SpecialCharTok{$}\NormalTok{nfidelid, mean, }\AttributeTok{na.rm =} \ConstantTok{TRUE}\NormalTok{)}
\NormalTok{desviaciones }\OtherTok{\textless{}{-}} \FunctionTok{tapply}\NormalTok{(hatco}\SpecialCharTok{$}\NormalTok{satisfac, hatco}\SpecialCharTok{$}\NormalTok{nfidelid, sd, }\AttributeTok{na.rm =} \ConstantTok{TRUE}\NormalTok{)}
\NormalTok{n }\OtherTok{\textless{}{-}} \FunctionTok{tapply}\NormalTok{(hatco}\SpecialCharTok{$}\NormalTok{satisfac[}\SpecialCharTok{!}\FunctionTok{is.na}\NormalTok{(hatco}\SpecialCharTok{$}\NormalTok{satisfac)], hatco}\SpecialCharTok{$}\NormalTok{nfidelid[}\SpecialCharTok{!}\FunctionTok{is.na}\NormalTok{(hatco}\SpecialCharTok{$}\NormalTok{satisfac)], length)}
\NormalTok{errores }\OtherTok{\textless{}{-}}\NormalTok{ desviaciones}\SpecialCharTok{/}\FunctionTok{sqrt}\NormalTok{(n)}
\FunctionTok{stripchart}\NormalTok{(hatco}\SpecialCharTok{$}\NormalTok{satisfac}\SpecialCharTok{\textasciitilde{}}\NormalTok{hatco}\SpecialCharTok{$}\NormalTok{nfidelid, }\AttributeTok{method=}\StringTok{\textquotesingle{}jitter\textquotesingle{}}\NormalTok{, }\AttributeTok{jit=}\FloatTok{0.01}\NormalTok{, }\AttributeTok{pch=}\DecValTok{18}\NormalTok{, }\AttributeTok{col=}\StringTok{\textquotesingle{}grey\textquotesingle{}}\NormalTok{, }\AttributeTok{vertical =} \ConstantTok{TRUE}\NormalTok{)}
\FunctionTok{arrows}\NormalTok{(}\DecValTok{1}\SpecialCharTok{:}\DecValTok{3}\NormalTok{, medias}\SpecialCharTok{+}\NormalTok{errores, }\DecValTok{1}\SpecialCharTok{:}\DecValTok{3}\NormalTok{, medias}\SpecialCharTok{{-}}\NormalTok{errores, }\AttributeTok{angle=}\DecValTok{90}\NormalTok{, }\AttributeTok{code=}\DecValTok{3}\NormalTok{, }\AttributeTok{lwd=}\DecValTok{2}\NormalTok{, }\AttributeTok{col=}\StringTok{\textquotesingle{}orange\textquotesingle{}}\NormalTok{)}
\FunctionTok{points}\NormalTok{(}\DecValTok{1}\SpecialCharTok{:}\DecValTok{3}\NormalTok{, medias, }\AttributeTok{pch=}\DecValTok{4}\NormalTok{, }\AttributeTok{lwd=}\DecValTok{2}\NormalTok{, }\AttributeTok{cex=}\DecValTok{2}\NormalTok{, }\AttributeTok{col=}\StringTok{\textquotesingle{}orange\textquotesingle{}}\NormalTok{)}
\end{Highlighting}
\end{Shaded}

\begin{center}\includegraphics[width=0.7\linewidth]{06-Inferencia_files/figure-latex/unnamed-chunk-33-1} \end{center}

\hypertarget{test-de-kruskal-wallis}{%
\subsection{Test de Kruskal-Wallis}\label{test-de-kruskal-wallis}}

Alternativa no paramétrica al análisis de la varianza con un factor

\begin{Shaded}
\begin{Highlighting}[]
\FunctionTok{kruskal.test}\NormalTok{(satisfac}\SpecialCharTok{\textasciitilde{}}\NormalTok{nfidelid, }\AttributeTok{data =}\NormalTok{ hatco)}
\end{Highlighting}
\end{Shaded}

\begin{verbatim}
## 
##  Kruskal-Wallis rank sum test
## 
## data:  satisfac by nfidelid
## Kruskal-Wallis chi-squared = 31.073, df = 2, p-value = 1.789e-07
\end{verbatim}

\hypertarget{modelado-de-datos}{%
\chapter{Modelado de datos}\label{modelado-de-datos}}

La realidad puede ser muy compleja por lo que es habitual emplear un
modelo para tratar de explicarla.

\begin{itemize}
\item
  Modelos estocásticos (con componente aleatoria).

  \begin{itemize}
  \item
    Tienen en cuenta la incertidumbre debida a no disponer de la suficiente información
    sobre las variables que influyen en el fenómeno en estudio.
  \item
    La inferencia estadística proporciona herramientas para ajustar y
    contrastar la validez del modelo a partir de los datos observados.
  \end{itemize}
\end{itemize}

Sin embargo resultaría muy extraño que la realidad coincida exactamente con un modelo concreto.

\begin{itemize}
\item
  \href{https://en.wikipedia.org/wiki/George_E._P._Box}{George Box} afirmó en su famoso aforismo:

  \begin{quote}
  En esencia, todos los modelos son falsos, pero algunos son útiles.
  \end{quote}
\item
  El objetivo de un modelo es disponer de una aproximación simple de la realidad que sea útil.
\end{itemize}

\hypertarget{modelos-de-regresiuxf3n}{%
\section{Modelos de regresión}\label{modelos-de-regresiuxf3n}}

Nos centraremos en los modelos de regresión:

\[Y=f(X_{1},\cdots,X_{p})+\varepsilon\]
donde:

\begin{itemize}
\item
  \(Y\equiv\) \textbf{variable respuesta} (o dependiente).
\item
  \(\left( X_{1},\cdots,X_{p}\right) \equiv\) \textbf{variables
  explicativas} (independientes, o covariables).
\item
  \(\varepsilon\equiv\) \textbf{error aleatorio.}
\end{itemize}

\hypertarget{herramientas-disponibles-en-r}{%
\subsection{\texorpdfstring{Herramientas disponibles en \texttt{R}}{Herramientas disponibles en R}}\label{herramientas-disponibles-en-r}}

\texttt{R} dispone de múltiples herramientas para trabajar con modelos de este tipo. Algunas de las funciones y paquetes disponibles se muestran a continuación:

\begin{itemize}
\item
  Modelos paramétricos:

  \begin{itemize}
  \item
    Modelos lineales:

    \begin{itemize}
    \item
      Regresión lineal: \texttt{lm()} (\texttt{aov()}, \texttt{lme()}, \texttt{biglm}, \ldots).
    \item
      Regresión lineal robusta: \texttt{MASS::rlm()}.
    \item
      Métodos de regularización (Ridge regression, Lasso): \texttt{glmnet}, \ldots{}
    \end{itemize}
  \item
    Modelos lineales generalizados: \texttt{glm()} (\texttt{bigglm}, \ldots).
  \item
    Modelos paramétricos no lineales: \texttt{nls()} (\texttt{nlme}, \ldots).
  \end{itemize}
\item
  Modelos no paramétricos:

  \begin{itemize}
  \item
    Regresión local (métodos de suavizado): \texttt{loess()}, \texttt{KernSmooth}, \texttt{sm}, \ldots{}
  \item
    Modelos aditivos generalizados (GAM): \texttt{gam}, \texttt{mgcv}, \ldots{}
  \item
    Arboles de decisión (Random Forest, Boosting): \texttt{rpart}, \texttt{randomForest}, \texttt{xgboost}, \ldots{}
  \item
    Redes neuronales, \ldots{}
  \end{itemize}
\end{itemize}

Desde el punto de vista de la programación, con todos estos modelos se trabaja de una forma muy similar en \texttt{R}.

\hypertarget{fuxf3rmulas}{%
\section{Fórmulas}\label{fuxf3rmulas}}

En \texttt{R} para especificar un modelo estadístico (realmente una familia) se suelen emplear fórmulas (también para generar gráficos).
Son de la forma:

\begin{Shaded}
\begin{Highlighting}[]
\NormalTok{respuesta }\SpecialCharTok{\textasciitilde{}}\NormalTok{ modelo}
\end{Highlighting}
\end{Shaded}

\texttt{modelo} especifica los ``términos'' mediante operadores (tienen un significado especial en este contexto):

\begin{longtable}[]{@{}ll@{}}
\toprule
Operador & Descripción \\
\midrule
\endhead
\texttt{a+b} & incluye \texttt{a} y \texttt{b} (efectos principales) \\
\texttt{-b} & excluye \texttt{b} del modelo \\
\texttt{a:b} & interacción de \texttt{a} y \texttt{b} \\
\textbackslash{} & \texttt{b\ \%in\%\ a} efectos de \texttt{b} anidados en \texttt{a} (\texttt{a:b}) \\
\textbackslash{} & \texttt{a/b\ =\ a\ +\ b\ \%in\%\ a\ =\ a\ +\ a:b} \\
\texttt{a*b\ =\ a+b+a:b} & efectos principales más interacciones \\
\texttt{\^{}n} & interacciones hasta nivel \texttt{n} (\texttt{(a+b)\^{}2\ =\ a+b+a:b}) \\
\texttt{poly(a,\ n)} & polinomios de \texttt{a} hasta grado \texttt{n} \\
\texttt{1} & término constante \\
\texttt{.} & todas las variables disponibles o modelo actual en actualizaciones \\
\bottomrule
\end{longtable}

Para realizar operaciones aritméticas (que incluyan \texttt{+}, \texttt{-}, \texttt{*}, \texttt{\^{}}, \texttt{1}, \ldots)
es necesario ``aislar'' la operación
dentro una función (e.g.~\texttt{log(abs(x)\ +\ 1)}).
Por ejemplo, para realizar un ajuste cuadrático se debería utilizar \texttt{y\ \textasciitilde{}\ x\ +\ I(x\^{}2)}, ya que \texttt{y\ \textasciitilde{}\ x\ +\ x\^{}2\ =\ y\ \textasciitilde{}\ x} (la interacción \texttt{x:x\ =\ x}).

\begin{itemize}
\tightlist
\item
  \texttt{I()} función identidad.
\end{itemize}

\hypertarget{ejemplo-regresiuxf3n-lineal-simple}{%
\section{Ejemplo: regresión lineal simple}\label{ejemplo-regresiuxf3n-lineal-simple}}

Introducido en descriptiva y con referencias al tema siguiente

\hypertarget{modelos-lineales}{%
\chapter{Modelos lineales}\label{modelos-lineales}}

Suponen que la función de regresión es lineal:

\[Y=\beta_{0}+\beta_{1}X_{1}+\beta_{2}X_{2}+\cdots+\beta_{p}X_{p}+\varepsilon\]

El efecto de las variables explicativas sobre la respuesta es simple (proporcional a su valor).

\hypertarget{ejemplo}{%
\section{Ejemplo}\label{ejemplo}}

El fichero \emph{hatco.RData} contiene observaciones de clientes de la compañía de
distribución industrial (Compañía Hair, Anderson y Tatham).
Las variables se pueden clasificar en tres grupos:

\begin{Shaded}
\begin{Highlighting}[]
\FunctionTok{load}\NormalTok{(}\StringTok{\textquotesingle{}datos/hatco.RData\textquotesingle{}}\NormalTok{)}
\FunctionTok{as.data.frame}\NormalTok{(}\FunctionTok{attr}\NormalTok{(hatco, }\StringTok{"variable.labels"}\NormalTok{))}
\end{Highlighting}
\end{Shaded}

\begin{verbatim}
##          attr(hatco, "variable.labels")
## empresa                         Empresa
## tamano             Tamaño de la empresa
## adquisic      Estructura de adquisición
## tindustr              Tipo de industria
## tsitcomp    Tipo de situación de compra
## velocida           Velocidad de entrega
## precio                 Nivel de precios
## flexprec        Flexibilidad de precios
## imgfabri          Imagen del fabricante
## servconj              Servicio conjunto
## imgfvent     Imagen de fuerza de ventas
## calidadp            Calidad de producto
## fidelida   Porcentaje de compra a HATCO
## satisfac            Satisfacción global
## nfidelid        Nivel de compra a HATCO
## nsatisfa          Nivel de satisfacción
\end{verbatim}

Consideraremos como respuesta la variable \emph{fidelida} y como variables explicativas
el resto de variables continuas menos \emph{satisfac}.

\begin{Shaded}
\begin{Highlighting}[]
\NormalTok{datos }\OtherTok{\textless{}{-}}\NormalTok{ hatco[, }\DecValTok{6}\SpecialCharTok{:}\DecValTok{13}\NormalTok{]  }\CommentTok{\# Nota: realmente no copia el objeto...}
\FunctionTok{plot}\NormalTok{(datos)}
\end{Highlighting}
\end{Shaded}

\includegraphics{08-ModelosLineales_files/figure-latex/unnamed-chunk-2-1.pdf}

\begin{Shaded}
\begin{Highlighting}[]
\CommentTok{\# cor(datos, use = "complete") \# Por defecto 8 decimales...}
\FunctionTok{print}\NormalTok{(}\FunctionTok{cor}\NormalTok{(datos, }\AttributeTok{use =} \StringTok{"complete"}\NormalTok{), }\AttributeTok{digits =} \DecValTok{2}\NormalTok{)}
\end{Highlighting}
\end{Shaded}

\begin{verbatim}
##          velocida precio flexprec imgfabri servconj imgfvent calidadp fidelida
## velocida    1.000 -0.354    0.519    0.049    0.609    0.081   -0.490    0.674
## precio     -0.354  1.000   -0.486    0.272    0.511    0.189    0.468    0.077
## flexprec    0.519 -0.486    1.000   -0.115    0.075   -0.038   -0.445    0.578
## imgfabri    0.049  0.272   -0.115    1.000    0.298    0.790    0.199    0.224
## servconj    0.609  0.511    0.075    0.298    1.000    0.246   -0.062    0.698
## imgfvent    0.081  0.189   -0.038    0.790    0.246    1.000    0.181    0.267
## calidadp   -0.490  0.468   -0.445    0.199   -0.062    0.181    1.000   -0.204
## fidelida    0.674  0.077    0.578    0.224    0.698    0.267   -0.204    1.000
\end{verbatim}

\hypertarget{ajuste-funciuxf3n-lm}{%
\section{\texorpdfstring{Ajuste: función \texttt{lm}}{Ajuste: función lm}}\label{ajuste-funciuxf3n-lm}}

Para el ajuste (estimación de los parámetros) de un modelo lineal a un conjunto de datos (por mínimos cuadrados) se emplea la función \texttt{lm}:

\begin{Shaded}
\begin{Highlighting}[]
\NormalTok{ajuste }\OtherTok{\textless{}{-}} \FunctionTok{lm}\NormalTok{(formula, datos, seleccion, pesos, na.action)}
\end{Highlighting}
\end{Shaded}

\begin{itemize}
\tightlist
\item
  \texttt{formula} fórmula que especifica el modelo.
\item
  \texttt{datos} data.frame opcional con las variables de la formula.
\item
  \texttt{seleccion} especificación opcional de un subconjunto de observaciones.
\item
  \texttt{pesos} vector opcional de pesos (WLS).
\item
  \texttt{na.action} opción para manejar los datos faltantes (\texttt{na.omit}).
\end{itemize}

\begin{Shaded}
\begin{Highlighting}[]
\NormalTok{modelo }\OtherTok{\textless{}{-}} \FunctionTok{lm}\NormalTok{(fidelida }\SpecialCharTok{\textasciitilde{}}\NormalTok{ servconj, datos)}
\NormalTok{modelo}
\end{Highlighting}
\end{Shaded}

\begin{verbatim}
## 
## Call:
## lm(formula = fidelida ~ servconj, data = datos)
## 
## Coefficients:
## (Intercept)     servconj  
##       21.98         8.30
\end{verbatim}

Al imprimir el ajuste resultante se muestra un pequeño resumen del ajuste (aunque el objeto que contiene los resultados es una lista).

Para obtener un resumen más completo se puede utilizar la función \texttt{summary()}.

\begin{Shaded}
\begin{Highlighting}[]
\FunctionTok{summary}\NormalTok{(modelo)}
\end{Highlighting}
\end{Shaded}

\begin{verbatim}
## 
## Call:
## lm(formula = fidelida ~ servconj, data = datos)
## 
## Residuals:
##      Min       1Q   Median       3Q      Max 
## -14.1956  -4.0655   0.2944   4.5945  11.9744 
## 
## Coefficients:
##             Estimate Std. Error t value Pr(>|t|)    
## (Intercept)  21.9754     2.6086   8.424 3.34e-13 ***
## servconj      8.3000     0.8645   9.601 9.76e-16 ***
## ---
## Signif. codes:  0 '***' 0.001 '**' 0.01 '*' 0.05 '.' 0.1 ' ' 1
## 
## Residual standard error: 6.432 on 97 degrees of freedom
##   (1 observation deleted due to missingness)
## Multiple R-squared:  0.4872, Adjusted R-squared:  0.482 
## F-statistic: 92.17 on 1 and 97 DF,  p-value: 9.765e-16
\end{verbatim}

\begin{Shaded}
\begin{Highlighting}[]
\FunctionTok{plot}\NormalTok{(fidelida }\SpecialCharTok{\textasciitilde{}}\NormalTok{ servconj, datos)}
\FunctionTok{abline}\NormalTok{(modelo)}
\end{Highlighting}
\end{Shaded}

\includegraphics{08-ModelosLineales_files/figure-latex/unnamed-chunk-7-1.pdf}

\hypertarget{extracciuxf3n-de-informaciuxf3n}{%
\subsection{Extracción de información}\label{extracciuxf3n-de-informaciuxf3n}}

Para la extracción de información se pueden acceder a los componentes del modelo ajustado o emplear funciones (genéricas). Algunas de las más utilizadas son las siguientes:

\begin{longtable}[]{@{}
  >{\raggedright\arraybackslash}p{(\columnwidth - 2\tabcolsep) * \real{0.12}}
  >{\raggedright\arraybackslash}p{(\columnwidth - 2\tabcolsep) * \real{0.88}}@{}}
\toprule
Función & Descripción \\
\midrule
\endhead
\texttt{fitted} & valores ajustados \\
\texttt{coef} & coeficientes estimados (y errores estándar) \\
\texttt{confint} & intervalos de confianza para los coeficientes \\
\texttt{residuals} & residuos \\
\texttt{plot} & gráficos de diagnóstico \\
\texttt{termplot} & gráfico de efectos parciales \\
\texttt{anova} & calcula tablas de análisis de varianza (también permite comparar modelos) \\
\texttt{predict} & calcula predicciones para nuevos datos \\
\bottomrule
\end{longtable}

Ejemplo:

\begin{Shaded}
\begin{Highlighting}[]
\NormalTok{modelo2 }\OtherTok{\textless{}{-}} \FunctionTok{lm}\NormalTok{(fidelida }\SpecialCharTok{\textasciitilde{}}\NormalTok{ servconj }\SpecialCharTok{+}\NormalTok{ flexprec, }\AttributeTok{data =}\NormalTok{ hatco)}
\FunctionTok{summary}\NormalTok{(modelo2)}
\end{Highlighting}
\end{Shaded}

\begin{verbatim}
## 
## Call:
## lm(formula = fidelida ~ servconj + flexprec, data = hatco)
## 
## Residuals:
##      Min       1Q   Median       3Q      Max 
## -10.2549  -2.2850   0.3411   3.3260   7.0853 
## 
## Coefficients:
##             Estimate Std. Error t value Pr(>|t|)    
## (Intercept)  -3.4617     2.9734  -1.164    0.247    
## servconj      7.8287     0.5897  13.276   <2e-16 ***
## flexprec      3.4017     0.3191  10.661   <2e-16 ***
## ---
## Signif. codes:  0 '***' 0.001 '**' 0.01 '*' 0.05 '.' 0.1 ' ' 1
## 
## Residual standard error: 4.375 on 96 degrees of freedom
##   (1 observation deleted due to missingness)
## Multiple R-squared:  0.7652, Adjusted R-squared:  0.7603 
## F-statistic: 156.4 on 2 and 96 DF,  p-value: < 2.2e-16
\end{verbatim}

\begin{Shaded}
\begin{Highlighting}[]
\FunctionTok{confint}\NormalTok{(modelo2)}
\end{Highlighting}
\end{Shaded}

\begin{verbatim}
##                 2.5 %   97.5 %
## (Intercept) -9.363813 2.440344
## servconj     6.658219 8.999274
## flexprec     2.768333 4.035030
\end{verbatim}

\begin{Shaded}
\begin{Highlighting}[]
\FunctionTok{anova}\NormalTok{(modelo2)}
\end{Highlighting}
\end{Shaded}

\begin{verbatim}
## Analysis of Variance Table
## 
## Response: fidelida
##           Df Sum Sq Mean Sq F value    Pr(>F)    
## servconj   1 3813.6  3813.6  199.23 < 2.2e-16 ***
## flexprec   1 2175.6  2175.6  113.66 < 2.2e-16 ***
## Residuals 96 1837.6    19.1                      
## ---
## Signif. codes:  0 '***' 0.001 '**' 0.01 '*' 0.05 '.' 0.1 ' ' 1
\end{verbatim}

\begin{Shaded}
\begin{Highlighting}[]
\CommentTok{\# anova(modelo2, modelo)}
\CommentTok{\# termplot(modelo2, partial.resid = TRUE)}
\end{Highlighting}
\end{Shaded}

Muchas de estas funciones genéricas son válidas para otros tipos de
modelos (glm, \ldots).

Algunas funciones como \texttt{summary()} devuelven información adicional:

\begin{Shaded}
\begin{Highlighting}[]
\NormalTok{res }\OtherTok{\textless{}{-}} \FunctionTok{summary}\NormalTok{(modelo2)}
\FunctionTok{names}\NormalTok{(res)}
\end{Highlighting}
\end{Shaded}

\begin{verbatim}
##  [1] "call"          "terms"         "residuals"     "coefficients" 
##  [5] "aliased"       "sigma"         "df"            "r.squared"    
##  [9] "adj.r.squared" "fstatistic"    "cov.unscaled"  "na.action"
\end{verbatim}

\begin{Shaded}
\begin{Highlighting}[]
\NormalTok{res}\SpecialCharTok{$}\NormalTok{sigma}
\end{Highlighting}
\end{Shaded}

\begin{verbatim}
## [1] 4.375074
\end{verbatim}

\begin{Shaded}
\begin{Highlighting}[]
\NormalTok{res}\SpecialCharTok{$}\NormalTok{adj.r.squared}
\end{Highlighting}
\end{Shaded}

\begin{verbatim}
## [1] 0.7603292
\end{verbatim}

\hypertarget{predicciuxf3n}{%
\section{Predicción}\label{predicciuxf3n}}

Para calcular predicciones (estimaciones de la media condicionada) se puede emplear la función \texttt{predict()} (ejecutar \texttt{help(predict.lm)} para ver todas las opciones disponibles).
Por defecto obtiene las predicciones correspondientes a las observaciones (\texttt{modelo\$fitted.values}). Para otros casos hay que emplear el argumento \texttt{newdata}:

\begin{itemize}
\tightlist
\item
  data.frame con los valores de (todas) las covariables, sus nombres
  deben coincidir con los originales.
\end{itemize}

Ejemplo:

\begin{Shaded}
\begin{Highlighting}[]
\NormalTok{valores }\OtherTok{\textless{}{-}} \DecValTok{0}\SpecialCharTok{:}\DecValTok{5}
\NormalTok{pred }\OtherTok{\textless{}{-}} \FunctionTok{predict}\NormalTok{(modelo, }\AttributeTok{newdata =} \FunctionTok{data.frame}\NormalTok{(}\AttributeTok{servconj =}\NormalTok{ valores))}
\NormalTok{pred}
\end{Highlighting}
\end{Shaded}

\begin{verbatim}
##        1        2        3        4        5        6 
## 21.97544 30.27548 38.57552 46.87556 55.17560 63.47564
\end{verbatim}

\begin{Shaded}
\begin{Highlighting}[]
\FunctionTok{plot}\NormalTok{(fidelida }\SpecialCharTok{\textasciitilde{}}\NormalTok{ servconj, datos)}
\FunctionTok{lines}\NormalTok{(valores, pred)}
\end{Highlighting}
\end{Shaded}

\includegraphics{08-ModelosLineales_files/figure-latex/unnamed-chunk-10-1.pdf}

Esta función también permite obtener intervalos de confianza y de predicción:

\begin{Shaded}
\begin{Highlighting}[]
\NormalTok{valores }\OtherTok{\textless{}{-}} \FunctionTok{seq}\NormalTok{(}\DecValTok{0}\NormalTok{, }\DecValTok{5}\NormalTok{, }\AttributeTok{len =} \DecValTok{100}\NormalTok{)}
\NormalTok{newdata }\OtherTok{\textless{}{-}} \FunctionTok{data.frame}\NormalTok{(}\AttributeTok{servconj =}\NormalTok{ valores)}
\NormalTok{pred }\OtherTok{\textless{}{-}} \FunctionTok{predict}\NormalTok{(modelo, }\AttributeTok{newdata =}\NormalTok{ newdata, }\AttributeTok{interval =} \FunctionTok{c}\NormalTok{(}\StringTok{"confidence"}\NormalTok{))}
\FunctionTok{head}\NormalTok{(pred)}
\end{Highlighting}
\end{Shaded}

\begin{verbatim}
##        fit      lwr      upr
## 1 21.97544 16.79816 27.15272
## 2 22.39463 17.30126 27.48800
## 3 22.81383 17.80427 27.82338
## 4 23.23302 18.30718 28.15886
## 5 23.65221 18.80999 28.49444
## 6 24.07141 19.31269 28.83013
\end{verbatim}

\begin{Shaded}
\begin{Highlighting}[]
\FunctionTok{plot}\NormalTok{(fidelida }\SpecialCharTok{\textasciitilde{}}\NormalTok{ servconj, datos)}
\FunctionTok{matlines}\NormalTok{(valores, pred, }\AttributeTok{lty =} \FunctionTok{c}\NormalTok{(}\DecValTok{1}\NormalTok{, }\DecValTok{2}\NormalTok{, }\DecValTok{2}\NormalTok{), }\AttributeTok{col =} \DecValTok{1}\NormalTok{)}
\NormalTok{pred2 }\OtherTok{\textless{}{-}} \FunctionTok{predict}\NormalTok{(modelo, }\AttributeTok{newdata =}\NormalTok{ newdata, }\AttributeTok{interval =} \FunctionTok{c}\NormalTok{(}\StringTok{"prediction"}\NormalTok{))}
\FunctionTok{matlines}\NormalTok{(valores, pred2[, }\SpecialCharTok{{-}}\DecValTok{1}\NormalTok{], }\AttributeTok{lty =} \DecValTok{3}\NormalTok{, }\AttributeTok{col =} \DecValTok{1}\NormalTok{)}
\FunctionTok{legend}\NormalTok{(}\StringTok{"topleft"}\NormalTok{, }\FunctionTok{c}\NormalTok{(}\StringTok{"Ajuste"}\NormalTok{, }\StringTok{"Int. confianza"}\NormalTok{, }\StringTok{"Int. predicción"}\NormalTok{), }\AttributeTok{lty =} \FunctionTok{c}\NormalTok{(}\DecValTok{1}\NormalTok{, }\DecValTok{2}\NormalTok{, }\DecValTok{3}\NormalTok{))}
\end{Highlighting}
\end{Shaded}

\includegraphics{08-ModelosLineales_files/figure-latex/unnamed-chunk-11-1.pdf}

\hypertarget{selecciuxf3n-de-variables-explicativas}{%
\section{Selección de variables explicativas}\label{selecciuxf3n-de-variables-explicativas}}

Cuando se dispone de un conjunto grande de posibles variables explicativas
suele ser especialmente importante determinar cuales de estas deberían ser
incluidas en el modelo de regresión.
Si alguna de las variables no contiene información relevante sobre la respuesta
no se debería incluir (se simplificaría la interpretación del modelo, aumentaría
la precisión de la estimación y se evitarían problemas como la multicolinealidad).
Se trataría entonces de conseguir un buen ajuste con el menor número de variables explicativas posible.

Para actualizar un modelo (p.e. eliminando o añadiendo variables) se puede emplear la función \texttt{update}:

\begin{Shaded}
\begin{Highlighting}[]
\NormalTok{modelo.completo }\OtherTok{\textless{}{-}} \FunctionTok{lm}\NormalTok{(fidelida }\SpecialCharTok{\textasciitilde{}}\NormalTok{ . , }\AttributeTok{data =}\NormalTok{ datos)}
\FunctionTok{summary}\NormalTok{(modelo.completo)}
\end{Highlighting}
\end{Shaded}

\begin{verbatim}
## 
## Call:
## lm(formula = fidelida ~ ., data = datos)
## 
## Residuals:
##      Min       1Q   Median       3Q      Max 
## -13.3351  -2.0733   0.5224   2.9218   6.7106 
## 
## Coefficients:
##             Estimate Std. Error t value Pr(>|t|)    
## (Intercept)  -9.5935     4.8213  -1.990   0.0496 *  
## velocida     -0.6023     1.9590  -0.307   0.7592    
## precio       -1.0771     2.0283  -0.531   0.5967    
## flexprec      3.4616     0.3997   8.660 1.62e-13 ***
## imgfabri     -0.1735     0.6472  -0.268   0.7892    
## servconj      9.0919     3.8023   2.391   0.0189 *  
## imgfvent      1.5596     0.9221   1.691   0.0942 .  
## calidadp      0.4874     0.3451   1.412   0.1613    
## ---
## Signif. codes:  0 '***' 0.001 '**' 0.01 '*' 0.05 '.' 0.1 ' ' 1
## 
## Residual standard error: 4.281 on 91 degrees of freedom
##   (1 observation deleted due to missingness)
## Multiple R-squared:  0.7869, Adjusted R-squared:  0.7705 
## F-statistic:    48 on 7 and 91 DF,  p-value: < 2.2e-16
\end{verbatim}

\begin{Shaded}
\begin{Highlighting}[]
\NormalTok{modelo.reducido }\OtherTok{\textless{}{-}} \FunctionTok{update}\NormalTok{(modelo.completo, . }\SpecialCharTok{\textasciitilde{}}\NormalTok{ . }\SpecialCharTok{{-}}\NormalTok{ imgfabri)}
\FunctionTok{summary}\NormalTok{(modelo.reducido)}
\end{Highlighting}
\end{Shaded}

\begin{verbatim}
## 
## Call:
## lm(formula = fidelida ~ velocida + precio + flexprec + servconj + 
##     imgfvent + calidadp, data = datos)
## 
## Residuals:
##      Min       1Q   Median       3Q      Max 
## -13.2195  -2.0022   0.4724   2.9514   6.8328 
## 
## Coefficients:
##             Estimate Std. Error t value Pr(>|t|)    
## (Intercept)  -9.9900     4.5656  -2.188   0.0312 *  
## velocida     -0.5207     1.9254  -0.270   0.7874    
## precio       -1.0017     1.9986  -0.501   0.6174    
## flexprec      3.4709     0.3962   8.761 9.23e-14 ***
## servconj      8.9111     3.7230   2.394   0.0187 *  
## imgfvent      1.3699     0.5883   2.329   0.0221 *  
## calidadp      0.4844     0.3432   1.411   0.1615    
## ---
## Signif. codes:  0 '***' 0.001 '**' 0.01 '*' 0.05 '.' 0.1 ' ' 1
## 
## Residual standard error: 4.26 on 92 degrees of freedom
##   (1 observation deleted due to missingness)
## Multiple R-squared:  0.7867, Adjusted R-squared:  0.7728 
## F-statistic: 56.56 on 6 and 92 DF,  p-value: < 2.2e-16
\end{verbatim}

Para obtener el modelo ``óptimo'' lo ideal sería evaluar todos los modelos posibles.

\hypertarget{buxfasqueda-exhaustiva}{%
\subsection{Búsqueda exhaustiva}\label{buxfasqueda-exhaustiva}}

La función \texttt{regsubsets} del paquete \texttt{leaps} permite seleccionar los mejores modelos
fijando el número de variables explicativas.
Por defecto, evalúa todos los modelos posibles con un determinado número de
parámetros (variando desde 1 hasta un máximo de \texttt{nvmax=8})
y selecciona el mejor (\texttt{nbest=1}).

\begin{Shaded}
\begin{Highlighting}[]
\FunctionTok{library}\NormalTok{(leaps)}
\NormalTok{res }\OtherTok{\textless{}{-}} \FunctionTok{regsubsets}\NormalTok{(fidelida }\SpecialCharTok{\textasciitilde{}}\NormalTok{ . , }\AttributeTok{data =}\NormalTok{ datos)}
\FunctionTok{summary}\NormalTok{(res)}
\end{Highlighting}
\end{Shaded}

\begin{verbatim}
## Subset selection object
## Call: regsubsets.formula(fidelida ~ ., data = datos)
## 7 Variables  (and intercept)
##          Forced in Forced out
## velocida     FALSE      FALSE
## precio       FALSE      FALSE
## flexprec     FALSE      FALSE
## imgfabri     FALSE      FALSE
## servconj     FALSE      FALSE
## imgfvent     FALSE      FALSE
## calidadp     FALSE      FALSE
## 1 subsets of each size up to 7
## Selection Algorithm: exhaustive
##          velocida precio flexprec imgfabri servconj imgfvent calidadp
## 1  ( 1 ) " "      " "    " "      " "      "*"      " "      " "     
## 2  ( 1 ) " "      " "    "*"      " "      "*"      " "      " "     
## 3  ( 1 ) " "      " "    "*"      " "      "*"      "*"      " "     
## 4  ( 1 ) " "      " "    "*"      " "      "*"      "*"      "*"     
## 5  ( 1 ) " "      "*"    "*"      " "      "*"      "*"      "*"     
## 6  ( 1 ) "*"      "*"    "*"      " "      "*"      "*"      "*"     
## 7  ( 1 ) "*"      "*"    "*"      "*"      "*"      "*"      "*"
\end{verbatim}

\begin{Shaded}
\begin{Highlighting}[]
\CommentTok{\# names(summary(res))}
\end{Highlighting}
\end{Shaded}

Al representar el resultado se obtiene un gráfico con los mejores modelos ordenados
según el criterio determinado por el argumento \texttt{scale\ =\ c("bic",\ "Cp",\ "adjr2",\ "r2")}.
Por ejemplo, en este caso, empleando el coeficiente de determinación ajustado, obtendríamos:

\begin{Shaded}
\begin{Highlighting}[]
\FunctionTok{plot}\NormalTok{(res, }\AttributeTok{scale =} \StringTok{"adjr2"}\NormalTok{)}
\end{Highlighting}
\end{Shaded}

\includegraphics{08-ModelosLineales_files/figure-latex/unnamed-chunk-14-1.pdf}

En este caso (considerando que una mejora del 2\% no es significativa), el modelo resultante sería:

\begin{Shaded}
\begin{Highlighting}[]
\FunctionTok{lm}\NormalTok{(fidelida }\SpecialCharTok{\textasciitilde{}}\NormalTok{ servconj }\SpecialCharTok{+}\NormalTok{ flexprec, }\AttributeTok{data =}\NormalTok{ hatco)}
\end{Highlighting}
\end{Shaded}

\begin{verbatim}
## 
## Call:
## lm(formula = fidelida ~ servconj + flexprec, data = hatco)
## 
## Coefficients:
## (Intercept)     servconj     flexprec  
##      -3.462        7.829        3.402
\end{verbatim}

\textbf{Notas}:

\begin{itemize}
\item
  Si se emplea alguno de los criterios habituales, el mejor modelo con un determinado
  número de variables no depende del criterio empleado.
  Pero estos criterios pueden diferir al comparar modelos con distinto número de
  variables explicativas.
\item
  Si el número de variables explicativas es grande, en lugar de emplear una
  búsqueda exhaustiva se puede emplear un criterio por pasos, mediante el argumento
  \texttt{method\ =\ c("backward",\ "forward",\ "seqrep")}, pero puede ser recomendable
  emplear el paquete \texttt{MASS} para obtener directamente el modelo final.
\end{itemize}

\hypertarget{selecciuxf3n-por-pasos}{%
\subsection{Selección por pasos}\label{selecciuxf3n-por-pasos}}

Si el número de variables es grande (no sería práctico evaluar todas las posibilidades)
se suele utilizar alguno (o varios) de los siguientes métodos:

\begin{itemize}
\item
  \textbf{Selección progresiva} (forward): Se parte de una situación en la
  que no hay ninguna variable y en cada paso se incluye una aplicando
  un \textbf{criterio de entrada} (hasta que ninguna de las restantes lo
  verifican).
\item
  \textbf{Eliminación progresiva} (backward): Se parte del modelo con todas
  las variables y en cada paso se elimina una aplicando un \textbf{criterio
  de salida} (hasta que ninguna de las incluidas lo verifican).
\item
  \textbf{Regresión paso a paso} (stepwise): El más utilizado, se combina
  un criterio de entrada y uno de salida. Normalmente se parte sin
  ninguna variable y \textbf{en cada paso puede haber una inclusión y una
  exclusión} (forward/backward).
\end{itemize}

La función \texttt{stepAIC} del paquete \texttt{MASS} permite seleccionar el modelo por pasos,
hacia delante o hacia atrás según criterio AIC o BIC (también esta disponible una función \texttt{step} del paquete base \texttt{stats} con menos opciones).
La función \texttt{stepwise} del paquete \texttt{RcmdrMisc} es una interfaz de \texttt{stepAIC}
que facilita su uso:

\begin{Shaded}
\begin{Highlighting}[]
\FunctionTok{library}\NormalTok{(MASS)}
\FunctionTok{library}\NormalTok{(RcmdrMisc)}
\NormalTok{modelo }\OtherTok{\textless{}{-}} \FunctionTok{stepwise}\NormalTok{(modelo.completo, }\AttributeTok{direction =} \StringTok{"forward/backward"}\NormalTok{, }\AttributeTok{criterion =} \StringTok{"BIC"}\NormalTok{)}
\end{Highlighting}
\end{Shaded}

\begin{verbatim}
## 
## Direction:  forward/backward
## Criterion:  BIC 
## 
## Start:  AIC=437.24
## fidelida ~ 1
## 
##            Df Sum of Sq    RSS    AIC
## + servconj  1    3813.6 4013.2 375.71
## + velocida  1    3558.5 4268.2 381.81
## + flexprec  1    2615.5 5211.3 401.57
## + imgfvent  1     556.9 7269.9 434.53
## + imgfabri  1     394.2 7432.5 436.72
## <none>                  7826.8 437.24
## + calidadp  1     325.8 7501.0 437.63
## + precio    1      46.2 7780.6 441.25
## 
## Step:  AIC=375.71
## fidelida ~ servconj
## 
##            Df Sum of Sq    RSS    AIC
## + flexprec  1    2175.6 1837.6 302.97
## + precio    1     831.5 3181.7 357.32
## + velocida  1     772.3 3240.9 359.15
## + calidadp  1     203.8 3809.4 375.15
## <none>                  4013.2 375.71
## + imgfvent  1      74.8 3938.4 378.44
## + imgfabri  1       2.3 4010.9 380.25
## - servconj  1    3813.6 7826.8 437.24
## 
## Step:  AIC=302.97
## fidelida ~ servconj + flexprec
## 
##            Df Sum of Sq    RSS    AIC
## + imgfvent  1     129.8 1707.7 300.31
## <none>                  1837.6 302.97
## + imgfabri  1      69.3 1768.3 303.76
## + calidadp  1      50.7 1786.9 304.80
## + precio    1       0.2 1837.4 307.56
## + velocida  1       0.0 1837.5 307.57
## - flexprec  1    2175.6 4013.2 375.71
## - servconj  1    3373.7 5211.3 401.57
## 
## Step:  AIC=300.31
## fidelida ~ servconj + flexprec + imgfvent
## 
##            Df Sum of Sq    RSS    AIC
## <none>                  1707.7 300.31
## - imgfvent  1    129.82 1837.6 302.97
## + calidadp  1     24.70 1683.0 303.47
## + precio    1      0.96 1706.8 304.85
## + imgfabri  1      0.66 1707.1 304.87
## + velocida  1      0.41 1707.3 304.88
## - flexprec  1   2230.67 3938.4 378.44
## - servconj  1   2850.14 4557.9 392.91
\end{verbatim}

\begin{Shaded}
\begin{Highlighting}[]
\FunctionTok{summary}\NormalTok{(modelo)}
\end{Highlighting}
\end{Shaded}

\begin{verbatim}
## 
## Call:
## lm(formula = fidelida ~ servconj + flexprec + imgfvent, data = datos)
## 
## Residuals:
##      Min       1Q   Median       3Q      Max 
## -12.9301  -2.1395   0.0695   2.9632   7.4286 
## 
## Coefficients:
##             Estimate Std. Error t value Pr(>|t|)    
## (Intercept)  -6.7761     3.1343  -2.162   0.0331 *  
## servconj      7.4320     0.5902  12.592   <2e-16 ***
## flexprec      3.4503     0.3097  11.140   <2e-16 ***
## imgfvent      1.5369     0.5719   2.687   0.0085 ** 
## ---
## Signif. codes:  0 '***' 0.001 '**' 0.01 '*' 0.05 '.' 0.1 ' ' 1
## 
## Residual standard error: 4.24 on 95 degrees of freedom
##   (1 observation deleted due to missingness)
## Multiple R-squared:  0.7818, Adjusted R-squared:  0.7749 
## F-statistic: 113.5 on 3 and 95 DF,  p-value: < 2.2e-16
\end{verbatim}

Los métodos disponibles son \texttt{"backward/forward"}, \texttt{"forward/backward"}, \texttt{"backward"} y \texttt{"forward"}.

Cuando el número de variables explicativas es muy grande (o si el tamaño de la muestra es pequeño en comparación) pueden aparecer problemas al emplear los métodos anteriores (incluso pueden no ser aplicables). Una alternativa son los métodos de regularización (Ridge regression, Lasso) disponibles en el paquete \texttt{glmnet}.

\hypertarget{regresiuxf3n-con-variables-categuxf3ricas}{%
\section{Regresión con variables categóricas}\label{regresiuxf3n-con-variables-categuxf3ricas}}

La función \texttt{lm()} admite también variables categóricas (factores), lo que equivaldría a modelos de análisis de la varianza o de la covarianza.

Como ejemplo, en el resto del tema emplearemos los datos de empleados:

\begin{Shaded}
\begin{Highlighting}[]
\FunctionTok{load}\NormalTok{(}\StringTok{"datos/empleados.RData"}\NormalTok{)}
\NormalTok{datos }\OtherTok{\textless{}{-}} \FunctionTok{with}\NormalTok{(empleados, }\FunctionTok{data.frame}\NormalTok{(}\AttributeTok{lnsal =} \FunctionTok{log}\NormalTok{(salario), }\AttributeTok{lnsalini =} \FunctionTok{log}\NormalTok{(salini), catlab, sexo))}
\end{Highlighting}
\end{Shaded}

Al incluir variables categóricas la función \texttt{lm()} genera las variables indicadoras (variables dummy) que sean necesarias.
Por ejemplo, la función \texttt{model.matrix()} construye la denominada matriz de diseño \(X\) de un modelo lineal:
\[\mathbf{Y}=X\mathbf{\beta}+\mathbf{\varepsilon}\]
En el caso de una variable categórica, por defecto se toma la primera categoría como referencia y se generan variables indicadoras del resto de categorías:

\begin{Shaded}
\begin{Highlighting}[]
\NormalTok{X }\OtherTok{\textless{}{-}} \FunctionTok{model.matrix}\NormalTok{(lnsal }\SpecialCharTok{\textasciitilde{}}\NormalTok{ catlab, datos)}
\FunctionTok{head}\NormalTok{(X)}
\end{Highlighting}
\end{Shaded}

\begin{verbatim}
##   (Intercept) catlabSeguridad catlabDirectivo
## 1           1               0               1
## 2           1               0               0
## 3           1               0               0
## 4           1               0               0
## 5           1               0               0
## 6           1               0               0
\end{verbatim}

En el correspondiente ajuste (análisis de la varianza de un factor):

\begin{Shaded}
\begin{Highlighting}[]
\NormalTok{modelo }\OtherTok{\textless{}{-}} \FunctionTok{lm}\NormalTok{(lnsal }\SpecialCharTok{\textasciitilde{}}\NormalTok{ catlab, datos)}
\FunctionTok{summary}\NormalTok{(modelo)}
\end{Highlighting}
\end{Shaded}

\begin{verbatim}
## 
## Call:
## lm(formula = lnsal ~ catlab, data = datos)
## 
## Residuals:
##      Min       1Q   Median       3Q      Max 
## -0.58352 -0.15983 -0.01012  0.13277  1.08725 
## 
## Coefficients:
##                 Estimate Std. Error t value Pr(>|t|)    
## (Intercept)     10.20254    0.01280 797.245  < 2e-16 ***
## catlabSeguridad  0.13492    0.04864   2.774  0.00576 ** 
## catlabDirectivo  0.82709    0.02952  28.017  < 2e-16 ***
## ---
## Signif. codes:  0 '***' 0.001 '**' 0.01 '*' 0.05 '.' 0.1 ' ' 1
## 
## Residual standard error: 0.2438 on 471 degrees of freedom
## Multiple R-squared:  0.625,  Adjusted R-squared:  0.6234 
## F-statistic: 392.6 on 2 and 471 DF,  p-value: < 2.2e-16
\end{verbatim}

el nivel de referencia no tiene asociado un coeficiente (su efecto se corresponde con \texttt{(Intercept)}). Los coeficientes del resto de niveles miden el cambio que se produce en la media al cambiar desde la categoría de referencia (diferencias de efectos respecto al nivel de referencia).

Para contrastar el efecto de los factores, es preferible emplear la función \texttt{anova}:

\begin{Shaded}
\begin{Highlighting}[]
\NormalTok{modelo }\OtherTok{\textless{}{-}} \FunctionTok{lm}\NormalTok{(lnsal }\SpecialCharTok{\textasciitilde{}}\NormalTok{ catlab }\SpecialCharTok{+}\NormalTok{ sexo, datos)}
\FunctionTok{anova}\NormalTok{(modelo)}
\end{Highlighting}
\end{Shaded}

\begin{verbatim}
## Analysis of Variance Table
## 
## Response: lnsal
##            Df Sum Sq Mean Sq F value    Pr(>F)    
## catlab      2 46.674 23.3372  489.59 < 2.2e-16 ***
## sexo        1  5.596  5.5965  117.41 < 2.2e-16 ***
## Residuals 470 22.404  0.0477                      
## ---
## Signif. codes:  0 '***' 0.001 '**' 0.01 '*' 0.05 '.' 0.1 ' ' 1
\end{verbatim}

\textbf{Notas}:

\begin{itemize}
\item
  Para centrarse en las efectos de los factores, se puede emplear la función
  \texttt{aov} (analysis of variance; ver también \texttt{model.tables()} y \texttt{TukeyHSD()}). Esta
  función llama internamente a \texttt{lm()} (utilizando la misma parametrización).
\item
  Para utilizar distintas parametrizaciones de los efectos se puede emplear
  el argumento \texttt{contrasts\ =\ c("contr.treatment",\ "contr.poly")}
  (ver \texttt{help(contrasts)}).
\end{itemize}

\hypertarget{interacciones}{%
\section{Interacciones}\label{interacciones}}

Al emplear el operador \texttt{+} se considera que los efectos de las covariables son aditivos (independientes):

\begin{Shaded}
\begin{Highlighting}[]
\NormalTok{modelo }\OtherTok{\textless{}{-}} \FunctionTok{lm}\NormalTok{(lnsal }\SpecialCharTok{\textasciitilde{}}\NormalTok{ lnsalini }\SpecialCharTok{+}\NormalTok{ catlab, datos)}
\FunctionTok{anova}\NormalTok{(modelo)}
\end{Highlighting}
\end{Shaded}

\begin{verbatim}
## Analysis of Variance Table
## 
## Response: lnsal
##            Df Sum Sq Mean Sq  F value    Pr(>F)    
## lnsalini    1 58.668  58.668 1901.993 < 2.2e-16 ***
## catlab      2  1.509   0.755   24.465 7.808e-11 ***
## Residuals 470 14.497   0.031                       
## ---
## Signif. codes:  0 '***' 0.001 '**' 0.01 '*' 0.05 '.' 0.1 ' ' 1
\end{verbatim}

\begin{Shaded}
\begin{Highlighting}[]
\FunctionTok{plot}\NormalTok{(lnsal }\SpecialCharTok{\textasciitilde{}}\NormalTok{ lnsalini, }\AttributeTok{data =}\NormalTok{ datos, }\AttributeTok{pch =} \FunctionTok{as.numeric}\NormalTok{(catlab), }\AttributeTok{col =} \StringTok{\textquotesingle{}darkgray\textquotesingle{}}\NormalTok{)}
\NormalTok{parest }\OtherTok{\textless{}{-}} \FunctionTok{coef}\NormalTok{(modelo)}
\FunctionTok{abline}\NormalTok{(}\AttributeTok{a =}\NormalTok{ parest[}\DecValTok{1}\NormalTok{], }\AttributeTok{b =}\NormalTok{ parest[}\DecValTok{2}\NormalTok{], }\AttributeTok{lty =} \DecValTok{1}\NormalTok{)}
\FunctionTok{abline}\NormalTok{(}\AttributeTok{a =}\NormalTok{ parest[}\DecValTok{1}\NormalTok{] }\SpecialCharTok{+}\NormalTok{ parest[}\DecValTok{3}\NormalTok{], }\AttributeTok{b =}\NormalTok{ parest[}\DecValTok{2}\NormalTok{], }\AttributeTok{lty =} \DecValTok{2}\NormalTok{)}
\FunctionTok{abline}\NormalTok{(}\AttributeTok{a =}\NormalTok{ parest[}\DecValTok{1}\NormalTok{] }\SpecialCharTok{+}\NormalTok{ parest[}\DecValTok{4}\NormalTok{], }\AttributeTok{b =}\NormalTok{ parest[}\DecValTok{2}\NormalTok{], }\AttributeTok{lty =} \DecValTok{3}\NormalTok{)}
\FunctionTok{legend}\NormalTok{(}\StringTok{"bottomright"}\NormalTok{, }\FunctionTok{levels}\NormalTok{(datos}\SpecialCharTok{$}\NormalTok{catlab), }\AttributeTok{pch =} \DecValTok{1}\SpecialCharTok{:}\DecValTok{3}\NormalTok{, }\AttributeTok{lty =} \DecValTok{1}\SpecialCharTok{:}\DecValTok{3}\NormalTok{)}
\end{Highlighting}
\end{Shaded}

\includegraphics{08-ModelosLineales_files/figure-latex/unnamed-chunk-21-1.pdf}

Para especificar que el efecto de una covariable depende de otra (interacción),
se pueden emplear los operadores \texttt{*} ó \texttt{:}.

\begin{Shaded}
\begin{Highlighting}[]
\NormalTok{modelo2 }\OtherTok{\textless{}{-}} \FunctionTok{lm}\NormalTok{(lnsal }\SpecialCharTok{\textasciitilde{}}\NormalTok{ lnsalini}\SpecialCharTok{*}\NormalTok{catlab, datos)}
\FunctionTok{summary}\NormalTok{(modelo2)}
\end{Highlighting}
\end{Shaded}

\begin{verbatim}
## 
## Call:
## lm(formula = lnsal ~ lnsalini * catlab, data = datos)
## 
## Residuals:
##      Min       1Q   Median       3Q      Max 
## -0.37440 -0.11335 -0.00524  0.10459  0.97018 
## 
## Coefficients:
##                          Estimate Std. Error t value Pr(>|t|)    
## (Intercept)               1.66865    0.43820   3.808 0.000159 ***
## lnsalini                  0.89512    0.04595  19.479  < 2e-16 ***
## catlabSeguridad           8.31808    3.01827   2.756 0.006081 ** 
## catlabDirectivo           3.01268    0.79509   3.789 0.000171 ***
## lnsalini:catlabSeguridad -0.85864    0.31392  -2.735 0.006470 ** 
## lnsalini:catlabDirectivo -0.27713    0.07924  -3.497 0.000515 ***
## ---
## Signif. codes:  0 '***' 0.001 '**' 0.01 '*' 0.05 '.' 0.1 ' ' 1
## 
## Residual standard error: 0.1727 on 468 degrees of freedom
## Multiple R-squared:  0.8131, Adjusted R-squared:  0.8111 
## F-statistic: 407.3 on 5 and 468 DF,  p-value: < 2.2e-16
\end{verbatim}

\begin{Shaded}
\begin{Highlighting}[]
\FunctionTok{anova}\NormalTok{(modelo2)}
\end{Highlighting}
\end{Shaded}

\begin{verbatim}
## Analysis of Variance Table
## 
## Response: lnsal
##                  Df Sum Sq Mean Sq   F value    Pr(>F)    
## lnsalini          1 58.668  58.668 1967.6294 < 2.2e-16 ***
## catlab            2  1.509   0.755   25.3090 3.658e-11 ***
## lnsalini:catlab   2  0.543   0.272    9.1097 0.0001315 ***
## Residuals       468 13.954   0.030                        
## ---
## Signif. codes:  0 '***' 0.001 '**' 0.01 '*' 0.05 '.' 0.1 ' ' 1
\end{verbatim}

En este caso las pendientes también varían dependiendo del nivel del factor:

\begin{Shaded}
\begin{Highlighting}[]
\FunctionTok{plot}\NormalTok{(lnsal }\SpecialCharTok{\textasciitilde{}}\NormalTok{ lnsalini, }\AttributeTok{data =}\NormalTok{ datos, }\AttributeTok{pch =} \FunctionTok{as.numeric}\NormalTok{(catlab), }\AttributeTok{col =} \StringTok{\textquotesingle{}darkgray\textquotesingle{}}\NormalTok{)}
\NormalTok{parest }\OtherTok{\textless{}{-}} \FunctionTok{coef}\NormalTok{(modelo2)}
\FunctionTok{abline}\NormalTok{(}\AttributeTok{a =}\NormalTok{ parest[}\DecValTok{1}\NormalTok{], }\AttributeTok{b =}\NormalTok{ parest[}\DecValTok{2}\NormalTok{], }\AttributeTok{lty =} \DecValTok{1}\NormalTok{)}
\FunctionTok{abline}\NormalTok{(}\AttributeTok{a =}\NormalTok{ parest[}\DecValTok{1}\NormalTok{] }\SpecialCharTok{+}\NormalTok{ parest[}\DecValTok{3}\NormalTok{], }\AttributeTok{b =}\NormalTok{ parest[}\DecValTok{2}\NormalTok{] }\SpecialCharTok{+}\NormalTok{ parest[}\DecValTok{5}\NormalTok{], }\AttributeTok{lty =} \DecValTok{2}\NormalTok{)}
\FunctionTok{abline}\NormalTok{(}\AttributeTok{a =}\NormalTok{ parest[}\DecValTok{1}\NormalTok{] }\SpecialCharTok{+}\NormalTok{ parest[}\DecValTok{4}\NormalTok{], }\AttributeTok{b =}\NormalTok{ parest[}\DecValTok{2}\NormalTok{] }\SpecialCharTok{+}\NormalTok{ parest[}\DecValTok{6}\NormalTok{], }\AttributeTok{lty =} \DecValTok{3}\NormalTok{)}
\FunctionTok{legend}\NormalTok{(}\StringTok{"bottomright"}\NormalTok{, }\FunctionTok{levels}\NormalTok{(datos}\SpecialCharTok{$}\NormalTok{catlab), }\AttributeTok{pch =} \DecValTok{1}\SpecialCharTok{:}\DecValTok{3}\NormalTok{, }\AttributeTok{lty =} \DecValTok{1}\SpecialCharTok{:}\DecValTok{3}\NormalTok{)}
\end{Highlighting}
\end{Shaded}

\includegraphics{08-ModelosLineales_files/figure-latex/unnamed-chunk-23-1.pdf}

Por ejemplo, empleando la fórmula \texttt{lnsal\ \textasciitilde{}\ lnsalini:catlab} se considerarían distintas pendientes pero el mismo término independiente.

\hypertarget{diagnosis-del-modelo}{%
\section{Diagnosis del modelo}\label{diagnosis-del-modelo}}

Las conclusiones obtenidas con este método se basan en las hipótesis básicas del modelo:

\begin{itemize}
\item
  Linealidad.
\item
  Normalidad (y homogeneidad).
\item
  Homocedasticidad.
\item
  Independencia.
\item
  Ninguna de las variables explicativas es combinación lineal de
  las demás.
\end{itemize}

Si alguna de estas hipótesis no es cierta, las conclusiones obtenidas pueden no ser
fiables, o incluso totalmente erróneas. En el caso de regresión múltiple es
de especial interés el fenómeno de la multicolinealidad (o colinearidad)
relacionado con la última de estas hipótesis.

En esta sección consideraremos como ejemplo el modelo:

\begin{Shaded}
\begin{Highlighting}[]
\NormalTok{modelo }\OtherTok{\textless{}{-}} \FunctionTok{lm}\NormalTok{(salario }\SpecialCharTok{\textasciitilde{}}\NormalTok{ salini }\SpecialCharTok{+}\NormalTok{ expprev, }\AttributeTok{data =}\NormalTok{ empleados)}
\FunctionTok{summary}\NormalTok{(modelo)   }
\end{Highlighting}
\end{Shaded}

\begin{verbatim}
## 
## Call:
## lm(formula = salario ~ salini + expprev, data = empleados)
## 
## Residuals:
##    Min     1Q Median     3Q    Max 
## -32263  -4219  -1332   2673  48571 
## 
## Coefficients:
##               Estimate Std. Error t value Pr(>|t|)    
## (Intercept) 3850.71760  900.63287   4.276 2.31e-05 ***
## salini         1.92291    0.04548  42.283  < 2e-16 ***
## expprev      -22.44482    3.42240  -6.558 1.44e-10 ***
## ---
## Signif. codes:  0 '***' 0.001 '**' 0.01 '*' 0.05 '.' 0.1 ' ' 1
## 
## Residual standard error: 7777 on 471 degrees of freedom
## Multiple R-squared:  0.7935, Adjusted R-squared:  0.7926 
## F-statistic: 904.8 on 2 and 471 DF,  p-value: < 2.2e-16
\end{verbatim}

\hypertarget{gruxe1ficas-buxe1sicas-de-diagnuxf3stico}{%
\subsection{Gráficas básicas de diagnóstico}\label{gruxe1ficas-buxe1sicas-de-diagnuxf3stico}}

Con la función \texttt{plot} se pueden generar gráficos de interés para la diagnosis del modelo:

\begin{Shaded}
\begin{Highlighting}[]
\NormalTok{oldpar }\OtherTok{\textless{}{-}} \FunctionTok{par}\NormalTok{( }\AttributeTok{mfrow=}\FunctionTok{c}\NormalTok{(}\DecValTok{2}\NormalTok{,}\DecValTok{2}\NormalTok{))}
\FunctionTok{plot}\NormalTok{(modelo)}
\end{Highlighting}
\end{Shaded}

\includegraphics{08-ModelosLineales_files/figure-latex/unnamed-chunk-25-1.pdf}

\begin{Shaded}
\begin{Highlighting}[]
\FunctionTok{par}\NormalTok{(oldpar)}
\end{Highlighting}
\end{Shaded}

Por defecto se muestran cuatro gráficos (ver \texttt{help(plot.lm)} para más detalles). El primero (residuos frente a predicciones) permite detectar falta de
linealidad o heterocedasticidad (o el efecto de un factor omitido: mala
especificación del modelo), lo ideal sería no observar ningún patrón.

El segundo gráfico (gráfico QQ), permite diagnosticar la normalidad, los puntos del deberían estar cerca de la diagonal.

El tercer gráfico de dispersión-nivel permite detectar heterocedasticidad y ayudar a seleccionar una transformación para corregirla (más adelante, en la sección \emph{Alternativas}, se tratará este tema), la pendiente de los datos debería ser nula.

El último gráfico permite detectar valores atípicos o influyentes. Representa los residuos estandarizados en función del valor de influencia (a priori) o leverage (\(hii\) que depende de los valores de las variables explicativas, debería ser \(< 2(p+1)/2\)) y señala las observaciones atípicas (residuos fuera de {[}-2,2{]}) e influyentes a posteriori (estadístico de Cook \textgreater0.5 y \textgreater1).

Si las conclusiones obtenidas dependen en gran medida de una
observación (normalmente atípica), esta se denomina influyente (a
posteriori) y debe ser examinada con cuidado por el experimentador.
Para recalcular el modelo sin una de las observaciones puede ser útil la función update:

\begin{Shaded}
\begin{Highlighting}[]
\CommentTok{\# which.max(cooks.distance(modelo))}
\NormalTok{modelo2 }\OtherTok{\textless{}{-}} \FunctionTok{update}\NormalTok{(modelo, }\AttributeTok{data =}\NormalTok{ empleados[}\SpecialCharTok{{-}}\DecValTok{29}\NormalTok{, ])}
\end{Highlighting}
\end{Shaded}

Si hay datos atípicos o influyentes, puede ser recomendable emplear regresión lineal robusta, por ejemplo mediante la función \texttt{rlm} del paquete \texttt{MASS}.

En el ejemplo anterior, se observa claramente heterogeneidad de varianzas y falta de normalidad. Aparentemente no hay observaciones influyentes (a posteriori) aunque si algún dato atípico.

\hypertarget{gruxe1ficos-parciales-de-residuos}{%
\subsection{Gráficos parciales de residuos}\label{gruxe1ficos-parciales-de-residuos}}

En regresión lineal múltiple, en lugar de generar gráficos de dispersión simple
(p.e. gráficos de dispersión matriciales) para detectar problemas (falta de
linealidad, \ldots) y analizar los efectos de las variables explicativas,
se pueden generar gráficos parciales de residuos, por ejemplo con el comando:

\begin{Shaded}
\begin{Highlighting}[]
\FunctionTok{termplot}\NormalTok{(modelo, }\AttributeTok{partial.resid =} \ConstantTok{TRUE}\NormalTok{)}
\end{Highlighting}
\end{Shaded}

Aunque puede ser preferible emplear las funciones \texttt{crPlots} ó \texttt{avPlots} del paquete \texttt{car}:

\begin{Shaded}
\begin{Highlighting}[]
\FunctionTok{library}\NormalTok{(car)}
\FunctionTok{crPlots}\NormalTok{(modelo)}
\end{Highlighting}
\end{Shaded}

\includegraphics{08-ModelosLineales_files/figure-latex/unnamed-chunk-28-1.pdf}

\begin{Shaded}
\begin{Highlighting}[]
\CommentTok{\# avPlots(modelo)}
\end{Highlighting}
\end{Shaded}

Estas funciones permitirían además detectar puntos atípicos o influyentes
(mediante los argumentos \texttt{id.method} e \texttt{id.n}).

\hypertarget{estaduxedsticos}{%
\subsection{Estadísticos}\label{estaduxedsticos}}

Para obtener medidas de diagnosis o resúmenes numéricos de interés se pueden emplear
las siguientes funciones:

\begin{longtable}[]{@{}
  >{\raggedright\arraybackslash}p{(\columnwidth - 2\tabcolsep) * \real{0.10}}
  >{\raggedright\arraybackslash}p{(\columnwidth - 2\tabcolsep) * \real{0.90}}@{}}
\toprule
Función & Descripción \\
\midrule
\endhead
rstandard & residuos estandarizados \\
rstudent & residuos estudentizados (eliminados) \\
cooks.distance & valores del estadístico de Cook \\
influence & valores de influencia, cambios en coeficientes y varianza residual al eliminar cada dato. \\
\bottomrule
\end{longtable}

Ejecutar \texttt{help(influence.measures)} para ver un listado de medidas de diagnóstico adicionales.

Hay muchas herramientas adicionales disponibles en otros paquetes.
Por ejemplo, para la detección de multicolinealidad, se puede emplear la función
\texttt{vif} del paquete \texttt{car} para calcular los factores de inflación de varianza para
las variables del modelo:

\begin{Shaded}
\begin{Highlighting}[]
\CommentTok{\# library(car)}
\FunctionTok{vif}\NormalTok{(modelo)}
\end{Highlighting}
\end{Shaded}

\begin{verbatim}
##   salini  expprev 
## 1.002041 1.002041
\end{verbatim}

Valores grandes, por ejemplo \textgreater{} 10, indican la posible presencia de multicolinealidad.

\textbf{Nota}: Las tolerancias (proporciones de variabilidad no explicada por las demás covariables) se pueden calcular con \texttt{1/vif(modelo)}.

\hypertarget{contrastes-1}{%
\subsection{Contrastes}\label{contrastes-1}}

\hypertarget{normalidad-1}{%
\subsubsection{Normalidad}\label{normalidad-1}}

Para realizar el contraste de normalidad de Shapiro-Wilk se puede emplear:

\begin{Shaded}
\begin{Highlighting}[]
\FunctionTok{shapiro.test}\NormalTok{(}\FunctionTok{residuals}\NormalTok{(modelo))}
\end{Highlighting}
\end{Shaded}

\begin{verbatim}
## 
##  Shapiro-Wilk normality test
## 
## data:  residuals(modelo)
## W = 0.85533, p-value < 2.2e-16
\end{verbatim}

\begin{Shaded}
\begin{Highlighting}[]
\FunctionTok{hist}\NormalTok{(}\FunctionTok{residuals}\NormalTok{(modelo))}
\end{Highlighting}
\end{Shaded}

\includegraphics{08-ModelosLineales_files/figure-latex/unnamed-chunk-30-1.pdf}

\hypertarget{homocedasticidad}{%
\subsubsection{Homocedasticidad}\label{homocedasticidad}}

La librería \texttt{lmtest} proporciona herramientas adicionales para la diagnosis de modelos lineales, por ejemplo el test de Breusch-Pagan para heterocedasticidad:

\begin{Shaded}
\begin{Highlighting}[]
\FunctionTok{library}\NormalTok{(lmtest)}
\FunctionTok{bptest}\NormalTok{(modelo, }\AttributeTok{studentize =} \ConstantTok{FALSE}\NormalTok{)}
\end{Highlighting}
\end{Shaded}

\begin{verbatim}
## 
##  Breusch-Pagan test
## 
## data:  modelo
## BP = 290.37, df = 2, p-value < 2.2e-16
\end{verbatim}

Si el p-valor es grande aceptaríamos que hay igualdad de varianzas.

\hypertarget{autocorrelaciuxf3n}{%
\subsubsection{Autocorrelación}\label{autocorrelaciuxf3n}}

Contraste de Durbin-Watson para detectar si hay correlación serial entre los errores:

\begin{Shaded}
\begin{Highlighting}[]
\FunctionTok{dwtest}\NormalTok{(modelo, }\AttributeTok{alternative=} \StringTok{"two.sided"}\NormalTok{)}
\end{Highlighting}
\end{Shaded}

\begin{verbatim}
## 
##  Durbin-Watson test
## 
## data:  modelo
## DW = 1.8331, p-value = 0.06702
## alternative hypothesis: true autocorrelation is not 0
\end{verbatim}

Si el p-valor es pequeño rechazaríamos la hipótesis de independencia.

\hypertarget{muxe9todos-de-regularizaciuxf3n}{%
\section{Métodos de regularización}\label{muxe9todos-de-regularizaciuxf3n}}

{[}{[}Pasar a selección de variables explicativas?{]}{]}

Estos métodos emplean también un modelo lineal:
\[Y=\beta_{0}+\beta_{1}X_{1}+\beta_{2}X_{2}+\cdots+\beta_{p}X_{p}+\varepsilon\]

En lugar de ajustarlo por mínimos cuadrados (estándar), minimizando:
\[ RSS = \sum\limits_{i=1}^{n}\left(  y_{i} - \beta_0 - \beta_1 x_{1i} - \cdots - \beta_p x_{pi} \right)^{2}\]

Se imponen restricciones adicionales a los parámetros que los
``retraen'' (shrink) hacia cero:

\begin{itemize}
\item
  Produce una reducción en la varianza de predicción (a
  costa del sesgo).
\item
  En principio se consideran todas las variables explicativas.
\end{itemize}

\textbf{Ridge regression}

\begin{itemize}
\tightlist
\item
  Penalización cuadrática: \(RSS+\lambda\sum_{j=1}^{p}\beta_{j}^{2}\).
\end{itemize}

\textbf{Lasso}

\begin{itemize}
\item
  Penalización en valor absoluto: \(RSS+\lambda\sum_{j=1}^{p}|\beta_{j}|\).
\item
  Normalmente asigna peso nulo a algunas variables
  (selección de variables).
\end{itemize}

El parámetro de penalización se selecciona por \textbf{validación cruzada}.

\begin{itemize}
\tightlist
\item
  Normalmente estandarizan las variables explicativas
  (coeficientes en la misma escala).
\end{itemize}

\hypertarget{datos}{%
\subsection{Datos}\label{datos}}

El fichero \emph{hatco.RData} contiene observaciones de clientes de la compañía de
distribución industrial (Compañía Hair, Anderson y Tatham).
Las variables se pueden clasificar en tres grupos:

\begin{Shaded}
\begin{Highlighting}[]
\FunctionTok{load}\NormalTok{(}\StringTok{\textquotesingle{}datos/hatco.RData\textquotesingle{}}\NormalTok{)}
\FunctionTok{as.data.frame}\NormalTok{(}\FunctionTok{attr}\NormalTok{(hatco, }\StringTok{"variable.labels"}\NormalTok{))}
\end{Highlighting}
\end{Shaded}

\begin{verbatim}
##          attr(hatco, "variable.labels")
## empresa                         Empresa
## tamano             Tamaño de la empresa
## adquisic      Estructura de adquisición
## tindustr              Tipo de industria
## tsitcomp    Tipo de situación de compra
## velocida           Velocidad de entrega
## precio                 Nivel de precios
## flexprec        Flexibilidad de precios
## imgfabri          Imagen del fabricante
## servconj              Servicio conjunto
## imgfvent     Imagen de fuerza de ventas
## calidadp            Calidad de producto
## fidelida   Porcentaje de compra a HATCO
## satisfac            Satisfacción global
## nfidelid        Nivel de compra a HATCO
## nsatisfa          Nivel de satisfacción
\end{verbatim}

Consideraremos como respuesta la variable \emph{fidelida} y como variables explicativas
el resto de variables continuas menos \emph{satisfac}.

\begin{Shaded}
\begin{Highlighting}[]
\FunctionTok{library}\NormalTok{(glmnet)}
\end{Highlighting}
\end{Shaded}

El paquete \texttt{glmnet} no emplea formulación de modelos, hay que establecer la respuesta
\texttt{y} y las variables explicativas \texttt{x} (se puede emplear la función \texttt{model.matrix()} para construir \texttt{x},
la matriz de diseño, a partir de una fórmula).
En este caso, eliminamos también la última fila por tener datos faltantes:

\begin{Shaded}
\begin{Highlighting}[]
\NormalTok{x }\OtherTok{\textless{}{-}} \FunctionTok{as.matrix}\NormalTok{(hatco[}\SpecialCharTok{{-}}\DecValTok{100}\NormalTok{, }\DecValTok{6}\SpecialCharTok{:}\DecValTok{12}\NormalTok{])}
\NormalTok{y }\OtherTok{\textless{}{-}}\NormalTok{ hatco}\SpecialCharTok{$}\NormalTok{fidelida[}\SpecialCharTok{{-}}\DecValTok{100}\NormalTok{]}
\end{Highlighting}
\end{Shaded}

\hypertarget{ridge-regression}{%
\subsection{Ridge Regression}\label{ridge-regression}}

Ajustamos un modelo de regresión ridge con la función \texttt{glmnet} con \texttt{alpha=0} (ridge penalty).

\begin{Shaded}
\begin{Highlighting}[]
\NormalTok{fit.ridge }\OtherTok{\textless{}{-}} \FunctionTok{glmnet}\NormalTok{(x, y, }\AttributeTok{alpha =} \DecValTok{0}\NormalTok{)}
\FunctionTok{plot}\NormalTok{(fit.ridge, }\AttributeTok{xvar =} \StringTok{"lambda"}\NormalTok{, }\AttributeTok{label =} \ConstantTok{TRUE}\NormalTok{)}
\end{Highlighting}
\end{Shaded}

\includegraphics{08-ModelosLineales_files/figure-latex/unnamed-chunk-36-1.pdf}

Para seleccionar el parámetro de penalización por validación cruzada se puede emplear
la función \texttt{cv.glmnet}.

\begin{Shaded}
\begin{Highlighting}[]
\NormalTok{cv.ridge }\OtherTok{\textless{}{-}} \FunctionTok{cv.glmnet}\NormalTok{(x, y, }\AttributeTok{alpha =} \DecValTok{0}\NormalTok{)}
\FunctionTok{plot}\NormalTok{(cv.ridge)}
\end{Highlighting}
\end{Shaded}

\includegraphics{08-ModelosLineales_files/figure-latex/unnamed-chunk-37-1.pdf}

En este caso el parámetro sería:

\begin{Shaded}
\begin{Highlighting}[]
\NormalTok{cv.ridge}\SpecialCharTok{$}\NormalTok{lambda}\FloatTok{.1}\NormalTok{se}
\end{Highlighting}
\end{Shaded}

\begin{verbatim}
## [1] 3.635163
\end{verbatim}

y el modelo resultante contiene todas las variables explicativas:

\begin{Shaded}
\begin{Highlighting}[]
\FunctionTok{coef}\NormalTok{(cv.ridge)}
\end{Highlighting}
\end{Shaded}

\begin{verbatim}
## 8 x 1 sparse Matrix of class "dgCMatrix"
##                     s1
## (Intercept) 5.26333438
## velocida    1.58051175
## precio      0.70395775
## flexprec    2.24798481
## imgfabri    0.31897738
## servconj    3.76988236
## imgfvent    1.07304993
## calidadp    0.06641356
\end{verbatim}

\hypertarget{lasso}{%
\subsection{Lasso}\label{lasso}}

Ajustamos un modelo lasso también con la función \texttt{glmnet} (con la opción por defecto \texttt{alpha=1}, lasso penalty).

\begin{Shaded}
\begin{Highlighting}[]
\NormalTok{fit.lasso }\OtherTok{\textless{}{-}} \FunctionTok{glmnet}\NormalTok{(x,y)}
\FunctionTok{plot}\NormalTok{(fit.lasso, }\AttributeTok{xvar =} \StringTok{"lambda"}\NormalTok{, }\AttributeTok{label =} \ConstantTok{TRUE}\NormalTok{)}
\end{Highlighting}
\end{Shaded}

\includegraphics{08-ModelosLineales_files/figure-latex/unnamed-chunk-40-1.pdf}

Seleccionamos el parámetro de penalización por validación cruzada.

\begin{Shaded}
\begin{Highlighting}[]
\NormalTok{cv.lasso }\OtherTok{\textless{}{-}} \FunctionTok{cv.glmnet}\NormalTok{(x,y)}
\FunctionTok{plot}\NormalTok{(cv.lasso)}
\end{Highlighting}
\end{Shaded}

\includegraphics{08-ModelosLineales_files/figure-latex/unnamed-chunk-41-1.pdf}

En este caso el modelo resultante solo contiene 4 variables explicativas:

\begin{Shaded}
\begin{Highlighting}[]
\FunctionTok{coef}\NormalTok{(cv.lasso)}
\end{Highlighting}
\end{Shaded}

\begin{verbatim}
## 8 x 1 sparse Matrix of class "dgCMatrix"
##                    s1
## (Intercept) 4.4757712
## velocida    0.1020531
## precio      .        
## flexprec    2.7202485
## imgfabri    .        
## servconj    6.4044378
## imgfvent    0.4651076
## calidadp    .
\end{verbatim}

\hypertarget{alternativas}{%
\section{Alternativas}\label{alternativas}}

\hypertarget{transformaciuxf3n-modelos-linealizables}{%
\subsection{Transformación (modelos linealizables)}\label{transformaciuxf3n-modelos-linealizables}}

Cuando no se satisfacen los supuestos básicos se puede intentar
transformar los datos para corregir la falta de
linealidad, la heterocedasticidad y/o la falta de normalidad
(normalmente estas últimas ``suelen ocurrir en la misma escala'').
Por ejemplo, la función \texttt{boxcox} del paquete \texttt{MASS} permite seleccionar la transformación de Box-Cox
más adecuada:
\[Y^{(\lambda)} =
\begin{cases}
\dfrac{Y^\lambda - 1}{\lambda} & \text{si } \lambda \neq 0 \\
\ln{(Y)} & \text{si } \lambda = 0
\end{cases}\]

\begin{Shaded}
\begin{Highlighting}[]
\CommentTok{\# library(MASS)}
\FunctionTok{boxcox}\NormalTok{(modelo)}
\end{Highlighting}
\end{Shaded}

\includegraphics{08-ModelosLineales_files/figure-latex/unnamed-chunk-43-1.pdf}

En este caso una transformación logarítmica parece adecuada.

En ocasiones para obtener una relación lineal (o heterocedasticidad) también es
necesario transformar las covariables además de la respuesta. Algunas de
las relaciones fácilmente linealizables se muestran a continuación:

\begin{longtable}[]{@{}llll@{}}
\toprule
modelo & ecuación & covariable & respuesta \\
\midrule
\endhead
logarítmico & \(y = a + b\text{ }log(x)\) & \(log(x)\) & \_ \\
inverso & \(y = a + b/x\) & \(1/x\) & \_ \\
potencial & \(y = ax^b\) & \(log(x)\) & \(log(y)\) \\
exponencial & \(y = ae^{bx}\) & \_ & \(log(y)\) \\
curva-S & \(y = ae^{b/x}\) & \(1/x\) & \(log(y)\) \\
\bottomrule
\end{longtable}

\hypertarget{ejemplo-1}{%
\subsubsection{Ejemplo:}\label{ejemplo-1}}

\begin{Shaded}
\begin{Highlighting}[]
\FunctionTok{plot}\NormalTok{(salario }\SpecialCharTok{\textasciitilde{}}\NormalTok{ salini, }\AttributeTok{data =}\NormalTok{ empleados, }\AttributeTok{col =} \StringTok{\textquotesingle{}darkgray\textquotesingle{}}\NormalTok{)}

\CommentTok{\# Ajuste lineal}
\FunctionTok{abline}\NormalTok{(}\FunctionTok{lm}\NormalTok{(salario }\SpecialCharTok{\textasciitilde{}}\NormalTok{ salini, }\AttributeTok{data =}\NormalTok{ empleados)) }

\CommentTok{\# Modelo exponencial}
\NormalTok{modelo1 }\OtherTok{\textless{}{-}} \FunctionTok{lm}\NormalTok{(}\FunctionTok{log}\NormalTok{(salario) }\SpecialCharTok{\textasciitilde{}}\NormalTok{ salini, }\AttributeTok{data =}\NormalTok{ empleados)}
\NormalTok{parest }\OtherTok{\textless{}{-}} \FunctionTok{coef}\NormalTok{(modelo1)}
\FunctionTok{curve}\NormalTok{(}\FunctionTok{exp}\NormalTok{(parest[}\DecValTok{1}\NormalTok{] }\SpecialCharTok{+}\NormalTok{ parest[}\DecValTok{2}\NormalTok{]}\SpecialCharTok{*}\NormalTok{x), }\AttributeTok{lty =} \DecValTok{2}\NormalTok{, }\AttributeTok{add =} \ConstantTok{TRUE}\NormalTok{)}

\CommentTok{\# Modelo logarítmico}
\NormalTok{modelo2 }\OtherTok{\textless{}{-}} \FunctionTok{lm}\NormalTok{(}\FunctionTok{log}\NormalTok{(salario) }\SpecialCharTok{\textasciitilde{}} \FunctionTok{log}\NormalTok{(salini), }\AttributeTok{data =}\NormalTok{ empleados)}
\NormalTok{parest }\OtherTok{\textless{}{-}} \FunctionTok{coef}\NormalTok{(modelo2)}
\FunctionTok{curve}\NormalTok{(}\FunctionTok{exp}\NormalTok{(parest[}\DecValTok{1}\NormalTok{]) }\SpecialCharTok{*}\NormalTok{ x}\SpecialCharTok{\^{}}\NormalTok{parest[}\DecValTok{2}\NormalTok{], }\AttributeTok{lty =} \DecValTok{3}\NormalTok{, }\AttributeTok{add =} \ConstantTok{TRUE}\NormalTok{)}

\FunctionTok{legend}\NormalTok{(}\StringTok{"bottomright"}\NormalTok{, }\FunctionTok{c}\NormalTok{(}\StringTok{"Lineal"}\NormalTok{,}\StringTok{"Exponencial"}\NormalTok{,}\StringTok{"Logarítmico"}\NormalTok{), }\AttributeTok{lty =} \DecValTok{1}\SpecialCharTok{:}\DecValTok{3}\NormalTok{)}
\end{Highlighting}
\end{Shaded}

\includegraphics{08-ModelosLineales_files/figure-latex/unnamed-chunk-44-1.pdf}

Con estos datos de ejemplo, el principal problema es la falta de homogeneidad de varianzas (y de normalidad) y se corrige sustancialmente con el segundo modelo:

\begin{Shaded}
\begin{Highlighting}[]
\FunctionTok{plot}\NormalTok{(}\FunctionTok{log}\NormalTok{(salario) }\SpecialCharTok{\textasciitilde{}} \FunctionTok{log}\NormalTok{(salini), }\AttributeTok{data =}\NormalTok{ empleados)}
\FunctionTok{abline}\NormalTok{(modelo2)}
\end{Highlighting}
\end{Shaded}

\includegraphics{08-ModelosLineales_files/figure-latex/unnamed-chunk-45-1.pdf}

\hypertarget{ajuste-polinuxf3mico}{%
\subsection{Ajuste polinómico}\label{ajuste-polinuxf3mico}}

En este apartado utilizaremos como ejemplo el conjunto de datos \texttt{Prestige} de la librería \texttt{car}. Al tratar de explicar \texttt{prestige} (puntuación de ocupaciones obtenidas a partir de una encuesta ) a partir de \texttt{income} (media de ingresos en la ocupación), un ajuste cuadrático puede parecer razonable:

\begin{Shaded}
\begin{Highlighting}[]
\CommentTok{\# library(car)}
\FunctionTok{plot}\NormalTok{(prestige }\SpecialCharTok{\textasciitilde{}}\NormalTok{ income, }\AttributeTok{data =}\NormalTok{ Prestige, }\AttributeTok{col =} \StringTok{\textquotesingle{}darkgray\textquotesingle{}}\NormalTok{)}
\CommentTok{\# Ajuste lineal}
\FunctionTok{abline}\NormalTok{(}\FunctionTok{lm}\NormalTok{(prestige }\SpecialCharTok{\textasciitilde{}}\NormalTok{ income, }\AttributeTok{data =}\NormalTok{ Prestige)) }
\CommentTok{\# Ajuste cuadrático}
\NormalTok{modelo }\OtherTok{\textless{}{-}} \FunctionTok{lm}\NormalTok{(prestige }\SpecialCharTok{\textasciitilde{}}\NormalTok{ income }\SpecialCharTok{+} \FunctionTok{I}\NormalTok{(income}\SpecialCharTok{\^{}}\DecValTok{2}\NormalTok{), }\AttributeTok{data =}\NormalTok{ Prestige)}
\NormalTok{parest }\OtherTok{\textless{}{-}} \FunctionTok{coef}\NormalTok{(modelo)}
\FunctionTok{curve}\NormalTok{(parest[}\DecValTok{1}\NormalTok{] }\SpecialCharTok{+}\NormalTok{ parest[}\DecValTok{2}\NormalTok{]}\SpecialCharTok{*}\NormalTok{x }\SpecialCharTok{+}\NormalTok{ parest[}\DecValTok{3}\NormalTok{]}\SpecialCharTok{*}\NormalTok{x}\SpecialCharTok{\^{}}\DecValTok{2}\NormalTok{, }\AttributeTok{lty =} \DecValTok{2}\NormalTok{, }\AttributeTok{add =} \ConstantTok{TRUE}\NormalTok{)}

\FunctionTok{legend}\NormalTok{(}\StringTok{"bottomright"}\NormalTok{, }\FunctionTok{c}\NormalTok{(}\StringTok{"Lineal"}\NormalTok{,}\StringTok{"Cuadrático"}\NormalTok{), }\AttributeTok{lty =} \DecValTok{1}\SpecialCharTok{:}\DecValTok{2}\NormalTok{)}
\end{Highlighting}
\end{Shaded}

\includegraphics{08-ModelosLineales_files/figure-latex/unnamed-chunk-46-1.pdf}

Alternativamente se podría emplear la función \texttt{poly}:

\begin{Shaded}
\begin{Highlighting}[]
\FunctionTok{plot}\NormalTok{(prestige }\SpecialCharTok{\textasciitilde{}}\NormalTok{ income, }\AttributeTok{data =}\NormalTok{ Prestige, }\AttributeTok{col =} \StringTok{\textquotesingle{}darkgray\textquotesingle{}}\NormalTok{)}
\CommentTok{\# Ajuste cúbico}
\NormalTok{modelo }\OtherTok{\textless{}{-}} \FunctionTok{lm}\NormalTok{(prestige }\SpecialCharTok{\textasciitilde{}} \FunctionTok{poly}\NormalTok{(income, }\DecValTok{3}\NormalTok{), }\AttributeTok{data =}\NormalTok{ Prestige)}
\NormalTok{valores }\OtherTok{\textless{}{-}} \FunctionTok{seq}\NormalTok{(}\DecValTok{0}\NormalTok{, }\DecValTok{26000}\NormalTok{, }\AttributeTok{len =} \DecValTok{100}\NormalTok{)}
\NormalTok{pred }\OtherTok{\textless{}{-}} \FunctionTok{predict}\NormalTok{(modelo, }\AttributeTok{newdata =} \FunctionTok{data.frame}\NormalTok{(}\AttributeTok{income =}\NormalTok{ valores))}
\FunctionTok{lines}\NormalTok{(valores, pred, }\AttributeTok{lty =} \DecValTok{3}\NormalTok{) }
\end{Highlighting}
\end{Shaded}

\includegraphics{08-ModelosLineales_files/figure-latex/unnamed-chunk-47-1.pdf}

\hypertarget{ajuste-polinuxf3mico-local-robusto}{%
\subsection{Ajuste polinómico local (robusto)}\label{ajuste-polinuxf3mico-local-robusto}}

Si no se logra un buen ajuste empleando los modelos anteriores se puede pensar en
utilizar métodos no paramétricos (p.e. regresión aditiva no paramétrica). Por ejemplo,
en\texttt{R} es habitual emplear la función \texttt{loess} (sobre todo en gráficos):

\begin{Shaded}
\begin{Highlighting}[]
\FunctionTok{plot}\NormalTok{(prestige }\SpecialCharTok{\textasciitilde{}}\NormalTok{ income, Prestige, }\AttributeTok{col =} \StringTok{\textquotesingle{}darkgray\textquotesingle{}}\NormalTok{)}
\NormalTok{fit }\OtherTok{\textless{}{-}} \FunctionTok{loess}\NormalTok{(prestige }\SpecialCharTok{\textasciitilde{}}\NormalTok{ income, Prestige, }\AttributeTok{span =} \FloatTok{0.75}\NormalTok{)}
\NormalTok{valores }\OtherTok{\textless{}{-}} \FunctionTok{seq}\NormalTok{(}\DecValTok{0}\NormalTok{, }\DecValTok{25000}\NormalTok{, }\DecValTok{100}\NormalTok{)}
\NormalTok{pred }\OtherTok{\textless{}{-}} \FunctionTok{predict}\NormalTok{(fit, }\AttributeTok{newdata =} \FunctionTok{data.frame}\NormalTok{(}\AttributeTok{income =}\NormalTok{ valores))}
\FunctionTok{lines}\NormalTok{(valores, pred)}
\end{Highlighting}
\end{Shaded}

\includegraphics{08-ModelosLineales_files/figure-latex/unnamed-chunk-48-1.pdf}

Este tipo de modelos los trataremos con detalle más adelante\ldots{}

\hypertarget{modelos-lineales-generalizados}{%
\chapter{Modelos lineales generalizados}\label{modelos-lineales-generalizados}}

Los modelos lineales generalizados son una extensión de los modelos lineales para el caso de que la distribución condicional de la variable respuesta no sea normal (por ejemplo discreta: Bernouilli, Binomial, Poisson, \ldots)

En los modelo lineales se supone que:
\[E( Y | \mathbf{X} ) = \beta_{0}+\beta_{1}X_{1}+\beta_{2}X_{2}+\cdots+\beta_{p}X_{p}\]
En los modelos lineales generalizados se introduce una función invertible \emph{g}, denominada función enlace (o link):
\[g\left(E(Y | \mathbf{X} )\right) = \beta_{0}+\beta_{1}X_{1}+\beta_{2}X_{2}+\cdots+\beta_{p}X_{p}\]

\hypertarget{ajuste-funciuxf3n-glm}{%
\section{\texorpdfstring{Ajuste: función \texttt{glm}}{Ajuste: función glm}}\label{ajuste-funciuxf3n-glm}}

Para el ajuste (estimación de los parámetros) de un modelo lineal generalizado a un conjunto de datos (por máxima verosimilitud) se emplea la función \texttt{glm}:

\begin{Shaded}
\begin{Highlighting}[]
\NormalTok{ajuste }\OtherTok{\textless{}{-}} \FunctionTok{glm}\NormalTok{(formula, }\AttributeTok{family =}\NormalTok{ gaussian, datos, ...)}
\end{Highlighting}
\end{Shaded}

El parámetro \texttt{family} indica la distribución y el link. Por ejemplo:

\begin{itemize}
\item
  \texttt{gaussian(link\ =\ "identity")}, \texttt{gaussian(link\ =\ "log")}
\item
  \texttt{binomial(link\ =\ "logit")}, \texttt{binomial(link\ =\ "probit")}
\item
  \texttt{poisson(link\ =\ "log")}
\item
  \texttt{Gamma(link\ =\ "inverse")}
\end{itemize}

Para cada distribución se toma por defecto una función link (mostrada en primer lugar; ver \texttt{help(family)} para más detalles).

Muchas de las herramientas y funciones genéricas disponibles para los modelos lineales son válidas
también para este tipo de modelos: \texttt{summary}, \texttt{coef}, \texttt{confint}, \texttt{predict}, \texttt{anova}, \ldots.

Veremos con más detalle el caso particular de la regresión logística.

\hypertarget{regresiuxf3n-loguxedstica}{%
\section{Regresión logística}\label{regresiuxf3n-loguxedstica}}

\hypertarget{ejemplo-2}{%
\subsection{Ejemplo}\label{ejemplo-2}}

Como ejemplo emplearemos los datos de clientes de la compañía de distribución industrial (Compañía Hair, Anderson y Tatham).

\begin{Shaded}
\begin{Highlighting}[]
\FunctionTok{load}\NormalTok{(}\StringTok{"datos/hatco.RData"}\NormalTok{)}
\FunctionTok{as.data.frame}\NormalTok{(}\FunctionTok{attr}\NormalTok{(hatco, }\StringTok{"variable.labels"}\NormalTok{))}
\end{Highlighting}
\end{Shaded}

\begin{verbatim}
##          attr(hatco, "variable.labels")
## empresa                         Empresa
## tamano             Tamaño de la empresa
## adquisic      Estructura de adquisición
## tindustr              Tipo de industria
## tsitcomp    Tipo de situación de compra
## velocida           Velocidad de entrega
## precio                 Nivel de precios
## flexprec        Flexibilidad de precios
## imgfabri          Imagen del fabricante
## servconj              Servicio conjunto
## imgfvent     Imagen de fuerza de ventas
## calidadp            Calidad de producto
## fidelida   Porcentaje de compra a HATCO
## satisfac            Satisfacción global
## nfidelid        Nivel de compra a HATCO
## nsatisfa          Nivel de satisfacción
\end{verbatim}

Consideraremos como respuesta la variable \emph{nsatisfa} y como variables explicativas
el resto de variables continuas menos \emph{fidelida} y \emph{satisfac}.
Eliminamos también la última fila por tener datos faltantes (realmente no sería necesario).

\begin{Shaded}
\begin{Highlighting}[]
\NormalTok{datos }\OtherTok{\textless{}{-}}\NormalTok{ hatco[}\SpecialCharTok{{-}}\DecValTok{100}\NormalTok{, }\FunctionTok{c}\NormalTok{(}\DecValTok{6}\SpecialCharTok{:}\DecValTok{12}\NormalTok{, }\DecValTok{16}\NormalTok{)]}
\FunctionTok{plot}\NormalTok{(datos, }\AttributeTok{pch =} \FunctionTok{as.numeric}\NormalTok{(datos}\SpecialCharTok{$}\NormalTok{nsatisfa), }\AttributeTok{col =} \FunctionTok{as.numeric}\NormalTok{(datos}\SpecialCharTok{$}\NormalTok{nsatisfa))}
\end{Highlighting}
\end{Shaded}

\includegraphics{09-ModelosGLM_files/figure-latex/unnamed-chunk-3-1.pdf}

\hypertarget{ajuste-de-un-modelo-de-regresiuxf3n-loguxedstica}{%
\subsection{Ajuste de un modelo de regresión logística}\label{ajuste-de-un-modelo-de-regresiuxf3n-loguxedstica}}

Se emplea la función \texttt{glm} seleccionando \texttt{family\ =\ binomial} (la función de enlace por defecto será \emph{logit}):

\begin{Shaded}
\begin{Highlighting}[]
\NormalTok{modelo }\OtherTok{\textless{}{-}} \FunctionTok{glm}\NormalTok{(nsatisfa }\SpecialCharTok{\textasciitilde{}}\NormalTok{ velocida }\SpecialCharTok{+}\NormalTok{ imgfabri , }\AttributeTok{family =}\NormalTok{ binomial, }\AttributeTok{data =}\NormalTok{ datos)}
\NormalTok{modelo}
\end{Highlighting}
\end{Shaded}

\begin{verbatim}
## 
## Call:  glm(formula = nsatisfa ~ velocida + imgfabri, family = binomial, 
##     data = datos)
## 
## Coefficients:
## (Intercept)     velocida     imgfabri  
##     -10.127        1.203        1.058  
## 
## Degrees of Freedom: 98 Total (i.e. Null);  96 Residual
## Null Deviance:       136.4 
## Residual Deviance: 88.64     AIC: 94.64
\end{verbatim}

La razón de ventajas (OR) permite cuantificar el efecto de las variables explicativas en la respuesta
(Incremento proporcional en la ventaja o probabilidad de éxito, al aumentar una unidad la variable manteniendo las demás fijas):

\begin{Shaded}
\begin{Highlighting}[]
\FunctionTok{exp}\NormalTok{(}\FunctionTok{coef}\NormalTok{(modelo))  }\CommentTok{\# Razones de ventajas ("odds ratios")}
\end{Highlighting}
\end{Shaded}

\begin{verbatim}
##  (Intercept)     velocida     imgfabri 
## 3.997092e-05 3.329631e+00 2.881619e+00
\end{verbatim}

\begin{Shaded}
\begin{Highlighting}[]
\FunctionTok{exp}\NormalTok{(}\FunctionTok{confint}\NormalTok{(modelo))}
\end{Highlighting}
\end{Shaded}

\begin{verbatim}
## Waiting for profiling to be done...
\end{verbatim}

\begin{verbatim}
##                    2.5 %      97.5 %
## (Intercept) 3.828431e-07 0.001621259
## velocida    2.061302e+00 5.976208357
## imgfabri    1.737500e+00 5.247303813
\end{verbatim}

Para obtener un resumen más completo del ajuste también se utiliza \texttt{summary()}

\begin{Shaded}
\begin{Highlighting}[]
\FunctionTok{summary}\NormalTok{(modelo)}
\end{Highlighting}
\end{Shaded}

\begin{verbatim}
## 
## Call:
## glm(formula = nsatisfa ~ velocida + imgfabri, family = binomial, 
##     data = datos)
## 
## Deviance Residuals: 
##     Min       1Q   Median       3Q      Max  
## -1.8941  -0.6697  -0.2098   0.7865   2.3378  
## 
## Coefficients:
##             Estimate Std. Error z value Pr(>|z|)    
## (Intercept) -10.1274     2.1062  -4.808 1.52e-06 ***
## velocida      1.2029     0.2685   4.479 7.49e-06 ***
## imgfabri      1.0584     0.2792   3.790 0.000151 ***
## ---
## Signif. codes:  0 '***' 0.001 '**' 0.01 '*' 0.05 '.' 0.1 ' ' 1
## 
## (Dispersion parameter for binomial family taken to be 1)
## 
##     Null deviance: 136.42  on 98  degrees of freedom
## Residual deviance:  88.64  on 96  degrees of freedom
## AIC: 94.64
## 
## Number of Fisher Scoring iterations: 5
\end{verbatim}

La desvianza (deviance) es una medida de la bondad del ajuste de un modelo lineal generalizado (sería equivalente a la suma de cuadrados residual de un modelo lineal; valores más altos indican peor ajuste). La \emph{Null deviance} se correspondería con un modelo solo con la constante y la \emph{Residual deviance} con el modelo ajustado.
En este caso hay una reducción de 47.78 con una pérdida de 2 grados de libertad (una reducción significativa).

Para contrastar globalmente el efecto de las covariables también podemos emplear:

\begin{Shaded}
\begin{Highlighting}[]
\NormalTok{modelo.null }\OtherTok{\textless{}{-}} \FunctionTok{glm}\NormalTok{(nsatisfa }\SpecialCharTok{\textasciitilde{}} \DecValTok{1}\NormalTok{, binomial, datos)}
\FunctionTok{anova}\NormalTok{(modelo.null, modelo, }\AttributeTok{test =} \StringTok{"Chi"}\NormalTok{)}
\end{Highlighting}
\end{Shaded}

\begin{verbatim}
## Analysis of Deviance Table
## 
## Model 1: nsatisfa ~ 1
## Model 2: nsatisfa ~ velocida + imgfabri
##   Resid. Df Resid. Dev Df Deviance  Pr(>Chi)    
## 1        98     136.42                          
## 2        96      88.64  2   47.783 4.207e-11 ***
## ---
## Signif. codes:  0 '***' 0.001 '**' 0.01 '*' 0.05 '.' 0.1 ' ' 1
\end{verbatim}

\hypertarget{predicciuxf3n-1}{%
\section{Predicción}\label{predicciuxf3n-1}}

Las predicciones se obtienen también con la función \texttt{predict}:

\begin{Shaded}
\begin{Highlighting}[]
\NormalTok{p.est }\OtherTok{\textless{}{-}} \FunctionTok{predict}\NormalTok{(modelo, }\AttributeTok{type =} \StringTok{"response"}\NormalTok{)}
\end{Highlighting}
\end{Shaded}

El parámetro \texttt{type\ =\ "response"} permite calcular las probabilidades estimadas de la segunda categoría.

Podríamos obtener una tabla de clasificación:

\begin{Shaded}
\begin{Highlighting}[]
\NormalTok{cat.est }\OtherTok{\textless{}{-}} \FunctionTok{as.numeric}\NormalTok{(p.est }\SpecialCharTok{\textgreater{}} \FloatTok{0.5}\NormalTok{)}
\NormalTok{tabla }\OtherTok{\textless{}{-}} \FunctionTok{table}\NormalTok{(datos}\SpecialCharTok{$}\NormalTok{nsatisfa, cat.est)}
\NormalTok{tabla}
\end{Highlighting}
\end{Shaded}

\begin{verbatim}
##       cat.est
##         0  1
##   bajo 44 10
##   alto  7 38
\end{verbatim}

\begin{Shaded}
\begin{Highlighting}[]
\FunctionTok{print}\NormalTok{(}\DecValTok{100}\SpecialCharTok{*}\FunctionTok{prop.table}\NormalTok{(tabla), }\AttributeTok{digits =} \DecValTok{2}\NormalTok{)}
\end{Highlighting}
\end{Shaded}

\begin{verbatim}
##       cat.est
##           0    1
##   bajo 44.4 10.1
##   alto  7.1 38.4
\end{verbatim}

Por defecto \texttt{predict} obtiene las predicciones correspondientes a las observaciones (\texttt{modelo\$fitted.values}). Para otros casos hay que emplear el argumento \texttt{newdata}.

\hypertarget{selecciuxf3n-de-variables-explicativas-1}{%
\section{Selección de variables explicativas}\label{selecciuxf3n-de-variables-explicativas-1}}

El objetivo sería conseguir un buen ajuste con el menor número de variables explicativas posible.

Para actualizar un modelo (p.e. eliminando o añadiendo variables) se puede emplear la función \texttt{update}:

\begin{Shaded}
\begin{Highlighting}[]
\NormalTok{modelo.completo }\OtherTok{\textless{}{-}} \FunctionTok{glm}\NormalTok{(nsatisfa }\SpecialCharTok{\textasciitilde{}}\NormalTok{ . , }\AttributeTok{family =}\NormalTok{ binomial, }\AttributeTok{data =}\NormalTok{ datos)}
\FunctionTok{summary}\NormalTok{(modelo.completo)}
\end{Highlighting}
\end{Shaded}

\begin{verbatim}
## 
## Call:
## glm(formula = nsatisfa ~ ., family = binomial, data = datos)
## 
## Deviance Residuals: 
##      Min        1Q    Median        3Q       Max  
## -2.01370  -0.31260  -0.02826   0.35423   1.74741  
## 
## Coefficients:
##             Estimate Std. Error z value Pr(>|z|)    
## (Intercept) -32.6317     7.7121  -4.231 2.32e-05 ***
## velocida      3.9980     2.3362   1.711 0.087019 .  
## precio        3.6042     2.3184   1.555 0.120044    
## flexprec      1.5769     0.4433   3.557 0.000375 ***
## imgfabri      2.1669     0.6857   3.160 0.001576 ** 
## servconj     -4.2655     4.3526  -0.980 0.327096    
## imgfvent     -1.1496     0.8937  -1.286 0.198318    
## calidadp      0.1506     0.2495   0.604 0.546147    
## ---
## Signif. codes:  0 '***' 0.001 '**' 0.01 '*' 0.05 '.' 0.1 ' ' 1
## 
## (Dispersion parameter for binomial family taken to be 1)
## 
##     Null deviance: 136.424  on 98  degrees of freedom
## Residual deviance:  60.807  on 91  degrees of freedom
## AIC: 76.807
## 
## Number of Fisher Scoring iterations: 7
\end{verbatim}

\begin{Shaded}
\begin{Highlighting}[]
\NormalTok{modelo.reducido }\OtherTok{\textless{}{-}} \FunctionTok{update}\NormalTok{(modelo.completo, . }\SpecialCharTok{\textasciitilde{}}\NormalTok{ . }\SpecialCharTok{{-}}\NormalTok{ calidadp)}
\FunctionTok{summary}\NormalTok{(modelo.reducido)}
\end{Highlighting}
\end{Shaded}

\begin{verbatim}
## 
## Call:
## glm(formula = nsatisfa ~ velocida + precio + flexprec + imgfabri + 
##     servconj + imgfvent, family = binomial, data = datos)
## 
## Deviance Residuals: 
##     Min       1Q   Median       3Q      Max  
## -2.0920  -0.3518  -0.0280   0.3876   1.7885  
## 
## Coefficients:
##             Estimate Std. Error z value Pr(>|z|)    
## (Intercept) -31.6022     7.3962  -4.273 1.93e-05 ***
## velocida      4.1831     2.2077   1.895 0.058121 .  
## precio        3.8872     2.1685   1.793 0.073044 .  
## flexprec      1.5452     0.4361   3.543 0.000396 ***
## imgfabri      2.1984     0.6746   3.259 0.001119 ** 
## servconj     -4.6985     4.0597  -1.157 0.247125    
## imgfvent     -1.1387     0.8784  -1.296 0.194849    
## ---
## Signif. codes:  0 '***' 0.001 '**' 0.01 '*' 0.05 '.' 0.1 ' ' 1
## 
## (Dispersion parameter for binomial family taken to be 1)
## 
##     Null deviance: 136.424  on 98  degrees of freedom
## Residual deviance:  61.171  on 92  degrees of freedom
## AIC: 75.171
## 
## Number of Fisher Scoring iterations: 7
\end{verbatim}

Para obtener el modelo ``óptimo'' lo ideal sería evaluar todos los modelos posibles.
En este caso no se puede emplear la función \texttt{regsubsets} del paquete \texttt{leaps} (sólo para modelos lineales),
pero por ejemplo el paquete
\href{https://cran.r-project.org/web/packages/bestglm/vignettes/bestglm.pdf}{\texttt{bestglm}}
proporciona una herramienta equivalente (\texttt{bestglm()}).

\hypertarget{selecciuxf3n-por-pasos-1}{%
\subsection{Selección por pasos}\label{selecciuxf3n-por-pasos-1}}

La función \texttt{stepwise} del paquete \texttt{RcmdrMisc} (interfaz de \texttt{stepAIC} del paquete \texttt{MASS})
permite seleccionar el modelo por pasos según criterio AIC o BIC:

\begin{Shaded}
\begin{Highlighting}[]
\FunctionTok{library}\NormalTok{(MASS)}
\FunctionTok{library}\NormalTok{(RcmdrMisc)}
\NormalTok{modelo }\OtherTok{\textless{}{-}} \FunctionTok{stepwise}\NormalTok{(modelo.completo, }\AttributeTok{direction=}\StringTok{\textquotesingle{}backward/forward\textquotesingle{}}\NormalTok{, }\AttributeTok{criterion=}\StringTok{\textquotesingle{}BIC\textquotesingle{}}\NormalTok{)}
\end{Highlighting}
\end{Shaded}

\begin{verbatim}
## 
## Direction:  backward/forward
## Criterion:  BIC 
## 
## Start:  AIC=97.57
## nsatisfa ~ velocida + precio + flexprec + imgfabri + servconj + 
##     imgfvent + calidadp
## 
##            Df Deviance     AIC
## - calidadp  1   61.171  93.337
## - servconj  1   61.565  93.730
## - imgfvent  1   62.668  94.834
## - precio    1   62.712  94.878
## - velocida  1   63.105  95.271
## <none>          60.807  97.568
## - imgfabri  1   76.251 108.416
## - flexprec  1   82.443 114.609
## 
## Step:  AIC=93.34
## nsatisfa ~ velocida + precio + flexprec + imgfabri + servconj + 
##     imgfvent
## 
##            Df Deviance     AIC
## - servconj  1   62.205  89.776
## - imgfvent  1   63.055  90.625
## - precio    1   63.698  91.269
## - velocida  1   63.983  91.554
## <none>          61.171  93.337
## + calidadp  1   60.807  97.568
## - imgfabri  1   77.823 105.394
## - flexprec  1   82.461 110.032
## 
## Step:  AIC=89.78
## nsatisfa ~ velocida + precio + flexprec + imgfabri + imgfvent
## 
##            Df Deviance     AIC
## - imgfvent  1   64.646  87.622
## <none>          62.205  89.776
## + servconj  1   61.171  93.337
## + calidadp  1   61.565  93.730
## - imgfabri  1   78.425 101.401
## - precio    1   79.699 102.675
## - flexprec  1   82.978 105.954
## - velocida  1   88.731 111.706
## 
## Step:  AIC=87.62
## nsatisfa ~ velocida + precio + flexprec + imgfabri
## 
##            Df Deviance     AIC
## <none>          64.646  87.622
## + imgfvent  1   62.205  89.776
## + servconj  1   63.055  90.625
## + calidadp  1   63.890  91.460
## - precio    1   80.474  98.854
## - flexprec  1   83.663 102.044
## - imgfabri  1   85.208 103.588
## - velocida  1   89.641 108.021
\end{verbatim}

\begin{Shaded}
\begin{Highlighting}[]
\FunctionTok{summary}\NormalTok{(modelo)}
\end{Highlighting}
\end{Shaded}

\begin{verbatim}
## 
## Call:
## glm(formula = nsatisfa ~ velocida + precio + flexprec + imgfabri, 
##     family = binomial, data = datos)
## 
## Deviance Residuals: 
##      Min        1Q    Median        3Q       Max  
## -1.99422  -0.36209  -0.03932   0.44249   1.80432  
## 
## Coefficients:
##             Estimate Std. Error z value Pr(>|z|)    
## (Intercept) -28.0825     6.4767  -4.336 1.45e-05 ***
## velocida      1.6268     0.4268   3.812 0.000138 ***
## precio        1.3749     0.4231   3.250 0.001155 ** 
## flexprec      1.3364     0.3785   3.530 0.000415 ***
## imgfabri      1.5168     0.4252   3.567 0.000361 ***
## ---
## Signif. codes:  0 '***' 0.001 '**' 0.01 '*' 0.05 '.' 0.1 ' ' 1
## 
## (Dispersion parameter for binomial family taken to be 1)
## 
##     Null deviance: 136.424  on 98  degrees of freedom
## Residual deviance:  64.646  on 94  degrees of freedom
## AIC: 74.646
## 
## Number of Fisher Scoring iterations: 6
\end{verbatim}

\hypertarget{diagnosis-del-modelo-1}{%
\section{Diagnosis del modelo}\label{diagnosis-del-modelo-1}}

\hypertarget{gruxe1ficas-buxe1sicas-de-diagnuxf3stico-1}{%
\subsection{Gráficas básicas de diagnóstico}\label{gruxe1ficas-buxe1sicas-de-diagnuxf3stico-1}}

Con la función \texttt{plot} se pueden generar gráficos de interés para la diagnosis del modelo:

\begin{Shaded}
\begin{Highlighting}[]
\NormalTok{oldpar }\OtherTok{\textless{}{-}} \FunctionTok{par}\NormalTok{( }\AttributeTok{mfrow=}\FunctionTok{c}\NormalTok{(}\DecValTok{2}\NormalTok{,}\DecValTok{2}\NormalTok{))}
\FunctionTok{plot}\NormalTok{(modelo)}
\end{Highlighting}
\end{Shaded}

\includegraphics{09-ModelosGLM_files/figure-latex/unnamed-chunk-12-1.pdf}

\begin{Shaded}
\begin{Highlighting}[]
\FunctionTok{par}\NormalTok{(oldpar)}
\end{Highlighting}
\end{Shaded}

Aunque su interpretación difiere un poco de la de los modelos lineales\ldots{}

\hypertarget{gruxe1ficos-parciales-de-residuos-1}{%
\subsection{Gráficos parciales de residuos}\label{gruxe1ficos-parciales-de-residuos-1}}

Se pueden generar gráficos parciales de residuos (p.e. \texttt{crPlots()} del paquete \texttt{car}):

\begin{Shaded}
\begin{Highlighting}[]
\CommentTok{\# library(car)}
\FunctionTok{crPlots}\NormalTok{(modelo)}
\end{Highlighting}
\end{Shaded}

\includegraphics{09-ModelosGLM_files/figure-latex/unnamed-chunk-13-1.pdf}

\hypertarget{estaduxedsticos-1}{%
\subsection{Estadísticos}\label{estaduxedsticos-1}}

Se pueden emplear las mismas funciones vistas en los modelos lineales para obtener medidas de diagnosis de interés (ver \texttt{help(influence.measures)}). Por ejemplo:

\begin{Shaded}
\begin{Highlighting}[]
\FunctionTok{residuals}\NormalTok{(model, }\AttributeTok{type =} \StringTok{"deviance"}\NormalTok{) }
\end{Highlighting}
\end{Shaded}

proporciona los residuos deviance.

En general, muchas de las herramientas para modelos lineales son también válidas para estos modelos. Por ejemplo:

\begin{Shaded}
\begin{Highlighting}[]
\CommentTok{\# library(car)}
\FunctionTok{vif}\NormalTok{(modelo)}
\end{Highlighting}
\end{Shaded}

\begin{verbatim}
## velocida   precio flexprec imgfabri 
## 2.088609 2.653934 2.520042 1.930409
\end{verbatim}

\hypertarget{alternativas-1}{%
\section{Alternativas}\label{alternativas-1}}

Además de considerar ajustes polinómicos, pueden ser de interés emplear métodos no paramétricos. Por ejemplo, puede ser recomendable la función \texttt{gam} del paquete \texttt{mgcv}.

\hypertarget{regresiuxf3n-no-paramuxe9trica}{%
\chapter{Regresión no paramétrica}\label{regresiuxf3n-no-paramuxe9trica}}

No se supone ninguna forma concreta en el efecto de las variables explicativas:
\[Y=f\left(  \mathbf{X}\right)  +\varepsilon,\]
con \emph{f} función ``cualquiera'' (suave).

\begin{itemize}
\item
  Métodos disponibles en \texttt{R}:

  \begin{itemize}
  \item
    Regresión local (métodos de suavizado): \texttt{loess()}, \texttt{KernSmooth}, \texttt{sm}, \ldots{}
  \item
    Modelos aditivos generalizados (GAM): \texttt{gam}, \texttt{mgcv}, \ldots{}
  \item
    \ldots{}
  \end{itemize}
\end{itemize}

\hypertarget{modelos-aditivos}{%
\section{Modelos aditivos}\label{modelos-aditivos}}

Se supone que:
\[Y=\beta_{0}+f_{1}\left(  \mathbf{X}_{1}\right)  +f_{2}\left(  \mathbf{X}_{2}\right)  +\cdots+f_{p}\left(  \mathbf{X}_{p}\right)  +\varepsilon\text{,}\]
con \(f_{i},\) \(i=1,...,p,\) funciones cualesquiera.

\begin{itemize}
\item
  Los modelos lineales son un caso particular considerando \(f_{i}(x) = \beta_{i}·x\).
\item
  Adicionalmente se puede considerar una función link: \textbf{Modelos aditivos generalizados} (GAM)

  \begin{itemize}
  \item
    Hastie, T.J. y Tibshirani, R.J. (1990). Generalized Additive Models. Chapman \& Hall.
  \item
    Wood, S. N. (2006). Generalized Additive Models: An Introduction with R. Chapman \& Hall/CRC
  \end{itemize}
\end{itemize}

\hypertarget{ajuste-funciuxf3n-gam}{%
\subsection{\texorpdfstring{Ajuste: función \texttt{gam}}{Ajuste: función gam}}\label{ajuste-funciuxf3n-gam}}

La función \texttt{gam} del paquete \texttt{mgcv} permite ajustar modelos aditivos (generalizados) empleando regresión por splines (ver \texttt{help("mgcv-package")}):

\begin{Shaded}
\begin{Highlighting}[]
\FunctionTok{library}\NormalTok{(mgcv)}
\NormalTok{ajuste }\OtherTok{\textless{}{-}} \FunctionTok{gam}\NormalTok{(formula, }\AttributeTok{family =}\NormalTok{ gaussian, datos, pesos, seleccion, na.action, ...)}
\end{Highlighting}
\end{Shaded}

Algunas posibilidades de uso son las que siguen:

\begin{itemize}
\item
  Modelo lineal:

\begin{Shaded}
\begin{Highlighting}[]
\NormalTok{ajuste }\OtherTok{\textless{}{-}} \FunctionTok{gam}\NormalTok{(y }\SpecialCharTok{\textasciitilde{}}\NormalTok{ x1 }\SpecialCharTok{+}\NormalTok{ x2 }\SpecialCharTok{+}\NormalTok{ x3)}
\end{Highlighting}
\end{Shaded}
\item
  Modelo aditivo con efectos no paramétricos para x1 y x2, y un efecto lineal para x3:

\begin{Shaded}
\begin{Highlighting}[]
\NormalTok{ajuste }\OtherTok{\textless{}{-}} \FunctionTok{gam}\NormalTok{(y }\SpecialCharTok{\textasciitilde{}} \FunctionTok{s}\NormalTok{(x1) }\SpecialCharTok{+} \FunctionTok{s}\NormalTok{(x2) }\SpecialCharTok{+}\NormalTok{ x3)}
\end{Highlighting}
\end{Shaded}
\item
  Modelo no aditivo (con interacción):

\begin{Shaded}
\begin{Highlighting}[]
\NormalTok{ajuste }\OtherTok{\textless{}{-}} \FunctionTok{gam}\NormalTok{(y }\SpecialCharTok{\textasciitilde{}} \FunctionTok{s}\NormalTok{(x1, x2))}
\end{Highlighting}
\end{Shaded}
\item
  Modelo con distintas combinaciones:

\begin{Shaded}
\begin{Highlighting}[]
\NormalTok{ajuste }\OtherTok{\textless{}{-}} \FunctionTok{gam}\NormalTok{(y }\SpecialCharTok{\textasciitilde{}} \FunctionTok{s}\NormalTok{(x1, x2) }\SpecialCharTok{+} \FunctionTok{s}\NormalTok{(x3) }\SpecialCharTok{+}\NormalTok{ x4)}
\end{Highlighting}
\end{Shaded}
\end{itemize}

\hypertarget{ejemplo-3}{%
\subsection{Ejemplo}\label{ejemplo-3}}

En esta sección utilizaremos como ejemplo el conjunto de datos \texttt{Prestige} de la librería \texttt{car}.
Se tratará de explicar \texttt{prestige} (puntuación de ocupaciones obtenidas a partir de una encuesta )
a partir de \texttt{income} (media de ingresos en la ocupación) y \texttt{education} (media de los años de
educación).

\begin{Shaded}
\begin{Highlighting}[]
\FunctionTok{library}\NormalTok{(mgcv)}
\FunctionTok{library}\NormalTok{(car)}
\NormalTok{modelo }\OtherTok{\textless{}{-}} \FunctionTok{gam}\NormalTok{(prestige }\SpecialCharTok{\textasciitilde{}} \FunctionTok{s}\NormalTok{(income) }\SpecialCharTok{+} \FunctionTok{s}\NormalTok{(education), }\AttributeTok{data =}\NormalTok{ Prestige)}
\FunctionTok{summary}\NormalTok{(modelo)}
\end{Highlighting}
\end{Shaded}

\begin{verbatim}
## 
## Family: gaussian 
## Link function: identity 
## 
## Formula:
## prestige ~ s(income) + s(education)
## 
## Parametric coefficients:
##             Estimate Std. Error t value Pr(>|t|)    
## (Intercept)  46.8333     0.6889   67.98   <2e-16 ***
## ---
## Signif. codes:  0 '***' 0.001 '**' 0.01 '*' 0.05 '.' 0.1 ' ' 1
## 
## Approximate significance of smooth terms:
##                edf Ref.df     F p-value    
## s(income)    3.118  3.877 14.61  <2e-16 ***
## s(education) 3.177  3.952 38.78  <2e-16 ***
## ---
## Signif. codes:  0 '***' 0.001 '**' 0.01 '*' 0.05 '.' 0.1 ' ' 1
## 
## R-sq.(adj) =  0.836   Deviance explained = 84.7%
## GCV = 52.143  Scale est. = 48.414    n = 102
\end{verbatim}

En este caso la función \texttt{plot} representa los efectos (parciales) estimados de cada covariable:

\begin{Shaded}
\begin{Highlighting}[]
\NormalTok{par.old }\OtherTok{\textless{}{-}} \FunctionTok{par}\NormalTok{(}\AttributeTok{mfrow =} \FunctionTok{c}\NormalTok{(}\DecValTok{1}\NormalTok{, }\DecValTok{2}\NormalTok{))}
\FunctionTok{plot}\NormalTok{(modelo, }\AttributeTok{shade =} \ConstantTok{TRUE}\NormalTok{) }\CommentTok{\# }
\end{Highlighting}
\end{Shaded}

\includegraphics{10-ModelosNP_files/figure-latex/unnamed-chunk-7-1.pdf}

\begin{Shaded}
\begin{Highlighting}[]
\FunctionTok{par}\NormalTok{(par.old)}
\end{Highlighting}
\end{Shaded}

\hypertarget{superficie-de-predicciuxf3n}{%
\subsection{Superficie de predicción}\label{superficie-de-predicciuxf3n}}

Las predicciones se obtienen también con la función \texttt{predict}:

\begin{Shaded}
\begin{Highlighting}[]
\NormalTok{pred }\OtherTok{\textless{}{-}} \FunctionTok{predict}\NormalTok{(modelo)}
\end{Highlighting}
\end{Shaded}

Por defecto \texttt{predict} obtiene las predicciones correspondientes a las observaciones (\texttt{modelo\$fitted.values}). Para otros casos hay que emplear el argumento \texttt{newdata}.

Para representar las estimaciones (la superficie de predicción) obtenidas con el modelo se puede
utilizar la función \texttt{persp}. Esta función necesita que los valores (x,y) de entrada estén
dispuestos en una rejilla bidimensional. Para generar esta rejilla se puede emplear la función \texttt{expand.grid(x,y)} que crea todas las combinaciones de los puntos dados en x e y.

\begin{Shaded}
\begin{Highlighting}[]
\NormalTok{inc }\OtherTok{\textless{}{-}} \FunctionTok{with}\NormalTok{(Prestige, }\FunctionTok{seq}\NormalTok{(}\FunctionTok{min}\NormalTok{(income), }\FunctionTok{max}\NormalTok{(income), }\AttributeTok{len =} \DecValTok{25}\NormalTok{))}
\NormalTok{ed }\OtherTok{\textless{}{-}} \FunctionTok{with}\NormalTok{(Prestige, }\FunctionTok{seq}\NormalTok{(}\FunctionTok{min}\NormalTok{(education), }\FunctionTok{max}\NormalTok{(education), }\AttributeTok{len =} \DecValTok{25}\NormalTok{))}
\NormalTok{newdata }\OtherTok{\textless{}{-}} \FunctionTok{expand.grid}\NormalTok{(}\AttributeTok{income =}\NormalTok{ inc, }\AttributeTok{education =}\NormalTok{ ed)}
\CommentTok{\# Representamos la rejilla}
\FunctionTok{plot}\NormalTok{(income }\SpecialCharTok{\textasciitilde{}}\NormalTok{ education, Prestige, }\AttributeTok{pch =} \DecValTok{16}\NormalTok{)}
\FunctionTok{abline}\NormalTok{(}\AttributeTok{h =}\NormalTok{ inc, }\AttributeTok{v =}\NormalTok{ ed, }\AttributeTok{col =} \StringTok{"grey"}\NormalTok{)}
\end{Highlighting}
\end{Shaded}

\includegraphics{10-ModelosNP_files/figure-latex/unnamed-chunk-9-1.pdf}

\begin{Shaded}
\begin{Highlighting}[]
\CommentTok{\# Se calculan las predicciones}
\NormalTok{pred }\OtherTok{\textless{}{-}} \FunctionTok{predict}\NormalTok{(modelo, newdata)}
\CommentTok{\# Se representan}
\NormalTok{pred }\OtherTok{\textless{}{-}} \FunctionTok{matrix}\NormalTok{(pred, }\AttributeTok{nrow =} \DecValTok{25}\NormalTok{)}
\FunctionTok{persp}\NormalTok{(inc, ed, pred, }\AttributeTok{theta =} \SpecialCharTok{{-}}\DecValTok{40}\NormalTok{, }\AttributeTok{phi =} \DecValTok{30}\NormalTok{)}
\end{Highlighting}
\end{Shaded}

\includegraphics{10-ModelosNP_files/figure-latex/unnamed-chunk-9-2.pdf}

Alternativamente se podría emplear la función \texttt{contour} o \texttt{filled.contour}:

\begin{Shaded}
\begin{Highlighting}[]
\CommentTok{\# contour(inc, ed, pred, xlab = "Income", ylab = "Education")}
\FunctionTok{filled.contour}\NormalTok{(inc, ed, pred, }\AttributeTok{xlab =} \StringTok{"Income"}\NormalTok{, }\AttributeTok{ylab =} \StringTok{"Education"}\NormalTok{, }\AttributeTok{key.title =} \FunctionTok{title}\NormalTok{(}\StringTok{"Prestige"}\NormalTok{))}
\end{Highlighting}
\end{Shaded}

\includegraphics{10-ModelosNP_files/figure-latex/unnamed-chunk-10-1.pdf}

Puede ser más cómodo emplear el paquete \href{https://github.com/hadley/modelr}{\texttt{modelr}} junto a los gráficos \texttt{ggplot2} para trabajar con modelos y predicciones.

\hypertarget{comparaciuxf3n-de-modelos}{%
\subsection{Comparación de modelos}\label{comparaciuxf3n-de-modelos}}

Además de las medidas de bondad de ajuste como el coeficiente de determinación ajustado, también se puede emplear la función \texttt{anova} para la comparación de modelos.
Por ejemplo, viendo el gráfico de los efectos se podría pensar que el efecto de \texttt{education} podría ser lineal:

\begin{Shaded}
\begin{Highlighting}[]
\CommentTok{\# plot(modelo)}
\NormalTok{modelo0 }\OtherTok{\textless{}{-}} \FunctionTok{gam}\NormalTok{(prestige }\SpecialCharTok{\textasciitilde{}} \FunctionTok{s}\NormalTok{(income) }\SpecialCharTok{+}\NormalTok{ education, }\AttributeTok{data =}\NormalTok{ Prestige)}
\FunctionTok{summary}\NormalTok{(modelo0)}
\end{Highlighting}
\end{Shaded}

\begin{verbatim}
## 
## Family: gaussian 
## Link function: identity 
## 
## Formula:
## prestige ~ s(income) + education
## 
## Parametric coefficients:
##             Estimate Std. Error t value Pr(>|t|)    
## (Intercept)   4.2240     3.7323   1.132    0.261    
## education     3.9681     0.3412  11.630   <2e-16 ***
## ---
## Signif. codes:  0 '***' 0.001 '**' 0.01 '*' 0.05 '.' 0.1 ' ' 1
## 
## Approximate significance of smooth terms:
##            edf Ref.df    F p-value    
## s(income) 3.58  4.441 13.6  <2e-16 ***
## ---
## Signif. codes:  0 '***' 0.001 '**' 0.01 '*' 0.05 '.' 0.1 ' ' 1
## 
## R-sq.(adj) =  0.825   Deviance explained = 83.3%
## GCV = 54.798  Scale est. = 51.8      n = 102
\end{verbatim}

\begin{Shaded}
\begin{Highlighting}[]
\FunctionTok{anova}\NormalTok{(modelo0, modelo, }\AttributeTok{test=}\StringTok{"F"}\NormalTok{)}
\end{Highlighting}
\end{Shaded}

\begin{verbatim}
## Analysis of Deviance Table
## 
## Model 1: prestige ~ s(income) + education
## Model 2: prestige ~ s(income) + s(education)
##   Resid. Df Resid. Dev     Df Deviance      F Pr(>F)  
## 1    95.559     4994.6                                
## 2    93.171     4585.0 2.3886   409.58 3.5418 0.0257 *
## ---
## Signif. codes:  0 '***' 0.001 '**' 0.01 '*' 0.05 '.' 0.1 ' ' 1
\end{verbatim}

En este caso aceptaríamos que el modelo original es significativamente mejor.

Alternativamente, podríamos pensar que hay interacción:

\begin{Shaded}
\begin{Highlighting}[]
\NormalTok{modelo2 }\OtherTok{\textless{}{-}} \FunctionTok{gam}\NormalTok{(prestige }\SpecialCharTok{\textasciitilde{}} \FunctionTok{s}\NormalTok{(income, education), }\AttributeTok{data =}\NormalTok{ Prestige)}
\FunctionTok{summary}\NormalTok{(modelo2)}
\end{Highlighting}
\end{Shaded}

\begin{verbatim}
## 
## Family: gaussian 
## Link function: identity 
## 
## Formula:
## prestige ~ s(income, education)
## 
## Parametric coefficients:
##             Estimate Std. Error t value Pr(>|t|)    
## (Intercept)  46.8333     0.7138   65.61   <2e-16 ***
## ---
## Signif. codes:  0 '***' 0.001 '**' 0.01 '*' 0.05 '.' 0.1 ' ' 1
## 
## Approximate significance of smooth terms:
##                      edf Ref.df     F p-value    
## s(income,education) 4.94  6.303 75.41  <2e-16 ***
## ---
## Signif. codes:  0 '***' 0.001 '**' 0.01 '*' 0.05 '.' 0.1 ' ' 1
## 
## R-sq.(adj) =  0.824   Deviance explained = 83.3%
## GCV = 55.188  Scale est. = 51.974    n = 102
\end{verbatim}

\begin{Shaded}
\begin{Highlighting}[]
\CommentTok{\# plot(modelo2, se = FALSE)}
\end{Highlighting}
\end{Shaded}

En este caso el coeficiente de determinación ajustado es menor\ldots{}

\hypertarget{diagnosis-del-modelo-2}{%
\subsection{Diagnosis del modelo}\label{diagnosis-del-modelo-2}}

La función \texttt{gam.check} realiza una diagnosis del modelo:

\begin{Shaded}
\begin{Highlighting}[]
\FunctionTok{gam.check}\NormalTok{(modelo)}
\end{Highlighting}
\end{Shaded}

\includegraphics{10-ModelosNP_files/figure-latex/unnamed-chunk-13-1.pdf}

\begin{verbatim}
## 
## Method: GCV   Optimizer: magic
## Smoothing parameter selection converged after 4 iterations.
## The RMS GCV score gradient at convergence was 9.783945e-05 .
## The Hessian was positive definite.
## Model rank =  19 / 19 
## 
## Basis dimension (k) checking results. Low p-value (k-index<1) may
## indicate that k is too low, especially if edf is close to k'.
## 
##                k'  edf k-index p-value
## s(income)    9.00 3.12    0.98    0.44
## s(education) 9.00 3.18    1.03    0.57
\end{verbatim}

Lo ideal sería observar normalidad en los dos gráficos de la izquierda, falta de patrón en el superior derecho, y ajuste a una recta en el inferior derecho. En este caso parece que el modelo se comporta adecuadamente.

\hypertarget{programacion}{%
\chapter{Programación}\label{programacion}}

En este capítulo se introducirán los comandos básicos de programación en R\ldots{}

\hypertarget{funciones}{%
\section{Funciones}\label{funciones}}

El lenguaje \texttt{R} permite al usuario
definir sus propias funciones. El esquema de una función es el que
sigue:

\begin{Shaded}
\begin{Highlighting}[]
\NormalTok{nombre }\OtherTok{\textless{}{-}} \ControlFlowTok{function}\NormalTok{(arg1, arg2, ... ) \{expresión\}}
\end{Highlighting}
\end{Shaded}

\begin{itemize}
\item
  En la expresión anterior \texttt{arg1,\ arg2,\ ...} son los
  argumentos de entrada (también llamados parámetros).
\item
  La \texttt{expresión} está compuesta de comandos que utilizan los
  argumentos de entrada para dar la \textbf{salida} deseada.
\item
  La salida de una función puese ser un número, un vector, una
  grafica, un mensaje, etc.
\end{itemize}

\hypertarget{ejemplo-progresiuxf3n-geomuxe9trica}{%
\subsection{Ejemplo: progresión geométrica}\label{ejemplo-progresiuxf3n-geomuxe9trica}}

Para introducirnos en las
funciones, vamos a escribir una función que permita trabajar con las
llamadas \textbf{progresiones geométricas}.

Una progresión geométrica es una sucesión de números \(a_1, a_2, a_3\ldots\)
tales que cada uno de ellos (salvo el primero) es igual al anterior
multiplicado por una constante llamada \textbf{razón}, que representaremos
por \(r\). Ejemplos:

\begin{itemize}
\item
  \(a_1=1\), \(r=2\):

  1, 2, 4, 8, 16,\ldots{}
\item
  \(a_1=-1\), \(r=-2\):

  1, -2, 4, -8, 16,\ldots{}
\end{itemize}

Según la definición anterior, se verifica que:
\[a_2=a_1\cdot r; \quad a_3=a_2\cdot r=a_1\cdot r^2; \quad ...\] y
generalizando este proceso se obtiene el llamado término general:

\[a_n=a_1\cdot r^{n-1}\]

También se puede comprobar que la suma de los \(n\) términos de la
progresión es:

\[S_n=a_1+\ldots_+a_n=\frac{a_1(r^n-1)}{r-1}\]

La siguiente función, que llamaremos \texttt{an} calcula el término
\(a_n\) de una progresión geométrica pasando como entrada el primer
elemento \texttt{a1}, la razón \texttt{r} y el valor \texttt{n}:

\begin{Shaded}
\begin{Highlighting}[]
\NormalTok{an }\OtherTok{\textless{}{-}} \ControlFlowTok{function}\NormalTok{(a1, r, n) \{}
\NormalTok{        a1 }\SpecialCharTok{*}\NormalTok{ r}\SpecialCharTok{\^{}}\NormalTok{(n }\SpecialCharTok{{-}} \DecValTok{1}\NormalTok{)}
\NormalTok{      \}}
\end{Highlighting}
\end{Shaded}

A continuación algún ejemplo para comprobar su funcionamiento:

\begin{Shaded}
\begin{Highlighting}[]
\FunctionTok{an}\NormalTok{(}\AttributeTok{a1 =} \DecValTok{1}\NormalTok{, }\AttributeTok{r =} \DecValTok{2}\NormalTok{, }\AttributeTok{n =} \DecValTok{5}\NormalTok{)}
\end{Highlighting}
\end{Shaded}

\begin{verbatim}
## [1] 16
\end{verbatim}

\begin{Shaded}
\begin{Highlighting}[]
\FunctionTok{an}\NormalTok{(}\AttributeTok{a1 =} \DecValTok{4}\NormalTok{, }\AttributeTok{r =} \SpecialCharTok{{-}}\DecValTok{2}\NormalTok{, }\AttributeTok{n =} \DecValTok{6}\NormalTok{)}
\end{Highlighting}
\end{Shaded}

\begin{verbatim}
## [1] -128
\end{verbatim}

\begin{Shaded}
\begin{Highlighting}[]
\FunctionTok{an}\NormalTok{(}\AttributeTok{a1 =} \SpecialCharTok{{-}}\DecValTok{50}\NormalTok{, }\AttributeTok{r =} \DecValTok{4}\NormalTok{, }\AttributeTok{n =} \DecValTok{6}\NormalTok{)}
\end{Highlighting}
\end{Shaded}

\begin{verbatim}
## [1] -51200
\end{verbatim}

Con la función anterior se pueden obtener, con una sola llamada, varios valores de
la progresión:

\begin{Shaded}
\begin{Highlighting}[]
\FunctionTok{an}\NormalTok{(}\AttributeTok{a1 =} \DecValTok{1}\NormalTok{, }\AttributeTok{r =} \DecValTok{2}\NormalTok{, }\AttributeTok{n =} \DecValTok{1}\SpecialCharTok{:}\DecValTok{5}\NormalTok{)    }\CommentTok{\# a1, ..., a5}
\end{Highlighting}
\end{Shaded}

\begin{verbatim}
## [1]  1  2  4  8 16
\end{verbatim}

\begin{Shaded}
\begin{Highlighting}[]
\FunctionTok{an}\NormalTok{(}\AttributeTok{a1 =} \DecValTok{1}\NormalTok{, }\AttributeTok{r =} \DecValTok{2}\NormalTok{, }\AttributeTok{n =} \DecValTok{10}\SpecialCharTok{:}\DecValTok{15}\NormalTok{)  }\CommentTok{\# a10, ..., a15}
\end{Highlighting}
\end{Shaded}

\begin{verbatim}
## [1]   512  1024  2048  4096  8192 16384
\end{verbatim}

La función \texttt{Sn} calcula la suma de los primeros \texttt{n}
elementos de la progresión:

\begin{Shaded}
\begin{Highlighting}[]
\NormalTok{Sn }\OtherTok{\textless{}{-}} \ControlFlowTok{function}\NormalTok{(a1, r, n) \{}
\NormalTok{        a1 }\SpecialCharTok{*}\NormalTok{ (r}\SpecialCharTok{\^{}}\NormalTok{n }\SpecialCharTok{{-}} \DecValTok{1}\NormalTok{) }\SpecialCharTok{/}\NormalTok{ (r }\SpecialCharTok{{-}} \DecValTok{1}\NormalTok{)}
\NormalTok{      \}}
  
\FunctionTok{Sn}\NormalTok{(}\AttributeTok{a1 =} \DecValTok{1}\NormalTok{, }\AttributeTok{r =} \DecValTok{2}\NormalTok{, }\AttributeTok{n =} \DecValTok{5}\NormalTok{)}
\end{Highlighting}
\end{Shaded}

\begin{verbatim}
## [1] 31
\end{verbatim}

\begin{Shaded}
\begin{Highlighting}[]
\FunctionTok{an}\NormalTok{(}\AttributeTok{a1 =} \DecValTok{1}\NormalTok{, }\AttributeTok{r =} \DecValTok{2}\NormalTok{, }\AttributeTok{n =} \DecValTok{1}\SpecialCharTok{:}\DecValTok{5}\NormalTok{)    }\CommentTok{\# Valores de la progresión}
\end{Highlighting}
\end{Shaded}

\begin{verbatim}
## [1]  1  2  4  8 16
\end{verbatim}

\begin{Shaded}
\begin{Highlighting}[]
\FunctionTok{Sn}\NormalTok{(}\AttributeTok{a1 =} \DecValTok{1}\NormalTok{, }\AttributeTok{r =} \DecValTok{2}\NormalTok{, }\AttributeTok{n =} \DecValTok{1}\SpecialCharTok{:}\DecValTok{5}\NormalTok{)    }\CommentTok{\# Suma de los valores}
\end{Highlighting}
\end{Shaded}

\begin{verbatim}
## [1]  1  3  7 15 31
\end{verbatim}

\begin{Shaded}
\begin{Highlighting}[]
\CommentTok{\# cumsum(an(a1 = 1, r = 2, n = 1:5))}
\end{Highlighting}
\end{Shaded}

\hypertarget{argumentos-de-entrada}{%
\subsection{Argumentos de entrada}\label{argumentos-de-entrada}}

Como ya hemos comentado, los
argumentos son los valores de entrada de una función.

\begin{itemize}
\item
  Por ejemplo, en la función anterior:

\begin{Shaded}
\begin{Highlighting}[]
\NormalTok{an }\OtherTok{\textless{}{-}} \ControlFlowTok{function}\NormalTok{(a1, r, n) \{a1 }\SpecialCharTok{*}\NormalTok{ r}\SpecialCharTok{\^{}}\NormalTok{(n }\SpecialCharTok{{-}} \DecValTok{1}\NormalTok{)\}}
\end{Highlighting}
\end{Shaded}

  los argumentos de entrada son \texttt{a1}, \texttt{r} y \texttt{n}.
\end{itemize}

Veamos alguna consideración sobre los argumentos:

\begin{itemize}
\item
  No es necesario utilizar el nombre de los argumentos. En este caso
  es obligatorio mantener el orden de entrada.
  Por ejemplo, las siguientes llamadas son equivalentes:

\begin{Shaded}
\begin{Highlighting}[]
\FunctionTok{an}\NormalTok{(}\DecValTok{1}\NormalTok{, }\DecValTok{2}\NormalTok{, }\DecValTok{5}\NormalTok{)}
\end{Highlighting}
\end{Shaded}

\begin{verbatim}
## [1] 16
\end{verbatim}

\begin{Shaded}
\begin{Highlighting}[]
\FunctionTok{an}\NormalTok{(}\AttributeTok{a1 =} \DecValTok{1}\NormalTok{, }\AttributeTok{r =} \DecValTok{2}\NormalTok{, }\AttributeTok{n =} \DecValTok{5}\NormalTok{)}
\end{Highlighting}
\end{Shaded}

\begin{verbatim}
## [1] 16
\end{verbatim}
\item
  Si se nombran los argumentos, se pueden pasar en cualquier orden:

\begin{Shaded}
\begin{Highlighting}[]
\FunctionTok{an}\NormalTok{(}\AttributeTok{r =} \DecValTok{2}\NormalTok{, }\AttributeTok{n =} \DecValTok{5}\NormalTok{, }\AttributeTok{a1 =} \DecValTok{1}\NormalTok{)}
\end{Highlighting}
\end{Shaded}

\begin{verbatim}
## [1] 16
\end{verbatim}

\begin{Shaded}
\begin{Highlighting}[]
\FunctionTok{an}\NormalTok{(}\AttributeTok{n =} \DecValTok{5}\NormalTok{, }\AttributeTok{r =} \DecValTok{2}\NormalTok{, }\AttributeTok{a1 =} \DecValTok{1}\NormalTok{)}
\end{Highlighting}
\end{Shaded}

\begin{verbatim}
## [1] 16
\end{verbatim}
\end{itemize}

\hypertarget{argumentos-por-defecto}{%
\subsubsection{Argumentos por defecto}\label{argumentos-por-defecto}}

En muchas ocasiones
resulta muy interesante que las funciones tengan argumentos por defecto.

Por ejemplo, si se quiere que en una función:

\begin{Shaded}
\begin{Highlighting}[]
\NormalTok{nombre }\OtherTok{\textless{}{-}} \ControlFlowTok{function}\NormalTok{(arg1, arg2, arg3, arg4, ...) \{ expresión \}}
\end{Highlighting}
\end{Shaded}

los argumento \texttt{arg2} y \texttt{arg3} tomen por defecto los valores \texttt{a} y \texttt{b}
respectivamentebastaría con escribir:

\begin{Shaded}
\begin{Highlighting}[]
\NormalTok{nombre }\OtherTok{\textless{}{-}} \ControlFlowTok{function}\NormalTok{(arg1, }\AttributeTok{arg2 =}\NormalTok{ a, }\AttributeTok{arg3 =}\NormalTok{ b, arg4, ...) \{ expresión \}}
\end{Highlighting}
\end{Shaded}

Para comprender mejor esto considérese el siguiente ejemplo ilustrativo:

\begin{Shaded}
\begin{Highlighting}[]
\NormalTok{xy2 }\OtherTok{\textless{}{-}} \ControlFlowTok{function}\NormalTok{(}\AttributeTok{x =} \DecValTok{2}\NormalTok{, }\AttributeTok{y =} \DecValTok{3}\NormalTok{) \{ x }\SpecialCharTok{*}\NormalTok{ y}\SpecialCharTok{\^{}}\DecValTok{2}\NormalTok{ \}}
\FunctionTok{xy2}\NormalTok{()}
\end{Highlighting}
\end{Shaded}

\begin{verbatim}
## [1] 18
\end{verbatim}

\begin{Shaded}
\begin{Highlighting}[]
\FunctionTok{xy2}\NormalTok{(}\AttributeTok{x =} \DecValTok{1}\NormalTok{, }\AttributeTok{y =} \DecValTok{4}\NormalTok{)}
\end{Highlighting}
\end{Shaded}

\begin{verbatim}
## [1] 16
\end{verbatim}

\begin{Shaded}
\begin{Highlighting}[]
\FunctionTok{xy2}\NormalTok{(}\AttributeTok{y =} \DecValTok{4}\NormalTok{)}
\end{Highlighting}
\end{Shaded}

\begin{verbatim}
## [1] 32
\end{verbatim}

\hypertarget{el-argumento-...}{%
\subsubsection{\texorpdfstring{El argumento \texttt{...}}{El argumento ...}}\label{el-argumento-...}}

El argumento ``\texttt{...}'' permite
pasar de manera ``libre'' argumentos adicionales para ser utilizados por otra ``subfunción''
dentro de la función principal.

Por ejemplo, en la función:

\begin{Shaded}
\begin{Highlighting}[]
\NormalTok{Density.Plot }\OtherTok{\textless{}{-}} \ControlFlowTok{function}\NormalTok{(datos, ...) \{ }\FunctionTok{plot}\NormalTok{(}\FunctionTok{density}\NormalTok{(datos), ...) \}}
\end{Highlighting}
\end{Shaded}

a partir del primer argumento, los argumentos se incluirán en \texttt{...}
y serán utilizados por la función \texttt{plot}.

\begin{Shaded}
\begin{Highlighting}[]
\FunctionTok{data}\NormalTok{(cars)}
\FunctionTok{Density.Plot}\NormalTok{(cars}\SpecialCharTok{$}\NormalTok{speed)}
\end{Highlighting}
\end{Shaded}

\begin{center}\includegraphics[width=0.7\linewidth]{11-Programacion_files/figure-latex/unnamed-chunk-14-1} \end{center}

\begin{Shaded}
\begin{Highlighting}[]
\FunctionTok{Density.Plot}\NormalTok{(cars}\SpecialCharTok{$}\NormalTok{speed, }\AttributeTok{col =} \StringTok{\textquotesingle{}red\textquotesingle{}}\NormalTok{, }\AttributeTok{xlab =} \StringTok{"velocidad"}\NormalTok{, }\AttributeTok{ylab =} \StringTok{"distancia"}\NormalTok{)}
\end{Highlighting}
\end{Shaded}

\begin{center}\includegraphics[width=0.7\linewidth]{11-Programacion_files/figure-latex/unnamed-chunk-14-2} \end{center}

Los argumentos de entrada de una función se obtienen ejecutando \texttt{args(funcion)}:

\begin{Shaded}
\begin{Highlighting}[]
\FunctionTok{args}\NormalTok{(an)}
\end{Highlighting}
\end{Shaded}

\begin{verbatim}
## function (a1, r, n) 
## NULL
\end{verbatim}

\begin{Shaded}
\begin{Highlighting}[]
\FunctionTok{args}\NormalTok{(xy2)}
\end{Highlighting}
\end{Shaded}

\begin{verbatim}
## function (x = 2, y = 3) 
## NULL
\end{verbatim}

\begin{Shaded}
\begin{Highlighting}[]
\FunctionTok{str}\NormalTok{(}\FunctionTok{args}\NormalTok{(Density.Plot))}
\end{Highlighting}
\end{Shaded}

\begin{verbatim}
## function (datos, ...)
\end{verbatim}

Por otro lado, al escribir el nombre de una función se obtiene su
contenido:

\begin{Shaded}
\begin{Highlighting}[]
\NormalTok{an}
\end{Highlighting}
\end{Shaded}

\begin{verbatim}
## function(a1, r, n) {
##         a1 * r^(n - 1)
##       }
## <bytecode: 0x000000001d6ff5f0>
\end{verbatim}

\hypertarget{salida}{%
\subsection{Salida}\label{salida}}

El valor que devolverá una función será:

\begin{itemize}
\item
  el último objeto evaluado dentro de ella, o
\item
  lo indicado dentro de la sentencia \texttt{return}.
\end{itemize}

Como las funciones pueden devolver objetos de varios tipos es hatibual
que la salida sea una lista.

\begin{Shaded}
\begin{Highlighting}[]
\NormalTok{an }\OtherTok{\textless{}{-}} \ControlFlowTok{function}\NormalTok{(a1, r, n) \{ a1 }\SpecialCharTok{*}\NormalTok{ r}\SpecialCharTok{\^{}}\NormalTok{(n }\SpecialCharTok{{-}} \DecValTok{1}\NormalTok{) \}}
\NormalTok{Sn }\OtherTok{\textless{}{-}} \ControlFlowTok{function}\NormalTok{(a1, r, n) \{ a1 }\SpecialCharTok{*}\NormalTok{ (r}\SpecialCharTok{\^{}}\NormalTok{n }\SpecialCharTok{{-}} \DecValTok{1}\NormalTok{) }\SpecialCharTok{/}\NormalTok{ (r }\SpecialCharTok{{-}} \DecValTok{1}\NormalTok{) \}}
  
\NormalTok{asn }\OtherTok{\textless{}{-}} \ControlFlowTok{function}\NormalTok{(}\AttributeTok{a1 =} \DecValTok{1}\NormalTok{, }\AttributeTok{r =} \DecValTok{2}\NormalTok{, }\AttributeTok{n =} \DecValTok{5}\NormalTok{) \{}
\NormalTok{  A }\OtherTok{\textless{}{-}} \FunctionTok{an}\NormalTok{(a1, r, n)}
\NormalTok{  S }\OtherTok{\textless{}{-}} \FunctionTok{Sn}\NormalTok{(a1, r, n)}
\NormalTok{  ii }\OtherTok{\textless{}{-}} \DecValTok{1}\SpecialCharTok{:}\NormalTok{n}
\NormalTok{  AA }\OtherTok{\textless{}{-}} \FunctionTok{an}\NormalTok{(a1, r, ii)}
\NormalTok{  SS }\OtherTok{\textless{}{-}} \FunctionTok{Sn}\NormalTok{(a1, r, ii)}
  \FunctionTok{return}\NormalTok{(}\FunctionTok{list}\NormalTok{(}\AttributeTok{an =}\NormalTok{ A, }\AttributeTok{Sn =}\NormalTok{ S, }\AttributeTok{salida =} \FunctionTok{data.frame}\NormalTok{(}\AttributeTok{valores =}\NormalTok{ AA, }\AttributeTok{suma =}\NormalTok{ SS)))}
\NormalTok{\}}
\end{Highlighting}
\end{Shaded}

La función \texttt{asn} utiliza las funiones \texttt{an} y \texttt{Sn}
programadas antes y devuelve como salida una lista con las siguientes
componentes:

\begin{itemize}
\tightlist
\item
  \texttt{an}: valor de \(a_n\)
\item
  \texttt{Sn}: valor de \(S_n\)
\item
  \texttt{salida}: data.frame con dos variables

  \begin{itemize}
  \tightlist
  \item
    \texttt{salida}: vector con las \(n\) primeras componentes de la
    progresión
  \item
    \texttt{suma}: suma de las \(n\) primeras componentes
  \end{itemize}
\end{itemize}

\begin{Shaded}
\begin{Highlighting}[]
\FunctionTok{asn}\NormalTok{()}
\end{Highlighting}
\end{Shaded}

\begin{verbatim}
## $an
## [1] 16
## 
## $Sn
## [1] 31
## 
## $salida
##   valores suma
## 1       1    1
## 2       2    3
## 3       4    7
## 4       8   15
## 5      16   31
\end{verbatim}

La salida de la función anterior es una lista y se puede
acceder a los elementos de la misma:

\begin{Shaded}
\begin{Highlighting}[]
\NormalTok{res }\OtherTok{\textless{}{-}} \FunctionTok{asn}\NormalTok{()}
\NormalTok{res}\SpecialCharTok{$}\NormalTok{an}
\end{Highlighting}
\end{Shaded}

\begin{verbatim}
## [1] 16
\end{verbatim}

\begin{Shaded}
\begin{Highlighting}[]
\NormalTok{res}\SpecialCharTok{$}\NormalTok{Sn}
\end{Highlighting}
\end{Shaded}

\begin{verbatim}
## [1] 31
\end{verbatim}

\begin{Shaded}
\begin{Highlighting}[]
\NormalTok{res}\SpecialCharTok{$}\NormalTok{salida}
\end{Highlighting}
\end{Shaded}

\begin{verbatim}
##   valores suma
## 1       1    1
## 2       2    3
## 3       4    7
## 4       8   15
## 5      16   31
\end{verbatim}

\hypertarget{otros-ejemplos}{%
\subsection{Otros ejemplos}\label{otros-ejemplos}}

\hypertarget{ejemplo-letra-del-dni}{%
\subsubsection{Ejemplo: letra del DNI}\label{ejemplo-letra-del-dni}}

A continuación se
calculará la letra del DNI a partir de su correspondiente número. El
método utilizado para obtener la letra del DNI consiste en dividir el
número entre 23 y según el resto obtenido adjudicar la letra que figura
en la siguiente tabla:

\begin{longtable}[]{@{}llllllll@{}}
\toprule
resto & letra & & resto & letra & & resto & letra \\
\midrule
\endhead
0 & T & & 8 & P & & 16 & Q \\
1 & R & & 9 & D & & 17 & V \\
2 & W & & 10 & X & & 18 & H \\
3 & A & & 11 & B & & 19 & L \\
4 & G & & 12 & N & & 20 & C \\
5 & M & & 13 & J & & 21 & K \\
6 & Y & & 14 & Z & & 22 & E \\
7 & F & & 15 & S & & & \\
\bottomrule
\end{longtable}

La siguiente función permite obtener la letra del DNI:

\begin{Shaded}
\begin{Highlighting}[]
\NormalTok{DNI }\OtherTok{\textless{}{-}} \ControlFlowTok{function}\NormalTok{(numero) \{}
\NormalTok{  letras }\OtherTok{\textless{}{-}} \FunctionTok{c}\NormalTok{(}\StringTok{"T"}\NormalTok{, }\StringTok{"R"}\NormalTok{, }\StringTok{"W"}\NormalTok{, }\StringTok{"A"}\NormalTok{, }\StringTok{"G"}\NormalTok{, }\StringTok{"M"}\NormalTok{, }\StringTok{"Y"}\NormalTok{, }\StringTok{"F"}\NormalTok{, }
              \StringTok{"P"}\NormalTok{, }\StringTok{"D"}\NormalTok{, }\StringTok{"X"}\NormalTok{, }\StringTok{"B"}\NormalTok{, }\StringTok{"N"}\NormalTok{, }\StringTok{"J"}\NormalTok{, }\StringTok{"Z"}\NormalTok{, }\StringTok{"S"}\NormalTok{, }
              \StringTok{"Q"}\NormalTok{, }\StringTok{"V"}\NormalTok{, }\StringTok{"H"}\NormalTok{, }\StringTok{"L"}\NormalTok{, }\StringTok{"C"}\NormalTok{, }\StringTok{"K"}\NormalTok{, }\StringTok{"E"}\NormalTok{)}
  \FunctionTok{return}\NormalTok{(letras[numero }\SpecialCharTok{\%\%} \DecValTok{23} \SpecialCharTok{+} \DecValTok{1}\NormalTok{]) }
\NormalTok{\}}

\FunctionTok{DNI}\NormalTok{(}\DecValTok{50247828}\NormalTok{)}
\end{Highlighting}
\end{Shaded}

\begin{verbatim}
## [1] "G"
\end{verbatim}

\hypertarget{ejemplo-simulaciuxf3n-del-lanzamiento-de-un-dado}{%
\subsubsection{Ejemplo: simulación del lanzamiento de un dado}\label{ejemplo-simulaciuxf3n-del-lanzamiento-de-un-dado}}

La siguiente función simula \(n\) (por defecto \(n=100\)) lanzamientos de un
dado. La función devuelve la tabla de frecuencias y realiza el
correspondiente gráfico:

\begin{Shaded}
\begin{Highlighting}[]
\NormalTok{dado }\OtherTok{\textless{}{-}} \ControlFlowTok{function}\NormalTok{(}\AttributeTok{n =} \DecValTok{100}\NormalTok{) \{}
\NormalTok{  lanzamientos }\OtherTok{\textless{}{-}} \FunctionTok{sample}\NormalTok{(}\DecValTok{1}\SpecialCharTok{:}\DecValTok{6}\NormalTok{, n, }\AttributeTok{rep =} \ConstantTok{TRUE}\NormalTok{)}
\NormalTok{  frecuencias }\OtherTok{\textless{}{-}} \FunctionTok{table}\NormalTok{(lanzamientos) }\SpecialCharTok{/}\NormalTok{ n}
  \FunctionTok{barplot}\NormalTok{(frecuencias, }\AttributeTok{main =} \FunctionTok{paste}\NormalTok{(}\StringTok{"Número de lanzamientos="}\NormalTok{, n))}
  \FunctionTok{abline}\NormalTok{(}\AttributeTok{h =} \DecValTok{1} \SpecialCharTok{/} \DecValTok{6}\NormalTok{, }\AttributeTok{col =} \StringTok{\textquotesingle{}red\textquotesingle{}}\NormalTok{, }\AttributeTok{lwd =} \DecValTok{2}\NormalTok{)}
  \FunctionTok{return}\NormalTok{(frecuencias)}
\NormalTok{\}}
\end{Highlighting}
\end{Shaded}

A continuación se muestran los resultados obtenidos para
varias simulaciones:

\begin{Shaded}
\begin{Highlighting}[]
\FunctionTok{dado}\NormalTok{(}\DecValTok{100}\NormalTok{)}
\end{Highlighting}
\end{Shaded}

\begin{center}\includegraphics[width=0.7\linewidth]{11-Programacion_files/figure-latex/unnamed-chunk-23-1} \end{center}

\begin{verbatim}
## lanzamientos
##    1    2    3    4    5    6 
## 0.20 0.24 0.13 0.12 0.11 0.20
\end{verbatim}

\begin{Shaded}
\begin{Highlighting}[]
\FunctionTok{dado}\NormalTok{(}\DecValTok{500}\NormalTok{)}
\end{Highlighting}
\end{Shaded}

\begin{center}\includegraphics[width=0.7\linewidth]{11-Programacion_files/figure-latex/unnamed-chunk-23-2} \end{center}

\begin{verbatim}
## lanzamientos
##     1     2     3     4     5     6 
## 0.168 0.174 0.162 0.162 0.188 0.146
\end{verbatim}

\begin{Shaded}
\begin{Highlighting}[]
\FunctionTok{dado}\NormalTok{(}\DecValTok{10000}\NormalTok{)}
\end{Highlighting}
\end{Shaded}

\begin{center}\includegraphics[width=0.7\linewidth]{11-Programacion_files/figure-latex/unnamed-chunk-23-3} \end{center}

\begin{verbatim}
## lanzamientos
##      1      2      3      4      5      6 
## 0.1690 0.1695 0.1657 0.1655 0.1629 0.1674
\end{verbatim}

Se puede comprobar que al aumentar el valor de \(n\) las frecuencias se
aproximan al valor teórico \(1/6=0.1667\).

\hypertarget{variables-locales-y-globales}{%
\subsection{Variables locales y globales}\label{variables-locales-y-globales}}

En \texttt{R} no es
necesario declarar las variables usadas dentro de una función. Se
utiliza la regla llamada ``ámbito lexicográfico'' para decidir si un
objeto es local a una función o global.

Para entender mejor esto se consideran los siguientes ejemplos:

\begin{Shaded}
\begin{Highlighting}[]
\NormalTok{fun }\OtherTok{\textless{}{-}} \ControlFlowTok{function}\NormalTok{() }\FunctionTok{print}\NormalTok{(x)}
\NormalTok{x }\OtherTok{\textless{}{-}} \DecValTok{1}
\FunctionTok{fun}\NormalTok{()}
\end{Highlighting}
\end{Shaded}

\begin{verbatim}
## [1] 1
\end{verbatim}

La variable \texttt{x} no está definida dentro de \texttt{fun}, así que \texttt{R} busca
\texttt{x} en el entorno en el que se llamó a la función e imprimirá su valor.

Si \texttt{x} es utilizado como el nombre de un objeto dentro de la
función, el valor de \texttt{x} en el ambiente global (fuera de la función) no
cambia.

\begin{Shaded}
\begin{Highlighting}[]
\NormalTok{x }\OtherTok{\textless{}{-}} \DecValTok{1}
\NormalTok{fun2 }\OtherTok{\textless{}{-}} \ControlFlowTok{function}\NormalTok{() \{}
\NormalTok{    x }\OtherTok{\textless{}{-}} \DecValTok{2}
    \FunctionTok{print}\NormalTok{(x)}
\NormalTok{\}}

\FunctionTok{fun2}\NormalTok{()}
\end{Highlighting}
\end{Shaded}

\begin{verbatim}
## [1] 2
\end{verbatim}

\begin{Shaded}
\begin{Highlighting}[]
\NormalTok{x}
\end{Highlighting}
\end{Shaded}

\begin{verbatim}
## [1] 1
\end{verbatim}

Para que el valor ``global'' de una variable pueda ser cambidado dentro de
una función se utiliza la doble asignación \texttt{\textless{}\textless{}-}.

\begin{Shaded}
\begin{Highlighting}[]
\NormalTok{x }\OtherTok{\textless{}{-}} \DecValTok{1}
\NormalTok{y }\OtherTok{\textless{}{-}} \DecValTok{3}
\NormalTok{fun2 }\OtherTok{\textless{}{-}} \ControlFlowTok{function}\NormalTok{() \{}
\NormalTok{    x }\OtherTok{\textless{}{-}} \DecValTok{2}
\NormalTok{    y }\OtherTok{\textless{}\textless{}{-}} \DecValTok{5}
    \FunctionTok{print}\NormalTok{(x)}
    \FunctionTok{print}\NormalTok{(y)}
\NormalTok{\}}

\FunctionTok{fun2}\NormalTok{()}
\end{Highlighting}
\end{Shaded}

\begin{verbatim}
## [1] 2
## [1] 5
\end{verbatim}

\begin{Shaded}
\begin{Highlighting}[]
\NormalTok{x }\CommentTok{\# No cambió su valor}
\end{Highlighting}
\end{Shaded}

\begin{verbatim}
## [1] 1
\end{verbatim}

\begin{Shaded}
\begin{Highlighting}[]
\NormalTok{y }\CommentTok{\# Cambió su valor}
\end{Highlighting}
\end{Shaded}

\begin{verbatim}
## [1] 5
\end{verbatim}

\hypertarget{ejecuciuxf3n-condicional}{%
\section{Ejecución condicional}\label{ejecuciuxf3n-condicional}}

Para hacer ejecuciones
condicionales de código se usa el comando \texttt{if} con sintaxis:

\begin{Shaded}
\begin{Highlighting}[]
\ControlFlowTok{if}\NormalTok{ (condicion1) \{expresión1\} }\ControlFlowTok{else}\NormalTok{ \{expresión2\}}
\end{Highlighting}
\end{Shaded}

La siguiente función comprueba si un número es múltiplo de dos:

\begin{Shaded}
\begin{Highlighting}[]
\NormalTok{multiplo2 }\OtherTok{=} \ControlFlowTok{function}\NormalTok{(x) \{}
  \ControlFlowTok{if}\NormalTok{ (x }\SpecialCharTok{\%\%} \DecValTok{2} \SpecialCharTok{==} \DecValTok{0}\NormalTok{) \{}
    \FunctionTok{print}\NormalTok{(}\FunctionTok{paste}\NormalTok{(x,}\StringTok{\textquotesingle{}es múltiplo de dos\textquotesingle{}}\NormalTok{))}
\NormalTok{  \} }\ControlFlowTok{else}\NormalTok{ \{}
    \FunctionTok{print}\NormalTok{(}\FunctionTok{paste}\NormalTok{(x,}\StringTok{\textquotesingle{}no es múltiplo de dos\textquotesingle{}}\NormalTok{))}
\NormalTok{  \}}
\NormalTok{\}}
  
\FunctionTok{multiplo2}\NormalTok{(}\DecValTok{5}\NormalTok{)}
\end{Highlighting}
\end{Shaded}

\begin{verbatim}
## [1] "5 no es múltiplo de dos"
\end{verbatim}

\begin{Shaded}
\begin{Highlighting}[]
\FunctionTok{multiplo2}\NormalTok{(}\SpecialCharTok{{-}}\FloatTok{2.3}\NormalTok{)}
\end{Highlighting}
\end{Shaded}

\begin{verbatim}
## [1] "-2.3 no es múltiplo de dos"
\end{verbatim}

\begin{Shaded}
\begin{Highlighting}[]
\FunctionTok{multiplo2}\NormalTok{(}\DecValTok{10}\NormalTok{)}
\end{Highlighting}
\end{Shaded}

\begin{verbatim}
## [1] "10 es múltiplo de dos"
\end{verbatim}

\hypertarget{bucles-y-vectorizaciuxf3n}{%
\section{Bucles y vectorización}\label{bucles-y-vectorizaciuxf3n}}

\hypertarget{bucles}{%
\subsection{Bucles}\label{bucles}}

\texttt{R} permite crear bucles repetitivos
(loops) y la ejecución condicional de sentencias. \texttt{R} admite bucles
\texttt{for}, \texttt{repeat} and \texttt{while}.

\hypertarget{el-bucle-for}{%
\subsubsection{\texorpdfstring{El bucle \texttt{for}}{El bucle for}}\label{el-bucle-for}}

La sintaxis de un bucle \texttt{for} es la que sigue:

\begin{Shaded}
\begin{Highlighting}[]
\ControlFlowTok{for}\NormalTok{ (i }\ControlFlowTok{in}\NormalTok{ lista\_de\_valores)  \{ expresión \}}
\end{Highlighting}
\end{Shaded}

Por ejemplo, dado un vector \(x\) se puede calcular \(y=x^2\) con el código:

\begin{Shaded}
\begin{Highlighting}[]
\NormalTok{x }\OtherTok{\textless{}{-}} \FunctionTok{seq}\NormalTok{(}\SpecialCharTok{{-}}\DecValTok{2}\NormalTok{, }\DecValTok{2}\NormalTok{, }\FloatTok{0.5}\NormalTok{)}
\NormalTok{n }\OtherTok{\textless{}{-}} \FunctionTok{length}\NormalTok{(x)}
\NormalTok{y }\OtherTok{\textless{}{-}} \FunctionTok{numeric}\NormalTok{(n) }\CommentTok{\# Es necesario crear el objeto para acceder a los componentes...}
\ControlFlowTok{for}\NormalTok{ (i }\ControlFlowTok{in} \DecValTok{1}\SpecialCharTok{:}\NormalTok{n) \{ y[i] }\OtherTok{\textless{}{-}}\NormalTok{ x[i] }\SpecialCharTok{\^{}} \DecValTok{2}\NormalTok{ \}}
\NormalTok{x}
\end{Highlighting}
\end{Shaded}

\begin{verbatim}
## [1] -2.0 -1.5 -1.0 -0.5  0.0  0.5  1.0  1.5  2.0
\end{verbatim}

\begin{Shaded}
\begin{Highlighting}[]
\NormalTok{y}
\end{Highlighting}
\end{Shaded}

\begin{verbatim}
## [1] 4.00 2.25 1.00 0.25 0.00 0.25 1.00 2.25 4.00
\end{verbatim}

\begin{Shaded}
\begin{Highlighting}[]
\NormalTok{x}\SpecialCharTok{\^{}}\DecValTok{2}
\end{Highlighting}
\end{Shaded}

\begin{verbatim}
## [1] 4.00 2.25 1.00 0.25 0.00 0.25 1.00 2.25 4.00
\end{verbatim}

Otro ejemplo:

\begin{Shaded}
\begin{Highlighting}[]
\ControlFlowTok{for}\NormalTok{(i }\ControlFlowTok{in} \DecValTok{1}\SpecialCharTok{:}\DecValTok{5}\NormalTok{) }\FunctionTok{print}\NormalTok{(i)}
\end{Highlighting}
\end{Shaded}

\begin{verbatim}
## [1] 1
## [1] 2
## [1] 3
## [1] 4
## [1] 5
\end{verbatim}

El siguiente código simula gráficamente el segundero de un
reloj:

\begin{Shaded}
\begin{Highlighting}[]
\NormalTok{angulo }\OtherTok{\textless{}{-}} \FunctionTok{seq}\NormalTok{(}\DecValTok{0}\NormalTok{, }\DecValTok{360}\NormalTok{, }\AttributeTok{by =} \DecValTok{6}\NormalTok{)}
\NormalTok{radianes }\OtherTok{\textless{}{-}}\NormalTok{ angulo }\SpecialCharTok{*}\NormalTok{ pi }\SpecialCharTok{/} \DecValTok{180}
\NormalTok{x }\OtherTok{\textless{}{-}} \FunctionTok{sin}\NormalTok{(radianes)}
\NormalTok{y }\OtherTok{\textless{}{-}} \FunctionTok{cos}\NormalTok{(radianes)}

\NormalTok{sec }\OtherTok{\textless{}{-}} \FunctionTok{seq}\NormalTok{(}\DecValTok{6}\NormalTok{, }\DecValTok{61}\NormalTok{, }\AttributeTok{by =} \DecValTok{5}\NormalTok{)}
\ControlFlowTok{for}\NormalTok{ (i }\ControlFlowTok{in} \DecValTok{1}\SpecialCharTok{:}\DecValTok{61}\NormalTok{) \{}
  \FunctionTok{plot}\NormalTok{(x, y, }\AttributeTok{axes =} \ConstantTok{FALSE}\NormalTok{, }\AttributeTok{xlab =} \StringTok{""}\NormalTok{, }\AttributeTok{ylab =} \StringTok{""}\NormalTok{, }\AttributeTok{type =} \StringTok{\textquotesingle{}l\textquotesingle{}}\NormalTok{, }\AttributeTok{col =} \StringTok{\textquotesingle{}grey\textquotesingle{}}\NormalTok{)}
  \FunctionTok{points}\NormalTok{(x[i], y[i])}
  \CommentTok{\# Añadir "decoración"}
  \FunctionTok{text}\NormalTok{(x[sec]}\SpecialCharTok{*}\FloatTok{0.9}\NormalTok{, y[sec]}\SpecialCharTok{*}\FloatTok{0.9}\NormalTok{, }\AttributeTok{labels =}\NormalTok{ sec }\SpecialCharTok{{-}} \DecValTok{1}\NormalTok{)}
  \FunctionTok{arrows}\NormalTok{(}\DecValTok{0}\NormalTok{, }\DecValTok{0}\NormalTok{, x[i]}\SpecialCharTok{*}\FloatTok{0.85}\NormalTok{, y[i]}\SpecialCharTok{*}\FloatTok{0.85}\NormalTok{, }\AttributeTok{col =} \StringTok{\textquotesingle{}blue\textquotesingle{}}\NormalTok{)}
  \CommentTok{\# Esperar un segundo}
  \FunctionTok{Sys.sleep}\NormalTok{(}\DecValTok{1}\NormalTok{)}
\NormalTok{\}}
\end{Highlighting}
\end{Shaded}

\hypertarget{el-bucle-while}{%
\subsubsection{\texorpdfstring{El bucle \texttt{while}}{El bucle while}}\label{el-bucle-while}}

La sintaxis del bucle \texttt{while}
es la que sigue:

\begin{Shaded}
\begin{Highlighting}[]
\ControlFlowTok{while}\NormalTok{ (condición lógica)  \{ expresión \}}
\end{Highlighting}
\end{Shaded}

Por ejemplo, si queremos calcular el primer número entero positivo cuyo
cuadrado no excede de 5000, podemos hacer:

\begin{Shaded}
\begin{Highlighting}[]
\NormalTok{cuadrado }\OtherTok{\textless{}{-}} \DecValTok{0}
\NormalTok{n }\OtherTok{\textless{}{-}} \DecValTok{0}
\ControlFlowTok{while}\NormalTok{ (cuadrado }\SpecialCharTok{\textless{}=} \DecValTok{5000}\NormalTok{) \{}
\NormalTok{  n }\OtherTok{\textless{}{-}}\NormalTok{ n }\SpecialCharTok{+} \DecValTok{1}
\NormalTok{  cuadrado }\OtherTok{\textless{}{-}}\NormalTok{ n}\SpecialCharTok{\^{}}\DecValTok{2}
\NormalTok{\}}
\NormalTok{cuadrado}
\end{Highlighting}
\end{Shaded}

\begin{verbatim}
## [1] 5041
\end{verbatim}

\begin{Shaded}
\begin{Highlighting}[]
\NormalTok{n}
\end{Highlighting}
\end{Shaded}

\begin{verbatim}
## [1] 71
\end{verbatim}

\begin{Shaded}
\begin{Highlighting}[]
\NormalTok{n}\SpecialCharTok{\^{}}\DecValTok{2}
\end{Highlighting}
\end{Shaded}

\begin{verbatim}
## [1] 5041
\end{verbatim}

\textbf{Nota:} Dentro de un bucle se puede emplear el comando \texttt{break} para terminarlo y el comando \texttt{next} para saltar a la siguiente iteración.

\hypertarget{vectorizaciuxf3n}{%
\subsection{Vectorización}\label{vectorizaciuxf3n}}

Como hemos visto en \texttt{R} se pueden
hacer bucles. Sin embargo, es preferible evitar este tipo de estructuras
y tratar de utilizar \textbf{operaciones vectorizadas} que son mucho más
eficientes desde el punto de vista computacional.

Por ejemplo para sumar dos vectores se puede hacer con un \texttt{for}:

\begin{Shaded}
\begin{Highlighting}[]
\NormalTok{x }\OtherTok{\textless{}{-}} \FunctionTok{c}\NormalTok{(}\DecValTok{1}\NormalTok{, }\DecValTok{2}\NormalTok{, }\DecValTok{3}\NormalTok{, }\DecValTok{4}\NormalTok{)}
\NormalTok{y }\OtherTok{\textless{}{-}} \FunctionTok{c}\NormalTok{(}\DecValTok{0}\NormalTok{, }\DecValTok{0}\NormalTok{, }\DecValTok{5}\NormalTok{, }\DecValTok{1}\NormalTok{)}
\NormalTok{n }\OtherTok{\textless{}{-}} \FunctionTok{length}\NormalTok{(x)}
\NormalTok{z }\OtherTok{\textless{}{-}} \FunctionTok{numeric}\NormalTok{(n)}
\ControlFlowTok{for}\NormalTok{ (i }\ControlFlowTok{in} \DecValTok{1}\SpecialCharTok{:}\NormalTok{n) \{}
\NormalTok{  z[i] }\OtherTok{\textless{}{-}}\NormalTok{ x[i] }\SpecialCharTok{+}\NormalTok{ y[i]}
\NormalTok{\}}
\NormalTok{z}
\end{Highlighting}
\end{Shaded}

\begin{verbatim}
## [1] 1 2 8 5
\end{verbatim}

Sin embargo, la operación anterior se podría hacer de modo más eficiente
en modo vectorial:

\begin{Shaded}
\begin{Highlighting}[]
\NormalTok{z }\OtherTok{\textless{}{-}}\NormalTok{ x }\SpecialCharTok{+}\NormalTok{ y}
\NormalTok{z}
\end{Highlighting}
\end{Shaded}

\begin{verbatim}
## [1] 1 2 8 5
\end{verbatim}

\hypertarget{funciones-apply}{%
\subsection{\texorpdfstring{Funciones \texttt{apply}}{Funciones apply}}\label{funciones-apply}}

\hypertarget{la-funciuxf3n-apply}{%
\subsubsection{\texorpdfstring{La función \texttt{apply}}{La función apply}}\label{la-funciuxf3n-apply}}

Una forma de evitar la
utilización de bucles es utilizando la función \texttt{apply()} que permite
evaluar una misma función en todas las filas, columnas, \ldots. de un array
de forma simultánea.

La sintaxis de esta función es:

\begin{Shaded}
\begin{Highlighting}[]
\FunctionTok{apply}\NormalTok{(X, MARGIN, FUN, ...)}
\end{Highlighting}
\end{Shaded}

\begin{itemize}
\tightlist
\item
  \texttt{X}: matriz (o array)
\item
  \texttt{MARGIN}: Un vector indicando las dimensiones donde se aplicará
  la función. 1 indica filas, 2 indica columnas, y \texttt{c(1,2)} indica
  filas y columnas.
\item
  \texttt{FUN}: función que será aplicada.
\item
  \texttt{...}: argumentos opcionales que serán usados por \texttt{FUN}.
\end{itemize}

Veamos la utilización de la función \texttt{apply} con un ejemplo:

\begin{Shaded}
\begin{Highlighting}[]
\NormalTok{x }\OtherTok{\textless{}{-}} \FunctionTok{matrix}\NormalTok{(}\DecValTok{1}\SpecialCharTok{:}\DecValTok{9}\NormalTok{, }\AttributeTok{nrow =} \DecValTok{3}\NormalTok{)}
\NormalTok{x}
\end{Highlighting}
\end{Shaded}

\begin{verbatim}
##      [,1] [,2] [,3]
## [1,]    1    4    7
## [2,]    2    5    8
## [3,]    3    6    9
\end{verbatim}

\begin{Shaded}
\begin{Highlighting}[]
\FunctionTok{apply}\NormalTok{(x, }\DecValTok{1}\NormalTok{, sum)    }\CommentTok{\# Suma por filas}
\end{Highlighting}
\end{Shaded}

\begin{verbatim}
## [1] 12 15 18
\end{verbatim}

\begin{Shaded}
\begin{Highlighting}[]
\FunctionTok{apply}\NormalTok{(x, }\DecValTok{2}\NormalTok{, sum)    }\CommentTok{\# Suma por columnas}
\end{Highlighting}
\end{Shaded}

\begin{verbatim}
## [1]  6 15 24
\end{verbatim}

\begin{Shaded}
\begin{Highlighting}[]
\FunctionTok{apply}\NormalTok{(x, }\DecValTok{2}\NormalTok{, min)    }\CommentTok{\# Mínimo de las columnas}
\end{Highlighting}
\end{Shaded}

\begin{verbatim}
## [1] 1 4 7
\end{verbatim}

\begin{Shaded}
\begin{Highlighting}[]
\FunctionTok{apply}\NormalTok{(x, }\DecValTok{2}\NormalTok{, range)  }\CommentTok{\# Rango (mínimo y máximo) de las columnas}
\end{Highlighting}
\end{Shaded}

\begin{verbatim}
##      [,1] [,2] [,3]
## [1,]    1    4    7
## [2,]    3    6    9
\end{verbatim}

\hypertarget{la-funciuxf3n-tapply}{%
\subsubsection{\texorpdfstring{La función \texttt{tapply}}{La función tapply}}\label{la-funciuxf3n-tapply}}

La function \texttt{tapply} es
similar a la función \texttt{apply} y permite aplicar una función a los datos desagregados,
utilizando como criterio los distintos niveles de una variable factor.
La sintaxis de esta función es como sigue:

\begin{Shaded}
\begin{Highlighting}[]
    \FunctionTok{tapply}\NormalTok{(X, INDEX, FUN, ...,)}
\end{Highlighting}
\end{Shaded}

\begin{itemize}
\tightlist
\item
  \texttt{X}: matriz (o array).
\item
  \texttt{INDEX}: factor indicando los grupos (niveles).
\item
  \texttt{FUN}: función que será aplicada.
\item
  \texttt{...}: argumentos opcionales .
\end{itemize}

Consideremos, por ejemplo, el data.frame \texttt{ChickWeight} con datos de un
experimento relacionado con la repercusión de varias dietas en el peso
de pollos.

\begin{Shaded}
\begin{Highlighting}[]
\FunctionTok{data}\NormalTok{(ChickWeight)}
\FunctionTok{head}\NormalTok{(ChickWeight)}
\end{Highlighting}
\end{Shaded}

\begin{verbatim}
##   weight Time Chick Diet
## 1     42    0     1    1
## 2     51    2     1    1
## 3     59    4     1    1
## 4     64    6     1    1
## 5     76    8     1    1
## 6     93   10     1    1
\end{verbatim}

\begin{Shaded}
\begin{Highlighting}[]
\NormalTok{peso }\OtherTok{\textless{}{-}}\NormalTok{ ChickWeight}\SpecialCharTok{$}\NormalTok{weight}
\NormalTok{dieta }\OtherTok{\textless{}{-}}\NormalTok{ ChickWeight}\SpecialCharTok{$}\NormalTok{Diet}
\FunctionTok{levels}\NormalTok{(dieta) }\OtherTok{\textless{}{-}} \FunctionTok{c}\NormalTok{(}\StringTok{"Dieta 1"}\NormalTok{, }\StringTok{"Dieta 2"}\NormalTok{, }\StringTok{"Dieta 3"}\NormalTok{, }\StringTok{"Dieta 4"}\NormalTok{)}
\FunctionTok{tapply}\NormalTok{(peso, dieta, mean)  }\CommentTok{\# Peso medio por dieta}
\end{Highlighting}
\end{Shaded}

\begin{verbatim}
##  Dieta 1  Dieta 2  Dieta 3  Dieta 4 
## 102.6455 122.6167 142.9500 135.2627
\end{verbatim}

\begin{Shaded}
\begin{Highlighting}[]
\FunctionTok{tapply}\NormalTok{(peso, dieta, summary)}
\end{Highlighting}
\end{Shaded}

\begin{verbatim}
## $`Dieta 1`
##    Min. 1st Qu.  Median    Mean 3rd Qu.    Max. 
##   35.00   57.75   88.00  102.65  136.50  305.00 
## 
## $`Dieta 2`
##    Min. 1st Qu.  Median    Mean 3rd Qu.    Max. 
##    39.0    65.5   104.5   122.6   163.0   331.0 
## 
## $`Dieta 3`
##    Min. 1st Qu.  Median    Mean 3rd Qu.    Max. 
##    39.0    67.5   125.5   142.9   198.8   373.0 
## 
## $`Dieta 4`
##    Min. 1st Qu.  Median    Mean 3rd Qu.    Max. 
##   39.00   71.25  129.50  135.26  184.75  322.00
\end{verbatim}

Otro ejemplo:

\begin{Shaded}
\begin{Highlighting}[]
\NormalTok{provincia }\OtherTok{\textless{}{-}} \FunctionTok{as.factor}\NormalTok{(}\FunctionTok{c}\NormalTok{(}\DecValTok{1}\NormalTok{, }\DecValTok{3}\NormalTok{, }\DecValTok{4}\NormalTok{, }\DecValTok{2}\NormalTok{, }\DecValTok{4}\NormalTok{, }\DecValTok{3}\NormalTok{, }\DecValTok{2}\NormalTok{, }\DecValTok{1}\NormalTok{, }\DecValTok{4}\NormalTok{, }\DecValTok{3}\NormalTok{, }\DecValTok{2}\NormalTok{))}
\FunctionTok{levels}\NormalTok{(provincia) }\OtherTok{=} \FunctionTok{c}\NormalTok{(}\StringTok{"A Coruña"}\NormalTok{, }\StringTok{"Lugo"}\NormalTok{, }\StringTok{"Orense"}\NormalTok{, }\StringTok{"Pontevedra"}\NormalTok{)}
\NormalTok{hijos }\OtherTok{\textless{}{-}} \FunctionTok{c}\NormalTok{(}\DecValTok{1}\NormalTok{, }\DecValTok{2}\NormalTok{, }\DecValTok{0}\NormalTok{, }\DecValTok{3}\NormalTok{, }\DecValTok{4}\NormalTok{, }\DecValTok{1}\NormalTok{, }\DecValTok{0}\NormalTok{, }\DecValTok{0}\NormalTok{, }\DecValTok{2}\NormalTok{, }\DecValTok{3}\NormalTok{, }\DecValTok{1}\NormalTok{)}
\FunctionTok{data.frame}\NormalTok{(provincia, hijos)}
\end{Highlighting}
\end{Shaded}

\begin{verbatim}
##     provincia hijos
## 1    A Coruña     1
## 2      Orense     2
## 3  Pontevedra     0
## 4        Lugo     3
## 5  Pontevedra     4
## 6      Orense     1
## 7        Lugo     0
## 8    A Coruña     0
## 9  Pontevedra     2
## 10     Orense     3
## 11       Lugo     1
\end{verbatim}

\begin{Shaded}
\begin{Highlighting}[]
\FunctionTok{tapply}\NormalTok{(hijos, provincia, mean) }\CommentTok{\# Número medio de hijos por provincia}
\end{Highlighting}
\end{Shaded}

\begin{verbatim}
##   A Coruña       Lugo     Orense Pontevedra 
##   0.500000   1.333333   2.000000   2.000000
\end{verbatim}

\hypertarget{aplicaciuxf3n-validaciuxf3n-cruzada}{%
\section{Aplicación: validación cruzada}\label{aplicaciuxf3n-validaciuxf3n-cruzada}}

Si deseamos evaluar la calidad predictiva de un modelo, lo ideal es disponer de
suficientes datos para poder hacer dos grupos con ellos: una muestra de entrenamiento
y otra de validación. Cuando hacer esto no es posible, disponemos como alternativa
de la \emph{validación cruzada}, una herramienta que permite estimar los errores de
predicción utilizando una única muestra de datos. En su versión más simple (llamada
en inglés \emph{leave-one-out}):

\begin{itemize}
\item
  se utilizan todos los datos menos uno para realizar el ajuste, y se mide su error de
  predicción en el único dato no utilizado;
\item
  a continuación se repite el proceso
  utilizando, uno a uno, todos los puntos de la muestra de datos;
\item
  y finalmente se combinan todos los errores en un único error de predicción.
\end{itemize}

El proceso anterior se puede generalizar repartiendo los datos en distintos grupos,
más o menos del mismo tamaño, y sustituyendo en la explicación anterior dato por grupo.

\hypertarget{primer-ejemplo}{%
\subsection{Primer ejemplo}\label{primer-ejemplo}}

Cuando disponemos de unos datos y los queremos ajustar utilizando un modelo que
depende de un parámetro, por ejemplo un modelo
de regresión polinómico que depende del grado del polinomio, podemos utilizar
la validación cruzada para seleccionar el grado del polinomio que debemos utilizar.

Veámoslo utilizando las variables \emph{income} y \emph{prestige} de la
base de datos \emph{Prestige}, incluida en el paquete \emph{car}.

\begin{Shaded}
\begin{Highlighting}[]
\FunctionTok{library}\NormalTok{(car)}
\FunctionTok{plot}\NormalTok{(prestige }\SpecialCharTok{\textasciitilde{}}\NormalTok{ income, }\AttributeTok{data =}\NormalTok{ Prestige, }\AttributeTok{col =} \StringTok{\textquotesingle{}darkgray\textquotesingle{}}\NormalTok{)}
\end{Highlighting}
\end{Shaded}

\begin{center}\includegraphics[width=0.7\linewidth]{11-Programacion_files/figure-latex/unnamed-chunk-42-1} \end{center}

Representemos, gráficamente, los ajustes lineal, cuadrático y cúbico.

\begin{Shaded}
\begin{Highlighting}[]
\FunctionTok{plot}\NormalTok{(prestige }\SpecialCharTok{\textasciitilde{}}\NormalTok{ income, }\AttributeTok{data =}\NormalTok{ Prestige, }\AttributeTok{col =} \StringTok{\textquotesingle{}darkgray\textquotesingle{}}\NormalTok{)}
\CommentTok{\# Ajuste lineal}
\FunctionTok{abline}\NormalTok{(}\FunctionTok{lm}\NormalTok{(prestige }\SpecialCharTok{\textasciitilde{}}\NormalTok{ income, }\AttributeTok{data =}\NormalTok{ Prestige))}
\CommentTok{\# Ajuste cuadrático}
\NormalTok{modelo }\OtherTok{\textless{}{-}} \FunctionTok{lm}\NormalTok{(prestige }\SpecialCharTok{\textasciitilde{}}\NormalTok{ income }\SpecialCharTok{+} \FunctionTok{I}\NormalTok{(income}\SpecialCharTok{\^{}}\DecValTok{2}\NormalTok{), }\AttributeTok{data =}\NormalTok{ Prestige)}
\NormalTok{parest }\OtherTok{\textless{}{-}} \FunctionTok{coef}\NormalTok{(modelo)}
\FunctionTok{curve}\NormalTok{(parest[}\DecValTok{1}\NormalTok{] }\SpecialCharTok{+}\NormalTok{ parest[}\DecValTok{2}\NormalTok{]}\SpecialCharTok{*}\NormalTok{x }\SpecialCharTok{+}\NormalTok{ parest[}\DecValTok{3}\NormalTok{]}\SpecialCharTok{*}\NormalTok{x}\SpecialCharTok{\^{}}\DecValTok{2}\NormalTok{, }\AttributeTok{lty =} \DecValTok{2}\NormalTok{, }\AttributeTok{add =} \ConstantTok{TRUE}\NormalTok{)}
\CommentTok{\# Ajuste cúbico}
\NormalTok{modelo }\OtherTok{\textless{}{-}} \FunctionTok{lm}\NormalTok{(prestige }\SpecialCharTok{\textasciitilde{}} \FunctionTok{poly}\NormalTok{(income, }\DecValTok{3}\NormalTok{), }\AttributeTok{data =}\NormalTok{ Prestige)}
\NormalTok{valores }\OtherTok{\textless{}{-}} \FunctionTok{seq}\NormalTok{(}\DecValTok{0}\NormalTok{, }\DecValTok{26000}\NormalTok{, }\AttributeTok{len =} \DecValTok{100}\NormalTok{)}
\NormalTok{pred }\OtherTok{\textless{}{-}} \FunctionTok{predict}\NormalTok{(modelo, }\AttributeTok{newdata =} \FunctionTok{data.frame}\NormalTok{(}\AttributeTok{income =}\NormalTok{ valores))}
\FunctionTok{lines}\NormalTok{(valores, pred, }\AttributeTok{lty =} \DecValTok{3}\NormalTok{)}
\FunctionTok{legend}\NormalTok{(}\StringTok{"bottomright"}\NormalTok{, }\FunctionTok{c}\NormalTok{(}\StringTok{"Lineal"}\NormalTok{,}\StringTok{"Cuadrático"}\NormalTok{,}\StringTok{"Cúbico"}\NormalTok{), }\AttributeTok{lty =} \DecValTok{1}\SpecialCharTok{:}\DecValTok{3}\NormalTok{)}
\end{Highlighting}
\end{Shaded}

\begin{center}\includegraphics[width=0.7\linewidth]{11-Programacion_files/figure-latex/unnamed-chunk-43-1} \end{center}

Vamos a escribir una función que nos devuelva, para cada dato (fila) de
\emph{Prestige}, la predicción en ese punto ajustando el modelo con todos los demás puntos.

\begin{Shaded}
\begin{Highlighting}[]
\NormalTok{cv.lm }\OtherTok{\textless{}{-}} \ControlFlowTok{function}\NormalTok{(formula, datos) \{}
\NormalTok{      n }\OtherTok{\textless{}{-}} \FunctionTok{nrow}\NormalTok{(datos)}
\NormalTok{      cv.pred }\OtherTok{\textless{}{-}} \FunctionTok{numeric}\NormalTok{(n)}
      \ControlFlowTok{for}\NormalTok{ (i }\ControlFlowTok{in} \DecValTok{1}\SpecialCharTok{:}\NormalTok{n) \{}
\NormalTok{          modelo }\OtherTok{\textless{}{-}} \FunctionTok{lm}\NormalTok{(formula, datos[}\SpecialCharTok{{-}}\NormalTok{i, ])}
\NormalTok{          cv.pred[i] }\OtherTok{\textless{}{-}} \FunctionTok{predict}\NormalTok{(modelo, }\AttributeTok{newdata =}\NormalTok{ datos[i, ])}
\NormalTok{      \}}
      \FunctionTok{return}\NormalTok{(cv.pred)}
\NormalTok{\}}
\end{Highlighting}
\end{Shaded}

Por último, calculamos el error de predicción (en este caso el \emph{error cuadrático medio})
en los datos de validación. Repetimos el proceso para cada valor del parámetro (grado del
ajuste polinómico) y minimizamos.

\begin{Shaded}
\begin{Highlighting}[]
\NormalTok{grado }\OtherTok{\textless{}{-}} \DecValTok{1}\SpecialCharTok{:}\DecValTok{5}
\NormalTok{cv.error }\OtherTok{\textless{}{-}} \FunctionTok{numeric}\NormalTok{(}\DecValTok{5}\NormalTok{)}
\ControlFlowTok{for}\NormalTok{(p }\ControlFlowTok{in}\NormalTok{ grado)\{}
\NormalTok{  cv.pred }\OtherTok{\textless{}{-}} \FunctionTok{cv.lm}\NormalTok{(prestige }\SpecialCharTok{\textasciitilde{}} \FunctionTok{poly}\NormalTok{(income, p), Prestige)}
\NormalTok{  cv.error[p] }\OtherTok{\textless{}{-}} \FunctionTok{mean}\NormalTok{((cv.pred }\SpecialCharTok{{-}}\NormalTok{ Prestige}\SpecialCharTok{$}\NormalTok{prestige)}\SpecialCharTok{\^{}}\DecValTok{2}\NormalTok{)}
\NormalTok{\}}
\FunctionTok{plot}\NormalTok{(grado, cv.error, }\AttributeTok{pch=}\DecValTok{16}\NormalTok{)}
\end{Highlighting}
\end{Shaded}

\begin{center}\includegraphics[width=0.7\linewidth]{11-Programacion_files/figure-latex/unnamed-chunk-45-1} \end{center}

\begin{Shaded}
\begin{Highlighting}[]
\NormalTok{grado[}\FunctionTok{which.min}\NormalTok{(cv.error)]}
\end{Highlighting}
\end{Shaded}

\begin{verbatim}
## [1] 2
\end{verbatim}

\hypertarget{segundo-ejemplo}{%
\subsection{Segundo ejemplo}\label{segundo-ejemplo}}

En este segundo ejemplo vamos a aplicar una técnica de modelado \emph{local} al problema de
regresión del ejemplo anterior.
El enfoque es \emph{data-analytic} en el sentido de que no nos limitamos a una
familia de funciones que dependen de unos parámetros (enfoque paramétrico),
que son los que tenemos que determinar, sino que las funciones de
regresión están determinadas por los datos. Aun así, sigue habiendo un
parámetro que controla el proceso, cuyo valor debemos fijar siguiendo
algún criterio de optimalidad.

Vamos a realizar, utilizando la función \texttt{loess}, un \emph{ajuste polinómico local
robusto}, que depende del parámetro \texttt{span}, que podemos interpretar como la
proporción de datos empleada en el ajuste.

Utilizando un valor \texttt{span=0.25}:

\begin{Shaded}
\begin{Highlighting}[]
\FunctionTok{plot}\NormalTok{(prestige }\SpecialCharTok{\textasciitilde{}}\NormalTok{ income, Prestige, }\AttributeTok{col =} \StringTok{\textquotesingle{}darkgray\textquotesingle{}}\NormalTok{)}
\NormalTok{fit }\OtherTok{\textless{}{-}} \FunctionTok{loess}\NormalTok{(prestige }\SpecialCharTok{\textasciitilde{}}\NormalTok{ income, Prestige, }\AttributeTok{span =} \FloatTok{0.25}\NormalTok{)}
\NormalTok{valores }\OtherTok{\textless{}{-}} \FunctionTok{seq}\NormalTok{(}\DecValTok{0}\NormalTok{, }\DecValTok{25000}\NormalTok{, }\DecValTok{100}\NormalTok{)}
\NormalTok{pred }\OtherTok{\textless{}{-}} \FunctionTok{predict}\NormalTok{(fit, }\AttributeTok{newdata =} \FunctionTok{data.frame}\NormalTok{(}\AttributeTok{income =}\NormalTok{ valores))}
\FunctionTok{lines}\NormalTok{(valores, pred)}
\end{Highlighting}
\end{Shaded}

\begin{center}\includegraphics[width=0.7\linewidth]{11-Programacion_files/figure-latex/unnamed-chunk-46-1} \end{center}

Si utilizamos \texttt{span=0.5}:

\begin{Shaded}
\begin{Highlighting}[]
\FunctionTok{plot}\NormalTok{(prestige }\SpecialCharTok{\textasciitilde{}}\NormalTok{ income, Prestige, }\AttributeTok{col =} \StringTok{\textquotesingle{}darkgray\textquotesingle{}}\NormalTok{)}
\NormalTok{fit }\OtherTok{\textless{}{-}} \FunctionTok{loess}\NormalTok{(prestige }\SpecialCharTok{\textasciitilde{}}\NormalTok{ income, Prestige, }\AttributeTok{span =} \FloatTok{0.5}\NormalTok{)}
\NormalTok{valores }\OtherTok{\textless{}{-}} \FunctionTok{seq}\NormalTok{(}\DecValTok{0}\NormalTok{, }\DecValTok{25000}\NormalTok{, }\DecValTok{100}\NormalTok{)}
\NormalTok{pred }\OtherTok{\textless{}{-}} \FunctionTok{predict}\NormalTok{(fit, }\AttributeTok{newdata =} \FunctionTok{data.frame}\NormalTok{(}\AttributeTok{income =}\NormalTok{ valores))}
\FunctionTok{lines}\NormalTok{(valores, pred)}
\end{Highlighting}
\end{Shaded}

\begin{center}\includegraphics[width=0.7\linewidth]{11-Programacion_files/figure-latex/unnamed-chunk-47-1} \end{center}

Nuestro objetivo es seleccionar un valor razonable para \texttt{span}, y lo vamos a
hacer utilizando validación cruzada y minimizando el error cuadrático medio
de la predicción en los datos de validación.

Utilizando la función

\begin{Shaded}
\begin{Highlighting}[]
\NormalTok{cv.loess }\OtherTok{\textless{}{-}} \ControlFlowTok{function}\NormalTok{(formula, datos, p) \{}
\NormalTok{  n }\OtherTok{\textless{}{-}} \FunctionTok{nrow}\NormalTok{(datos)}
\NormalTok{  cv.pred }\OtherTok{\textless{}{-}} \FunctionTok{numeric}\NormalTok{(n)}
  \ControlFlowTok{for}\NormalTok{ (i }\ControlFlowTok{in} \DecValTok{1}\SpecialCharTok{:}\NormalTok{n) \{}
\NormalTok{    modelo }\OtherTok{\textless{}{-}} \FunctionTok{loess}\NormalTok{(formula, datos[}\SpecialCharTok{{-}}\NormalTok{i, ], }\AttributeTok{span =}\NormalTok{ p, }
                    \AttributeTok{control =} \FunctionTok{loess.control}\NormalTok{(}\AttributeTok{surface =} \StringTok{"direct"}\NormalTok{))}
    \CommentTok{\# control = loess.control(surface = "direct") permite extrapolaciones}
\NormalTok{    cv.pred[i] }\OtherTok{\textless{}{-}} \FunctionTok{predict}\NormalTok{(modelo, }\AttributeTok{newdata =}\NormalTok{ datos[i, ])}
\NormalTok{  \}}
  \FunctionTok{return}\NormalTok{(cv.pred)}
\NormalTok{\}}
\end{Highlighting}
\end{Shaded}

y procediendo de modo similar al caso anterior:

\begin{Shaded}
\begin{Highlighting}[]
\NormalTok{ventanas }\OtherTok{\textless{}{-}} \FunctionTok{seq}\NormalTok{(}\FloatTok{0.2}\NormalTok{, }\DecValTok{1}\NormalTok{, }\AttributeTok{len =} \DecValTok{10}\NormalTok{)}
\NormalTok{np }\OtherTok{\textless{}{-}} \FunctionTok{length}\NormalTok{(ventanas)}
\NormalTok{cv.error }\OtherTok{\textless{}{-}} \FunctionTok{numeric}\NormalTok{(np)}
\ControlFlowTok{for}\NormalTok{(p }\ControlFlowTok{in} \DecValTok{1}\SpecialCharTok{:}\NormalTok{np)\{}
\NormalTok{  cv.pred }\OtherTok{\textless{}{-}} \FunctionTok{cv.loess}\NormalTok{(prestige }\SpecialCharTok{\textasciitilde{}}\NormalTok{ income, Prestige, ventanas[p])}
\NormalTok{  cv.error[p] }\OtherTok{\textless{}{-}} \FunctionTok{mean}\NormalTok{((cv.pred }\SpecialCharTok{{-}}\NormalTok{ Prestige}\SpecialCharTok{$}\NormalTok{prestige)}\SpecialCharTok{\^{}}\DecValTok{2}\NormalTok{)}
  \CommentTok{\# cv.error[p] \textless{}{-} median(abs(cv.pred {-} Prestige$prestige))}
\NormalTok{\}}

\FunctionTok{plot}\NormalTok{(ventanas, cv.error)}
\end{Highlighting}
\end{Shaded}

\begin{center}\includegraphics[width=0.7\linewidth]{11-Programacion_files/figure-latex/unnamed-chunk-49-1} \end{center}

obtenemos la ventana ``óptima'' (en este caso el valor máximo):

\begin{Shaded}
\begin{Highlighting}[]
\NormalTok{span }\OtherTok{\textless{}{-}}\NormalTok{ ventanas[}\FunctionTok{which.min}\NormalTok{(cv.error)]}
\NormalTok{span}
\end{Highlighting}
\end{Shaded}

\begin{verbatim}
## [1] 1
\end{verbatim}

y la correspondiente estimación:

\begin{Shaded}
\begin{Highlighting}[]
\FunctionTok{plot}\NormalTok{(prestige }\SpecialCharTok{\textasciitilde{}}\NormalTok{ income, Prestige, }\AttributeTok{col =} \StringTok{\textquotesingle{}darkgray\textquotesingle{}}\NormalTok{)}
\NormalTok{fit }\OtherTok{\textless{}{-}} \FunctionTok{loess}\NormalTok{(prestige }\SpecialCharTok{\textasciitilde{}}\NormalTok{ income, Prestige, }\AttributeTok{span =}\NormalTok{ span)}
\NormalTok{valores }\OtherTok{\textless{}{-}} \FunctionTok{seq}\NormalTok{(}\DecValTok{0}\NormalTok{, }\DecValTok{25000}\NormalTok{, }\DecValTok{100}\NormalTok{)}
\NormalTok{pred }\OtherTok{\textless{}{-}} \FunctionTok{predict}\NormalTok{(fit, }\AttributeTok{newdata =} \FunctionTok{data.frame}\NormalTok{(}\AttributeTok{income =}\NormalTok{ valores))}
\FunctionTok{lines}\NormalTok{(valores, pred)}
\end{Highlighting}
\end{Shaded}

\begin{center}\includegraphics[width=0.7\linewidth]{11-Programacion_files/figure-latex/unnamed-chunk-51-1} \end{center}

\hypertarget{informes}{%
\chapter{Generación de informes}\label{informes}}

Una versión más completa de este capítulo está disponible en el
\href{https://rubenfcasal.github.io/bookdown_intro/rmarkdown.html}{apéndice} del libro
\href{https://rubenfcasal.github.io/bookdown_intro}{Escritura de libros con bookdown}.

\hypertarget{r-markdown}{%
\section{R Markdown}\label{r-markdown}}

R-Markdown es recomendable para difundir análisis realizados con \texttt{R} en formato HTML, PDF y DOCX (Word), entre otros.

\hypertarget{introducciuxf3n-1}{%
\subsection{Introducción}\label{introducciuxf3n-1}}

R-Markdown permite combinar Markdown con \texttt{R}. \href{http://daringfireball.net/projects/markdown/}{Markdown} se diseñó inicialmente para la creación de páginas web a partir de documentos de texto de forma muy sencilla y rápida (tiene unas reglas sintácticas muy simples). Actualmente gracias a múltiples herramientas como \href{http://pandoc.org/}{pandoc} permite generar múltiples tipos de documentos (incluido LaTeX; ver \href{http://rmarkdown.rstudio.com/authoring_pandoc_markdown.html}{Pandoc Markdown})

Para más detalles ver \url{http://rmarkdown.rstudio.com}.

También se dispone de información en la ayuda de \emph{RStudio}:

\begin{itemize}
\item
  \emph{Help \textgreater{} Markdown Quick Reference}
\item
  \emph{Help \textgreater{} Cheatsheets \textgreater{} R Markdown Cheat Sheet}
\item
  \emph{Help \textgreater{} Cheatsheets \textgreater{} R Markdown Reference Guide}
\end{itemize}

Al renderizar un fichero rmarkdown se generará un documento que incluye el código \texttt{R}
y los resultados incrustados en el documento.
En \emph{RStudio} basta con hacer clic en el botón \textbf{Knit HTML}.
En \texttt{R} se puede emplear la funcion \texttt{render} del paquete \emph{rmarkdown}
(por ejemplo: \texttt{render("8-Informes.Rmd")}).
También se puede abrir directamente el informe generado:

\begin{verbatim}
library(rmarkdown)
browseURL(url = render("8-Informes.Rmd"))
\end{verbatim}

\hypertarget{inclusiuxf3n-de-cuxf3digo-r}{%
\subsection{Inclusión de código R}\label{inclusiuxf3n-de-cuxf3digo-r}}

Se puede incluir código R entre los delimitadores \texttt{\textasciigrave{}\textasciigrave{}\textasciigrave{}\{r\}} y \texttt{\textasciigrave{}\textasciigrave{}\textasciigrave{}}. Por defecto, se mostrará el código, se evaluará y se mostrarán los resultados justo a continuación:

\begin{Shaded}
\begin{Highlighting}[]
\FunctionTok{head}\NormalTok{(mtcars[}\DecValTok{1}\SpecialCharTok{:}\DecValTok{3}\NormalTok{])}
\end{Highlighting}
\end{Shaded}

\begin{verbatim}
##                    mpg cyl disp
## Mazda RX4         21.0   6  160
## Mazda RX4 Wag     21.0   6  160
## Datsun 710        22.8   4  108
## Hornet 4 Drive    21.4   6  258
## Hornet Sportabout 18.7   8  360
## Valiant           18.1   6  225
\end{verbatim}

\begin{Shaded}
\begin{Highlighting}[]
\FunctionTok{summary}\NormalTok{(mtcars[}\DecValTok{1}\SpecialCharTok{:}\DecValTok{3}\NormalTok{])}
\end{Highlighting}
\end{Shaded}

\begin{verbatim}
##       mpg             cyl             disp      
##  Min.   :10.40   Min.   :4.000   Min.   : 71.1  
##  1st Qu.:15.43   1st Qu.:4.000   1st Qu.:120.8  
##  Median :19.20   Median :6.000   Median :196.3  
##  Mean   :20.09   Mean   :6.188   Mean   :230.7  
##  3rd Qu.:22.80   3rd Qu.:8.000   3rd Qu.:326.0  
##  Max.   :33.90   Max.   :8.000   Max.   :472.0
\end{verbatim}

En \emph{RStudio} pulsando ``Ctrl + Alt + I'' o en el icono correspondiente se incluye un trozo de código.

Se puede incluir código en línea empleando \texttt{\textasciigrave{}r\ código\textasciigrave{}},
por ejemplo \texttt{\textasciigrave{}r\ 2\ +\ 2\textasciigrave{}} produce 4.

\hypertarget{inclusiuxf3n-de-gruxe1ficos}{%
\subsection{Inclusión de gráficos}\label{inclusiuxf3n-de-gruxe1ficos}}

Se pueden generar gráficos:

\begin{center}\includegraphics[width=0.7\linewidth]{12-Informes_files/figure-latex/figura1-1} \end{center}

Los trozos de código pueden tener nombre y opciones, se establecen en la cabecera de la forma
\texttt{\textasciigrave{}\textasciigrave{}\textasciigrave{}\{r\ nombre,\ op1,\ op2\}}
(en el caso anterior no se muestra el código, al haber empleado \texttt{\textasciigrave{}\textasciigrave{}\textasciigrave{}\{r,\ echo=FALSE\}}).
Para un listado de las opciones disponibles ver \url{http://yihui.name/knitr/options}.

En \emph{RStudio} se puede pulsar en los iconos a la derecha del chunk para establecer opciones,
ejecutar todo el código anterior o sólo el correspondiente trozo.

\hypertarget{inclusiuxf3n-de-tablas}{%
\subsection{Inclusión de tablas}\label{inclusiuxf3n-de-tablas}}

Las tablas en markdown son de la forma:

\begin{verbatim}
| First Header  | Second Header |
| ------------- | ------------- |
| Row1 Cell1    | Row1 Cell2    |
| Row2 Cell1    | Row2 Cell2    |
\end{verbatim}

Por ejemplo:

\begin{longtable}[]{@{}ll@{}}
\toprule
Variable & Descripción \\
\midrule
\endhead
mpg & Millas / galón (EE.UU.) \\
cyl & Número de cilindros \\
disp & Desplazamiento (pulgadas cúbicas) \\
hp & Caballos de fuerza bruta \\
drat & Relación del eje trasero \\
wt & Peso (miles de libras) \\
qsec & Tiempo de 1/4 de milla \\
vs & Cilindros en V/Straight (0 = cilindros en V, 1 = cilindros en línea) \\
am & Tipo de transmisión (0 = automático, 1 = manual) \\
gear & Número de marchas (hacia adelante) \\
carb & Número de carburadores \\
\bottomrule
\end{longtable}

Para convertir resultados de R en tablas de una forma simple se puede emplear la función \texttt{kable} del paquete \emph{knitr}:

\begin{Shaded}
\begin{Highlighting}[]
\NormalTok{knitr}\SpecialCharTok{::}\FunctionTok{kable}\NormalTok{(}
  \FunctionTok{head}\NormalTok{(mtcars), }
  \AttributeTok{caption =} \StringTok{"Una kable knitr"}
\NormalTok{)}
\end{Highlighting}
\end{Shaded}

\begin{table}

\caption{\label{tab:kable}Una kable knitr}
\centering
\begin{tabular}[t]{l|r|r|r|r|r|r|r|r|r|r|r}
\hline
  & mpg & cyl & disp & hp & drat & wt & qsec & vs & am & gear & carb\\
\hline
Mazda RX4 & 21.0 & 6 & 160 & 110 & 3.90 & 2.620 & 16.46 & 0 & 1 & 4 & 4\\
\hline
Mazda RX4 Wag & 21.0 & 6 & 160 & 110 & 3.90 & 2.875 & 17.02 & 0 & 1 & 4 & 4\\
\hline
Datsun 710 & 22.8 & 4 & 108 & 93 & 3.85 & 2.320 & 18.61 & 1 & 1 & 4 & 1\\
\hline
Hornet 4 Drive & 21.4 & 6 & 258 & 110 & 3.08 & 3.215 & 19.44 & 1 & 0 & 3 & 1\\
\hline
Hornet Sportabout & 18.7 & 8 & 360 & 175 & 3.15 & 3.440 & 17.02 & 0 & 0 & 3 & 2\\
\hline
Valiant & 18.1 & 6 & 225 & 105 & 2.76 & 3.460 & 20.22 & 1 & 0 & 3 & 1\\
\hline
\end{tabular}
\end{table}

Otros paquetes proporcionan opciones adicionales: \emph{xtable}, \emph{stargazer}, \emph{pander}, \emph{tables} y \emph{ascii}.

\hypertarget{extracciuxf3n-del-cuxf3digo-r}{%
\subsection{Extracción del código R}\label{extracciuxf3n-del-cuxf3digo-r}}

Para generar un fichero con el código R se puede emplear la función \texttt{purl} del paquete \emph{knitr}. Por ejemplo:

\begin{verbatim}
purl("8-Informes.Rmd")
\end{verbatim}

Si se quiere además el texto rmarkdown como comentarios tipo \emph{spin}, se puede emplear:

\begin{verbatim}
purl("8-Informes.Rmd", documentation = 2)
\end{verbatim}

\hypertarget{spin}{%
\section{Spin}\label{spin}}

Una forma rápida de crear este tipo de informes a partir de un fichero de código R es emplear la funcion
\texttt{spin} del paquete \emph{knitr} (ver p.e. \url{http://yihui.name/knitr/demo/stitch}).

Para ello se debe comentar todo lo que no sea código R de una forma especial:

\begin{itemize}
\tightlist
\item
  El texto rmarkdown se comenta con \texttt{\#\textquotesingle{}}. Por ejemplo:
  \texttt{\#\textquotesingle{}\ \#\ Este\ es\ un\ título\ de\ primer\ nivel\ \ \ \ \ \#\textquotesingle{}\ \#\#\ Este\ es\ un\ título\ de\ segundo\ nivel}
\item
  Las opciones de un trozo de código se comentan con \texttt{\#+}. Por ejemplo:
  \texttt{\#+\ setup,\ include=FALSE\ \ \ \ \ opts\_chunk\$set(comment=NA,\ prompt=TRUE,\ dev=\textquotesingle{}svg\textquotesingle{},\ fig.height=6,\ fig.width=6)}
\end{itemize}

Para generar el informe se puede emplear la funcion \texttt{spin} del paquete \emph{knitr}. Por ejemplo: \texttt{spin("Ridge\_Lasso.R"))}.
También se podría abrir directamente el informe generado:

\begin{verbatim}
browseURL(url = knitr::spin("Ridge_Lasso.R"))
\end{verbatim}

Pero puede ser recomendable renderizarlo con rmarkdown:

\begin{verbatim}
library(rmarkdown)
browseURL(url = render(knitr::spin("Ridge_Lasso.R", knit = FALSE)))
\end{verbatim}

En \emph{RStudio} basta con pulsar ``Ctrl + Shift + K'' o seleccionar \emph{File \textgreater{} Knit Document} (en las últimas versiones también \emph{File \textgreater{} Compile Notebook} o hacer clic en el icono correspondiente).

\hypertarget{referencias}{%
\chapter*{Referencias}\label{referencias}}
\addcontentsline{toc}{chapter}{Referencias}

Fernández-Casal R., Costa J. y Oviedo de la Fuente, M. (2021). \emph{\href{https://rubenfcasal.github.io/aprendizaje_estadistico}{Aprendizaje Estadístico}}. \href{https://github.com/rubenfcasal/aprendizaje_estadistico}{github}.

Gil Bellosta C.J. (2018). \emph{\href{https://www.datanalytics.com/libro_r/index.html}{R para profesionales de los datos: una introducción}}.

Grolemund, G. (2014). \emph{\href{https://rstudio-education.github.io/hopr}{Hands-on programming with R: Write your own functions and simulations}}, \href{http://shop.oreilly.com/product/0636920028574.do}{O'Reilly}.

Matloff, N. (2011). \emph{The art of R programming: A tour of statistical software design}, \href{https://www.nostarch.com/artofr.htm}{No Starch Press}.

Quintela del Rio A. (2019). \emph{\href{https://bookdown.org/aquintela/EBE}{Estadística Básica Edulcorada}}

Wickham, H., y Grolemund, G. (2016). \emph{\href{http://r4ds.had.co.nz}{R for data science: import, tidy, transform, visualize, and model data}}, \href{https://es.r4ds.hadley.nz}{online-castellano}, \href{http://shop.oreilly.com/product/0636920034407.do}{O'Reilly}.

\hypertarget{links}{%
\section*{Enlaces}\label{links}}
\addcontentsline{toc}{section}{Enlaces}

\textbf{\emph{Repositorio}}: \href{https://github.com/rubenfcasal/intror}{rubenfcasal/intror}

\textbf{Recursos para el aprendizaje de R}: En este \href{https://rubenfcasal.github.io/post/ayuda-y-recursos-para-el-aprendizaje-de-r}{post} se muestran algunos recursos que pueden ser útiles para el aprendizaje de R y la obtención de ayuda.

\href{https://bookdown.org}{\textbf{\emph{Bookdown}}}:

\begin{itemize}
\item
  Fernández-Casal, R. y Cotos-Yáñez, T.R. (2018). \emph{\href{https://rubenfcasal.github.io/bookdown_intro}{Escritura de libros con bookdown}}, \href{https://github.com/rubenfcasal/bookdown_intro}{github}. Incluye un apéndice con una \href{https://rubenfcasal.github.io/bookdown_intro/rmarkdown.html}{Introducción a RMarkdown}.
\item
  Kuhn, M. y Silge, J. (2022). \emph{\href{https://www.tmwr.org}{Tidy Modeling with R}}. \href{https://amzn.to/35Hn96s}{O'Reill}.
\item
  Wickham, H. (2015). \emph{\href{http://r-pkgs.had.co.nz/}{R packages: organize, test, document, and share your code}} (actualmente 2ª edición en desarrollo con H. Bryan), \href{http://shop.oreilly.com/product/0636920034421.do}{O'Reilly, 1ª edición}.
\item
  Wickham, H. (2019). \emph{\href{https://adv-r.hadley.nz/}{Advanced R, 2ª edición}}, \href{https://www.amazon.com/dp/0815384572}{Chapman \& Hall}, \href{http://adv-r.had.co.nz/}{1ª edición.}.
\end{itemize}

\href{https://posit.co}{*\textbf{Posit (RStudio)}}

\begin{itemize}
\item
  \href{https://posit.co/blog}{Blog}
\item
  \href{https://posit.co/resources/videos}{Videos}
\item
  \href{https://posit.co/resources/cheatsheets}{Chuletas (Cheatsheets)}
\item
  \href{https://www.tidyverse.org/}{\textbf{\emph{tidyverse}}}:

  \begin{itemize}
  \item
    \href{https://dplyr.tidyverse.org}{dplyr}
  \item
    \href{https://tibble.tidyverse.org}{tibble}
  \item
    \href{https://tidyr.tidyverse.org}{tidyr}
  \item
    \href{https://stringr.tidyverse.org}{stringr}
  \item
    \href{https://readr.tidyverse.org}{readr}
  \item
    \href{https://solutions.posit.co/connections/db}{Best Practices in Working with Databases}
  \end{itemize}
\item
  \href{https://www.tidymodels.org}{tidymodels}
\item
  \href{https://spark.rstudio.com/}{sparklyr}
\item
  \href{http://shiny.rstudio.com}{shiny}
\end{itemize}

\hypertarget{bibliografuxeda-complementaria}{%
\section*{Bibliografía complementaria}\label{bibliografuxeda-complementaria}}
\addcontentsline{toc}{section}{Bibliografía complementaria}

Beeley (2015). \emph{Web Application Development with R Using Shiny}.
Packt Publishing.

Bivand \emph{et al.} (2008). \emph{Applied Spatial Data Analysis with R}. Springer.

James \emph{et al.} (2008). \emph{An Introduction to Statistical Learning: with
Aplications in R}. Springer.

Kolaczyk y Csárdi (2014). \emph{Statistical analysis of network data with R}. Springer.

Munzert \emph{et al.} (2014). \emph{Automated Data Collection with R:
A Practical Guide to Web Scraping and Text Mining}. Wiley.

Ramsay \emph{et al.} (2009). \emph{Functional Data Analysis with R and MATLAB}.
Springer.

Van der Loo y de Jonge (2012). \emph{Learning RStudio for R Statistical
Computing}. Packt Publishing.

Williams (2011). \emph{Data Mining with Rattle and R}. Springer.

Wood (2006). \emph{Generalized Additive Models: An Introduction with R}. Chapman.

Yihui Xie (2015). \emph{Dynamic Documents with R and knitr}. Chapman.

\hypertarget{appendix-apendices}{%
\appendix}


\hypertarget{instalacion}{%
\chapter{Instalación de R}\label{instalacion}}

En la web del proyecto R (\href{http://www.r-project.org}{www.r-project.org}) está disponible mucha información sobre este entorno estadístico.

\begin{figure}[!htb]

{\centering \includegraphics[width=0.45\linewidth]{figuras/rproject} \includegraphics[width=0.45\linewidth]{figuras/cran} 

}

\caption{Web de [R-project](https://r-project.org) y [CRAN](https://cran.r-project.org).}\label{fig:rproject}
\end{figure}

Las descargas se realizan a través de la web del CRAN (The Comprehensive R Archive Network), con múltiples \href{https://cran.r-project.org/mirrors.html}{mirrors}:

\begin{itemize}
\tightlist
\item
  \emph{Oficina de Software Libre (A Coruña)} (CIXUG): \href{http://ftp.cixug.es/CRAN/}{ftp.cixug.es/CRAN}.
\item
  \emph{Spanish National Research Network (Madrid)} (RedIRIS): \href{http://cran.es.r-project.org/}{cran.es.r-project.org}.
\end{itemize}

\hypertarget{windows}{%
\section{Instalación de R en Windows}\label{windows}}

Seleccionando \href{http://ftp.cixug.es/CRAN/bin/windows/}{Download R for Windows} y posteriormente \href{http://ftp.cixug.es/CRAN/bin/windows/base/}{base} accedemos al enlace con el instalador de R para Windows (actualmente de la versión \href{http://ftp.cixug.es/CRAN/bin/windows/base/R-4.2.2-win.exe}{4.2.2}).

\begin{figure}[!htb]

{\centering \includegraphics[width=0.5\linewidth]{figuras/R351} 

}

\caption{Web de [descarga de R para Windows](http://ftp.cixug.es/CRAN/bin/windows).}\label{fig:rdownload}
\end{figure}

\hypertarget{asistente-de-instalaciuxf3n}{%
\subsection{Asistente de instalación}\label{asistente-de-instalaciuxf3n}}

Durante el proceso de instalación la recomendación (para evitar posibles problemas) es seleccionar ventanas simples SDI en lugar de múltiples ventanas MDI (hay que utilizar \emph{Opciones de configuración}).

\begin{figure}[!htb]

{\centering \includegraphics[width=0.9\linewidth]{figuras/asistente} 

}

\caption{Pasos del asistente para instalación de R en Windows.}\label{fig:asistente}
\end{figure}

Una vez terminada la instalación, al abrir R aparece la ventana de la consola (simula una ventana de comandos de Unix) que permite ejecutar comandos de R.

Por defecto se instalan un conjunto de paquetes base de R (que se cargan automáticamente al iniciarlo) y un conjunto de \href{https://cran.r-project.org/src/contrib/4.1.2/Recommended}{paquetes recomendados} (que se pueden cargar empleando el comando \texttt{library()}),
pero hay disponibles miles de paquetes que cubren literalmente todos los campos del análisis de datos.
Ver por ejemplo:

\begin{itemize}
\item
  \href{https://cran.r-project.org/web/packages/index.html}{CRAN: Packages}
\item
  \href{https://cran.r-project.org/web/views}{CRAN: Task Views}
\end{itemize}

\hypertarget{paquetes-win}{%
\subsection{Instalación de paquetes}\label{paquetes-win}}

Después de la instalación de R suele ser necesario instalar paquetes adicionales (puede ser recomendable \emph{Ejecutar como administrador} R para evitar problemas de permiso de escritura en la carpeta \emph{C:\textbackslash Program Files\textbackslash R\textbackslash R-X.Y.Z\textbackslash library}, o cambiar previamente los permisos de esta carpeta como se indica \protect\hyperlink{library}{aquí}).

Para ejecutar los ejemplos mostrados en el libro sería necesario tener instalados los siguientes paquetes:
\href{https://CRAN.R-project.org/package=lattice}{\texttt{lattice}}, \href{https://CRAN.R-project.org/package=ggplot2}{\texttt{ggplot2}}, \href{https://CRAN.R-project.org/package=foreign}{\texttt{foreign}}, \href{https://CRAN.R-project.org/package=car}{\texttt{car}}, \href{https://CRAN.R-project.org/package=leaps}{\texttt{leaps}}, \href{https://CRAN.R-project.org/package=MASS}{\texttt{MASS}}, \href{https://CRAN.R-project.org/package=RcmdrMisc}{\texttt{RcmdrMisc}}, \href{https://CRAN.R-project.org/package=lmtest}{\texttt{lmtest}}, \href{https://CRAN.R-project.org/package=glmnet}{\texttt{glmnet}}, \href{https://CRAN.R-project.org/package=mgcv}{\texttt{mgcv}}, \href{https://CRAN.R-project.org/package=rmarkdown}{\texttt{rmarkdown}}, \href{https://CRAN.R-project.org/package=knitr}{\texttt{knitr}}, \href{https://CRAN.R-project.org/package=dplyr}{\texttt{dplyr}}, \href{https://CRAN.R-project.org/package=tidyr}{\texttt{tidyr}}.
Por ejemplo mediante los siguientes comandos:

\begin{Shaded}
\begin{Highlighting}[]
\NormalTok{pkgs }\OtherTok{\textless{}{-}} \FunctionTok{c}\NormalTok{(}\StringTok{"lattice"}\NormalTok{, }\StringTok{"ggplot2"}\NormalTok{, }\StringTok{"foreign"}\NormalTok{, }\StringTok{"car"}\NormalTok{, }\StringTok{"leaps"}\NormalTok{, }\StringTok{"MASS"}\NormalTok{, }\StringTok{"RcmdrMisc"}\NormalTok{, }
          \StringTok{"lmtest"}\NormalTok{, }\StringTok{"glmnet"}\NormalTok{, }\StringTok{"mgcv"}\NormalTok{, }\StringTok{"rmarkdown"}\NormalTok{, }\StringTok{"knitr"}\NormalTok{, }\StringTok{"dplyr"}\NormalTok{, }\StringTok{"tidyr"}\NormalTok{)}
\FunctionTok{install.packages}\NormalTok{(}\FunctionTok{setdiff}\NormalTok{(pkgs, }\FunctionTok{installed.packages}\NormalTok{()[,}\StringTok{"Package"}\NormalTok{]), }\AttributeTok{dependencies =} \ConstantTok{TRUE}\NormalTok{)}
\end{Highlighting}
\end{Shaded}

(puede que haya que seleccionar el repositorio de descarga, e.g.~\emph{Spain (Madrid)}).

El código anterior no reinstala los paquetes ya instalados, por lo que podrían aparecer problemas debidos a incompatibilidades entre versiones (aunque no suele ocurrir, salvo que nuestra instalación de R esté muy desactualizada).
Si es el caso, en lugar de la última línea se puede ejecutar:

\begin{Shaded}
\begin{Highlighting}[]
\FunctionTok{install.packages}\NormalTok{(pkgs, }\AttributeTok{dependencies =} \ConstantTok{TRUE}\NormalTok{) }\CommentTok{\# Instala todos...}
\end{Highlighting}
\end{Shaded}

\hypertarget{library}{%
\subsubsection{\texorpdfstring{Cambiar los permisos de la carpeta \emph{library} (opcional)}{Cambiar los permisos de la carpeta library (opcional)}}\label{library}}

Para evitar problemas con la instalación de paquetes en Windows (y evitar también que los paquetes se instalen en \emph{Documentos\textbackslash R\textbackslash win-library\textbackslash X.Y}) se puede dar permiso de \emph{control total} a los usuarios del equipo en el subdirectorio \emph{library} de la instalación de R.
Para ello, pulsar con el botón derecho en esta carpeta (e.g.~\emph{C:\textbackslash Program Files\textbackslash R\textbackslash R-4.2.2\textbackslash library}), seleccionar \emph{Propiedades \textgreater{} Seguridad \textgreater{} Editar}, seleccionar los \emph{Usuarios} del equipo, marcar \emph{Control total} y \emph{Aplicar}.

\begin{figure}[!htb]

{\centering \includegraphics[width=0.45\linewidth]{figuras/propiedades} \includegraphics[width=0.45\linewidth]{figuras/permisos} 

}

\caption{Pasos en Windows para cambiar permisos en la carpeta library.}\label{fig:permisos-library}
\end{figure}

\hypertarget{ide}{%
\subsection{Instalación de RStudio Desktop}\label{ide}}

Aunque la consola de R dispone de un editor básico de código (script), puede ser recomendable trabajar con un editor de comandos más cómodo y flexible.
El entorno de desarrollo (\emph{Integrated Development Environment}, IDE) recomendado es \href{https://posit.co/products/open-source/rstudio}{RStudio}.
Está disponible para la mayoría de plataformas\footnote{También hay una versión para servidores: \href{https://posit.co/download/rstudio-server}{RStudio Server}} e integra una gran cantidad de herramientas, que permiten desde la generación de informes, hasta la gestión de distintos tipos de proyectos, depuración de código, control de versiones, etc.
También es compatible con otros lenguajes, incluido \href{https://support.posit.co/hc/en-us/articles/1500007929061-Using-Python-with-the-RStudio-IDE}{Python}.

Una vez instalado R, para instalar RStudio Desktop basta con descargar el correspondiente archivo de instalación de \href{https://posit.co/download/rstudio-desktop}{\emph{https://posit.co/download/rstudio-desktop}} y seguir las instrucciones.

Este entorno se describe en la Sección \ref{rstudio}.

\hypertarget{op-rstudio-win}{%
\subsubsection{Configuración adicional de RStudio (opcional)}\label{op-rstudio-win}}

En lugar de emplear los visores de gráficos, ayuda y navegador web integrados, nos puede interesar que los gráficos se muestren en ventanas independientes y las páginas web en el navegador del equipo.
Esto se puede conseguir modificando los archivos de configuración (en el directorio \emph{C:\textbackslash Program Files\textbackslash RStudio\textbackslash R} en Windows y \emph{/Applications/RStudio.app/Contents/Resources/R} en Linux), que normalmente habrá que editar como administrador.

Por defecto los gráficos generados desde RStudio se mostrarán en la pestaña \emph{Plots} panel inferior derecho y por ejemplo puede aparecer errores si el área gráfica es demasiado pequeña.
Para utilizar el dispositivo gráfico de R habría que modificar las siguientes líneas de \emph{C:\textbackslash Program Files\textbackslash RStudio\textbackslash R\textbackslash Tools.R}:

\begin{Shaded}
\begin{Highlighting}[]
\CommentTok{\# set our graphics device as the default and cause it to be created/set}
\FunctionTok{.rs.addFunction}\NormalTok{( }\StringTok{"initGraphicsDevice"}\NormalTok{, }\ControlFlowTok{function}\NormalTok{()}
\NormalTok{\{}
   \CommentTok{\# options(device="RStudioGD")}
   \CommentTok{\# grDevices::deviceIsInteractive("RStudioGD")}
\NormalTok{  grDevices}\SpecialCharTok{::}\FunctionTok{deviceIsInteractive}\NormalTok{()}
\NormalTok{\})}
\end{Highlighting}
\end{Shaded}

El visor integrado de RStudio no resulta muy cómodo para navegar por la ayuda de las funciones (por ejemplo no permite hacer zoom o abrir múltiples ventanas).
Para utilizar en su lugar el navegador del equipo habría que comentar las siguientes líneas de \emph{C:\textbackslash Program Files\textbackslash RStudio\textbackslash R\textbackslash Options.R}:

\begin{Shaded}
\begin{Highlighting}[]
\CommentTok{\# \# custom browseURL implementation.}
\CommentTok{\# .rs.setOption("browser", function(url)}
\CommentTok{\# \{}
   \CommentTok{\# .Call("rs\_browseURL", url, PACKAGE = "(embedding)")}
\CommentTok{\# \})}
\end{Highlighting}
\end{Shaded}

\hypertarget{ubuntu}{%
\section{Instalación de R en Ubuntu/Devian}\label{ubuntu}}

Instalar dependencias:

\begin{Shaded}
\begin{Highlighting}[]
\FunctionTok{sudo}\NormalTok{ apt install libcurl4{-}gnutls{-}dev libgit2{-}dev libxml2{-}dev libssl{-}dev }
\end{Highlighting}
\end{Shaded}

Si aparecen problemas asegurarse de que los repositorios \emph{universe} y \emph{multiverse} están disponibles:

\begin{Shaded}
\begin{Highlighting}[]
\FunctionTok{sudo}\NormalTok{ add{-}apt{-}repository universe}
\FunctionTok{sudo}\NormalTok{ add{-}apt{-}repository multiverse}
\FunctionTok{sudo}\NormalTok{ apt update}
\end{Highlighting}
\end{Shaded}

Se puede instalar R desde estos repositorios, pero normalmente no será la versión más actualizada y no lo recomendaría.

\hypertarget{instalaciuxf3n-de-r-desde-cran}{%
\subsection{\texorpdfstring{Instalación de R desde \href{https://cran.r-project.org/bin/linux/ubuntu}{CRAN}}{Instalación de R desde CRAN}}\label{instalaciuxf3n-de-r-desde-cran}}

Añadir la llave de firma GPG, añadir el repositorio CRAN a la lista de fuentes (para ver la versión de ubuntu se puede ejecutar \texttt{lsb\_release\ -a}, el siguiente código ya la obtiene directamente) e instalar R:

\begin{Shaded}
\begin{Highlighting}[]
\CommentTok{\# Cambiar a root (alternativamente añadir \textasciigrave{}sudo\textasciigrave{} al principio de los comandos)}
\FunctionTok{sudo} \AttributeTok{{-}i}
\CommentTok{\# update indices}
\ExtensionTok{apt}\NormalTok{ update }\AttributeTok{{-}qq}
\CommentTok{\# install two helper packages we need}
\ExtensionTok{apt}\NormalTok{ install }\AttributeTok{{-}{-}no{-}install{-}recommends}\NormalTok{ software{-}properties{-}common dirmngr}
\CommentTok{\# add the signing key (by Michael Rutter) for these repos}
\CommentTok{\# To verify key, run gpg {-}{-}show{-}keys /etc/apt/trusted.gpg.d/cran\_ubuntu\_key.asc }
\CommentTok{\# Fingerprint: 298A3A825C0D65DFD57CBB651716619E084DAB9}
\FunctionTok{wget} \AttributeTok{{-}qO{-}}\NormalTok{ https://cloud.r{-}project.org/bin/linux/ubuntu/marutter\_pubkey.asc }\KeywordTok{|} \FunctionTok{sudo}\NormalTok{ tee }\AttributeTok{{-}a}\NormalTok{ /etc/apt/trusted.gpg.d/cran\_ubuntu\_key.asc}
\CommentTok{\# add the R 4.0 repo from CRAN {-}{-} adjust \textquotesingle{}focal\textquotesingle{} to \textquotesingle{}groovy\textquotesingle{} or \textquotesingle{}bionic\textquotesingle{} as needed}
\ExtensionTok{add{-}apt{-}repository} \StringTok{"deb https://cloud.r{-}project.org/bin/linux/ubuntu }\VariableTok{$(}\ExtensionTok{lsb\_release} \AttributeTok{{-}cs}\VariableTok{)}\StringTok{{-}cran40/"}
\ExtensionTok{apt{-}get}\NormalTok{ update}
\ExtensionTok{apt{-}get}\NormalTok{ install r{-}base r{-}base{-}dev}
\BuiltInTok{logout}
\end{Highlighting}
\end{Shaded}

\hypertarget{instalaciuxf3n-de-devtools-y-demuxe1s-paquetes}{%
\subsection{Instalación de devtools y demás paquetes}\label{instalaciuxf3n-de-devtools-y-demuxe1s-paquetes}}

Ejecutar R en modo administrador para que los paquetes que se instalen estén disponibles para todos los usuarios

\begin{Shaded}
\begin{Highlighting}[]
\FunctionTok{sudo} \AttributeTok{{-}i}\NormalTok{ R}
\end{Highlighting}
\end{Shaded}

Ejecutar en la consola de R

\begin{Shaded}
\begin{Highlighting}[]
\FunctionTok{install.packages}\NormalTok{(}\StringTok{\textquotesingle{}devtools\textquotesingle{}}\NormalTok{, }\AttributeTok{dependencies =} \ConstantTok{TRUE}\NormalTok{)}
\end{Highlighting}
\end{Shaded}

Si aparecen problemas mirar \href{http://stackoverflow.com/a/20924082}{stackoverflow - Problems installing the devtools package}.

Instalar el resto de paquetes como se muestra en la sección de \protect\hyperlink{paquetes-win}{Windows}.

\hypertarget{ayuda-html}{%
\subsection{Ayuda html}\label{ayuda-html}}

Si queremos la ayuda html (en un entorno gráfico con un navegador web instalado):

\begin{Shaded}
\begin{Highlighting}[]
\BuiltInTok{echo} \StringTok{"options(help\_type=\textquotesingle{}html\textquotesingle{})"} \KeywordTok{|} \FunctionTok{sudo}\NormalTok{ tee }\AttributeTok{{-}a}\NormalTok{ /etc/R/Rprofile.site}
\end{Highlighting}
\end{Shaded}

\hypertarget{actualizar-r}{%
\subsection{Actualizar R}\label{actualizar-r}}

\begin{Shaded}
\begin{Highlighting}[]
\FunctionTok{sudo}\NormalTok{ apt{-}get update}
\FunctionTok{sudo}\NormalTok{ apt{-}get upgrade}
\end{Highlighting}
\end{Shaded}

\hypertarget{instalacion-de-rstudio-desktop}{%
\subsection{Instalacion de RStudio Desktop}\label{instalacion-de-rstudio-desktop}}

Antes de nada, puede ser recomendable crear un directorio donde descargar el instalador:

\begin{Shaded}
\begin{Highlighting}[]
\FunctionTok{mkdir}\NormalTok{ installr}
\BuiltInTok{cd}\NormalTok{ installr}
\end{Highlighting}
\end{Shaded}

Buscar la versión actualizada de RStudio en \href{https://posit.co/download/rstudio-desktop}{Download RStudio} correspondiente a la versión de Ubuntu, en este caso emplearemos ``\url{https://download1.rstudio.org/electron/bionic/amd64/rstudio-2022.12.0-353-amd64.deb}'',
para Ubuntu 18+/Debian 10.

\begin{Shaded}
\begin{Highlighting}[]
\FunctionTok{sudo}\NormalTok{ apt{-}get install gdebi{-}core}
\FunctionTok{wget}\NormalTok{ https://download1.rstudio.org/electron/bionic/amd64/rstudio{-}2022.12.0{-}353{-}amd64.deb}
\FunctionTok{sudo}\NormalTok{ gdebi }\AttributeTok{{-}n}\NormalTok{ rstudio{-}2022.12.0{-}353{-}amd64.deb}
\end{Highlighting}
\end{Shaded}

Al igual que como se mostró para el caso de \protect\hyperlink{op-rstudio-win}{Windows}, nos puede interesar modificar la configuración de de RStudio para mostrar los gráficos en ventanas independientes y las páginas web en el navegador del equipo.
En este caso se procedería de forma idéntica, modificando los archivos de configuración en \emph{/Applications/RStudio.app/Contents/Resources/R}, editándolos como administrador.

\hypertarget{macosx}{%
\section{Instalación en Mac OS X}\label{macosx}}

Instalar R de \url{http://cran.es.r‐project.org/bin/macosx} siguiendo los pasos habituales.

Para instalar R-Commander (\url{https://socialsciences.mcmaster.ca/jfox/Misc/Rcmdr/installation-notes.html}) es necesario disponer de las librerías gráficas X11, como a partir de OS X Lion ya no están instaladas por defecto en el sistema, hay que instalar las librerías Open Source XQuartz \url{https://www.xquartz.org}.

Finalmente, para instalar \texttt{Rcmdr} ejecutar en la consola de R:

\begin{Shaded}
\begin{Highlighting}[]
\FunctionTok{install.packages}\NormalTok{(}\StringTok{"Rcmdr"}\NormalTok{, }\AttributeTok{dependencies =} \ConstantTok{TRUE}\NormalTok{)}
\end{Highlighting}
\end{Shaded}

\hypertarget{manipulaciuxf3n-de-datos-con-dplyr}{%
\chapter{Manipulación de datos con dplyr}\label{manipulaciuxf3n-de-datos-con-dplyr}}

\hypertarget{el-paquete-dplyr}{%
\section{\texorpdfstring{El paquete \textbf{dplyr}}{El paquete dplyr}}\label{el-paquete-dplyr}}

\begin{Shaded}
\begin{Highlighting}[]
\FunctionTok{library}\NormalTok{(dplyr)}
\end{Highlighting}
\end{Shaded}

\textbf{dplyr} Permite sustituir funciones base de R (como \texttt{split()}, \texttt{subset()},
\texttt{apply()}, \texttt{sapply()}, \texttt{lapply()}, \texttt{tapply()} y \texttt{aggregate()})
mediante una ``gramática'' más sencilla para la manipulación de datos:

\begin{itemize}
\tightlist
\item
  \texttt{select()} seleccionar variables/columnas (también \texttt{rename()}).
\item
  \texttt{mutate()} crear variables/columnas (también \texttt{transmute()}).
\item
  \texttt{filter()} seleccionar casos/filas (también \texttt{slice()}).
\item
  \texttt{arrange()} ordenar o organizar casos/filas.
\item
  \texttt{summarise()} resumir valores.
\item
  \texttt{group\_by()} permite operaciones por grupo empleando el concepto
  ``dividir-aplicar-combinar'' (\texttt{ungroup()} elimina el agrupamiento).
\end{itemize}

Puede trabajar con conjuntos de datos en distintos formatos:

\begin{itemize}
\tightlist
\item
  \texttt{data.frame}, \texttt{data.table}, \texttt{tibble}, \ldots{}
\item
  bases de datos relacionales (lenguaje SQL), \ldots{}
\item
  bases de datos \emph{Hadoop} (paquete \texttt{plyrmr}).
\end{itemize}

En lugar de operar sobre vectores como las funciones base,
opera sobre objetos de este tipo (solo nos centraremos en \texttt{data.frame}).

\hypertarget{datos-de-ejemplo-1}{%
\subsection{Datos de ejemplo}\label{datos-de-ejemplo-1}}

El fichero \emph{empleados.RData} contiene datos de empleados de un banco.
Supongamos por ejemplo que estamos interesados en estudiar si hay
discriminación por cuestión de sexo o raza.

\begin{Shaded}
\begin{Highlighting}[]
\FunctionTok{load}\NormalTok{(}\StringTok{"datos/empleados.RData"}\NormalTok{)}
\FunctionTok{data.frame}\NormalTok{(}\AttributeTok{Etiquetas =} \FunctionTok{attr}\NormalTok{(empleados, }\StringTok{"variable.labels"}\NormalTok{))  }\CommentTok{\# Listamos las etiquetas}
\end{Highlighting}
\end{Shaded}

\begin{verbatim}
##                              Etiquetas
## id                  Código de empleado
## sexo                              Sexo
## fechnac            Fecha de nacimiento
## educ            Nivel educativo (años)
## catlab               Categoría Laboral
## salario                 Salario actual
## salini                 Salario inicial
## tiempemp       Meses desde el contrato
## expprev     Experiencia previa (meses)
## minoria           Clasificación étnica
## sexoraza Clasificación por sexo y raza
\end{verbatim}

\begin{Shaded}
\begin{Highlighting}[]
\FunctionTok{attr}\NormalTok{(empleados, }\StringTok{"variable.labels"}\NormalTok{) }\OtherTok{\textless{}{-}} \ConstantTok{NULL}                  \CommentTok{\# Eliminamos las etiquetas para que no molesten...}
\end{Highlighting}
\end{Shaded}

\hypertarget{operaciones-con-variables-columnas}{%
\section{Operaciones con variables (columnas)}\label{operaciones-con-variables-columnas}}

\hypertarget{seleccionar-variables-con-select}{%
\subsection{\texorpdfstring{Seleccionar variables con \textbf{select()}}{Seleccionar variables con select()}}\label{seleccionar-variables-con-select}}

\begin{Shaded}
\begin{Highlighting}[]
\NormalTok{emplea2 }\OtherTok{\textless{}{-}} \FunctionTok{select}\NormalTok{(empleados, id, sexo, minoria, tiempemp, salini, salario)}
\FunctionTok{head}\NormalTok{(emplea2)}
\end{Highlighting}
\end{Shaded}

\begin{verbatim}
##   id   sexo minoria tiempemp salini salario
## 1  1 Hombre      No       98  27000   57000
## 2  2 Hombre      No       98  18750   40200
## 3  3  Mujer      No       98  12000   21450
## 4  4  Mujer      No       98  13200   21900
## 5  5 Hombre      No       98  21000   45000
## 6  6 Hombre      No       98  13500   32100
\end{verbatim}

Se puede cambiar el nombre (ver también \emph{?rename()})

\begin{Shaded}
\begin{Highlighting}[]
\FunctionTok{head}\NormalTok{(}\FunctionTok{select}\NormalTok{(empleados, sexo, }\AttributeTok{noblanca =}\NormalTok{ minoria, salario))}
\end{Highlighting}
\end{Shaded}

\begin{verbatim}
##     sexo noblanca salario
## 1 Hombre       No   57000
## 2 Hombre       No   40200
## 3  Mujer       No   21450
## 4  Mujer       No   21900
## 5 Hombre       No   45000
## 6 Hombre       No   32100
\end{verbatim}

Se pueden emplear los nombres de variables como índices:

\begin{Shaded}
\begin{Highlighting}[]
\FunctionTok{head}\NormalTok{(}\FunctionTok{select}\NormalTok{(empleados, sexo}\SpecialCharTok{:}\NormalTok{salario))}
\end{Highlighting}
\end{Shaded}

\begin{verbatim}
##     sexo    fechnac educ         catlab salario
## 1 Hombre 1952-02-03   15      Directivo   57000
## 2 Hombre 1958-05-23   16 Administrativo   40200
## 3  Mujer 1929-07-26   12 Administrativo   21450
## 4  Mujer 1947-04-15    8 Administrativo   21900
## 5 Hombre 1955-02-09   15 Administrativo   45000
## 6 Hombre 1958-08-22   15 Administrativo   32100
\end{verbatim}

\begin{Shaded}
\begin{Highlighting}[]
\FunctionTok{head}\NormalTok{(}\FunctionTok{select}\NormalTok{(empleados, }\SpecialCharTok{{-}}\NormalTok{(sexo}\SpecialCharTok{:}\NormalTok{salario)))}
\end{Highlighting}
\end{Shaded}

\begin{verbatim}
##   id salini tiempemp expprev minoria     sexoraza
## 1  1  27000       98     144      No Blanca varón
## 2  2  18750       98      36      No Blanca varón
## 3  3  12000       98     381      No Blanca mujer
## 4  4  13200       98     190      No Blanca mujer
## 5  5  21000       98     138      No Blanca varón
## 6  6  13500       98      67      No Blanca varón
\end{verbatim}

Hay opciones para considerar distintos criterios: \texttt{starts\_with()}, \texttt{ends\_with()},
\texttt{contains()}, \texttt{matches()}, \texttt{one\_of()} (ver \emph{?select}).

\begin{Shaded}
\begin{Highlighting}[]
\FunctionTok{head}\NormalTok{(}\FunctionTok{select}\NormalTok{(empleados, }\FunctionTok{starts\_with}\NormalTok{(}\StringTok{"s"}\NormalTok{)))}
\end{Highlighting}
\end{Shaded}

\begin{verbatim}
##     sexo salario salini     sexoraza
## 1 Hombre   57000  27000 Blanca varón
## 2 Hombre   40200  18750 Blanca varón
## 3  Mujer   21450  12000 Blanca mujer
## 4  Mujer   21900  13200 Blanca mujer
## 5 Hombre   45000  21000 Blanca varón
## 6 Hombre   32100  13500 Blanca varón
\end{verbatim}

\hypertarget{generar-nuevas-variables-con-mutate}{%
\subsection{\texorpdfstring{Generar nuevas variables con \textbf{mutate()}}{Generar nuevas variables con mutate()}}\label{generar-nuevas-variables-con-mutate}}

\begin{Shaded}
\begin{Highlighting}[]
\FunctionTok{head}\NormalTok{(}\FunctionTok{mutate}\NormalTok{(emplea2, }\AttributeTok{incsal =}\NormalTok{ salario }\SpecialCharTok{{-}}\NormalTok{ salini, }\AttributeTok{tsal =}\NormalTok{ incsal}\SpecialCharTok{/}\NormalTok{tiempemp ))}
\end{Highlighting}
\end{Shaded}

\begin{verbatim}
##   id   sexo minoria tiempemp salini salario incsal      tsal
## 1  1 Hombre      No       98  27000   57000  30000 306.12245
## 2  2 Hombre      No       98  18750   40200  21450 218.87755
## 3  3  Mujer      No       98  12000   21450   9450  96.42857
## 4  4  Mujer      No       98  13200   21900   8700  88.77551
## 5  5 Hombre      No       98  21000   45000  24000 244.89796
## 6  6 Hombre      No       98  13500   32100  18600 189.79592
\end{verbatim}

\hypertarget{operaciones-con-casos-filas}{%
\section{Operaciones con casos (filas)}\label{operaciones-con-casos-filas}}

\hypertarget{seleccionar-casos-con-filter}{%
\subsection{\texorpdfstring{Seleccionar casos con \textbf{filter()}}{Seleccionar casos con filter()}}\label{seleccionar-casos-con-filter}}

\begin{Shaded}
\begin{Highlighting}[]
\FunctionTok{head}\NormalTok{(}\FunctionTok{filter}\NormalTok{(emplea2, sexo }\SpecialCharTok{==} \StringTok{"Mujer"}\NormalTok{, minoria }\SpecialCharTok{==} \StringTok{"Sí"}\NormalTok{))}
\end{Highlighting}
\end{Shaded}

\begin{verbatim}
##   id  sexo minoria tiempemp salini salario
## 1 14 Mujer      Sí       98  16800   35100
## 2 23 Mujer      Sí       97  11100   24000
## 3 24 Mujer      Sí       97   9000   16950
## 4 25 Mujer      Sí       97   9000   21150
## 5 40 Mujer      Sí       96   9000   19200
## 6 41 Mujer      Sí       96  11550   23550
\end{verbatim}

\hypertarget{organizar-casos-con-arrange}{%
\subsection{\texorpdfstring{Organizar casos con \textbf{arrange()}}{Organizar casos con arrange()}}\label{organizar-casos-con-arrange}}

\begin{Shaded}
\begin{Highlighting}[]
\FunctionTok{head}\NormalTok{(}\FunctionTok{arrange}\NormalTok{(emplea2, salario))}
\end{Highlighting}
\end{Shaded}

\begin{verbatim}
##    id  sexo minoria tiempemp salini salario
## 1 378 Mujer      No       70  10200   15750
## 2 338 Mujer      No       74  10200   15900
## 3  90 Mujer      No       92   9750   16200
## 4 224 Mujer      No       82  10200   16200
## 5 411 Mujer      No       68  10200   16200
## 6 448 Mujer      Sí       66  10200   16350
\end{verbatim}

\begin{Shaded}
\begin{Highlighting}[]
\FunctionTok{head}\NormalTok{(}\FunctionTok{arrange}\NormalTok{(emplea2, }\FunctionTok{desc}\NormalTok{(salini), salario))}
\end{Highlighting}
\end{Shaded}

\begin{verbatim}
##    id   sexo minoria tiempemp salini salario
## 1  29 Hombre      No       96  79980  135000
## 2 343 Hombre      No       73  60000  103500
## 3 205 Hombre      No       83  52500   66750
## 4 160 Hombre      No       86  47490   66000
## 5 431 Hombre      No       66  45000   86250
## 6  32 Hombre      No       96  45000  110625
\end{verbatim}

\hypertarget{resumir-valores-con-summarise}{%
\section{\texorpdfstring{Resumir valores con \textbf{summarise()}}{Resumir valores con summarise()}}\label{resumir-valores-con-summarise}}

\begin{Shaded}
\begin{Highlighting}[]
\FunctionTok{summarise}\NormalTok{(empleados, }\AttributeTok{sal.med =} \FunctionTok{mean}\NormalTok{(salario), }\AttributeTok{n =} \FunctionTok{n}\NormalTok{())}
\end{Highlighting}
\end{Shaded}

\begin{verbatim}
##    sal.med   n
## 1 34419.57 474
\end{verbatim}

\hypertarget{agrupar-casos-con-group_by}{%
\section{\texorpdfstring{Agrupar casos con \textbf{group\_by()}}{Agrupar casos con group\_by()}}\label{agrupar-casos-con-group_by}}

\begin{Shaded}
\begin{Highlighting}[]
\FunctionTok{summarise}\NormalTok{(}\FunctionTok{group\_by}\NormalTok{(empleados, sexo, minoria), }\AttributeTok{sal.med =} \FunctionTok{mean}\NormalTok{(salario), }\AttributeTok{n =} \FunctionTok{n}\NormalTok{())}
\end{Highlighting}
\end{Shaded}

\begin{verbatim}
## # A tibble: 4 x 4
## # Groups:   sexo [2]
##   sexo   minoria sal.med     n
##   <fct>  <fct>     <dbl> <int>
## 1 Hombre No       44475.   194
## 2 Hombre Sí       32246.    64
## 3 Mujer  No       26707.   176
## 4 Mujer  Sí       23062.    40
\end{verbatim}

\hypertarget{operador-pipe-tuberuxeda-redirecciuxf3n}{%
\section{\texorpdfstring{Operador \emph{pipe} \textbf{\%\textgreater\% }(tubería, redirección)}{Operador pipe \%\textgreater\% (tubería, redirección)}}\label{operador-pipe-tuberuxeda-redirecciuxf3n}}

Este operador le permite canalizar la salida de una función a la entrada de otra función.
\texttt{segundo(primero(datos))} se traduce en \texttt{datos\ \%\textgreater{}\%\ primero\ \%\textgreater{}\%\ segundo}
(lectura de funciones de izquierda a derecha).

Ejemplos:

\begin{Shaded}
\begin{Highlighting}[]
\NormalTok{empleados }\SpecialCharTok{\%\textgreater{}\%}  \FunctionTok{filter}\NormalTok{(catlab }\SpecialCharTok{==} \StringTok{"Directivo"}\NormalTok{) }\SpecialCharTok{\%\textgreater{}\%}
          \FunctionTok{group\_by}\NormalTok{(sexo, minoria) }\SpecialCharTok{\%\textgreater{}\%}
          \FunctionTok{summarise}\NormalTok{(}\AttributeTok{sal.med =} \FunctionTok{mean}\NormalTok{(salario), }\AttributeTok{n =} \FunctionTok{n}\NormalTok{())}
\end{Highlighting}
\end{Shaded}

\begin{verbatim}
## # A tibble: 3 x 4
## # Groups:   sexo [2]
##   sexo   minoria sal.med     n
##   <fct>  <fct>     <dbl> <int>
## 1 Hombre No       65684.    70
## 2 Hombre Sí       76038.     4
## 3 Mujer  No       47214.    10
\end{verbatim}

\begin{Shaded}
\begin{Highlighting}[]
\NormalTok{empleados }\SpecialCharTok{\%\textgreater{}\%} \FunctionTok{select}\NormalTok{(sexo, catlab, salario) }\SpecialCharTok{\%\textgreater{}\%}
          \FunctionTok{filter}\NormalTok{(catlab }\SpecialCharTok{!=} \StringTok{"Seguridad"}\NormalTok{) }\SpecialCharTok{\%\textgreater{}\%}
          \FunctionTok{group\_by}\NormalTok{(catlab) }\SpecialCharTok{\%\textgreater{}\%}
          \FunctionTok{mutate}\NormalTok{(}\AttributeTok{saldif =}\NormalTok{ salario }\SpecialCharTok{{-}} \FunctionTok{mean}\NormalTok{(salario)) }\SpecialCharTok{\%\textgreater{}\%}
          \FunctionTok{ungroup}\NormalTok{() }\SpecialCharTok{\%\textgreater{}\%}
          \FunctionTok{boxplot}\NormalTok{(saldif }\SpecialCharTok{\textasciitilde{}}\NormalTok{ sexo}\SpecialCharTok{*}\FunctionTok{droplevels}\NormalTok{(catlab), }\AttributeTok{data =}\NormalTok{ .)}
\FunctionTok{abline}\NormalTok{(}\AttributeTok{h =} \DecValTok{0}\NormalTok{, }\AttributeTok{lty =} \DecValTok{2}\NormalTok{)}
\end{Highlighting}
\end{Shaded}

\begin{center}\includegraphics[width=0.7\linewidth]{24-dplyr_files/figure-latex/unnamed-chunk-13-1} \end{center}

\begin{center}\rule{0.5\linewidth}{0.5pt}\end{center}

Para mas información sobre \emph{dplyr} ver por ejemplo la `vignette' del paquete:\\
\href{http://cran.rstudio.com/web/packages/dplyr/vignettes/introduction.html}{Introduction to dplyr}.

\hypertarget{compauxf1uxedas-que-usan-r}{%
\chapter{\texorpdfstring{Compañías que usan \texttt{R}}{Compañías que usan R}}\label{compauxf1uxedas-que-usan-r}}

Cada vez son más las empresas que utilizan \texttt{R}.

\begin{itemize}
\item
  Grupo de empresas que apoyan a la Fundación R y a la comunidad R.

  \includegraphics[width=0.4\textwidth,height=\textheight]{figuras/rconsortium2.png}
\item
  Otras compañías:

  \begin{itemize}
  \tightlist
  \item
    Facebook, Twitter, Bank of America, Monsanto, \ldots{}
  \end{itemize}
\end{itemize}

\hypertarget{microsoft}{%
\section{Microsoft}\label{microsoft}}

\includegraphics[width=0.2\textwidth,height=\textheight]{figuras/Revolution.jpeg}

\begin{itemize}
\item
  Herramientas para entornos Big Data y computación de altas prestaciones.
\item
  Versión de R con rendimiento mejorado.

  \begin{itemize}
  \item
    Microsoft R Application Network:

    MRAN: \url{https://mran.microsoft.com}
  \end{itemize}
\item
  Integracion de R con: SQL Server, PowerBI, Azure y Cortana
  Analytics.
\end{itemize}

\hypertarget{rstudio-com}{%
\section{RStudio (Posit)}\label{rstudio-com}}

\includegraphics[width=0.4\textwidth,height=\textheight]{figuras/rstudio_stickers.png}

Además del entorno de desarrollo (IDE) con múltiples herramientas,
descrito en la Sección \ref{rstudio}:

\begin{itemize}
\item
  RStudio Server: permite ejecutar RStudio en un servidor mediante una interfaz web.

  \begin{itemize}
  \item
    Evita el movimiento de datos a los clientes.
  \item
    Ediciones Open Source y Professional (RStudio Workbench).
  \end{itemize}
\end{itemize}

\begin{itemize}
\item
  Compañía muy activa en el desarrollo de R:

  \begin{itemize}
  \item
    Múltiples paquetes: tidyverse (dplyr, tidyr, ggplot2, knitr, \ldots), tidymodels, shiny, rmarkdown, \ldots{}
  \item
    Hadley Wickham (Jefe científico de RStudio).
  \end{itemize}
\end{itemize}

En la Sección \protect\hyperlink{links}{Enlaces} de las Referencias se incluyen recursos adicionales.

  \bibliography{book.bib,packages.bib}

\end{document}
